\chapter{基于CIM的构件级洪灾损伤评估与三维可视化}

在前两章中,本文分别构建了面向城区的三维水动力风险评估体系和室内外一体化疏散路径规划模型,实现了“危险感知—路径优化”的实时联动。然而,灾害全过程治理不仅仅停留在风险评估和疏散规划层面,还需对洪水作用后的建筑构件损伤、经济损失以及修复优先级进行更为精细的评估。这一评估将为应急恢复与灾后风险复盘提供必要的支持。

洪灾损伤评估 Flood Damage Assessment,简称 FDA,是基于风险的洪水管理方法中的关键组成部分\cite{Thieken2005}。目前,FDA 方法往往忽视建筑物的独特性,采用基于类别的简化处理方式,这使得它难以满足需要逐案分析建筑损伤的详细应用需求。尤其是在使用不完整且质量较低的建筑数据输入时,这一局限性更加明显。此外,关于建筑几何形状和材料组成的假设及近似处理,也可能导致评估结果的不完整性和不确定性\cite{Merz2010}。

随着三维城市建模和建筑信息模型 BIM 在城市管理领域的广泛应用,本章在城市信息模型 CIM 框架下,借鉴了微尺度洪灾损伤评估的相关研究成果\cite{Amirebrahimi2016},提出了一个集成框架。该框架利用详细的三维建筑模型进行洪灾损伤评估,并实现三维可视化。通过结合第二章中输出的洪水参数与第三章的 BIM 与 MGNM 数据,框架成功构建了“风险评估$\rightarrow$疏散路径$\rightarrow$构件损伤”的多尺度闭环。

与传统 FDA 方法相比,本章提出的评估方法具有以下优势:
(1)通过利用 BIM 和 CityGML 的丰富语义信息,实现了建筑构件的精确几何建模和材料表征;
(2)采用时变洪水作用分析,揭示了洪水对建筑物的动态影响过程;
(3)构建了多物理场耦合的损伤判据,综合考虑了静水压力、动水压力、浮力和水接触等多种损伤机制;
(4)实现了构件级损伤的三维可视化,从而增强了决策支持的直观性和有效性。

本章提出的基于 CIM 的洪灾损伤评估框架主要包括数据准备、物理损伤评估、损失量化和可视化报告四个阶段,如图~\ref{fig:cim_framework} 所示。该框架充分利用了三维城市模型在几何精度和语义丰富度方面的优势,使得建筑物个体层面的损伤分析得以精细化实现。

在具体研究路径上,本章在第二章水动力风险评估和第三章室内外一体化疏散路径规划的基础上,引入构件级损伤分析作为后灾阶段的关键环节,实现了从外部洪水情景到建筑内部功能退化的持续跟踪。通过在 CIM 中显式表示“城市—建筑—房间—构件”的多层次对象体系,洪水模拟结果可以以统一的数据结构传递至构件级损伤模块,并进一步映射为经济损失与功能恢复指标。这种由上而下的逐级细化,使得宏观风险评估与微观构件响应之间建立起可追溯的逻辑链条。

此外,本章的构件级评估框架兼顾了工程可操作性和模型可扩展性。一方面,损伤判据与成本映射均尽量采用规范条文、设计图纸和现有定额中的参数,便于在工程实践中直接落地;另一方面,各模块之间通过 CIM 数据总线以松耦合方式衔接,未来可以在不改变总体架构的前提下,引入更精细的材料退化模型或概率风险分析方法,为后续扩展留出空间。由此生成的构件级损伤与经济损失结果将在第六章中作为多主体协同应急管理系统的核心输入,用于重点保护对象识别与资源优化配置。

\begin{figure}[htbp]
	\centering
	\includegraphics[width=0.8\textwidth]{PIC-5/fig5_framework.jpg}
	\caption{基于CIM的洪灾损伤评估框架}
	\caption*{Figure~\thefigure~ CIM-based Flood Damage Assessment Framework} 
	\label{fig:cim_framework}
\end{figure}

\section{数据准备与统一信息模型}

\subsection{建筑信息与语义扩展}

BIM 模型采用 IFC 2\,×\,3 标准,包含墙体、梁板、立面构件、门窗、楼梯等 612 个构件实例。为了适应损伤分析需求,针对 BIM 元素进行了以下语义扩展:
\begin{enumerate}
	\item 耐水性能等级:按照《建筑防水工程技术规范》将材料分为三级,分别为 A 级耐水、B 级短时耐水和 C 级易损,并将该信息附加到 IfcMaterial 层级。
	\item 力学参数:为承重墙、幕墙、门窗等构件附加密度、抗压强度、抗剪强度、弹性模量等力学参数,这些数据来源于设计图纸和标准手册。
	\item 连接关系:通过 IfcRelConnectsElements 和 IfcRelSpaceBoundary 重建构件之间的拓扑关系,用以识别洪水作用下的受力路径和渗透通道。
\end{enumerate}
图~\ref{fig:bim_model} 展示了语义增强后的 BIM 模型,后续的损伤可视化将基于该模型进行。

在此基础上,本研究进一步按照“结构构件—围护构件—机电及附属构件”的分层方式对 BIM 模型进行归类。结构构件包括剪力墙、梁柱、楼板和基础等,主要承受静水压力、动水压力和浮力引起的内力变化。围护构件包括外墙、幕墙、填充墙、门窗和屋面,是洪水入侵和水接触损伤的关键承载体。机电及附属构件涵盖配电箱、机房设备、应急照明、弱电终端、家具和物资柜等,其受损往往直接影响建筑功能的可恢复性。分层建模使不同类型构件在后续损伤判定和成本映射中可以采用差异化规则,避免粗粒度的一刀切处理。

在 CIM 层面,BIM 构件通过唯一标识例如 IFC GUID 与城市对象语义矩阵中的相应条目建立一一对应关系,使“构件—房间—建筑—街区”之间的层级结构在数据上保持一致。每个构件不仅保存其几何、材性和功能属性,还关联所在房间的用途、楼层标高、疏散节点邻接关系等信息,从而实现从构件追溯到场景的语义闭环。为保证评估结果的可靠性,本研究在语义扩展完成后引入几何合法性检查、属性完整性检查以及与资产台账的交叉核对,将语义缺失或异常构件的比例控制在较低水平,减少后续损伤推断的不确定性。

\begin{figure}[htbp]
	\centering
	\includegraphics[width=0.8\textwidth]{PIC-5/fig5_bim_model.jpg}
	\caption{滨海应急服务中心的BIM模型,语义增强之后}
	\caption*{Figure~\thefigure~ BIM Model of the Coastal Emergency Service Center after Semantic Enhancement} 
	\label{fig:bim_model}
\end{figure}

\subsection{洪水参数与时间序列}

第二章中,通过 MIKE 21 得到的水深 $h(x,y,t)$、流速 $U(x,y,t)$ 和持续时间 $T(x,y)$ 等数据,通过双线性插值映射至建筑足迹,并将数据沿入口与竖向交通节点投影至各楼层。时间维度采用 $\Delta t = 300\,\mathrm{s}$ 的等间隔离散,步长与 Kelman 渗透模型所需的步长一致。为了保证局部水动力效应的准确性,对建筑周边 20 m 范围内的模拟结果进行了 0.5 m 网格细化处理。

考虑到洪水演进通常经历上升、高位停留和消退三个阶段,本研究对 MIKE 21 输出结果进行特征时段划分,标记初次漫堤时刻、最大水深时刻以及水位回落到安全阈值的时刻,并在 CIM 平台中创建相应的洪水事件对象。这些事件以统一时间戳驱动构件状态随时间的更新,使损伤评估不再局限于单一时刻,而是能够刻画构件在整个淹没过程中的累积效应。

在空间映射方面,针对建筑外立面的冲刷作用,采用基于法向的局部插值策略。沿建筑外轮廓提取一圈采样点,将二维水深场根据地形和建筑高度投影到立面单元上,从而获得与墙面构件相匹配的水深和流速组合。对于室内空间,则利用第三章构建的室内外一体化通行网络,将门窗节点处的水位变化作为边界条件,通过简化的一维水柱模型估计各房间的水位和浸泡时间序列,在外部二维流场、建筑足迹与室内房间之间建立多级耦合关系。

\subsection{成本与维修库}

构件成本主要来源于以下几个方面:
\begin{itemize}
	\item 山东省 2023 年《建筑装饰工程消耗量定额》和《房屋修缮工程消耗量定额》,提供了常见装饰和设施的基准单价及折旧规则;
	\item 滨海应急服务中心资产台账,补充了专用设备例如配电箱、应急物资柜的采购与折旧信息;
	\item 威海市公共资源交易中心价格信息,用于校核市场价格波动。
\end{itemize}
成本库采用 MasterFormat 分类编码,并与 BIM 构件一一对应,使得损伤评估能够直接输出货币化的量化结果。

在构建成本与维修库时,本研究遵循来源可追溯、口径可统一和时点可更新的原则。对于室内墙面饰材、地面铺装等面积型构件,以单位面积造价为基础,结合折旧年限和残值率,将其折算为评估时点上的重置成本。对于门窗组件、机电设备等件数型构件,根据设备采购合同和资产盘点结果,区分完全损坏需要更换、可以修复但需停用以及仅外观受损三类情形,分别给出对应的维修或更换单价。

为反映不同功能区域的重要程度,成本库中引入功能权重系数。位于配电间、通信机房等关键空间内的构件在相同损伤等级下,其有效损失通过乘以权重系数进行放大,用以体现其对系统功能恢复的影响。成本库预留年度更新机制,当材料价格、人工成本或资产折旧政策发生变化时,可以通过批量更新方式调整单价和折旧参数,保证评估结果与实际经济环境保持一致性。

\section{构件级洪灾损伤评估模型}

\subsection{洪水作用计算与构件损伤判据}

损伤评估同时考虑水动力和水接触两类效应。参考 FEMA\cite{FEMA2012}、Kelman\cite{Kelman2002} 等规范,建立了如下计算流程。图~\ref{fig:damage_process} 展示了物理损伤评估的详细流程,涵盖洪水作用计算、构件阻力分析和损伤状态判定等关键环节。

\begin{figure}[htbp]
	\centering
	\includegraphics[width=0.8\textwidth]{PIC-5/fig5_damage_process.jpg}
	\caption{物理损伤评估过程流程图}
	\caption*{Figure~\thefigure~ Flowchart of the Physical Damage Assessment Process} 
	\label{fig:damage_process}
\end{figure}

\begin{enumerate}
	\item 水头差与渗透:根据室外水深 $h_{\mathrm{out}}$、室内水深 $h_{\mathrm{in}}$ 以及开口构件例如门、窗、通风口的渗透系数 $C_k$,采用 Kelman 渗透模型计算每个时间步的水量交换 $I_k(t)$,并更新室内水位。图~\ref{fig:water_levels} 展示了室内外水位变化过程,结果表明,由于建筑物开口的渗透作用,室内水位逐渐上升,最终与室外水位趋于平衡。
	\item 静水压力:对于每个墙体、幕墙等结构构件,采用公式
	\begin{equation}
		p_{\mathrm{hyd}} = \frac{1}{2} \rho g (h_{\mathrm{out}}^2 - h_{\mathrm{in}}^2)/t_w,
	\end{equation}
	估算净静水压力,其中 $t_w$ 为墙厚。如果 $p_{\mathrm{hyd}}$ 超过构件抗压强度的 60\%,判定为结构受限。
	\item 动水压力:依据流速分量 $U_\perp$,采用公式
	\begin{equation}
		p_{\mathrm{dyn}} = 0.6 \rho U_\perp^2 
	\end{equation}
	计算迎水面构件的附加压力,并考虑正压和负压差引起的吸附破坏。
	\item 浮力:对于地下或半地下空间构件,根据排水体积 $V$ 与外部水深,计算浮力 $F_{\mathrm{buoy}} = \rho g V$,如果浮力与结构自重之比超过 0.8,则标记为存在脱离风险。
	\item 水接触损伤:根据材料的耐水等级与浸泡时间 $t_{\mathrm{wet}}$,采用指数模型 $D = 1 - e^{-\alpha t_{\mathrm{wet}}}$ 评估材料损伤程度,其中参数 $\alpha$ 由材料等级确定,例如木质地板 $\alpha = 0.12$,PVC 地板 $\alpha = 0.04$。
\end{enumerate}

将上述判据综合,得到构件损伤状态,例如结构破坏、存在脱离风险、材料损伤以及安全状态,并记录触发的主导机制,便于后续可视化说明损伤成因。

在工程实现层面,水动力作用首先通过网格化方式分配到 BIM 构件表面单元。对于墙体和幕墙等平面构件,将水深沿高度方向离散为若干节点,分别计算对应的静水压力和动水压力,再采用等效均布方法将三角形分布荷载换算为作用于构件质心的等效压力。对于柱和桩等线性构件,在横向投影面积上计算迎水面压力,并考虑流向与构件轴线夹角对有效受力面积的影响。对于地下室外墙和底板,需要同时考虑浮力和反向水压力,构建上浮、开裂和渗漏等综合判据。

在材料退化方面,指数模型中的参数 $\alpha$ 随材料类型、施工质量和维护状况而变化。为避免参数不确定性对结果造成过大影响,本研究将材料按耐水性能划分为若干等级,为每个等级选取一组代表性参数区间,并在敏感性分析中评估参数变化对损伤率的影响。数值试验表明,在合理的参数范围内,构件损伤空间分布格局保持稳定,模型对关键区域的识别具有一定鲁棒性。

为便于后续损失量化,本研究将构件损伤划分为四个等级。记 $DS_0$ 为无损伤,$DS_1$ 为轻微损伤,仅外观受到影响,$DS_2$ 为中等损伤,功能受限但可以修复,$DS_3$ 为严重损伤,需要更换或大修。各构件的损伤等级在 CIM 中以属性形式维护,与构件几何和语义信息直接关联。

\begin{figure}[htbp]
	\centering
	\includegraphics[width=0.8\textwidth]{PIC-5/fig5_water_levels.jpg}
	\caption{建筑物室内外水位变化过程}
	\caption*{Figure~\thefigure~ Variation of Indoor and Outdoor Water Levels in the Building} 
	\label{fig:water_levels}
\end{figure}

\subsection{损失量化与关键指标}

损失量化遵循 Assembly-Based Vulnerability 思想,简称 ABV\cite{Porter2001},将构件损伤与成本库关联,计算公式为
\begin{equation}
	C_{\mathrm{loss}} = \sum_{i} \gamma_i \cdot c_i,
\end{equation}
其中 $c_i$ 为构件 $i$ 的维修成本或更换成本,$\gamma_i$ 为损伤系数,取值依据损伤等级,例如完全损坏时系数为 1,可修复时系数为 0.4,只有外观修复需求时系数为 0.1。

在本研究中,构件级物理损伤与经济损失之间通过损伤等级与损伤系数建立了显式映射关系。首先,根据上一小节中得到的构件损伤判定结果,将各构件划分为 $DS_0$~$DS_3$ 四类状态;随后针对不同构件类型(如墙面饰材、门窗、低位电气设备等)为各等级指定相应的损伤系数 $\gamma_i$,$\gamma_i$ 表征构件在当前损伤状态下需要维修或更换的成本比例。这样一来,损伤判据中体现的水动力效应与材料耐水性能通过 $\gamma_i$ 投射到经济维度,结合成本库中的单位造价 $c_i$,即可将离散的损伤状态映射为连续的经济损失量,并在构件、房间、楼层乃至整栋建筑等不同空间层级上进行聚合。

为支持韧性评估,定义以下指标:
\begin{itemize}
	\item 构件损伤率 $R_{\mathrm{comp}} = N_{\mathrm{dam}} / N_{\mathrm{tot}}$,表示达到损伤状态的构件数量占总构件数量的比例;
	\item 功能恢复时间 $T_{\mathrm{rec}}$,由配电、通信等关键设施的损伤程度和维修顺序映射得到;
	\item 投资回收比 $B\!C\!R = \frac{\text{改造成本}}{\text{预期损失差}}$,用于评估韧性改造方案的经济性。
\end{itemize}
表~\ref{tab:cost_basis} 汇总了典型构件的成本和折旧假设。

在具体应用中,构件损伤率可以按构件类型、楼层或功能分区进行分解,用于识别易损构件的空间集聚特征和薄弱楼层。功能恢复时间通过构件损伤组合与维修工艺顺序构建关键路径网络,并结合资源约束估算关键功能恢复所需时间。通过对比不同改造方案下构件损伤率、经济损失和功能恢复时间的变化情况,可以形成多指标韧性评估框架。

在多情景分析中,上述指标还可以扩展为潜在最大损失 Probable Maximum Loss,简称 PML,以及期望年损失 Expected Annual Loss,简称 EAL 等风险指标。潜在最大损失用于反映给定重现期洪水情景下的最大经济损失估计,期望年损失通过对不同情景损失与发生概率的加权计算得到。尽管本章案例主要聚焦单一设计情景,但本节提出的指标体系为将构件级损伤评估拓展到概率风险分析提供了基础。上述从损伤等级到经济损失及派生指标的映射,为后续在协同系统中识别高损失集聚区和重点保护对象、开展多方案比选提供了量化基础。

\begin{table}[htbp]
	\centering
	\caption{损失评估单价与折算依据}
	\caption*{Table~\thetable~ Cost and depreciation assumptions for loss assessment} 
	\label{tab:cost_basis}
	\begin{tabular}{@{}p{3cm}p{3.5cm}p{3cm}p{3cm}@{}}
		\toprule
		构件类别 & 单价及单位 & 折旧与折算方法 & 数据来源 \\
		\midrule
		室内墙面饰材 & 每平方米 320 元 & 5 年直线折旧,残值率 5\% & 山东省装饰工程定额 2023 年版 \\
		地面铺装 & 每平方米 450 元 & 5 年直线折旧,残值率 5\% & 山东省装饰工程定额 2023 年版 \\
		低位电气设备 & 每台 6800 元 & 以采购价的 80\% 计入,可再利用部分扣除残值 & 威海市公共资源交易中心价格信息 2023 年 \\
		门窗组件 & 每樘 950 元 & 按完好率估算,框体损坏时全部计入 & 威海市装饰建材市场调研 \\
		家具及应急物资 & 每件 420 元 & 使用寿命 3 年折旧,残值率 10\% & 滨海中心资产台账 2023 年 \\
		\bottomrule
	\end{tabular}
\end{table}

\section{三维可视化表达与协同应用}

损伤结果通过三维可视化进行表达,以提高跨部门沟通效率:
\begin{itemize}
	\item 构件着色:依据损伤等级,采用红色、橙色、黄色和绿色的色彩体系,红色表示需要更换,橙色表示结构受限,黄色表示局部维修,绿色表示安全;
	\item 机制标注:利用 BIM 属性显示主导损伤机制,标出静水压力、动水压力、浮力和水接触等因素,便于定位问题根源;
	\item 交互查询:结合 CityGML 的空间索引,实现按楼层、构件类型或损失金额的过滤和统计。
\end{itemize}
图~\ref{fig:damage_visualization} 展示了构件级损伤的三维表达方式。受损的门窗以红色高亮,墙面饰材因水接触损伤以黄色显示,未受影响的幕墙保持绿色,用户可以通过图层控制聚焦关键区域。

在视图组织方面,可视化系统采用城市层、建筑层和构件层三种视图联动。城市层叠加显示洪水淹没范围、建筑危险等级和关键基础设施分布,为区域级应急部署提供整体态势。建筑层通过半透明处理建筑外壳、采用楼层剖切和局部放大,将构件损伤信息嵌入到结构框架中。构件层允许用户选中任一构件,查看其损伤等级、主导损伤机制、预计维修成本和功能恢复时间等属性,实现直观查询和精细管理。

为支撑跨部门协同,可视化系统中设计了多种专题图层,如经济损失分布、关键设备受损和资产价值空间分布等。应急管理部门可以结合经济损失和功能恢复时间图层优化应急资源配置,设施管理部门可以利用构件损伤和维修成本图层制定分阶段修复计划,保险机构可以基于资产价值和损伤等级图层辅助完成定损和保费调整。同时,通过在界面中突出显示损伤严重且经济损失集中的构件群,可为后续协同系统中的重点保护对象管理提供直观的空间依据。

此外,本章构建的可视化模块与第三章的室内外疏散路径规划结果实现联动。在播放洪水情景演进时,系统可以同步高亮显示受损构件和可通行路径。当某些门窗或楼梯由于损伤而失效时,对应路径会自动标记为不可用,有助于指挥人员快速判断替代路径,形成风险、疏散和损伤信息一体化的三维表达。

\begin{figure}[htbp]
	\centering
	\includegraphics[width=0.9\textwidth]{PIC-5/fig5_damage_visualization.jpg}
	\caption{构件级洪灾损伤的三维可视化表达}
	\caption*{Figure~\thefigure~ 3D Visualization of Component-Level Flood Damage}
	\label{fig:damage_visualization}
\end{figure}

\section{案例研究:威海滨海应急服务中心构件级洪灾损伤评估}

\subsection{场景设定与模型实例化}

案例选择威海滨海应急服务中心首层和半地下功能区。场地紧邻北岸防潮堤,地面高程低于周边道路 0.4 m,容易在风暴潮与暴雨叠加条件下形成积水。选取 2021 年烟花台风在威海附近登陆时段的极端水位过程作为输入,外部水深峰值为 0.68 m,持续时间约 2.5 小时;降雨量采用对应暴雨站点一小时最大雨强 34 mm。第二章得到的 0.5 m 分辨率风险栅格与第三章建立的 MGNM 网络在 CIM 平台内完成坐标统一,并通过 IFC 空间边界建立首层 246 个构件和半地下 58 个构件之间的关联关系。时间维度沿用时间步长 $\Delta t = 300\,\mathrm{s}$ 的离散方案,总计 30 个时间步覆盖整个事件过程。

选择滨海应急服务中心作为试点对象,一方面在于其兼具应急指挥、物资储备和公共服务等多重功能,是区域洪涝灾害应对中的关键节点,另一方面该建筑的 BIM 和 CIM 数据较为完备,便于开展从水动力模拟、构件损伤评估到三维可视化的一体化实践。通过该案例可以较为全面地检验本章方法在复杂功能建筑上的适用性和可扩展性。

在数据映射过程中,首先根据地形与建筑轮廓将 MIKE 21 输出的二维水深场裁剪至研究区域,再通过建筑足迹多边形将水深和流速序列与建筑外墙和外广场面元建立对应关系。随后结合首层和半地下楼层标高、入口门槛高度以及排水沟渠布设,将外部水位过程转化为室内水位边界条件,并在 CIM 中以时间序列形式存储,为渗透计算与构件损伤评估提供输入。

\subsection{联动评估流程与系统运行}

案例验证遵循水动力响应、疏散约束和构件损伤三部分组成的闭环流程:
\begin{enumerate}
	\item 风暴潮与暴雨引起的外部水深、流速和压力通过三维 RANS 模型计算,并映射至建筑外墙、入口和地下空间的控制面;
	\item 第三章构建的风险驱动疏散图层提供入口通行能力和滞留时间等信息,这些信息作为门窗渗透系数和功能恢复优先级的约束条件;
	\item 依据 Kelman 渗透模型和多物理损伤判据,计算各构件的损伤状态,生成与 BIM 构件绑定的损伤标签、主导机制和维修成本;
	\item 将结果同步回写至 CIM 平台,生成洪水淹没体、构件损伤点云和业务提醒信息,例如机房存在浮力风险,供应急指挥系统联动调度使用。
\end{enumerate}

该流程在 CIM 数据总线中保持自动化,可以在约 14.6 分钟内完成一次完整演算,满足小时级滚动推演的需求。在计算实现层面,评估引擎采用按楼层和功能区划分任务块的方式进行多线程并行计算,每个任务块包含若干构件及其对应的水动力时间序列,由独立计算线程负责完成水荷载计算和损伤判据判断。水接触损伤部分利用向量化运算,对相同材料等级构件的浸泡时间序列进行批量处理。在当前案例规模下,并行化与向量化策略可以将构件级损伤评估时间压缩到分钟量级,为在应急演练和情景推演中实现近实时更新提供了可能。

\subsection{量化结果与模型验证}

损伤评估结果与现场资料和历史演练记录进行了对比,关键指标如表~\ref{tab:case_metrics} 所示。

\begin{table}[htbp]
	\centering
	\caption{威海滨海应急服务中心案例关键指标}
	\caption*{Table~\thetable~ Key Indicators of the Weihai Coastal Emergency Service Center Case} 
	\label{tab:case_metrics}
	\begin{tabular}{@{}p{4cm}p{5cm}p{5cm}@{}}
		\toprule
		指标 & 模型结果 & 参照与验证依据 \\
		\midrule
		构件损伤率 $R_{\mathrm{comp}}$ & 总体为 27\%,其中门窗构件损伤率为 62\%,墙面构件损伤率为 54\% & 2021 年 8 月积水事件照片和运维记录 \cite{WeihaiEM2022} \\
		静水压力峰值 & 北侧外墙达到 38.5\,kPa,对应抗压强度的 65\% & 结构设计图纸和验算资料 \\
		浮力系数 & 半地下机房浮力系数达到 0.78,接近临界状态 & 现场排水能力评估报告 \cite{WeihaiEM2022} \\
		直接经济损失 $C_{\mathrm{loss}}$ & 约为 51.4 万元 & 2022 年资产盘点单价与保险理赔清单 \\
		功能恢复时间 $T_{\mathrm{rec}}$ & 约为 21 天,瓶颈环节为配电和通信机柜 & 2020 年应急演练恢复时长 19 天 \cite{CN_Chu2022EvacuationDrill} \\
		疏散通行能力 & 入口 E3 受损后通行能力下降 48\% & 第三章疏散仿真中入口 E3 风险代价提升 52\% \\
		\bottomrule
	\end{tabular}
\end{table}

\begin{figure}[htbp]
	\centering
	\includegraphics[width=0.9\textwidth]{PIC-5/fig5_inundation.jpg}
	\caption{威海滨海应急服务中心洪水淹没体与构件损伤叠加可视化}
	\caption*{Figure~\thefigure~ Overlay Visualization of Flood Inundation and Component Damage at the Weihai Coastal Emergency Service Center}
	\label{fig:3d_visualization}
\end{figure}

图~\ref{fig:3d_visualization} 展示了洪水淹没体与构件损伤的叠加情况,可以直观辨识受损区域与水动力主导方向。结合图~\ref{fig:damage_visualization} 的构件着色结果,可以清晰看到首层门窗构件损伤集中于主要入水通道,墙面饰材损伤集中于长期被水浸泡区域,半地下机房围护结构存在明显浮力风险。与现场水位尺记录对比,模拟峰值水深误差控制在 0.06 m 以内,表明水动力和渗透模型在本案例中具有较高可信度。

从量化结果来看,构件损伤主要集中在首层外门窗、室内墙面饰材以及半地下机房的围护结构区域。首层门厅和走廊区域处于主要入水通道,门窗构件长期承受水头差和动水压力,导致损伤率较高。半地下机房在洪水高位停留阶段浮力系数接近临界值,在排水能力不足或防水构造存在缺陷时存在明显上浮风险。将模型结果与历史演练记录和运维检修记录对比可以发现,高损伤区域与实际易积水和高频维修区域高度重叠,从侧面验证了构件级损伤评估的合理性。

从损失构成角度看,首层外门窗和相邻墙体饰面虽然单体造价相对有限,但在数量上占比高、且多处于 $DS_2$~$DS_3$ 损伤等级,对总损失 $C_{\mathrm{loss}}$ 的贡献占据较大份额;半地下机房内的配电、通信等低位机电设备数量有限,但单体成本高、功能关键性强,即便损伤件数不多,其对应的 $\gamma_i c_i$ 项在经济损失和功能恢复时间 $T_{\mathrm{rec}}$ 中的边际影响仍然显著。这种“数量型构件”与“价值型构件”共同构成的损伤热点,为后续韧性提升和投资排序提供了清晰靶向。

综合构件损伤率、经济损失贡献和对关键业务的影响,可将首层门厅及走廊带、外门窗带以及半地下机房及其围护结构识别为本案例中的“高风险—高价值”区域。在下一章的协同系统中,这些区域及其所包含的关键构件将作为重点保护对象优先纳入监测、加固和快速恢复序列。

\subsection{韧性提升策略与业务协同}

基于损伤结果和敏感性分析,本研究提出以下业务协同建议:
\begin{itemize}
	\item 设施改造优先级方面,在入口 E3 增设高度为 0.9 m 的防水闸门和门窗密封条,可以将门窗损伤率降低到约 28\%,并在疏散模型中恢复约 85\% 的入口通行能力。在半地下机房布设抗浮锚杆并配置备用泵站后,浮力系数可以降至 0.42,功能恢复时间缩短到约 12 天;
	\item 应急演练方面,将洪水淹没体与损伤清单输出到市应急管理局指挥平台,形成预警、封控、排水和复盘等环节组成的演练脚本,弥补传统演练中缺乏量化损伤环节的不足;
	\item 保险与资产管理方面,构件级损失明细可以作为理赔和资产盘点的客观依据,并为年度韧性投资的成本收益分析提供数据支撑,有助于避免资源投入的主观性分配。
\end{itemize}

综合来看,案例验证表明所构建的损伤评估模块不仅能够与第二章和第三章的模型无缝衔接,还能够在业务层面支撑设施改造、应急演练和资产管理等多类决策,体现了 CIM 平台在实际应急场景中的可操作性。在此基础上,若进一步结合多情景风险分析和资金约束条件,可以将构件级损伤结果纳入年度韧性改造滚动计划,形成风险、损伤、投资和收益之间的闭环管理模式。通过对不同改造组合方案的成本收益比进行排序,可以量化单位投资带来的损失降低和功能恢复收益,从而为有限资金条件下的项目优先级排序提供科学依据。

需要指出的是,构件级损伤评估所揭示的高损伤率、高损失贡献和较长恢复时间的构件与区域,可直接转化为协同系统中的“重点保护对象”清单。通过在系统中对这些对象赋予更高的风险权重和调度优先级,可以在资源与时间均受约束的条件下,有针对性地优化封控布设、抢修顺序和物资投放,实现整体风险水平在可接受范围内的最大程度降低。

\section{本章小结}

本章在城市信息模型 CIM 框架下,围绕构件级洪灾损伤评估与三维可视化展开研究,并与前两章的三维水动力风险评估和室内外一体化疏散路径规划形成多尺度联动。通过在同一数据体系中集成水动力荷载、构件响应、经济损失和功能恢复等要素,本章方法为关键建筑的精细化风险认知提供了构件层面的技术支撑。主要结论如下:

\begin{enumerate}
	\item 构建了面向构件级洪灾损伤评估的 CIM–BIM 集成框架。通过对 IFC 模型进行语义扩展,补充材料耐水等级、力学参数和功能属性,并将第二章输出的水深、流速及持续时间场统一映射到构件表面与室内空间边界,形成“水动力荷载—构件响应”的完整链条。该框架在保持工程可操作性的同时,兼顾了模型的可扩展性,为引入更精细的材料退化模型或概率风险分析方法预留了接口。
	
	\item 采用 Assembly-Based Vulnerability 思想,将构件损伤等级 $DS_0$~$DS_3$ 与损伤系数 $\gamma_i$、单位造价 $c_i$ 建立显式映射,实现了从物理损伤状态到经济损失 $C_{\mathrm{loss}}$ 的逐级推演。进一步引入构件损伤率 $R_{\mathrm{comp}}$、功能恢复时间 $T_{\mathrm{rec}}$ 和投资回收比 $BCR$ 等指标,使得损伤程度、恢复过程与改造成本在统一指标体系下可比,为识别薄弱环节和评估韧性提升方案提供了多维度量化依据。
	
	\item 以威海滨海应急服务中心为对象,对构件级损伤分布和经济后果进行了验证。结果表明,总体构件损伤率约为 27\%,其中门窗构件损伤率约为 62\%,墙面饰材损伤率约为 54\%,直接经济损失约为 51.4 万元,功能恢复时间约为 21 天。损伤和损失主要集中于首层门厅及走廊的外门窗带以及半地下机房围护结构和低位机电设备区域,这些“高风险—高价值”区域与历史积水记录和运维台账高度吻合,说明所构建的构件级评估方法具备较好的可信度和解释力。
	
	\item 本章形成的构件级损伤评估与三维可视化结果,为后续协同系统提供了关键输入。一方面,通过在 CIM 平台上以专题图层形式呈现构件损伤等级、经济损失贡献和恢复时间,可直接支撑重点保护对象的识别及其空间管理;另一方面,与第三章疏散路径规划结果联动后,可在洪水演进过程中同步更新可通行路径与受损构件状态,为多部门联合会商、封控与排水调度以及韧性改造投资排序提供数据基础。
\end{enumerate}

总体而言,本章打通了从水动力荷载场到构件损伤、经济损失与功能恢复指标的多尺度通道,使城市关键建筑的风险分析从“建筑级”进一步细化到“构件级”。这一结果不仅完善了前两章提出的风险感知与疏散优化链条,也为下一章多主体协同应急管理系统的构建提供了可直接调用的损伤与损失信息,为实现防灾减灾与城市韧性提升的一体化管理奠定了基础。

