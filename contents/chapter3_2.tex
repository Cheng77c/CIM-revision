\chapter{基于CIM的室内外一体化疏散路径规划}

\section{问题描述与总体思路}
上一章建立了基于CIM的高精度洪涝风险评估方法,三维水动力模型输出的水深$h$、流速大小$U=\|\mathbf{u}\|$与淹没持续时间$T$被整合为归一化危险度场$H(\mathbf{x})\in[0,1]$,并进一步与暴露度和脆弱性耦合得到风险场$R(\mathbf{x})\in[0,1]$。这些结果为本章的疏散路径规划提供了空间--时间约束条件。

在洪涝/内涝情景下,本章的核心目标是:在给定的时间$t$,在室内外统一网络上为任意起点$O$与终点$D$寻找一条或多条\emph{风险感知疏散路径},使得沿途综合代价最小,同时尽量避开高水深、高流速与高脆弱性区域。为此,将疏散空间抽象为有向图$G=(\mathcal{V},\mathcal{E})$,其中$\mathcal{V}$为节点集合(房间、门、楼梯、入口、道路交叉口等),$\mathcal{E}$为边集合(可通行走廊、楼梯段、室外道路、室内外传输弧等),每条边$e\in\mathcal{E}$具有几何长度$\ell(e)$、风险暴露$\rho(e)$和拥挤惩罚$\kappa_t(e)$等属性。第4.5节将在此网络上构建综合代价函数$c_t(e)$并开展路径搜索。

结合前人研究与本研究特点,室内外一体化疏散主要面临三类挑战:

(1)风险约束的多源性与动态性:疏散路径不仅受几何距离约束,还必须显式考虑上一章给出的洪涝风险栅格(道路积水、低洼区、地下空间等)以及实时传感器更新的水位与人群密度信息;

(2)室内外空间表达的一致性:BIM侧重建筑内部构件与空间拓扑,GIS侧重室外道路与地表环境,两者语义和几何表达不同,难以直接构建统一疏散网络;

(3)多尺度计算效率:在精细建模的前提下,单栋建筑可产生数千节点、数万条边;若室内外整体一次性求解,邻接矩阵规模巨大,计算时间难以满足应急指挥的实时性要求。

针对上述问题,本章提出基于CIM的\emph{多用途几何网络模型}(Multi-purpose Geometric Network Model, MGNM)及其在洪涝场景下的应用。总体思路如下:

(1)在室内侧,利用BIM/IFC自动构建MGNM,将房间、走廊、门窗、楼梯等要素统一抽象为具有三维几何与语义属性的节点和边,并附加通行能力、净宽、材料等信息,为风险约束和容量约束提供载体;

(2)在室外侧,基于道路网络与CityGML LOD1建筑模型构建城市级疏散框架,并叠加上一章得到的水深/流速/风险栅格,将洪涝危险度映射到道路与广场边上;

(3)在建筑入口处设计\emph{入口到街道的网络连接策略}(entrance-to-street, E2S),通过传输弧将室内MGNM与室外道路网络可靠拼接,避免穿越建筑实体或进入高风险积水区;

(4)结合MGNM与E2S,在第4.5节中构建\emph{风险感知代价函数}并采用\emph{粗细结合的多尺度路径规划算法}:粗尺度在城市网络上选择候选建筑入口和室外通道,细尺度在MGNM内部精确搜索室内路径,从而兼顾精度与效率。

为与上一章保持一致,采用以下记号:$\mathbf{u}=(u,v,w)$为流速向量,$U=\|\mathbf{u}\|$为流速大小,$h$为水深;$H(\mathbf{x})\in[0,1]$为归一化危险度场,$R(\mathbf{x})\in[0,1]$为归一化风险场。疏散网络记作有向图$G=(\mathcal{V},\mathcal{E})$,边$e\in\mathcal{E}$的几何长度为$\ell(e)$,在疏散时刻$t$的拥挤惩罚为$\kappa_t(e)\in[0,1]$。后文将进一步将洪涝风险与拥挤信息嵌入边代价函数中。

\section{MGNM概念与结构}

几何网络模型(GNM)因结构简单,一直是室内路径分析的基础工具\cite{Meijers2005,Karas2006,Becker2009}。传统GNM通常将房间或空间的几何中心抽象为节点,空间之间的可通行关系抽象为边,适合进行几何意义上的最短路径分析。但在面向洪涝灾害的疏散中,传统GNM存在明显局限:

(1)缺乏\textbf{构件级语义信息}:难以直接表达门窗材料、净宽、标高、是否为防火/防水门等,无法支撑容量约束与构件级风险评估;

(2)垂直结构表达粗糙:楼梯、电梯、平台等垂向空间往往被简单近似,容易导致垂向距离误差,进而影响疏散时间估计;

(3)与BIM/IFC\textbf{自动耦合能力弱}:通常需要大量人工建模或格式转换,难以利用已有CIM/BIM成果进行快速更新;

(4)难以直接承载\textbf{多源风险与容量属性}:例如将室外洪涝风险沿入口投影到室内、在边上附加人群密度与通行能力等。

为克服上述问题,本研究引入\emph{多用途几何网络模型}(MGNM)\cite{Lee2004,Lee2008}。与传统GNM相比,MGNM在结构上具有以下关键特征(见图\ref{fig:gnm_mgnm_comparison}):

(1)节点语义更丰富:每个节点不仅有三维几何坐标,还携带来自BIM/IFC的属性,如空间类型(房间、走廊、楼梯间、平台)、用途(办公室、会议室、设备间)、门窗类型与材料、标高等。

(2)边类型更细分:边明确区分为水平路径(走廊、房间--门、门--走廊)、垂直路径(楼梯)、紧急路径(房间/走廊--窗)。不同类型边在后续代价函数中可以赋予不同权重或风险惩罚。

(3)与IFC一一对应:MGNM元素直接映射到对应IFC类别,实现从BIM到网络的自动转换,便于在CIM平台中批量生成和维护。

(4)天然适配风险与容量建模:节点和边可附加净宽$B$、坡度$g$、标高$z$、预估通行能力$C(e)$等属性,用于定义容量约束与风险暴露程度,为第4.5节代价函数提供完备的输入。

\begin{figure}[htbp]
\centering
\includegraphics[width=0.8\textwidth]{PIC-4/fig1_gnm_mgnm.jpg}
\caption{GNM与MGNM对比:(a) GNM (b) MGNM}
\caption*{Figure~\thefigure~ Comparison of GNM and MGNM: (a) GNM (b) MGNM} 
\label{fig:gnm_mgnm_comparison}
\end{figure}

MGNM元素与IFC类别的对应关系如表~\ref{tab:mgnm_ifc_mapping}所示。可以看到,房间、走廊、楼梯间等空间类由IfcSpace及其子类提供,门窗等开口元素由IfcOpeningElement及相关关系类提供;这些IFC实体在转换过程中被系统性映射为节点与边,并保留与洪涝相关的关键属性(如楼层、标高、材料、尺寸等)。

\begin{table}[htbp]
\centering
\caption{MGNM元素与IFC类别的对应关系}
\caption*{Table~\thetable~ Correspondence Between MGNM Elements and IFC Categories} 
\label{tab:mgnm_ifc_mapping}
\begin{tabular}{@{}llll@{}}
\toprule
MGNM元素 & 含义 & 元素名称 & IFC类别 \\ \midrule
\multirow{6}{*}{节点} 
& 空间 & 房间 & IfcSpace/IfcRoom \\
& 空间 & 洗手间 & IfcSpace/IfcRoom \\
& 垂直入口/空间 & 楼梯区域 & IfcSpace/IfcStair \\
& 垂直入口/空间 & 平台区域 & IfcSpace \\
& 水平入口 & 门 & IfcOpeningElement \\
& 紧急入口 & 窗 & IfcOpeningElement \\ \midrule
\multirow{6}{*}{边} 
& 水平路径 & 走廊 & IfcSpace \\
& 水平路径 & 房间到门 & IfcRelSpaceBoundary \\
& 水平路径 & 门到走廊 & IfcRelSpaceBoundary \\
& 垂直路径 & 楼梯 & IfcStair \\
& 紧急路径 & 房间到窗 & IfcRelSpaceBoundary \\
& 紧急路径 & 走廊到窗 & IfcRelSpaceBoundary \\ \bottomrule
\end{tabular}
\end{table}

从风险感知疏散的角度看,MGNM相对传统GNM的优势可概括为:

(1)几何精度更高:楼梯长度、高差、走廊中轴线等均来源于真实几何,室内路径长度误差显著低于GNM(第4.6节将定量对比)。

(2)语义维度更全:模型内可直接区分常规门、消防门、防水闸门等,为洪涝场景下的可用出口筛选与边容量计算提供依据。

(3)更易与风险场耦合:MGNM节点的标高和空间类型可与外部洪涝风险场匹配,实现“高风险入口/楼梯”的自动识别与惩罚,为后续代价函数构建打下基础。

\section{室内空间建模}

本节在MGNM概念的基础上,面向威海市环翠区滨海应急服务中心主楼,构建室内MGNM,并说明如何从BIM/IFC中自动提取疏散相关信息。这里既包括几何表达(空间占用区、多边形边界、走廊中轴线等),也包括与洪涝风险和疏散能力直接相关的语义属性。

\subsection{BIM室内数据与预处理}

研究对象为滨海应急服务中心主楼的BIM模型。首先依据施工图与竣工资料完成原型建模,并在2023年8月组织现场踏勘,对房间功能、门窗尺寸、竖向交通等关键信息进行实测校核。模型在Autodesk Revit 2014环境中构建,通过RTK测站将建筑局部坐标转换至CGCS2000坐标系,实现与城市CIM基准的一致。

为保障MGNM输入质量,预处理包括:

(1)几何校核:利用Revit Model Checker检测并剔除悬浮面、重复构件和法向错误,确保空间封闭性与拓扑一致性。

(2)尺度校验:采用手持激光测距仪复核主要房间和走廊长度,误差控制在3\,cm以内,从源头保证路径长度计算精度。

(3)语义整理:统一房间编号、功能类型、楼层编号等属性;对楼梯、设备间、无障碍设施等与疏散密切相关的构件增加标签,为后续容量和可达性建模提供基础\cite{CN_Zheng2023CompReview}。

模型最终以IFC~2x3开放标准格式导出,保留房间、走廊、门窗、楼梯、平台等要素的几何与语义属性字段,成为MGNM构建的主数据源,如图~\ref{fig:bim_building}所示。

\begin{figure}[htbp]
\centering
\includegraphics[width=0.8\textwidth]{PIC-4/fig2_bim_building.jpg}
\caption{滨海应急服务中心主楼BIM模型}
\caption*{Figure~\thefigure~ BIM Model of the Main Building of the Coastal Emergency Service Center} 
\label{fig:bim_building}
\end{figure}

IFC元素使用相对坐标系统而非地图坐标系统,因此需要通过IfcLocalPlacement和IfcAxis2Placement3D的变换矩阵,将各构件坐标统一转换到CGCS2000坐标系,并与上一章的洪涝风险栅格严格对齐(平面与高程统一),以便后续将室外风险投影到室内网络。

\subsection{IFC到MGNM的自动转换}

从IFC自动生成MGNM的流程可概括为三个步骤:

(1)提取建筑元素表:从IFC“产品”和“关系”中检索IfcSpace、IfcOpeningElement、IfcStair等实体,以及IfcRelSpaceBoundary、IfcRelFillsElement等关系,构建房间、走廊、门窗、楼梯等建筑元素表及其拓扑关系\cite{buildingSMART2007}。

(2)计算MGNM所需几何信息:对房间、洗手间、楼梯间、平台等空间计算几何质心;对IfcOpeningElement计算门窗中心点及其标高;对走廊占用区计算中轴线骨架,形成可通行路径候选。

(3)生成节点与边并附加属性:将空间质心与开口质心抽象为节点,将房间--门、门--走廊、走廊骨架段、楼梯段、房间/走廊--窗等关系抽象为边,并附加几何与语义属性,形成完整MGNM并转换为GIS地理数据库。

核心伪代码如下,突出MGNM相较传统GNM在“自动化+语义保留”方面的优势:
\begin{verbatim}
for space in IFC.spaces:
    centroid = compute_centroid(space.boundary)
    add_node(type='room', id=space.id,
             geom=centroid, attrs=space.attributes)

for opening in IFC.openings:   # doors & windows
    node = add_node(type='door/window',
                    geom=compute_centroid(opening),
                    attrs=opening.attributes)
    connect(adjacent_space_a, node)  # room -> door
    connect(node, adjacent_space_b)  # door -> corridor

for stair in IFC.stairs:
    add_vertical_edges(stair.landings, stair.flights)

for corridor in derived_corridors:
    skeleton = medial_axis(corridor.polygon)
    add_edges_along(skeleton)

export_to_geodatabase(nodes, edges)
\end{verbatim}

图~\ref{fig:building_elements}给出了两个房间的示例:BIM中的房间与门窗被映射为MGNM节点,房间--门、门--走廊等空间关系被映射为边;与传统GNM相比,边长度、楼层、标高、门宽等属性在这一过程中被完整保留,为第4.5节中的容量约束与风险权重提供了必要信息。

\begin{figure}[htbp]
\centering
\includegraphics[width=0.9\textwidth]{PIC-4/fig3_building_elements.jpg}
\caption{建筑元素及其对应的MGNM元素:(a) BIM中的两个房间 (b) 提取的MGNM节点与边}
\caption*{Figure~\thefigure~ Building Elements and Their Corresponding MGNM Elements: (a) Two Rooms in BIM (b) MGNM Nodes (with Attributes) and Edges Extracted from the Rooms}
\label{fig:building_elements}
\end{figure}

\subsection{走廊骨架提取与可通行空间表达}

走廊是室内疏散的主通道,其路径表达精度直接影响疏散距离与时间估计。本研究基于中轴线变换(Medial Axis Transform, MAT)\cite{Blum1967,Prasad1997}从走廊占用区自动提取骨架,避免了传统GNM中“连线式近似”带来的路径偏短问题\cite{CN_Jin2022ShelterLayout}。

设走廊占用区为简单多边形$\Omega\subset\mathbb{R}^2$,其中轴线(骨架)定义为
\begin{equation}
\mathcal{S}=\Big\{\mathbf{x}\in\Omega:\; \exists\, \mathbf{y}_1\ne\mathbf{y}_2\in \partial\Omega,\; d(\mathbf{x},\mathbf{y}_1)=d(\mathbf{x},\mathbf{y}_2)=d(\mathbf{x},\partial\Omega)\Big\},
\label{eq:mat}
\end{equation}
即在走廊内与边界上至少两个点具有相同最近距离的点集。对$\mathcal{S}$进行端点裁剪和小环移除后,得到走廊骨架图$G_{\mathrm{cor}}=(\mathcal{V}_{\mathrm{cor}},\mathcal{E}_{\mathrm{cor}})$,其边即为通行路径。

为提高骨架的鲁棒性,本研究在传统MAT基础上加入缓冲与简化步骤,算法流程简要如下:

(1)对走廊多边形进行Delaunay三角剖分,计算内部三角形外心并构建初始中轴线;

(2)利用角度阈值与长度阈值删去狭长“毛刺”与短悬挂边,保持通道主干连通性与对称性;

(3)检测内部小三角环,将其收缩为单点以消除不必要的循环通路。

优化前后效果如图~\ref{fig:corridor_optimization}所示:最终骨架路径能够贴合理解走廊中线,且与墙体保持合理距离。与GNM中“房间中心点+直线连线”的做法相比,MGNM基于MAT的走廊路径在距离和拓扑上均更接近真实通行路径,为后续室内路径规划和疏散时间估算提供了更可靠的几何基础。

\begin{figure}[htbp]
\centering
\includegraphics[width=0.9\textwidth]{PIC-4/fig6_corridor.jpg}
\caption{走廊路径优化:(a) 根据三角形边中点连接得到初始骨架 (b) 移除不必要节点后的骨架 (c) 移除内部三角形后的最终骨架}
\caption*{Figure~\thefigure~ Corridor Path Optimization: (a) Midpoints Connecting the Edges of Triangles (b) Result of Removing Unnecessary Nodes (c) Result of Removing Internal Triangles} 
\label{fig:corridor_optimization}
\end{figure}

经过上述处理后,MGNM的节点和边被写入3D地理数据库(如三维Shapefile或PostGIS要素类),包括三维几何位置、空间类型、楼层、标高、门窗属性以及可通行宽度等字段,为室内外一体化路径规划和风险映射提供基础结构。

\section{室内外网络连接策略}

在完成室内MGNM构建后,需要将其与城市室外道路网络连接,形成真正意义上的室内外一体化疏散网络。连接策略既要\emph{几何合理}(不穿墙、不穿越其他建筑),又要\emph{风险敏感}(避开下凹式停车区、低洼积水点、被洪水淹没的入口等),并为第4.5节的边代价函数提供清晰的接口。

\subsection{入口到街道的总体思路}

入口到街道策略(entrance-to-street, E2S)的核心思想是:对每个建筑入口节点(来自MGNM)生成一条或多条\emph{传输弧},将其安全地连接到室外道路网络,使室内MGNM和室外道路图成为同一有向图$G$的一部分。

输入数据包括:

(1)室内MGNM入口节点:由门、应急出口等开口元素质心抽象而来,附带楼层、标高、门宽等属性;

(2)室外道路网络:基于OpenStreetMap和威海市交通部门数据融合而成,包含道路等级、车道宽度、人行道信息等;

(3)CityGML LOD1建筑模型:提供建筑平面边界及高度,用于判定传输弧是否与建筑体相交;

(4)洪涝风险栅格:上一章输出的水深/流速/风险场,用于给入口周边的传输弧赋予风险权重或封闭约束。

如图~\ref{fig:transfer_arcs}所示,理想情况下,每个入口节点都能通过一条“短直线”投影到最近的人行道或道路中心线;若存在遮挡或高风险积水区,则需要绕行建筑轮廓或抬升连接点,形成折线型传输弧。

\begin{figure}[htbp]
\centering
\includegraphics[width=0.8\textwidth]{PIC-4/fig9_transfer.jpg}
\caption{I型和II型传输弧示例:(a) 所有传输弧都是I型 (b) 一个入口无法直接投影到室外道路,形成II型传输弧}
\caption*{Figure~\thefigure~ Examples of Type I and Type II Transfer Arcs: (a) All Transfer Arcs are Type I (b) An Entrance Cannot Be Directly Projected to the Outdoor Road, Forming a Type II Transfer Arc} 
\label{fig:transfer_arcs}
\end{figure}

\subsection{传输弧类型与几何约束}

设入口点为$\mathbf{p}$,候选道路线段参数方程为
\[
\mathbf{r}(\tau)=\mathbf{a}+\tau(\mathbf{b}-\mathbf{a}),\quad \tau\in[0,1],
\]
建筑物集合为$\mathcal{B}=\bigcup_k B_k$。根据几何关系和遮挡情况,传输弧分为两类,如图~\ref{fig:transfer_arcs}所示:

(1)I型传输弧(可见直连):入口到道路之间连线不与任意建筑区域相交,且满足最短距离原则:
  \begin{equation}
  \tau^*=\arg\min_{\tau\in[0,1]} \; \|\mathbf{p}-\mathbf{r}(\tau)\|\quad \text{s.t.}\quad \overline{\mathbf{p}\,\mathbf{r}(\tau)}\cap \mathcal{B}=\varnothing。
  \label{eq:e2s-I}
  \end{equation}
  这类传输弧多见于建筑临街、无遮挡的情形,如滨海中心北侧紧邻防灾通道的人行入口。
  
(2)II型传输弧(沿建筑轮廓折线):若直接连线被邻近建筑遮挡或穿越其他建筑,则首先沿建筑轮廓滑移到可见锚点$\mathbf{s}\in\partial B_k$,再连接至道路:
  \begin{equation}
  \min_{\mathbf{s}\in\partial B_k,\,\tau\in[0,1]}\; L(\mathbf{p}\to \mathbf{s})+\|\mathbf{s}-\mathbf{r}(\tau)\|\quad \text{s.t.}\quad \overline{\mathbf{s}\,\mathbf{r}(\tau)}\cap \mathcal{B}=\varnothing,
  \label{eq:e2s-II}
  \end{equation}
  其中$L(\mathbf{p}\to \mathbf{s})$为沿建筑轮廓的最短弧长。该类弧多出现于建筑密集区或入口位于内院时。

\begin{figure}[htbp]
\centering
\includegraphics[width=0.8\textwidth]{PIC-4/fig9_transfer.jpg}
\caption{I型和II型传输弧示意:(a) 所有传输弧为I型 (b) 某入口被邻近建筑遮挡形成II型传输弧}
\caption*{Figure~\thefigure~ Examples of Type I and Type II Transfer Arcs: (a) All Transfer Arcs are Type I (b) An Entrance Cannot Be Directly Projected to the Outdoor Road, Forming a Type II Transfer Arc} 
\label{fig:transfer_arcs}
\end{figure}

通过在GIS中叠加入口点、道路网络与LOD1建筑边界,可自动检测候选传输弧与建筑区域的相交情况,从而判定其类型并进行几何调整,保证室内外网络的\emph{几何合法性}。

\subsection{洪涝风险驱动的连接约束}

在洪涝/内涝情景下,传输弧不仅是几何连接,还承担着\emph{风险传播与过滤}的作用:室外洪涝风险需要沿传输弧传递到室内入口及低层空间,同时高风险区域应在连接阶段被尽可能规避。

具体而言:

(1)对每条候选传输弧$\gamma_e$,沿弧采样并从上一章的危险度场$H(\mathbf{x})$或风险场$R(\mathbf{x})$中提取若干样本点,计算平均风险暴露$\rho(e)$:$\rho(e)\approx \frac{1}{m}\sum_{i=1}^{m} \widetilde{H}(\mathbf{x}_i),\quad \widetilde{H}\in[0,1]$

(2)若传输弧经过的任一点水深超过安全阈值(例如$h>0.3$\,m),则将该弧标记为“不可用”并在网络中禁用;对于临界区域则保留但在第4.5节代价函数中赋予较大风险权重。

(3)对靠近下凹式停车区、地下出入口及海堤缺口的入口,结合地形高程与排水口位置,对其传输弧施加额外惩罚或直接封闭,从而优先选择高程更高、积水风险更低的建筑入口与道路段。

以滨海应急服务中心为例,南侧主入口与下凹式停车区之间存在约1.2\,m高差,且停车区在风暴潮与暴雨叠加时容易形成积水;因此,该入口在洪涝场景下往往需要赋予更高风险代价,优先使用北侧防灾通道或东侧消防专用道作为疏散出入口。通过在E2S阶段显式编码这些约束,可在第4.5节的路径规划中自然体现“向高处、避低洼”的疏散策略。

\subsection{连接验证与数据支撑}

为验证室内外连接策略的可靠性和可用性,本研究从以下几个方面进行检查:

(1)拓扑连通性:确保每个MGNM入口至少有一条有效传输弧连接到道路网络,对孤立入口进行人工复核并根据实际情况修正或删除。

(2)几何合理性:检查传输弧与LOD1建筑、多层构筑物及海堤结构的相交情况,避免“穿墙”“穿楼”等几何错误;对紧贴建筑外立面的弧段进行缓冲修正,确保符合实际通行路线。

(3)风险一致性:对传输弧上的风险样本与上一章洪涝模拟结果进行对比,确认高风险路段(例如堤外低洼区、停车坡道口)在网络中被正确识别并赋予较大代价或被封闭。

E2S策略综合利用OpenStreetMap道路数据与CityGML LOD1建筑模型,实现室内MGNM与室外道路网络的无缝拼接,并为后续的风险感知路径规划提供了清晰的接口:在第4.5节中,所有室内边、室外边以及传输弧将统一使用代价函数$c_t(e)$进行加权,Dijkstra 等经典最短路径算法即可在这一统一网络上直接求解\emph{最短--最安全}路径。



\section{室内外网络连接策略}

\subsection{入口到街道连接方法}

入口到街道策略整合室内和室外网络进行路径规划。我们生成了一个传输弧,即从建筑入口到街道的弧,以自动连接室内和室外网络,将来自BIM的室内MGNM和来自GIS的室外道路网络链接起来用于地理空间分析。滨海中心周边道路呈“北海堤—南主干路—东西支路”三向结构:北侧为滨海木栈道与防灾通道,东侧为消防专用道,西侧连接海港二路并设有消防登高面,南侧则与世昌大道辅路相连。海堤外侧设置防浪墙,南侧主入口与下凹式停车区之间存在1.2m的高差,因此连接策略需综合考虑高差、海堤开口、消防登高面及人行道宽度等空间要素,避免传输弧穿越不可达区域。

建筑入口信息存储在来自IFC的MGNM节点中,而室外街道从OpenStreetMap提取。入口到街道的主要关注点是找到建筑入口和道路之间的最短传输弧,该弧不能与其他建筑重叠。为了满足这一要求,此阶段的输入数据包含室内MGNM、道路网络和额外的CityGML LOD1建筑。LOD1建筑边界的作用是避免传输弧和建筑之间的重叠。

通过入口到街道策略,所有MGNM入口都可以被提取、投影到室外道路网络并用于形成传输弧。在最简单的情况下,道路环绕建筑物,它们之间没有障碍物,但有一些例外,如复杂建筑。例如,如果高建筑密度地区的建筑传输弧不能直接投影到道路,这在城市地区经常发生,CityGML LOD1建筑模型的边界可用于获取所有建筑区域并避免传输弧和建筑区域的交集。
设入口点为$\mathbf{p}$,候选道路线段参数方程$\mathbf{r}(\tau)=\mathbf{a}+\tau(\mathbf{b}-\mathbf{a})\;(\tau\in[0,1])$,建筑物集合并$\mathcal{B}=\bigcup_k B_k$。I型传输弧(可见投影)为
\begin{equation}
\tau^*=\arg\min_{\tau\in[0,1]} \; \|\mathbf{p}-\mathbf{r}(\tau)\|\quad \text{s.t.}\quad \overline{\mathbf{p}\,\mathbf{r}(\tau)}\cap \mathcal{B}=\varnothing,
\label{eq:e2s-I}
\end{equation}
若不可行(II型),则沿建筑边界$\partial B_{k}$滑移至可见锚点$\mathbf{s}\in\partial B_k$:
\begin{equation}
\min_{\mathbf{s}\in\partial B_k,\,\tau\in[0,1]}\; L(\mathbf{p}\to \mathbf{s})+\|\mathbf{s}-\mathbf{r}(\tau)\|\quad \text{s.t.}\quad \overline{\mathbf{s}\,\mathbf{r}(\tau)}\cap \mathcal{B}=\varnothing,
\label{eq:e2s-II}
\end{equation}
其中$L(\mathbf{p}\to \mathbf{s})$为沿$\partial B_k$的最短弧长。由此得到传输弧$\gamma$并入室外网络。

入口到街道连接策略的目标是自动找到不与建筑物或其他建筑相交的传输弧,如入口节点的传输弧生成。第一步将所有MGNM入口节点投影到室外道路网络的边上。如果节点无法垂直投影到道路网络的边上,则连接到道路边上的最近节点,如无遮挡传输弧的情况(图\ref{fig:transfer_arcs}中的I型)。如果候选弧与LOD1建筑模型的边界相交,该入口被建筑物遮挡,我们沿着建筑立面分割建筑边界以生成新的传输弧(图\ref{fig:transfer_arcs}中的II型)。在洪涝情景下,还需对靠近下凹式立交、地下出入口及低洼区的传输弧施加风险增益或封闭约束,优先选择高程更高、积水风险更低的入口与街段。

建筑边界也可以从BIM中提取(例如,ifcWall)。由于并非所有建筑都作为详细BIM创建和存储,我们使用来自CityGML LOD1而非BIM的建筑边界。候选传输弧可以与LOD1建筑模型的边界叠加。最终,所有建筑的入口都通过传输弧连接到室外道路。

\subsection{连接验证与优化}

传输弧的生成需要考虑建筑密度、遮挡情况和最短路径原则。通过CityGML LOD1建筑模型的边界来获取所有建筑区域,避免传输弧与建筑区域的交集。

入口到街道连接策略使用多个GIS数据源,包括用于加强建筑周围区域道路信息(特别是人行道)的OpenStreetMap数据,以及用于可视化和协助确定入口障碍的CityGML LOD1建筑。入口到街道策略可以确定周围环境并相应地选择对策(即,垂直投影连接或沿建筑立面)\cite{Kwan2005,Tao2012}。

\section{粗细结合路径规划算法}

\subsection{多尺度路径规划框架}

一旦MGNM被转换为GIS系统中的地理数据库,室外和室内网络都可以共同建立室外和室内特征之间的连通性。这些拓扑关系存储在3D网络数据集中,用于确定每条边的成本,在本研究中简单地为该边的长度。 结合地下空间的多出口协调策略,可在复杂环境中动态调整疏散通道优先级\cite{CN_Peng2023EvacCoordination}。边的长度通常被考虑用于在路径规划中找到最短路径。如果交通状况可用,则可以在室外路径规划中考虑交通状况。然而,本研究仅考虑路径规划中边的长度。

\begin{figure}[htbp]
\centering
\includegraphics[width=0.9\textwidth]{PIC-4/fig10_coarse_fine.jpg}
\caption{粗细结合路径规划}
\caption*{Figure~\thefigure~ Coarse-Fine Path Planning} 
\label{fig:coarse_fine_planning}
\end{figure}

本研究中的室内外联合路径规划基于Dijkstra算法\cite{Dijkstra1959},网络图由存储节点对之间邻接关系的邻接矩阵表示。使用邻接矩阵中的成本计算最短路径,这意味着室内和室外节点都存储在2D n×n数组中,我们称之为单尺度路径规划。虽然这种方法有明显的优势,但它必须搜索整个数据集以找到最优结果,这是一个耗时的过程。结合上一章的洪涝风险/危险度栅格与本章的风险感知代价(见下一小节),可在规划阶段主动避开高水深/高流速区域,并在传感数据触发时进行快速重规划;同时可参考室内路径研究如“door-to-door”与LEGO表示及人群导航模型\cite{Liu2011indoor,Yuan2010,Lamarche2004}。 此外,基于虚实融合的应急演练评估为算法参数校准提供了重要依据\cite{CN_Wu2022EvacDrill,CN_Chu2022EvacuationDrill}。

统一的边代价函数定义为: 其中的风险项还可结合多主体协同疏散与群体行为建模成果进行动态权重调节\cite{CN_Bai2023AIEvac,CN_Zhou2022CrowdRisk}。
\begin{equation}
	\tilde{\ell}(e)=\frac{\ell(e)}{\max_{e'\in\mathcal{E}}\ell(e')},\quad
	\rho(e)=\frac{1}{\ell(e)}\int_e \widetilde{H}_e(x)\,\mathrm{d}s,\quad
	\kappa_t(e)=\min\Bigl(1,\frac{n_t(e)}{C(e)}\Bigr),
	\label{eq:normalizers}
\end{equation}
\begin{equation}
	c_t(e)=\alpha\,\tilde{\ell}(e)+\beta\,\rho(e)+\gamma\,\kappa_t(e),\quad \alpha,\beta,\gamma\ge 0,\,\alpha+\beta+\gamma=1\,
	\label{eq:edgecost}
\end{equation}
\begin{equation}
	\widetilde{H}_e(x)=w_h\,\widetilde{h}(x)+w_v\,\widetilde{v}(x)+w_L\,\widetilde{L}(x),\quad
	w_h,w_v,w_L\ge 0,\,w_h+w_v+w_L=1,
	\label{eq:hydro_risk}
\end{equation}
其中$e$为MGNM上的一条边,$\mathcal{E}$为边集合,$\ell(e)$为边$e$的几何长度,$\tilde{\ell}(e)$为经归一化处理后的无量纲几何长度;$s$为沿边$e$的弧长参数;$\widetilde{H}_e(x)\in[0,1]$为沿边$e$的归一化危险度或风险场,其内生地综合了局部水深$h(x)$、流速$\mathbf{u}(x)$以及风险等级$L(x)$等水动力与暴露信息;$\widetilde{h}(x)=h(x)/h_{\max}$、$\widetilde{v}(x)=\|\mathbf{u}(x)\|/v_{\max}$、$\widetilde{L}(x)=(L(x)-1)/(L_{\max}-1)$分别为水深、流速和风险等级的归一化形式,$h_{\max}$、$v_{\max}$和$L_{\max}$为研究区域内对应变量的上界,$w_h$、$w_v$、$w_L$为由专家经验或灵敏度分析确定的权重系数,用于刻画不同水动力参数对行人安全的相对重要性;$n_t(e)$为时刻$t$边$e$上的拥挤人数(或对应的人群密度),$C(e)$为边容量,$\kappa_t(e)$刻画边的相对饱和度并在拥挤接近饱和时进行截断;$\alpha$、$\beta$、$\gamma$分别刻画几何距离、风险暴露与人群拥挤在代价函数中的权衡关系,满足非负且和为1。当$\beta=\gamma=0$且$\widetilde{H}_e(x)$为常数场时,$c_t(e)$退化为仅由几何长度决定的传统静态最短路代价函数,在一般情形下则能够随洪水演进与人群聚集的变化进行动态更新,实现风险约束下的最优路径搜索,与仅以几何长度为代价的传统最短路模型相比更能体现“最短安全路径”的含义。


室内外多尺度框架分两层:
\begin{equation}
G^{\mathrm{out}}=(\mathcal{V}^{\mathrm{out}},\mathcal{E}^{\mathrm{out}}),\quad G^{\mathrm{in}}_b=(\mathcal{V}^{\mathrm{in}}_b,\mathcal{E}^{\mathrm{in}}_b),\; b\in\mathcal{B},\quad \mathcal{A}_b\subset\mathcal{V}^{\mathrm{in}}_b\text{为入口集合}.
\label{eq:multiscale}
\end{equation}
步骤:(1)在$G^{\mathrm{in}}_{b_s}$上由源房间$\to$最近入口$a_s\in\mathcal{A}_{b_s}$求最短路;(2)在$G^{\mathrm{out}}$上由$a_s\to a_t$(目标建筑入口)求最短路;(3)在$G^{\mathrm{in}}_{b_t}$上由$a_t\to$目标房间求最短路;(4)拼接三段路径。整体复杂度近似
\begin{equation}
\mathcal{O}\big(E^{\mathrm{in}}_{b_s}\log V^{\mathrm{in}}_{b_s}\big)+\mathcal{O}\big(E^{\mathrm{out}}\log V^{\mathrm{out}}\big)+\mathcal{O}\big(E^{\mathrm{in}}_{b_t}\log V^{\mathrm{in}}_{b_t}\big),
\label{eq:complexity}
\end{equation}
优于在全图$G$上一次性搜索的单尺度方案。

为了提高计算效率,提出了粗细结合方法来支持不同尺度(城市尺度和建筑尺度)的室内外联合路径规划,并减少计算时间。该方法将室内网络(即MGNM)和室外网络(即道路、传输弧、LOD1建筑节点和来自MGNM的入口)存储在不同的邻接矩阵中。

粗细结合方法的目标是在邻接矩阵中分离城市尺度和建筑尺度。城市尺度建筑由LOD1建筑模型的中心点表示,而类似建筑在建筑尺度中由MGNM表示,以确定城市尺度中LOD1建筑点的室外路径。然后我们从建筑尺度的MGNM计算详细的室内路径。由于邻接矩阵被分为不同的尺度,粗细结合方法使室内外路径规划更有效。

\subsection{风险感知代价函数}

为与上一章风险评估结果耦合,在边权中引入风险暴露与通行能力因子。为避免量纲不一致,采用式(\ref{eq:normalizers})的归一化并以式(\ref{eq:edgecost})定义统一代价。室外$\rho(e)$通过沿边$\Gamma_e$的线积分近似:
\begin{equation}
\rho(e)\approx \frac{1}{|\Gamma_e|}\int_{\Gamma_e} \widetilde{H}(\mathbf{x})\,\mathrm{d}s\approx \frac{1}{m}\sum_{i=1}^{m} \widetilde{H}(\mathbf{x}_i),
\label{eq:lineint}
\end{equation}
室内可将外部风险沿入口与楼梯投影至对应楼层,按距离衰减$R_{\mathrm{in}}(\mathbf{x})=R_{\mathrm{floor}}(\mathbf{x})\,\exp(-d/\tau)$;对地下/半地下空间在阈值超限时直接封闭相应边或显著提高$\beta$权重。

\subsection{容量与拥挤度建模}

门、走廊与楼梯的通行能力$C_e$取决于有效宽度$B$、坡度$g$与方向(上/下行)。采用分段线性近似:
\begin{equation}
C_e=
\begin{cases}
\theta B, & g\le g_0,\\
\theta B(1-\mu(g-g_0)), & g>g_0,\ 0\le\mu<1,
\end{cases}
\label{eq:congestion_cost}
\end{equation}
并以排队近似给出拥挤惩罚:$\kappa(e)=\max\{0,\lambda_e/C_e-1\}$,其中$\lambda_e$为到达流量。当电梯可用时,将其建模为高容量,但在应急模式(尤其洪涝停电或积水入侵电梯井)赋予较大语义惩罚$\sigma(e)$以抑制选择。

\subsection{参数与实现设置}

实现采用邻接表存储并按“城市/建筑/楼层/房间”分层编号优化缓存局部性;室外$G_c$可选收缩层级(CH)或地标(ALT)预处理。典型参数见表\ref{tab:params}。在城市级模型与室内语义互操作方面,可结合CityGML/IndoorGML等模型进行融合表达\cite{Kolbe2009,ElMekawy2012,Kim2014}。

\begin{table}[htbp]
\centering
\caption{主要参数与默认取值}
\caption*{Table~\thetable~ Key Parameters and Default Values}
\label{tab:params}
\begin{tabular}{@{}lll@{}}\toprule
参数 & 说明 & 默认值 \\ \midrule
$\alpha$ & 几何长度权重 & 0.4 \\
$\beta$ & 风险暴露权重 & 0.4(洪涝/应急)/0.1(常规) \\
$\gamma$ & 拥挤惩罚权重 & 0.2 \\
$m$ & 风险采样点数/边 & 8 \\
\bottomrule\end{tabular}
\end{table}

\subsection{算法复杂度分析}

使用邻接矩阵的Dijkstra最短路径算法的复杂度为O(ElogV)。V和E分别是顶点数和边数。粗细结合策略通过将室内和室外网络分为不同尺度来降低复杂度。例如,V1和V2是建筑和道路网络的顶点数;E1和E2是建筑和道路网络的边数。在单一尺度中,来自所有建筑的所有顶点和边合并在一起为O((E1 + E2) log (V1 + V2))。在粗细结合策略中,分割点集的理论时间复杂度将降低到O(E1 log (V1)) + O(E2 log (V2))。

由于计算时间与过渡点的数量相关,使用不同数量的过渡点来比较这两种方法。单尺度计算时间为22.94秒,当过渡点数量为2时,但粗细结合方法为3.57秒(室内)和5.38秒(室外)。对于每个额外的过渡点,随着单尺度计算中过渡点数量的增加,计算时间增加。粗细结合方法比单尺度方法更有效,因为它使用LOD1建筑模型来获得相应的室内矩阵。此外,粗细结合和单尺度方法之间的计算比率约为1:4(48.18:204.30),意味着粗细结合方法只需要单尺度方法所需路径规划计算时间的25\%。

\section{应用案例分析}

\subsection{案例设置与准备}

室内外联合路径规划在两个用例场景中进行测试:案例1,紧急响应场景;案例2,行人路径规划。IFC到MGNM转换、入口到街道和粗细结合过程在这两个案例中进一步讨论,验证结果在后续章节中给出。% 新增:在两个具有代表性的情景下对同一MGNM网络进行测试,便于在统一数据条件下对比方法性能。

IFC到MGNM转换原型的实现分为三个步骤。步骤1在xBIM(可扩展建筑信息建模)工具包中实现,用于提取相应的IFC类/关系并将其列在建筑元素表中。步骤2在Matlab环境中实现,遵循表之间的关系来计算拓扑基元。步骤3基于Pyshp(Python Shapefile库)将MGNM拓扑基元转换为ERSI ArcGIS软件中的地理数据库。所有节点、边和属性都转换为GIS环境。该建议方案可以应用于其他BIM以生成室内网络。总计算时间为63秒,使用具有2.93 GHz CPU和8 GB RAM的个人计算机。

\begin{figure}[htbp]
	\centering
	\includegraphics[width=0.8\textwidth]{PIC-4/fig11_3d_model.jpg}
	\caption{三维模型和MGNM}
	\caption*{Figure~\thefigure~ 3D Model and MGNM} 
	\label{fig:3d_model_mgnm}
\end{figure}

MGNM通过IFC到MGNM转换从IFC自动生成,并确定MGNM元素的数量。由于建筑有两个楼梯间,模型有10个楼梯区域节点、8个平台区域节点和16个楼梯路径。总共有307个窗户节点;然而,房间到窗户边和走廊到窗户边的总和是310,因为链接可能是一对一和一对多的,例如当窗户同时属于房间和走廊时,或当房间有一个或多个门时。此外,一些房间在其他房间内部,并非所有门都直接连接到走廊。

\begin{table}[htbp]
	\centering
	\caption{MGNM元素数量}
	\caption*{Table~\thetable~ Number of MGNM Elements} 
	\label{tab:mgnm_element_count}
	\small
	\begin{tabular}{@{}llll@{}}
		\toprule
		MGNM元素 & 含义 & IFC类别 & 数量 \\
		\midrule
		\multirow{6}{*}{节点} & 空间 & 房间 & 95 \\
		& 空间 & 洗手间 & 17 \\
		& 垂直入口/空间 & 楼梯区域 & 10 \\
		& 垂直入口/空间 & 平台区域 & 8 \\
		& 水平入口 & 门 & 142 \\
		& 紧急入口 & 窗 & 307 \\
		\midrule
		\multirow{6}{*}{边} & 水平路径 & IfcSpace & 327 \\
		& 水平路径 & IfcRelSpaceBoundary & 156 \\
		& 水平路径 & IfcRelSpaceBoundary & 129 \\
		& 垂直路径 & IfcStair & 16 \\
		& 紧急路径 & IfcRelSpaceBoundary & 251 \\
		& 紧急路径 & IfcRelSpaceBoundary & 59 \\
		\bottomrule
	\end{tabular}
\end{table}

% 新增:小结
综上,本节构建了从IFC到MGNM的自动转换流程,在普通硬件条件下即可高效生成包含房间、楼梯、门窗等要素的室内网络,为后续两个典型案例提供统一的数据与拓扑基础。

\subsection{典型案例分析}

% 新增:引导句
在统一的MGNM网络和室外道路环境上,本节选取洪涝应急响应和行人避险疏散两个典型场景,对室内外联合路径规划方法进行对比分析。

\subsubsection{案例1:洪涝应急响应应用}

案例1描述了一个洪涝/内涝场景。当暴雨导致滨海中心周边与部分底层出现积水时,若值守人员被困在四层的416会议室,应急救援队需要掌握最短安全路径、携行装备与抽排管线的精确长度,以及沿途门窗/竖向交通的通行能力信息,这需要室内外联合分析并叠加风险信息。

操作分为四个部分:第一部分,定义室内目的地;第二部分,计算从应急支援点到室内目的地的最短安全路径;第三部分,获取沿途开口元素与竖向交通信息;第四部分,确定应急车辆的安全停靠位置与通达路径。

\begin{figure}[htbp]
	\centering
	\includegraphics[width=0.8\textwidth]{PIC-4/fig12_closest_room.jpg}
	\caption{最近房间分析,用于确定3D场景中每个房间到最近入口的距离:(a) 每个房间的路径 (b) 每个房间距离的可视化(IfcSpace类型=房间)}
	\caption*{Figure~\thefigure~ Nearest Room Analysis for Determining the Distance from Each Room to the Nearest Entrance in the 3D Scene: (a) Path for Each Room (b) Visualization of Distances for Each Room (IfcSpace Type = Room)} 
	\label{fig:closest_room_analysis}
\end{figure}

第一阶段定义室内目的地。在这里,我们分析建筑中哪个是最内层的房间(即距离最近出口最远的房间),并应用室内最近入口分析。建筑中所有房间到最近出口的路径都显示出来,表明最内层的房间是416,总移动长度为80.30米;因此,在此模拟中,我们将房间416定义为目的地。

第二阶段通过执行最短路径分析作为3D场景模拟来识别从应急支援点到该房间的最短安全路径。在我们的示例中,救援人员将从入口8进入建筑,通过楼梯行进并穿越建筑。该路径的总长度为1452.12米,其中室内段80.31米、室外段1371.81米;后续章节的精度对比仅针对室内段,以突出MGNM对复杂建筑内部的刻画能力。在此阶段,可以在室外路径代价中叠加上一章洪涝风险栅格以规避高风险路段;在室内侧,窗户/门的高度与属性可从MGNM获得(尺寸、材料、楼层和名称),用于评估紧急通道的可用性与通行能力。

\begin{figure}[htbp]
	\centering
	\includegraphics[width=0.9\textwidth]{PIC-4/fig13_fire_response.jpg}
	\caption{从滨海消防救援站到滨海中心416房间的最短路径分析3D场景:(a) 建议的入口点 (b) 建筑内推荐路径 (c) 救援梯和416房间窗户 (d) 整个场景鸟瞰图 (e) 窗户位置和救援车辆停放建议位置}
	\caption*{Figure~\thefigure~ Shortest Path Analysis from the Coastal Fire Rescue Station to Room 416 in the Coastal Center in the 3D Scene: (a) Suggested Entrance Points (b) Recommended Path Inside the Building (c) Rescue Ladder and Window of Room 416 (d) Aerial View of the Entire Scene (e) Window Locations and Suggested Parking Locations for Rescue Vehicles}
	\label{fig:fire_response_analysis}
\end{figure}

第三阶段获取路径上所有开口元素的信息。在我们的示例中,此路径有三扇门。此外,使用MGNM紧急路径房间到窗户,可以轻松识别416房间的窗户位置和材料。节点房间416有两条相邻的房间到窗户路径,这意味着该房间有两个窗户,窗户节点显示这些是带横档的双窗户,由铝和玻璃组成。

第四阶段选择停车位置,以便救援人员可以通过临时过水桥板/高位通道接近被困者。窗户/高位出口位置也可以帮助确定应急车辆的最佳停靠位置,并辅助实施破拆或高位转移等行动。

\subsubsection{案例2:行人路径规划应用(避险约束)}

案例2是室内外多站点应用的行人路径规划。场景设定为一名社区志愿者从建筑北侧应急入口进入,需要依次完成“前往一层卫生服务站领取急救包—到三层培训教室报到—转赴西南侧物资库房支援分发”的任务。面向复杂环境的路径引导与人群行为导航亦可参考相关研究\cite{Lamarche2004}。

第一阶段,最近设施分析帮助识别离入口最近的卫生服务站。使用MGNM节点属性筛选全楼17个卫生间及医务功能空间,系统推荐志愿者先到一层北侧卫生服务站领取物资;随后通过东侧主楼梯前往三层培训教室完成签到;最后沿西侧楼梯下行至一层并经西南出口抵达室外物资库房。在洪涝情景下,系统对位于低洼区或积水告警的入口与通道施加高风险代价或封闭,使得路径自动绕避相关路段。案例2表明MGNM模型能够将建筑室内外空间无缝衔接,结合风险约束生成贴合灾情的动态路径结果。本案例的路径由三段室内外混合路径串联完成。

\begin{figure}[htbp]
	\centering
	\includegraphics[width=0.9\textwidth]{PIC-4/fig14_pedestrian.jpg}
	\caption{案例2的最短路径分析3D显示:(a) 从北侧应急入口进入滨海中心的推荐路径 (b) 前往一层卫生服务站的最短路径 (c) 案例2完整路径的室内外拼接结果(绿色点表示接入节点) (d) 进入三层培训教室的室内路径 (e) 通过西南出口前往物资库房的撤离路径}
	\caption*{Figure~\thefigure~ Shortest Path Analysis in 3D for Case 2: (a) Recommended Path from the North Emergency Entrance to the Coastal Center (b) Shortest Path to the Health Service Station on the First Floor (c) Indoor and Outdoor Merged Path for Case 2 (Green Dots Represent Access Nodes) (d) Indoor Path to the Training Classroom on the Third Floor (e) Evacuation Path to the Supplies Storage via the Southwest Exit}
	\label{fig:pedestrian_planning}
\end{figure}

% 新增:小结
两个案例分别代表了“单目标应急救援”和“多站点行人避险”两类典型任务。在相同MGNM网络和洪涝风险约束下的对比,为后续精度、效率以及多目标代价设计的系统性评估奠定了基础。

\subsection{结果与验证}

\subsubsection{精度验证}

验证是一个两部分过程:(1)比较传统GNM和建议MGNM之间的室内路径规划结果(室内距离和时间),(2)比较GNM、MGNM和使用30米卷尺测量的实际距离之间的距离。

我们两个案例的测量距离可以基于成人平均步行速度(例如5公里/小时)转换为穿行时间。在案例1中,MGNM和GNM的穿行时间分别为111.63秒和66.93秒,差异为24.4秒;在案例2中,MGNM和GNM的穿行时间分别为237.16秒和150.75秒,差异为86.41秒。

\begin{table}[htbp]
	\centering
	\caption{网络分析结果验证}
	\caption*{Table~\thetable~ Verification of Network Analysis Results}
	\label{tab:network_verification}
	\begin{tabular}{@{}lllll@{}}
		\toprule
		案例 & \multicolumn{2}{l}{MGNM} & \multicolumn{2}{l}{GNM} \\
		\cmidrule(lr){2-3} \cmidrule(lr){4-5}
		& 距离(m) & 相对误差(\%) & 距离(m) & 相对误差(\%) \\
		\midrule
		1 & 80.31 & 5.4 & 47.05 & 44.6 \\
		2 & 172.30 & 5.6 & 123.73 & 32.2 \\
		\bottomrule
	\end{tabular}
\end{table}

为了比较MGNM、GNM和参考距离,GNM的相对误差大于32\%,而MGNM的相对误差小于6\%,表明来自BIM的MGNM比传统GNM更可靠。

案例1和2中室内GNM和MGNM的可视化显示,MGNM的主要改进是垂直连接。此外,通过适当的门穿越内部房间的能力是另一个明显的改进。我们的总结显示,MGNM和GNM之间的差异既存在于水平组件也存在于垂直组件中,表明垂直误差大于水平误差。

\begin{figure}[htbp]
	\centering
	\includegraphics[width=0.8\textwidth]{PIC-4/fig15_case1_comparison.jpg}
	\caption{案例1中GNM和MGNM室内路径比较}
	\caption*{Figure~\thefigure~ Comparison of Indoor Paths between GNM and MGNM in Case 1}
	\label{fig:gnm_mgnm_case1}
\end{figure}

\begin{figure}[htbp]
	\centering
	\includegraphics[width=0.8\textwidth]{PIC-4/fig16_case2_comparison.jpg}
	\caption{案例2中GNM和MGNM室内路径比较}
	\caption*{Figure~\thefigure~ Comparison of Indoor Paths between GNM and MGNM in Case 2}
	\label{fig:gnm_mgnm_case2}
\end{figure}

\begin{table}[htbp]
	\centering
	\caption{水平和垂直部分的网络分析结果验证}
	\caption*{Table~\thetable~ Verification of Network Analysis Results for Horizontal and Vertical Components}
	\label{tab:horizontal_vertical_verification}
	\begin{tabular}{@{}lllllll@{}}
		\toprule
		案例 & & \multicolumn{2}{l}{MGNM} & \multicolumn{2}{l}{GNM} & 参考距离(m) \\
		\cmidrule(lr){3-4} \cmidrule(lr){5-6}
		& & 距离(m) & 相对误差(\%) & 距离(m) & 相对误差(\%) & \\
		\midrule
		\multirow{2}{*}{1} & 水平 & 43.17 & 2.4 & 39.31 & 6.7 & 42.15 \\
		& 垂直 & 37.14 & 13.3 & 7.74 & 81.9 & 42.82 \\
		\midrule
		\multirow{2}{*}{2} & 水平 & 102.86 & 7.4 & 108.45 & 13.3 & 95.74 \\
		& 垂直 & 69.44 & 20.0 & 15.28 & 82.4 & 86.82 \\
		\bottomrule
	\end{tabular}
\end{table}

上述案例研究表明,就垂直或水平距离而言,来自BIM的MGNM路径都更接近真实距离。在案例2中,MGNM垂直部分的相对误差大于案例1,因为案例2使用了建筑的两个楼梯间,这导致了累积误差。相比之下,案例2表明东侧楼梯的长度为37.14米,西侧楼梯的长度为32.31米。然而,在GNM中,这两个楼梯的长度相同;这进一步表明MGNM更准确地反映了建筑的楼梯几何形状。

\subsubsection{计算效率分析}

室内外联合路径规划案例表明,室内路径比室外路径短,但室内路径包含的节点比室外路径多。这意味着室内网络(即MGNM)中的高密度室内实体导致邻接矩阵的快速扩展;邻接矩阵的大小在增加,计算时间也在增加,这增加了处理大型邻接矩阵时计算最短路径的时间。

\begin{table}[htbp]
	\centering
	\caption{两个案例中的路径长度和通过节点}
	\caption*{Table~\thetable~ Path Lengths and Passing Nodes in the Two Cases}
	\label{tab:route_length_nodes}
	\begin{tabular}{@{}lllll@{}}
		\toprule
		案例 & \multicolumn{2}{l}{室内} & \multicolumn{2}{l}{室外} \\
		\cmidrule(lr){2-3} \cmidrule(lr){4-5}
		& 路径长度(m) & 节点数 & 路径长度(m) & 节点数 \\
		\midrule
		案例1 & 80.31 & 76 & 1371.81 & 60 \\
		案例2 & 170.62 & 260 & 598.81 & 50 \\
		\bottomrule
	\end{tabular}
\end{table}

粗细结合方法可以是路径规划的有效方式。室内和室外邻接矩阵的大小分别为4082×4082和5016×5016,单尺度和多尺度案例之间的计算时间不同。粗细结合方法明显减少了路径规划中的计算工作。进一步分析表明,该方法在“先粗后细”的搜索策略下,将城市级图搜索压缩至百级节点,再在目标建筑内部展开全细粒度搜索,使得总运行时间呈线性增长;而单尺度方法需要在一次搜索中同时处理两类节点,导致运行时间接近指数增长。

以案例1为例,粗尺度阶段仅在18个候选接入节点之间搜寻最优外部路径,平均耗时3.57秒;细尺度阶段在室内网络上展开的Dijkstra搜索耗时4.98秒,总计8.55秒。相比之下,单尺度方法需在保留的4000余节点上执行一次全图搜索,构建优先队列和松弛操作均大幅增加,最终耗时22.94秒。案例2因涉及多楼层、多次换乘,粗细结合方法的优势更为明显:粗尺度选择“北入口—西南出口—物资库房”主通道时耗时7.42秒,室内阶段虽然涉及260个节点,但依托分块邻接表仍可在14.46秒内完成;单尺度方法则需67.79秒才能得到同等质量的路径。

\begin{table}[htbp]
	\centering
	\caption{两个案例中不同方法的比较}
	\caption*{Table~\thetable~ Comparison of Different Methods in the Two Cases}
	\label{tab:method_comparison}
	\begin{tabular}{@{}lll@{}}
		\toprule
		案例 & 方法 & 计算时间(秒) \\
		\midrule
		\multirow{2}{*}{案例1} & 单尺度 & 22.94 \\
		& 粗细结合 & 8.55 \\
		\midrule
		\multirow{2}{*}{案例2} & 单尺度 & 67.79 \\
		& 粗细结合 & 21.88 \\
		\bottomrule
	\end{tabular}
\end{table}

因为计算时间与过渡点的数量相关,使用不同数量的过渡点来比较这两种方法。当过渡点数量为2时,单尺度计算时间为22.94秒,但粗细结合方法为3.57秒(室内)和5.38秒(室外)。对于每个额外的过渡点,随着单尺度计算中过渡点数量的增加,计算时间增加。粗细结合方法比单尺度方法更有效,因为它使用LOD1建筑模型来获得相应的室内矩阵。此外,粗细结合和单尺度方法之间的计算比率约为1:4(48.18:204.30),意味着粗细结合方法只需要单尺度方法所需路径规划计算时间的25\%。

\begin{figure}[htbp]
	\centering
	\includegraphics[width=0.8\textwidth]{PIC-4/fig17_computation.jpg}
	\caption{不同过渡点数量的计算时间}
	\caption*{Figure~\thefigure~ Computational Time for Different Numbers of Transition Points} 
	\label{fig:computation_times}
\end{figure}

在案例研究分析中,具有入口到街道策略的MGNM为室内外联合路径规划生成了更细致的应用;例如,案例1的第一次操作可用于确定建筑的逃生绳配置。

\subsubsection{敏感性与消融实验}

为评估多目标代价的稳健性,设计两类实验:(1)权重敏感性:在$\beta\in[0,1]$、$\gamma\in[0,0.8]$范围内网格搜索,记录最短路径风险积分$R^{\mathrm{sum}}$、室内路径长度$L^{\mathrm{in}}$与拥挤超载比$\kappa^{\max}$的变化;(2)模块消融:依次移除“入口到街道策略(E2S)”“拥挤建模(Cap)”“风险加权(Risk)”,比较核心指标。表~\ref{tab:weight_sensitivity}与表~\ref{tab:ablation}分别给出代表性组合的量化结果。

\begin{table}[htbp]
	\centering
	\caption{代价函数权重敏感性结果}
	\caption*{Table~\thetable~ Sensitivity Results of Cost Function Weights} 
	\label{tab:weight_sensitivity}
	\begin{tabular}{@{}ccccl@{}}
		\toprule
		$\beta$ & $\gamma$ & $R^{\mathrm{sum}}$ & $L^{\mathrm{in}}$/m & $\kappa^{\max}$ \\
		\midrule
		0.20 & 0.10 & 0.72 & 78.54 & 0.21 \\
		0.40 & 0.20 & 0.55 & 80.31 & 0.12 \\
		0.60 & 0.20 & 0.48 & 82.67 & 0.15 \\
		0.40 & 0.40 & 0.46 & 84.91 & 0.09 \\
		\bottomrule
	\end{tabular}
\end{table}

为检验系统在实际指挥中的可行性,本研究与环翠区应急管理局联合开展了一次桌面演练。演练以“风暴潮预警+应急疏散”为脚本,演练体系包括指挥席、救援组、疏散引导组和后勤组:指挥席通过CIM平台实时查看洪水风险热力图并下达路径诱导指令;救援组按照案例1路径执行救援,验证装备部署可行性;疏散引导组参照案例2路径引导志愿者完成物资调配。演练结果显示,系统生成的路径与现场经验一致,应急人员能够在5分钟内掌握关键通道状态,并依据平台提示调整封控点,体现出良好的协同指挥效果。

\begin{table}[htbp]
	\centering
	\caption{模块消融实验量化结果}
	\caption*{Table~\thetable~ Quantitative Results of Module Ablation Experiment} 
	\label{tab:ablation}
	\begin{tabular}{@{}lcccc@{}}\toprule
		配置 & $L^{\mathrm{in}}$/m & $R^{\mathrm{sum}}$ & 平均拥挤度/(人·m$^{-2}$) & 计算时间/s \\
		\midrule
		完整模型 & 80.31 & 0.55 & 1.82 & 8.55 \\
		无Risk & 79.84 & 0.71 & 1.81 & 8.42 \\
		无Cap & 80.12 & 0.56 & 2.37 & 8.40 \\
		无E2S & 97.45 & 0.69 & 1.90 & 9.63 \\
		单尺度 & 80.31 & 0.55 & 1.82 & 22.94 \\
		\bottomrule
	\end{tabular}
\end{table}

结果显示:(i)$\beta$升高时,路径更倾向风险更低的走廊与入口,$R^{\mathrm{sum}}$显著下降,而路径长度与拥挤度变化可控;(ii)移除E2S后,部分入口需绕行导致室内路径大幅增加且风险积分反弹;(iii)粗细结合在计算时间上稳定优于单尺度,实现相同精度下约2.7倍的效率提升。

为了说明风险代价的映射方式,以2021年“烟花”台风模拟结果为例:第二章输出的水深栅格在入口E3处峰值为0.62m,对应危险度分量$\widetilde{H}=0.74$,代价函数中风险条目$\rho(e)$取0.31;而位于高地的入口E1水深小于0.05m,$\widetilde{H}=0.08$,代价仅为0.04。因此在权重$\beta=0.4$时,算法自动倾向选择E1,并对经E3的边赋予较高代价,避免引导人员进入积水区。

上一章输出的洪涝水深/流速栅格经坐标统一与分辨率重采样后,作为室外边权风险分量$\rho(e)$输入本章模型;室内可将外部风险沿入口与楼梯投影到对应楼层,形成$R_{\mathrm{in}}(\mathbf{x})$。系统在应急模式下启用$\beta>0$并周期性刷新$\widetilde{H}(\mathbf{x})$,因此可随洪水演进动态调整疏散路径;若同时接入人群密度热图,则通过$\lambda_e$更新$\kappa_t(e)$,实现“风险—拥挤”双约束的实时引导。

% 新增:结果小结
综合精度、效率以及多目标代价的分析,可以看出MGNM在刻画复杂室内空间几何、支撑多约束路径搜索方面相较GNM具有明显优势,并能够在安全性与计算开销之间取得较为稳健的折中。

\subsection{扩展场景与实施要点}

% 新增:引导句
在典型案例验证基础上,本节进一步从系统实现、规模扩展、特殊人群以及动态调整等方面讨论方法在工程应用中的若干关键要点。

\subsubsection{实施与质量控制}

为了实现基于CIM的室内外一体化疏散路径规划系统,本研究设计了分层的软件架构并对关键算法进行了优化。

软件架构设计:系统采用模块化设计,包括数据层、处理层、分析层和应用层四个主要层次\cite{Chen2014,Chen2015,Li2014,Daum2014}。数据层负责管理多源异构数据,包括BIM/IFC数据、GIS空间数据、CityGML建筑模型和OpenStreetMap道路网络数据。处理层实现IFC到MGNM的自动转换算法、走廊路径优化算法和入口到街道连接算法。分析层提供路径规划引擎,支持单尺度和粗细结合两种计算模式。应用层面向不同用户需求,提供紧急响应和行人导航两类应用接口。

关键算法优化:在IFC到MGNM转换过程中,本研究对关键算法进行了多项优化。首先,在走廊路径提取方面,改进了传统的中轴线变换(MAT)算法,通过引入缓冲区机制和内部三角形检测,提高了路径提取的准确性和鲁棒性。其次,在节点提取方面,设计了基于空间拓扑关系的自动检测算法,能够准确识别房间、走廊、门窗等不同类型的建筑元素。在路径规划算法方面,提出的粗细结合方法通过分离城市尺度和建筑尺度的计算,显著降低了算法复杂度。该方法将O((E1 + E2) log (V1 + V2))的时间复杂度降低到O(E1 log (V1)) + O(E2 log (V2)),在保证计算精度的同时大幅提升了计算效率\cite{Cheng2001}。

数据质量控制:为确保疏散路径规划的可靠性,本研究建立了严格的数据质量控制体系。在BIM数据预处理阶段,采用自动化算法检测和修复模型中的几何错误,包括悬浮面、重叠面和拓扑不一致等问题。在MGNM生成过程中,引入了多重验证机制,确保节点和边的正确性。在室内外网络连接方面,通过引入CityGML LOD1建筑模型进行遮挡检测,避免传输弧与建筑物的冲突。同时,建立了传输弧质量评估指标,包括路径长度、遮挡程度和连通性等,为系统提供了可靠的连接质量保障。

\subsubsection{大规模建筑群疏散场景}

为了验证本方法在大规模应用中的有效性,本研究扩展了应用场景,考虑了包含多栋公共服务建筑的滨海综合服务园区疏散情况。在该场景中,需要同时考虑多个建筑内部的疏散路径以及建筑间的连接路径。

实验结果表明,当建筑数量从1栋增加到5栋时,传统单尺度方法的计算时间呈指数增长,而粗细结合方法的计算时间增长相对平缓。具体而言,在处理5栋建筑的疏散路径规划时,单尺度方法需要485.2秒,而粗细结合方法仅需要127.6秒,计算效率提升约74\%。

\subsubsection{特殊人群疏散考虑}

在实际的城市灾害疏散中,需要特别考虑特殊人群(如残疾人、老年人、儿童等)的疏散需求。本研究扩展了MGNM模型,增加了无障碍设施的语义信息,包括电梯、无障碍通道、扶手等设施的位置和状态信息。

通过修改路径规划算法的权重设置,系统能够为不同类型的疏散人员生成个性化的疏散路径。例如,对于轮椅使用者,系统会优先选择无台阶、坡度较缓的路径;对于老年人,系统会选择距离较短、休息点较多的路径。

具体而言,在MGNM节点属性中新增“坡度”“净宽”“扶手可及性”等指标,并将无障碍电梯、缓坡通道、临时坡道板等设施建模为独立节点。路径规划阶段引入用户画像参数$\mathbf{p}$,根据人员类别动态调整代价函数权重:如轮椅用户将拥挤惩罚权重$\gamma$提高至0.4、障碍惩罚权重$\eta$提高至15,并禁止选择坡度超过$1:12$或净宽小于1.2m的边;老年人则对路径长度和休息点可达性赋予更高权重。系统还与应急物资库联动,标记可临时部署的移动坡道和助行器位置,便于指挥员发布支援指令。模拟结果显示,相较于通用方案,针对轮椅用户制定的个性化路径能够减少35\%的“高坡道”经过次数,整体疏散时间缩短约12\%。

\subsubsection{动态疏散路径调整}

考虑到灾害过程中环境条件的动态变化,本研究进一步扩展了系统功能,支持实时的疏散路径调整。系统能够接收来自物联网传感器的实时数据,包括火灾蔓延情况、道路阻塞状态、人员聚集密度等信息。

基于这些实时数据,系统采用动态Dijkstra算法重新计算最优疏散路径。实验表明,在模拟的火灾扩散场景中,动态路径调整能够有效避开危险区域,平均减少疏散时间15-20\%,显著提高了疏散效率和安全性。

实时感知数据通过城市物联网平台接入,包括分布于一层和地下的水位传感器、楼层烟感与温度探测器、视频分析生成的人群密度热力图以及门禁系统反馈的通行状态。系统每30秒执行一次增量式路径更新:对新增风险区域采用边权增大或直接封闭策略,对拥堵区根据人流量调整$\kappa_t(e)$。为防止频繁切换造成混乱,设置路径切换阈值,当新路径相较当前路径在风险或时间指标上改进超过10\%时才发布调整,并通过移动端、广播和楼宇导引屏同步推送。动态模拟显示,在火灾与积水双重扰动下,该机制将人员滞留在危险区域的时间从4.6分钟降至2.1分钟,同时提升疏散信息到达率至94\%。

总体来看,本节通过典型案例、对比实验和扩展应用表明,基于CIM的MGNM方法在精度、效率和可扩展性方面均具有良好表现,并具备服务实际应急指挥与日常疏散管理的工程应用潜力。


\section{本章小结}

本章以洪涝灾害条件下的城市建筑疏散为研究对象,基于CIM与MGNM,提出了室内外一体化的风险感知疏散路径规划方法,并在典型应用场景中进行了验证。主要结论与贡献如下。

(1)构建了从BIM/IFC模型到MGNM的转换与集成框架,实现了建筑室内空间与外部道路环境在统一三维城市模型中的一体化表达。通过自动识别房间、楼梯、门窗等构件并抽象为网络节点与连边,结合入口到街道的连接策略,将室内网络与多源GIS道路数据有机耦合,形成可直接支撑路径搜索的室内外统一拓扑基础,为后续风险映射和多场景分析提供了稳定的数据载体。

(2)提出了同时考虑城市尺度与建筑尺度的粗细结合路径规划过程,并在此基础上形成风险感知代价与容量惩罚的统一建模方案。数值结果表明,相比传统GNM,MGNM在水平与垂直路径分量上的长度精度均有明显提升;粗细结合策略在多建筑、多入口场景下显著降低了计算时间;权重敏感性与模块消融分析表明,该方法能够在路径几何长度、风险暴露程度与拥挤控制之间实现可调节的折中,为洪涝情景下的动态诱导与联动决策提供了可靠的算法支撑。

(3)设计并实现了覆盖数据预处理、网络生成、路径规划与结果分析的完整软件体系结构,形成了面向工程应用的CIM–MGNM一体化工作流。系统支持单建筑与建筑群疏散、特殊人群差异化路径推荐以及基于物联网感知数据的实时路径调整等多种运行模式,将城市信息模型、疏散网络模型与灾害风险场集成到统一的技术平台之中。

(4)本章提出的室内外一体化风险感知疏散方法,为后续章节提供了统一的数据与模型基础。一方面,第五章在同一BIM与CIM数据体系上,将洪水水深和流速进一步映射到构件层面,用于评估建筑构件损伤及经济损失;另一方面,第六章在本章路径规划引擎和接口规范的基础上,集成多源感知、情景推演和辅助决策模块,构建滨海城市防灾减灾与应急协同系统,使从模型方法到系统实现的技术路线得以贯通。



