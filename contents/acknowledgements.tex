在本论文即将完成之际,回首博士求学生涯,心中充满感激之情。谨向在此期间给予我关心、帮助和支持的所有领导、老师、同学、亲友,致以最诚挚的谢意。

首先,衷心感谢我的导师张顶立教授。在整个博士阶段的学习和科研过程中,导师在选题论证、研究思路、论文撰写等各个方面都给予了我悉心指导和耐心教诲。从科研方法到学术规范,从严谨求实的治学态度到脚踏实地的人生态度,导师都以言传身教的方式对我产生了深远影响,使我受益终身。在论文遇到瓶颈和困惑时,导师总能给予及时点拨和鼓励,让我重拾信心、坚持向前,在此谨致崇高的敬意和由衷的感谢。

感谢土木工程学院及相关学院的各位老师在课程学习、学术训练和能力培养方面给予我的帮助。多年来老师们在课堂内外的悉心指导,为我奠定了扎实的理论基础,也开拓了我在城市信息建模、洪涝风险评估与应急管理等领域的学术视野。

本论文的研究工作得到了“中国中铁科技研究开发计划(2023-重大-21)”项目的资助与支持,在此一并表示感谢。 项目为论文的开展提供了必要的经费保障和工程背景,使得相关理论方法得以在实际工程场景中验证和应用。

在具体研究过程中,感谢课题组的各位老师和同学在数据采集、模型实现、系统开发与论文修改过程中给予的无私帮助。无论是在三维城市模型构建、三维水动力数值模拟,还是在多用途几何网络模型设计、疏散算法实现与系统联调中,大家提供了大量宝贵的建议和耐心的协助。你们严谨踏实的工作作风、开放合作的学术氛围,使我在科研道路上始终受益良多。

同时,感谢论文评阅专家和答辩委员会各位教授在百忙之中审阅本论文,并提出了许多中肯而宝贵的意见和建议,这些意见不仅有助于完善本论文的工作,也对我今后的研究方向和学术思考具有重要启发作用。

此外,还要感谢北京交通大学为我提供的良好学习和科研环境,以及学校相关职能部门在学习、生活及实践环节中给予的关心与支持。也感谢在博士学习期间曾经给予我帮助的同学和朋友,与你们的交流与合作使我的研究和生活都更加充实而愉快。

最后,最深切的感谢献给我的家人。感谢父母长期以来的无私付出和默默支持,是你们包容、理解和鼓励,让我能够心无旁骛地完成学业;感谢爱人和家人在生活上的关心与照顾,在我面对压力和挫折时始终给予温暖和力量。你们是我坚持走完博士征程的最大动力。

谨以此文,向所有关心、帮助和支持过我的老师、同学、同事与亲友致以诚挚的谢意。若有未及一一具名者,亦在此一并深表感谢。