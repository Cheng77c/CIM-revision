本论文是在我的导师,我国交通基础设施工程领域著名专家张顶立教授的悉心指导下完成的。从选题论证、研究思路的凝练,到最终论文撰写等各方面均给予了悉心指导与耐心教诲,每一个环节都倾注了导师的心血与智慧。五年来,先生渊博的学识、严谨求实的治学态度、敢为人先的学术气魄,以及“德字为先”的谆谆教诲,使我受益匪浅,并将伴随我终身。尤为感激的是,先生始终为我营造宽松而自由的学术氛围,提供丰富的科研课题与工程实践机会,使我在理论与实践的深度融合中不断成长。值此论文收笔之际,谨向恩师致以最崇高的敬意与最诚挚的感谢!

衷心感谢土木建筑工程学院的房倩教授、陈铁林教授,人工智能学院的郑伟教授、网安学院的张大林副教授等,在论文研究、撰写及定稿过程中诸位博导及老师提出了诸多宝贵意见,拓宽了我的研究视野,在学习、生活上给予了悉心关怀与大力支持,在此一并致以诚挚谢意。

感谢中铁第六勘察设计院集团有限公司和中国中铁智慧城市研发中心提供的宝贵的研发机会以及众多的科研项目,为论文的顺利开展提供了必要的经费保障与工程背景,使相关理论方法得以在实际工程场景中验证与应用。感谢中铁六院范建国总工程师、中国中铁智慧城市研发中心王烨主任、付功云部长、魏子贤助理研究员及智慧城市研发中心全体同事;北京交通大学的李启明、王嘉琛等博士,刘朝辉、张子溢等硕士,在课题调研、研究、开发与现场监测阶段给予了无私帮助与数据支持,在资料共享、学术讨论、数值计算及论文排版等方面给予我极大帮助,在资料整理与学术启迪上的鼎力相助。谨向所有参与、关心和支持本课题的工程技术人员致以崇高敬意!

求学之路虽苦,幸有家人始终相伴。感谢妻子高蕾五载寒暑独撑家庭,抚育爱子李泽林,始终无怨无悔地支持我完成学业;感谢父母岳父母长期以来的无私付出与默默支持,你们虽年事已高,仍默默理解、包容、鼓励并支持我的求学选择;兄弟姐妹亦在生活上给予多方资助与鼓励。你们是我顺利完成博士学业的最大动力。谨以此文向所有家人致以我最深切的感激和最诚挚的谢意!

最后,感谢各位评审专家与答辩委员在百忙之中审阅本论文,您的宝贵意见将是我后续研究的指路明灯!

谨以此文,向所有关心、帮助与支持过我的师长、学友、同事及亲友,致以最诚挚的谢意。若有未能一一具名者,亦在此一并深表感谢。