% ********************************************************************
% ****************** Free to modify the content below ******************
% ********************************************************************

在全球气候变化与高速城市化耦合效应驱动下,极端降雨—风暴潮复合事件呈显著上升趋势,城市洪涝对人口安全、关键基础设施及城市运行系统造成的级联风险持续累积。然而,既有研究多基于二维地形与静态建筑单体开展风险评估与疏散设计,难以解析高密度建成环境中的三维水动力过程、跨建筑群人群疏散行为及构件级损伤机理,导致风险评估—疏散决策—损伤分析链条断裂,应急管理缺乏统一的数字底座与协同机制。针对上述科学瓶颈,本文以城市信息模型(City Information Modeling, CIM)为数字基座,提出面向城市洪涝灾害的多尺度一体化技术框架,实现洪涝风险评估、风险驱动智能疏散与灾后损伤评估的全流程贯通,并在典型城区完成系统集成与工程验证。

在理论与建模层面,本文引入复杂系统视角下的城市韧性治理与数字孪生城市理念,提出面向城市洪涝全过程管理的数字韧性治理总体框架,构建覆盖物质、空间、性能、文化和时间五维的5D-4V CIM 建模范式,从场景化、参数化、互动化与智能化四个视角系统组织城市对象及其行为语义。通过设计统一的城市语义矩阵与对象关系模型,实现建筑构件—单体—街区—城市的跨尺度一致表达,为三维水动力仿真、疏散网络构建与构件级损伤分析提供同源数据与知识底座。

在洪涝风险评估方面,本文融合无人机倾斜摄影、机载激光雷达、CityGML与 BIM/IFC 等多源空间信息,构建高精度城市三维地形与建筑语义模型,形成支持细尺度栅格运算的 CIM 数据底座。基于三维雷诺平均纳维-斯托克斯(RANS)方程,开发嵌套水动力求解器,精细刻画建筑群周边复杂流场;相较传统二维浅水模型,淹没范围与水压力估计误差分别降低约 10.8\% 和 16.0\%。在此基础上,进一步构建耦合物理暴露、基础设施脆弱性和社会经济敏感性的综合指标体系,生成时空连续的洪涝危险度与人口暴露图,为后续疏散路径规划和构件级损伤评估提供具有物理约束的定量基础。

在智能疏散方面,本文提出多用途几何网络模型(Multi-purpose Geometric Net- work Model,MGNM),从 IFC 模型中自动抽取房间、走廊、门窗、楼梯等语义对象及其拓扑关系,构建与建筑几何高度一致的室内通行网络,并通过出入口节点与城市道路网自动拼接,形成室内外一体化可达图。基于水深、流速与人群密度构建风险感知代价函数,设计粗细结合的并行路径求解策略,实现城市级疏散网络的动态更新与局部重规划。算例结果表明,在五栋互联建筑场景中,路径计算时间由 485.2 s 降至 127.6 s,计算效率提升约 74\%;在典型洪涝情景下,风险驱动疏散可使总疏散时间缩短约 15.3\%,处于高风险边上的路径占比降至 9.4\%,人员疏散的安全性和鲁棒性显著提升。

在灾后损伤评估与系统集成方面,本文拓展基于构件的易损性(Assembly-Based Vulnerability)理论,在 BIM 构件语义中耦合静水压力、动水压力与浮力等水荷载,建立水荷载—几何—材性参数间的定量映射,提出面向单构件的洪涝损伤判别准则及维修/更换成本映射方法,构建可按构件类型、损伤等级和经济损失进行过滤、统计与空间查询的三维可视化评估框架。以威海滨海应急服务中心为案例,系统识别出整体构件损伤率约为 27\%,其中门窗和墙面饰材损伤尤为突出,估算直接经济损失约 51.4 万元,关键功能预计在 21 天内恢复。在此基础上, 本文将高精度水动力模拟、MGNM 疏散网络与构件级损伤评估模块集成至统一 CIM 平台,构建数据采集—语义治理—风险评估—路径规划—损伤分析—协同指挥的闭环协同系统。系统采用分层微服务架构与事件驱动机制,可在约 14.6 min 内完成研究区域风险评估—疏散规划—损伤分析全流程演算,并通过三维可视化界面面向指挥员和公众终端同步展示风险态势与疏散引导路径,实际部署与多轮演练验证了系统在响应时效、跨部门协同与场景直观性方面的工程适用性。

综上,本文在理论上提出 CIM 驱动的城市洪涝数字韧性治理框架,在方法上形成高精度三维水动力风险评估—风险驱动室内外一体化疏散—构件级洪灾损伤评估的多尺度联动技术链,在工程上构建并验证可服务真实业务场景的城市应急管理原型系统。研究成果为构建集风险预警、智能疏散与灾后恢复于一体的数字化防灾减灾体系提供系统化技术支撑,也为多灾种耦合、实时数据同化及人群行为精细建模等后续研究奠定了方法与工程基础。
