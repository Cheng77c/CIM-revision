% ********************************************************************
% ****************** Free to modify the content below ******************
% ********************************************************************

在全球气候变化与高速城市化进程的双重驱动下,极端降雨—风暴潮复合灾害频发,城市高密度建成区面临着严峻的级联风险挑战。既有研究多受限于二维地形概化与静态建筑模型,难以在微观尺度上精准解析复杂三维流场特征、跨建筑疏散行为及构件级损伤演化机理,导致风险评估、应急决策与灾后恢复之间存在明显的“语义断层”与“模型孤岛”,缺乏统一的数字底座支撑。针对上述科学瓶颈,本文以城市信息模型(City Information Modeling, CIM)为数字基座,引入复杂系统视角下的韧性治理理念,提出了一套“机理—数据—知识—模型”(MDKM)四驱融合的城市洪涝数字韧性治理技术框架,实现了从宏观城市底座到微观构件损伤的全链条贯通与闭环验证。主要研究内容与成果如下:

第一,构建了CIM驱动的MDKM四驱融合理论与建模范式。 针对城市复杂系统的多源异构特性,提出了覆盖物质、空间、性能、文化及时间五维的“5D-4V” CIM建模体系。以此为基础,构建了MDKM方法论,通过对象—关系—事件(O-R-E)模型,建立了从建筑构件到城市街区的跨尺度语义映射机制。该框架为多物理场仿真、行为动力学模拟与知识推理提供了统一的时空基准与语义底座,有效解决了机理模型与数据驱动方法难以协同的难题。

第二,提出了基于高精度CIM底座的三维水动力风险评估方法。 融合无人机倾斜摄影、机载激光雷达、CityGML与BIM/IFC等多源空间信息,构建了厘米级精度的城市三维地形与建筑语义模型。基于三维雷诺平均纳维-斯托克斯(RANS)方程,开发了嵌套网格水动力求解器,精细刻画了密集建筑群周边的复杂流场与垂向涡旋特征。验证表明,相较于传统二维浅水模型,该方法在淹没范围与水压力估算上的误差分别降低了10.8%和16.0%,体现了“机理—数据”齿轮的深度啮合,为后续决策提供了具有物理约束的定量输入。

第三,建立了风险驱动的室内外一体化智能疏散模型。针对疏散路径规划中的空间割裂问题,提出了多用途几何网络模型(MGNM)。该模型通过自动解析IFC语义提取建筑拓扑,并构建入口—街道连接策略,实现了室内外通行网络的无缝衔接。引入融合水深、流速与人群密度的风险感知代价函数,设计了粗细粒度结合的并行路径求解策略,实现了“知识—模型”的协同驱动。实验显示,在典型洪涝情景下,风险驱动疏散使总疏散时间缩短约15.3%,高风险路径占比降至9.4%,显著提升了疏散效率与安全性。

第四,构建了多物理场耦合的构件级洪灾损伤评估体系。 拓展了基于构件的易损性(Assembly-Based Vulnerability, ABV)理论,在BIM语义中耦合了静水压力、动水压力、浮力及水接触等多物理作用机制。建立了水荷载—构件属性—经济损失的定量映射模型,实现了对建筑构件损伤状态的精细化判定与三维可视化。以威海滨海应急服务中心为例,系统识别出整体构件损伤率约为27%,并精准定位了门窗与饰面材料的损伤高发区,验证了“机理—知识”融合在灾后恢复评估中的有效性。

第五,研发了城市防灾应急协同系统并完成工程验证。 将上述高精度水动力模拟、MGNM疏散网络与构件级损伤评估模块集成至统一CIM平台,构建了“数据采集—语义治理—风险评估—路径规划—损伤分析—协同指挥”的闭环系统。系统采用分层微服务架构,可在约14.6分钟内完成全流程演算,并通过三维可视化界面实现多端协同。实际演练证明,该系统在响应时效性、跨部门协同能力及场景直观性方面具有显著的工程应用价值。

综上所述,本文在理论上提出了MDKM驱动的数字韧性治理框架,在方法上突破了跨尺度灾害建模与多模型耦合的关键技术,在工程上形成了一套可复制、可推广的城市应急管理解决方案,为构建智慧韧性城市提供了科学依据与技术支撑。

