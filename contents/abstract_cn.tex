% ********************************************************************
% ****************** Free to modify the content below ******************
% ********************************************************************

在气候变化与高强度城市化的双重驱动下,城市洪涝灾害呈现发生频率和损失强度同步上升的态势,高密度建成环境与复杂基础设施网络之间的耦合放大了极端降雨事件引发的级联效应与系统性风险。现有研究在城市信息建模(City Information Modeling, CIM)、水动力模拟、应急疏散与损失评估等方面虽已取得重要进展,但普遍存在模型孤立演化、语义耦合不足以及从风险识别到协同控制的机理链条不完整等问题,难以支撑复杂城市洪涝情景下的精细化风险演化刻画与跨部门应急协同决策。针对上述关键科学与工程问题,本研究以“城市洪涝灾害的风险演化机理与协同控制系统”为主线,在 CIM 语义建模、多尺度水动力—风险耦合、室内外一体化疏散、构件级损伤评估与应急协同系统集成等方面开展系统研究。

首先,在理论与方法层面,构建面向城市洪涝灾害的 CIM 5D–4V 语义建模体系与对象–关系–事件(Object–Relation–Event, O–R–E)语义结构,提出集成要素维、空间维、性能维、文化维与时间维的多维灾害知识模型,系统梳理“降雨—汇流—积涝—暴露—损伤”的风险演化链条,形成支撑数据汇聚、模型协同与业务驱动的总体框架,为后续各章方法提供统一的语义与机理基础(第 2 章)。

其次,在洪涝致灾过程定量表征方面,构建多尺度嵌套的三维雷诺时均 Navier–Stokes 方程与二维浅水方程水动力模拟体系,引入地形—建筑一体化高精度几何建模与边界条件同化策略,实现对关键区域流场结构与整体城市浅表径流过程的协同计算;在此基础上建立多指标洪涝危险度、暴露度与脆弱性度量方法,形成面向城市格网的综合风险评估技术,并在典型滨海城市案例中完成验证(第 3 章)。

然后,围绕洪涝情景下的群体疏散问题,提出基于多层广义网络模型(Multi-layer Generalized Network Model, MGNM)的室内外一体化疏散建模方法,将 IFC 等建筑信息模型解析得到的楼宇拓扑与城市道路网络在 CIM 平台中统一编码,引入源自动态洪涝场的风险感知代价函数与时间依赖通行约束,构建风险驱动的多目标疏散路径优化与人群演化仿真方法,用以评估不同预案与控制策略的安全性与效率(第 4 章)。

再次,为刻画洪涝作用下建成环境的细粒度损伤,提出基于属性–建筑–体素(Attribute–Building–Voxel, ABV)的一体化构件表达与损伤评估方法,将水深、流速等水动力参数映射至构件层级的受损状态与功能退化,结合脆弱性曲线与修复成本模型,构建构件级经济损失与服务性能损失评估体系,并在 CIM 三维场景中实现可视化表达,为韧性评估与恢复决策提供量化支撑(第 5 章)。

最后,在系统集成与应用层面,将上述关键模型与方法集成于分层微服务架构的城市防灾应急协同系统中,以数据层—模型层—业务层划分实现多源时空数据接入、模型组件化编排与应急业务流程的事件驱动协同,面向预警发布、避险引导、资源调度与灾后评估等典型业务构建一系列应用场景,并在真实城市案例中进行系统验证与运行示范(第 6 章)。

综合而言,本研究构建了面向城市洪涝灾害的 CIM 语义建模与风险演化分析新框架,提出多尺度水动力—风险耦合评估方法、风险驱动的室内外一体化疏散建模与优化方法以及构件级洪灾损伤评估与三维表达技术,并在此基础上实现了城市防灾应急协同控制系统的工程化落地。相关研究成果可为复杂城市洪涝灾害的机理认知、风险量化与协同防控提供系统化技术路径,对支撑韧性城市建设和流域—城市一体化防洪减灾实践具有重要理论价值与应用前景。