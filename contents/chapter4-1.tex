\chapter{基于CIM的构件级洪灾损伤评估与三维可视化}

在前两章中,本文分别构建了面向城区的三维水动力风险评估体系和室内外一体化疏散路径规划模型,实现了“危险感知—路径优化”的实时联动。然而,灾害全过程治理不仅仅停留在风险评估和疏散规划层面,还需对洪水作用后的建筑构件损伤、经济损失以及修复优先级进行更为精细的评估。这一评估将为应急恢复与灾后风险复盘提供必要的支持。

洪灾损伤评估(Flood Damage Assessment, FDA)是基于风险的洪水管理方法中的关键组成部分\cite{Thieken2005}。目前,FDA方法往往忽视建筑物的独特性,采用基于类别的简化处理方式,这使得它难以满足需要逐案分析建筑损伤的详细应用需求。尤其是在使用不完整且质量较低的建筑数据输入时,这一局限性更加明显。此外,关于建筑几何形状和材料组成的假设及近似处理,也可能导致评估结果的不完整性和不确定性\cite{Merz2010}。

随着三维城市建模和建筑信息模型(BIM)在城市管理领域的广泛应用,本章在城市信息模型(CIM)框架下,借鉴了微尺度洪灾损伤评估的相关研究成果\cite{Amirebrahimi2016},提出了一个集成框架。该框架利用详细的三维建筑模型进行洪灾损伤评估,并实现三维可视化。通过结合第二章中输出的洪水参数与第三章的BIM/MGNM数据,框架成功构建了"风险评估$\rightarrow$疏散路径$\rightarrow$构件损伤"的多尺度闭环。

与传统FDA方法相比,本章提出的评估方法具有以下优势:(1)通过利用BIM和CityGML的丰富语义信息,实现了建筑构件的精确几何建模和材料表征;(2)采用时变洪水作用分析,揭示了洪水对建筑物的动态影响过程;(3)构建了多物理场耦合的损伤判据,综合考虑了静水压力、动水压力、浮力和水接触等多种损伤机制;(4)实现了构件级损伤的三维可视化,从而增强了决策支持的直观性和有效性。

本章提出的基于CIM的洪灾损伤评估框架主要包括数据准备、物理损伤评估、损失量化和可视化报告四个阶段,如图~\ref{fig:cim_framework}所示。该框架充分利用了三维城市模型在几何精度和语义丰富度方面的优势,使得建筑物个体层面的损伤分析得以精细化实现。

\begin{figure}[htbp]
  \centering
  \includegraphics[width=0.8\textwidth]{PIC-5/fig5_framework.jpg}
  \caption{基于CIM的洪灾损伤评估框架}
  \caption*{Figure~\thefigure~ CIM-based Flood Damage Assessment Framework} 
  \label{fig:cim_framework}
\end{figure}

\section{数据准备与统一信息模型}

\subsection{建筑信息与语义扩展}

BIM 模型采用 IFC 2\,×\,3 标准,包含墙体、梁板、立面构件、门窗、楼梯等 612 个构件实例。为了适应损伤分析需求,针对 BIM 元素进行了以下语义扩展:
\begin{enumerate}
  \item 耐水性能等级:按照《建筑防水工程技术规范》将材料分为 A(耐水)、B(短时耐水)、C(易损)三级,并将该信息附加到 IfcMaterial 层级。
  \item 力学参数:为承重墙、幕墙、门窗等构件附加密度、抗压/抗剪强度、弹性模量等力学参数,这些数据来源于设计图纸和标准手册。
 \item 连接关系:通过 IfcRelConnectsElements和IfcRelSpaceBoundary 重建构件之间的拓扑关系,用以识别洪水作用下的受力路径和渗透通道。
\end{enumerate}
图~\ref{fig:bim_model} 展示了语义增强后的 BIM 模型,后续的损伤可视化将基于该模型进行。

\begin{figure}[htbp]
  \centering
  \includegraphics[width=0.8\textwidth]{PIC-5/fig5_bim_model.jpg}
  \caption{滨海应急服务中心的BIM模型(语义增强后)}
  \caption*{Figure~\thefigure~ BIM Model of the Coastal Emergency Service Center (Semantic Enhancement)} 
  \label{fig:bim_model}
\end{figure}

\subsection{洪水参数与时间序列}

第二章中,通过 MIKE 21 得到的水深 $h(x,y,t)$、流速 $U(x,y,t)$和持续时间 $T(x,y)$ 等数据,通过双线性插值映射至建筑足迹,并将数据沿入口与竖向交通节点投影至各楼层。时间维度采用 $\Delta t = 300\,\mathrm{s}$ 的等间隔离散,步长与 Kelman 渗透模型所需的步长一致。为了保证局部水动力效应的准确性,对建筑周边 20 m 范围内的模拟结果进行了 0.5 m 网格细化处理。

\subsection{成本与维修库}

构件成本主要来源于以下几个方面:
\begin{itemize}
  \item 山东省 2023 年《建筑装饰工程消耗量定额》与《房屋修缮工程消耗量定额》,提供了常见装饰和设施的基准单价及折旧规则;
  \item 滨海应急服务中心资产台账,补充了专用设备(如配电箱、应急物资柜)的采购与折旧信息;
  \item 威海市公共资源交易中心价格信息,用于校核市场价格波动。
\end{itemize}
成本库采用 MasterFormat 分类编码,并与 BIM 构件一一对应,使得损伤评估能够直接输出货币化的量化结果。

\section{洪水作用计算与构件损伤判据}

损伤评估同时考虑水动力和水接触两类效应。参考 FEMA\cite{FEMA2012}、Kelman\cite{Kelman2002} 等规范,建立了如下计算流程。图~\ref{fig:damage_process}展示了物理损伤评估的详细流程,包括洪水作用计算、构件阻力分析和损伤状态判定等关键环节。

\begin{figure}[htbp]
  \centering
  \includegraphics[width=0.8\textwidth]{PIC-5/fig5_damage_process.jpg}
  \caption{物理损伤评估过程流程图}
  \caption*{Figure~\thefigure~ Flowchart of the Physical Damage Assessment Process} 
  \label{fig:damage_process}
\end{figure}
\begin{enumerate}
  \item 水头差与渗透:根据室外水深 $h_{\mathrm{out}}$、室内水深 $h_{\mathrm{in}}$ 以及开口构件(门、窗、通风口)的渗透系数 $C_k$,采用 Kelman 渗透模型计算每个时间步的水量交换 $I_k(t)$,并更新室内水位。图~\ref{fig:water_levels}展示了室内外水位变化过程,结果表明,由于建筑物开口的渗透作用,室内水位逐渐上升,最终与室外水位趋于平衡。
  \item 静水压力:对于每个墙体、幕墙等结构构件,采用公式 
  \begin{equation}
    p_{\mathrm{hyd}} = \frac{1}{2} \rho g (h_{\mathrm{out}}^2 - h_{\mathrm{in}}^2)/t_w,
  \end{equation}
  估算净静水压力,其中 $t_w$ 为墙厚。如果 $p_{\mathrm{hyd}}$ 超过构件抗压强度的 60\%,判定为“结构受限”。
  \item 动水压力:依据流速分量 $U_\perp$,采用公式 
  \begin{equation}
    p_{\mathrm{dyn}} = 0.6 \rho U_\perp^2 
  \end{equation}
  计算迎水面构件的附加压力,并考虑正负压差引起的吸附破坏。
  \item 浮力:对于地下或半地下空间构件,根据排水体积 $V$ 与外部水深,计算浮力 $F_{\mathrm{buoy}} = \rho g V$,如果浮力与结构自重之比超过 0.8,则标记为“脱离风险”。
  \item 水接触损伤:根据材料的耐水等级与浸泡时间 $t_{\mathrm{wet}}$,采用指数模型 $D = 1 - e^{-\alpha t_{\mathrm{wet}}}$ 评估材料损伤程度,其中 $\alpha$ 由材料等级确定(如木质地板 $\alpha=0.12$,PVC 地板 $\alpha=0.04$)。
\end{enumerate}
将上述判据综合,得到构件损伤状态(结构破坏/脱离风险/材料损伤/安全),并记录触发的主导机制,便于后续可视化说明"损伤因何而起"。

\begin{figure}[htbp]
  \centering
  \includegraphics[width=0.8\textwidth]{PIC-5/fig5_water_levels.jpg}
  \caption{建筑物室内外水位变化过程}
  \caption*{Figure~\thefigure~ Variation of Indoor and Outdoor Water Levels in the Building} 
  \label{fig:water_levels}
\end{figure}

\section{损失量化与关键指标}

损失量化遵循 Assembly-Based Vulnerability(ABV)思想\cite{Porter2001},将构件损伤与成本库关联,计算公式为:
\begin{equation}
  C_{\mathrm{loss}} = \sum_{i} \gamma_i \cdot c_i,
\end{equation}
其中 $c_i$ 为构件 $i$ 的维修或更换成本,$\gamma_i$ 为损伤系数,取值依据损伤等级(如完全损坏 $\gamma=1$,可修复 $\gamma=0.4$,外观修复 $\gamma=0.1$)。此外,为支持韧性评估,定义以下指标:
\begin{itemize}
  \item 构件损伤率 $R_{\mathrm{comp}} = N_{\mathrm{dam}} / N_{\mathrm{tot}}$;
  \item 功能恢复时间 $T_{\mathrm{rec}}$:由关键设施(如配电、通信)的损伤程度映射;
  \item 投资回收比 $B\!C\!R = \frac{\text{改造成本}}{\text{预期损失差}}$,用于评估韧性改造方案的经济性。
\end{itemize}
表~\ref{tab:cost_basis} 汇总了典型构件的成本与折旧假设。

\begin{table}[htbp]
  \centering
  \caption{损失评估单价与折算依据}
  \caption*{Table~\thetable~ Cost and depreciation assumptions for loss assessment} 
  \label{tab:cost_basis}
  \begin{tabular}{@{}p{3cm}p{3.5cm}p{3cm}p{3cm}@{}}
    \toprule
    构件类别 & 单价/元$\,\mathrm{m}^{-2}$(或台件) & 折旧与折算方法 & 数据来源 \\
    \midrule
    室内墙面饰材 & 320 & 5年直线折旧,残值率5\% & 山东省装饰工程定额(2023) \\
    地面铺装 & 450 & 5年直线折旧,残值率5\% & 山东省装饰工程定额(2023) \\
    低位电气设备 & 6800/台 & 以采购价80\%计入,可再利用部分扣除残值 & 威海市公共资源交易中心价格信息(2023) \\
    门窗组件 & 950/樘 & 按完好率估算;框体损坏按100\%计入 & 威海市装饰建材市场调研 \\
    家具及应急物资 & 420/件 & 使用寿命3年折旧,残值率10\% & 滨海中心资产台账(2023) \\
    \bottomrule
  \end{tabular}
\end{table}

\section{三维可视化表达与协同应用}

损伤结果通过三维可视化表达,以提高跨部门沟通效率:
\begin{itemize}
  \item 构件着色:依据损伤等级,采用“红-橙-黄-绿”四色体系,红色表示需要更换,橙色表示结构受限,黄色表示局部维修,绿色表示安全;
  \item 机制标注:利用 BIM 属性显示主导损伤机制(静压力、动压力、浮力、水接触),便于定位问题根源;
  \item 交互查询:结合 CityGML 的空间索引,实现按楼层、构件类型或损失金额的过滤与统计。
\end{itemize}
图~\ref{fig:damage_visualization} 展示了构件级损伤的三维表达方式:受损的门窗以红色突出,墙面饰材因水接触损伤标记为黄色,未受影响的幕墙保持绿色,用户可通过图层控制聚焦关键区域。

\begin{figure}[htbp]
  \centering
  \includegraphics[width=0.9\textwidth]{PIC-5/fig5_damage_visualization.jpg}
  \caption{构件级洪灾损伤的三维可视化表达}
  \caption*{Figure~\thefigure~ 3D Visualization of Component-Level Flood Damage}
  \label{fig:damage_visualization}
\end{figure}

\section{案例验证:威海滨海应急服务中心}

\subsection{场景设定与数据映射}

案例选择威海滨海应急服务中心首层及半地下功能区,场地紧邻北岸防潮堤,且地面高程低于周边道路 $0.4\,\mathrm{m}$,容易在风暴潮与暴雨复合条件下积水。选取 2021 年“烟花”台风在威海登陆时段的极端水位过程作为输入,外部水深峰值为 $0.68\,\mathrm{m}$,持续 $2.5\,\mathrm{h}$;降雨量则采用对应暴雨站点 $1\,\mathrm{h}$ 最大雨强 $34\,\mathrm{mm}$。第二章的 $0.5\,\mathrm{m}$ 分辨率风险栅格与第三章的 MGNM 网络在 CIM 平台内统一坐标系下注册,并通过 IFC 空间边界建立了首层 246 个构件与半地下 58 个构件的关联关系。时间维度沿用 $\Delta t=300\,\mathrm{s}$ 的离散步长,总计 30 个时间步覆盖整个事件过程。

\subsection{联动评估流程}

案例验证遵循“水动力响应—疏散约束—构件损伤”闭环流程:
\begin{enumerate}
  \item 风暴潮与暴雨引起的外部水深、流速与压力通过三维 RANS 模型输出,并映射至建筑外墙、入口和地下空间的控制面;
  \item 第三章构建的风险驱动疏散图层提供了入口通行能力、滞留时间等信息,这些作为门窗渗透系数与功能恢复优先级的约束条件;
  \item 依据 Kelman 渗透模型与多物理损伤判据,计算各构件的损伤状态,并输出与 BIM 构件绑定的损伤标签、主导机制及维修成本;
  \item 结果同步回写至 CIM 平台,生成洪水淹没体、构件损伤点云与业务提醒(如“机房浮力风险”),供应急指挥系统联动调度使用。
\end{enumerate}
该流程在 CIM 数据总线中保持自动化,能够在 14.6 分钟内完成一次完整演算,满足小时级滚动推演需求。

\subsection{量化结果与验证}

损伤评估结果与现场资料、历史演练记录进行了比对,关键指标如表~\ref{tab:case_metrics} 所示。

\begin{table}[htbp]
  \centering
  \caption{威海滨海应急服务中心案例关键指标}
  \caption*{Table~\thetable~ Key Indicators of the Weihai Coastal Emergency Service Center Case} 
  \label{tab:case_metrics}
  \begin{tabular}{@{}p{4cm}p{5cm}p{5cm}@{}}
    \toprule
    指标 & 模型结果 & 参照/验证依据 \\
    \midrule
    构件损伤率 $R_{\mathrm{comp}}$ & 27\%,其中门窗 62\%、墙面 54\% & 2021 年 8 月积水事件照片、运维记录 \cite{WeihaiEM2022} \\
    静水压力峰值 & 北侧外墙 38.5\,kPa,达到抗压强度的 65\% & 结构设计图纸与验算书(内部资料) \\
    浮力系数 & 半地下机房 0.78(临界) & 现场排水能力评估报告 \cite{WeihaiEM2022} \\
    直接经济损失 $C_{\mathrm{loss}}$ & 51.4 万元 & 2022 年资产盘点单价、保险理赔清单 \\
    功能恢复时间 $T_{\mathrm{rec}}$ & 21 d(瓶颈:配电与通信机柜) & 2020 年应急演练恢复时长 19 d \cite{CN_Chu2022EvacuationDrill} \\
    疏散通行能力 & 入口 E3 受损后容量下降 48\% & 第三章疏散仿真中 E3 风险代价提升 52\% \\
    \bottomrule
  \end{tabular}
\end{table}

\begin{figure}[htbp]
  \centering
  \includegraphics[width=0.9\textwidth]{PIC-5/fig5_inundation.jpg}
  \caption{威海滨海应急服务中心洪水淹没体与构件损伤叠加可视化}
  \caption*{Figure~\thefigure~ Overlay Visualization of Flood Inundation and Component Damage at the Weihai Coastal Emergency Service Center}
  \label{fig:3d_visualization}
\end{figure}

图~\ref{fig:3d_visualization} 展示了洪水淹没体与构件损伤的叠加情况,可以直观辨识受损区域与水动力主导方向;结合图~\ref{fig:damage_visualization} 的构件着色结果,门窗组件的高损伤率以及机房围护结构的浮力风险得以精确定位。与现场水位尺记录对比,模拟峰值水深误差控制在 $0.06\,\mathrm{m}$ 以内,表明水动力与渗透模型在本案例中具有较高的可信度。

\subsection{韧性提升策略与业务协同}

基于损伤结果与敏感性分析,提出如下业务协同建议:
\begin{itemize}
  \item 设施改造优先级:在入口 E3 增设 $0.9\,\mathrm{m}$ 防水闸门与门窗密封条,可将门窗损伤率降至 28\%,并在疏散模型中恢复 85\% 的入口通行能力;在半地下机房布设抗浮锚杆与备用泵站后,浮力系数降至 0.42,功能恢复时间缩短至 12 d;
  \item 应急演练流程:将洪水淹没体与损伤清单输出至市应急管理局指挥平台,形成“预警—封控—排水—复盘”的演练脚本,弥补传统演练中缺乏量化损伤环节的不足;
  \item 保险与资产管理:构件级损失明细可作为理赔与资产盘点的客观依据,并支持年度韧性投资的成本-收益分析,避免资源投入的主观性分配。
\end{itemize}
综合来看,案例验证表明所构建的损伤评估模块不仅能与第二章和第三章的模型无缝衔接,还能在业务层面支撑设施改造、应急演练与资产管理等多个决策,体现了 CIM 平台在实际应急场景中的可操作性。

\section{本章小结}

本章在城市信息模型(CIM)框架下,构建了面向构件级的洪灾损伤评估与三维可视化模块,实现了与前两章的多尺度联动。该模块充分发挥了CIM在城市尺度和BIM在建筑尺度的技术优势,成功实现了跨尺度的无缝集成。主要结论如下:

\begin{enumerate}
  \item 建立了CIM-BIM集成的损伤评估框架:通过扩展IFC数据模型的语义信息,将建筑构件的力学参数和耐水性能等级集成到三维模型中,实现了"城市洪水风险—建筑构件响应"的多尺度关联。

  \item 构建了多物理场耦合的损伤判据体系:结合Kelman渗透模型、FEMA洪水作用计算方法和材料耐水模型,建立了静水压力、动水压力、浮力和水接触等多种损伤机制的定量判据,能够准确识别构件失效的主导因素。

  \item 实现了基于ABV的精细化损失量化:采用Assembly-Based Vulnerability方法,建立了构件损伤状态与维修/更换成本的映射关系,提出构件损伤率、功能恢复时间、投资回收比等关键指标,为灾后恢复决策提供量化依据。

  \item 开发了与CIM联动的三维可视化系统:基于CityGML和BIM的空间索引功能,实现了按构件类型、损伤程度和损失金额的动态过滤和统计查询,支持损伤机制的交互式标注和多层级可视化表达。

  \item 验证了方法在实际案例中的有效性:以威海滨海应急服务中心为例,识别出门窗组件(损伤率62\%)、墙面饰材(损伤率54\%)等高风险构件,估算总损失51.4万元,功能恢复时间21天,为韧性改造和风险管控提供了科学依据。
\end{enumerate}

本章构建的损伤评估模块与第二章的三维水动力风险评估、第三章的室内外一体化疏散规划形成有机闭环,实现了"风险识别$\rightarrow$应急响应$\rightarrow$损伤评估$\rightarrow$恢复优化"的全流程数字化管理。该技术路线不仅适用于滨海城市的洪灾治理,还可扩展应用于其他类型的城市灾害风险管理,为智慧城市的韧性建设提供技术支撑。
