\chapter{基于CIM的城市防灾应急协同系统构建}

\section{系统总体设计与功能框架}

\subsection{研究定位与总体目标}

本章意在将高精度洪涝风险评估、MGNM 室内外一体化疏散以及构件级损伤评估,与三维可视化能力一并融入统一的 CIM(City Information Modeling)平台,以构建面向城市防灾与应急处置的协同框架。我们认为,此过程更像是对既有成果的结构化整合,而非简单叠加,其核心在于让数据、模型与业务逻辑在同一体系内相互联通。框架设计立足系统工程方法,通过推动多类信息的深度耦合,为灾前、灾中与灾后的连续管理提供可运行的技术支撑。本文倾向于将这一机制理解为一种逐步形成的综合体系,需要在统一的 CIM 语义和空间坐标下协调多源数据的动态汇聚,并促使不同模型在约束一致的条件下协同求解。这样的架构虽然仍有进一步完善的空间,但现阶段的实现表明,它能够支持智能化决策生成,并在现场反馈的作用下形成相对稳定的闭环,从而增强防灾业务的时效性与适应能力。

系统设计强调分层解耦与职责明确。数据层、模型层与业务层在结构上相对独立,通过规范化接口保持松耦合关系,从而维持系统的可扩展性。所有数据对象均依托 CIM 语义进行统一建模,我们认为,这种做法虽需保持持续一致性,但有助于保障建筑、道路、管网和地形等多源空间要素在时空表达上的协调。系统支持模块的可插拔运行与计算编排,不同模型可根据场景需求灵活组合与调用。传感器、遥测与人机交互端的动态数据不断用于更新模型参数和路径代价函数,使系统呈现出一定的自适应能力。一个可能的解释是,这种动态校准机制在应对复杂环境变化时更具稳健性。为确保工程实现的可验证性,系统在设计阶段引入可观测指标体系,涵盖路径重算时间、系统响应时延、渲染帧率等性能指标,以及预测精度、疏散效率与资源到位率等业务指标,为后续的系统验证提供相对明确的量化依据。
\subsection{三层体系框架与模块边界}

系统总体结构采用“三层—四域”的架构形式,即数据支撑层、模型计算层和业务应用层,贯穿风险感知、应急决策、资源调度与可视化展示四个功能域。其总体组织关系可见图~\ref{fig:System_Structure}所示。

\begin{figure}[H]
\centering
\includegraphics[width=0.8\textwidth]{PIC-ins/1.png}
\caption{系统总体结构}
\caption*{Figure~\thefigure~ Overall system structure}
\label{fig:System_Structure}
\end{figure}

数据支撑层以CIM核心数据库为中心,整合BIM、GIS、DEM、气象、水文、遥感及IoT传感器等多源数据,通过统一的语义映射与空间基准来维持地形、水系、建筑与管网要素间的对应关系。我们认为,这种处理方式虽依赖持续维护,但在复杂场景下更能保证数据一致性。模型计算层封装各类关键算法模块,涵盖三维水动力学仿真、MGNM室内外路径规划、构件级损伤评估与风险分级预警,使跨尺度协同计算成为可能。一个可能的解释是,统一的调用机制降低了不同模型间的耦合度,从而提高了计算组织的灵活性。业务应用层以城市防灾任务为导向,提供态势监测、风险分析、决策推演、资源调度与三维可视化等功能,为应急管理流程提供相对完整的一体化支撑。

\subsection{系统功能构成与职责划分}

围绕“灾前预警—灾中响应—灾后评估”的防灾流程,系统功能被组织为四个相互关联的子域。风险感知子域依托多源数据与三维水动力学模型生成洪涝深度、流速及影响范围等时空分布,并在分级标准的约束下形成动态风险图。我们认为,这类基于模型的风险表达虽存在不确定性,但在支撑整体态势判断方面仍较为有效。应急决策子域以MGNM路径引擎为核心,将风险场结果转化为风险感知代价函数,进而计算多主体、多尺度的疏散与交通组织方案。一个可能的解释是,该机制能够在复杂情势下更好地协调不同主体的路径需求。资源调度子域在道路约束与需求预测的条件下完成车辆、人员与装备的时空分配,与路径计算过程保持滚动联动,以提升调度的可行性。可视化展示子域在三维CIM环境中呈现风险态势、路径规划、资源分布及构件级损伤等信息,为指挥中心、现场终端与公众平台提供分层且可交互的可视化服务,从而支持业务执行的连续性。

\subsection{运行逻辑与信息流闭环}

系统运行遵循“感知—求解—决策—反馈”的闭环逻辑,其信息流关系见图~\ref{fig:System_info}。系统首先从CIM核心库加载城市的静态空间基底,包括建筑、道路、地形与管网等要素,并持续接入气象、水文、遥感及IoT设备的实时数据流。我们认为,这种静态与动态数据的并行获取,为后续模型求解建立了较为稳固的输入条件。在此基础上,水动力学模型生成随时间变化的风险场,为MGNM路径规划提供代价输入;构件级模型则对重点设施进行受灾评估,使风险信息在空间和设施层面形成相对完整的集合。决策模块依据模型输出,为指挥中心、现场终端与公众平台等不同主体生成可执行的疏散与封控方案。一个可能的解释是,这种多主体适配机制能够在复杂条件下保持方案的可实施性。现场位置回传、道路通行状态及设施运行状况等实时反馈会进一步触发模型参数与路径代价的更新,使系统具备一定的自适应调节能力,从而维持闭环控制的连续运行。

\begin{figure}[H]
\centering
\includegraphics[width=0.8\textwidth]{PIC-ins/2.png}
\caption{系统整体信息流关系}
\caption*{Figure~\thefigure~ The overall information flow relationship of the system}
\label{fig:System_info}
\end{figure}

\subsection{关键接口与数据契约}

为支撑跨层级、跨模型的协同运行,系统在概要层设计了统一的接口与数据契约。所有模块共享一致的城市级空间坐标体系(如CGCS或UTM)与时间标准(UTC+偏移),并依据CIM核心对象Schema进行描述。核心对象涵盖建筑、道路、水动力网格与传感器等基础实体,每类对象均具备规范的几何、属性与状态字段。我们认为,这种一致化描述虽需持续维护,但在保证模型联通性方面具有实际意义。

模型间的数据交互遵循明确的输入输出约定:水动力学模型以地形、糙率、边界和降雨条件为输入,输出包含深度与速度的风险场;MGNM模型依赖空间网络、风险代价及约束条件,生成多主体路径与耗时指标;构件评估模型接收水位、持续时间与材料信息,计算损伤指数与结构状态。一个可能的解释是,这类标准化接口减少了模型间的耦合风险,使协同求解更易组织。系统内部的事件通信通过主题机制实现,如/risk/update、/route/recompute、/dispatch/assign等主题消息,用于各模块间的异步触发与数据流转,从而维持整体流程的连贯性。

\subsection{系统层可观测指标}

为衡量系统的整体性能与业务成效,本研究在概要层设置了两类可观测指标。工程性能指标用于评估系统的运行效率,包括路径重算时延、资源调度的收敛时间、三维场景的渲染帧率以及数据传输吞吐量等。我们认为,这些指标虽无法覆盖全部细节,但在刻画系统负载与响应能力方面具有实际参考价值。业务成效指标关注模型输出与业务需求之间的匹配度,涵盖风险预测误差、疏散到达时间分布、关键设施损伤判定的准确率,以及资源到位率与平均响应时间等内容。一个可能的解释是,此类指标能够从业务角度反映系统在不同约束条件下的适应性。上述指标将在后续章节的系统验证中用作量化依据,用以评估系统在多类灾害场景下的实际表现与工程可行性。

\section{系统架构与数据模型集成设计}

\subsection{架构设计思路与实现原则}

在前述总体框架之上,本节进一步阐述系统的实现性架构与数据—模型融合机制。系统采用分布式微服务作为核心组织方式,通过CIM平台实现数据、模型与业务模块的可插拔集成。我们认为,这类架构在保持组件自治的同时,借助统一的消息通信与数据访问协议实现全局协同,从而在计算性能与扩展能力之间取得了相对平衡。

架构设计以解耦和模块化为基本原则。模型计算、数据处理与可视化渲染均以独立微服务运行,并依赖标准化接口完成通信。各模块之间没有直接依赖关系,而是通过消息总线及接口规范维持松耦合交互。事件处理机制采用事件驱动与异步通信模式,当风险场更新、路径重算或调度命令触发时,事件以主题(Topic)发布给订阅模块,实现高并发条件下的即时响应。

在数据管理层,系统构建了统一的数据访问接口,所有模型计算均通过该层与 CIM 数据库交互。该机制为模型提供一致的空间语义与坐标基准,使各类算法在不直接处理底层数据结构的情况下完成计算与结果写回。我们认为,这一抽象层在降低模型间差异带来的适配成本方面具有现实意义。为增强运行稳定性,系统配置了容错与高可用机制,核心模块具备状态恢复与故障转移能力,即便在高负载或链路中断等情况下也能保持服务连续性。系统运行结构及各模块的交互关系见图~\ref{fig:System_topo},其中展示了数据访问层、模型算子、消息总线与应用端在实际部署中的通信路径与信息流转方式。

\begin{figure}[H]
\centering
\includegraphics[width=0.8\textwidth]{PIC-ins/3.png}
\caption{系统交互关系拓扑}
\caption*{Figure~\thefigure~ System interaction relationship topology}
\label{fig:System_topo}
\end{figure}

\subsection{数据与模型的一体化集成机制}

系统在CIM平台的支撑下实现了数据与模型的深度融合,其核心机制包括统一的数据服务体系、模型算子化封装以及数据到模型的动态映射。

首先,统一的数据服务体系依托分布式数据库结构,实现多源信息的集中管理。系统在城市语义模型(City Ontology)的约束下,将BIM、GIS、DEM等静态数据与气象、水文、遥感和传感器等动态数据整合到数据中台,形成可统一访问的数据服务层。该层承担数据注册、版本管理、坐标与时间对齐、索引缓存及权限控制等职责。我们认为,这种抽象机制在一定程度上减轻了模型对底层结构的依赖,使各模型能够在不关心具体数据组织方式的情况下完成访问与计算。

接着,模型集成机制通过算子(Operator)封装实现可插拔式部署。每个模型模块均以独立算子形式运行,具备明确的输入、输出及依赖项,并通过任务调度系统进行注册。当事件触发时,调度器自动调用相关算子完成计算。系统中的主要算子包括水动力学算子(生成风险场与流速场)、MGNM算子(执行室内外路径规划)、结构算子(进行构件级损伤评估)以及基于路径与需求预测的资源调度算子。算子之间通过数据引用与消息队列交互,而非直接调用,从而构成一个松耦合、可组合的计算环境。

最后,数据与模型的映射过程在运行时按照既定流程执行。数据准备阶段完成多源数据的同步、清洗与标准化,形成统一的数据包;模型调用阶段,各算子通过接口访问数据服务层并加载运行参数;在结果发布阶段,模型输出通过API网关或消息队列写回数据库或传递至下游模块。一个可能的解释是,该机制在语义、时空与数据一致性层面提供了可靠保障,使串行或并行的多模型协同计算得以顺利实现。

\subsection{多源信息融合与接口实现}

在多源信息融合层面,系统以CIM平台为枢纽,采用“集中注册—异步订阅—动态更新”的方式实现实时集成。各类数据源在注册中心登记元数据及访问方式,模型模块以订阅方式获取相关主题;当数据发生变化时,系统自动触发事件通知,使风险感知、路径规划与调度优化等模块能够保持联动并及时更新。我们认为,这种机制在一定程度上提升了数据流的灵活组织能力,使系统能够根据场景调整优先级和传输策略。

系统接口设计强调标准化与互操作性,主要分为内部接口、外部接口与管理接口三类。内部接口服务于模型算子间的通信,依托消息总线(如MQ或Kafka)完成事件与计算结果的异步传递;外部接口面向业务应用层,通过RESTful API或WebSocket支撑可视化平台与外部系统的交互;管理接口承担系统监控、日志收集和状态维护等任务。所有接口统一采用JSON或GeoJSON格式,并包含空间索引与时间戳标识,以保持跨平台的一致性与可追溯性。

数据安全与运行监测构成系统的重要保障机制。系统通过身份认证与访问控制限制不同角色的数据可见范围,并对位置、轨迹等敏感信息在公开展示前进行脱敏处理。接口调用会自动记录访问日志、执行时长与响应状态,为性能分析与溯源诊断提供依据。一个可能的解释是,这些安全与监控措施不仅提升了系统的可靠性与数据保密性,也为后续扩展与维护奠定了较为稳固的工程基础。

\subsection{关键技术实现与性能优化}

为保证系统在城市级场景下的实时性与稳定性,本研究在整体架构之上实现了一系列关键工程技术与性能优化策略。系统采用“容器化部署—消息驱动—前端增量渲染”的技术路线,在保持模块自治与可扩展性的前提下提升高并发条件下的数据吞吐与渲染效率。我们认为,这种组合式架构在大型场景中更具可操作性。

在部署层面,模型计算服务、数据访问服务与可视化服务均以Docker容器封装,并通过容器编排平台集中管理。各微服务经由服务注册与发现机制接入网关,网关基于Nginx提供负载均衡与反向代理,从而在洪涝演化与路径重算频繁触发时实现资源的弹性扩展。面向水动力学仿真与MGNM路径规划等计算密集型任务,系统将核心求解模块部署在GPU节点上,利用CUDA加速矩阵运算与图搜索过程。一个可能的解释是,相较于传统CPU实现,这一方案使平均计算时间缩短了约35\%,从而改善了整体计算响应能力。

在数据流转方面,系统依托Kafka构建统一消息总线,并采用“主题细分—分区并行”的策略管理不同类型的数据流。洪涝风险更新、路径重算请求与调度指令分别映射至/risk、/route与/dispatch等主题,各主题依据城市区域或业务类别再细分为多个分区,由独立的消费者组并行处理。我们认为,这种结构在一定程度上提升了高并发条件下的吞吐能力。为减轻数据库压力,系统在消息总线与数据库之间加入Redis缓存,用于存储高频访问的风险切片、路径片段与任务状态,缓存时长一般为30~60秒,从而避免短时间内重复查询同一批数据。

在三维可视化前端,系统采用多级瓦片切片与层次细节(LOD)结合的场景管理策略。底层地形与建筑模型预先切分为多级瓦片,并在Cesium中按需加载;视锥外瓦片不参与渲染,视锥边缘区域以低分辨率模型过渡,以减轻GPU负载。动态风险场与路径结果通过独立动态图层叠加,仅在新模型结果生成或用户触发操作时增量更新,而非重绘整个场景。对于高频交互操作(平移、缩放、时间轴拖动等),系统采用前后端协同的节流策略,将请求频率控制在5~10次/秒区间,以保持交互流畅,同时减少不必要的网络调用。

在运行监控与故障恢复方面,系统进行了相对细致的工程设计。所有微服务集成统一的日志采集与指标上报模块,监测项包含CPU/GPU利用率、Kafka消息堆积量、接口调用耗时和三维渲染帧率等。当指标超过阈值时,监控模块会自动触发告警并记录相应时间段的运行日志。核心服务支持状态快照与自动重启策略,当单个算子服务异常退出时,编排平台会在数秒内完成实例重建与状态恢复。一个可能的解释是,这些机制使系统在城市级应急任务中能够规避局部故障带来的连锁影响。依托上述关键技术与性能优化,系统在复杂工况下仍能维持洪涝更新、路径重算与三维可视化的整体稳定运行。

\section{应急响应与协同决策机制设计}

\subsection{动态风险驱动的预警与响应}

本节构建了一个以动态风险为核心驱动的预警与应急响应机制,实现从灾情监测到决策执行的全流程闭环。系统通过对气象、水文及传感器数据的实时监控,将外部环境变化以数据流的形式持续注入模型计算层。当监测参数触及预设阈值时,系统自动启动应急响应流程。整个逻辑流程包括风险识别、等级预警、动态推演及应急联动四个阶段,具体信息流可参见图~\ref{fig:System_emer}。


\begin{figure}[H]
\centering
\includegraphics[width=0.8\textwidth]{PIC-ins/4.png}
\caption{系统应急响应流程}
\caption*{Figure~\thefigure~ System emergency response procedure}
\label{fig:System_emer}
\end{figure}

在风险识别阶段,水动力学模型依据实时输入生成洪涝范围、深度与演化趋势,并输出动态风险场。系统按照既定风险等级标准对结果进行分级,当等级达到轻微、严重或极端等阈值时,会自动触发相应预警事件。我们认为,这种基于阈值的触发方式虽依赖参数设置,但在实际场景中具有一定可操作性。完成识别后进入动态推演阶段,模型快速执行预测与风险场更新,形成新的代价函数与风险区划,为路径计算与资源调度提供后续输入。最终,系统将预警信息同步推送至业务层与外部管理平台,以驱动路径重算与资源调度模块的协同运行。

整体机制基于事件驱动方式。当风险场更新完成后,系统发布/risk/update事件,调度器捕获该事件后立即启动路径重算与资源调度。若局部区域风险上升,系统执行局部代价调整并重算路径;若风险扩展至全局,则启动全域规划;若检测到设施失效或交通封闭,系统会更新路网结构并完成避障重构。一个可能的解释是,通过这种逐层触发与更新机制,系统能够在数秒内完成从风险识别到路径输出的响应,从而在突发情势下保持动态适应能力。

依据城市防灾体系的分级要求,系统设置三级响应模式。轻微风险阶段主要发布预警信息,引导公众与管理部门保持关注;严重阶段在洪涝深度超过0.5米且持续时间超过30分钟时,系统自动触发疏散路径规划与交通调整;极端阶段当关键设施受淹或通信受阻时进入应急调度模式,并切换备用通信链路以维持决策链的连续性。通过分级响应设计,系统构建了“风险监测—模型求解—策略执行”的闭环,为城市应急反应提供了具备一定自适应性与韧性的技术基础。

\subsection{基于MGNM的多主体路径规划}

在应急响应过程中,MGNM(Multi-purpose Geometric Network Model)模型承担核心路径求解任务,是衔接风险场与资源调度的关键计算环节。MGNM 通过融合室内 BIM 语义与室外道路拓扑,实现室内外网络的统一表达,并以风险感知代价函数为驱动,在不同空间尺度下完成多主体路径规划。我们认为,这种跨域建模方式在复杂城区环境中具有较强的适应性。

系统先基于城市 GIS 数据构建室外道路网络的拓扑结构,生成节点与边的集合;随后解析建筑 BIM 信息,识别楼层、走廊与出口等要素,并映射为室内节点。通过建筑出入口、地铁通道与廊桥等关键连接点,系统建立室内外网络的跨域映射,确保全域连通性。动态更新机制使模型能够及时反映外部状态变化:当水动力学模型检测到道路封闭或积水深度超阈时,相应边会被自动禁用并触发路径重算。图~\ref{fig:Network}给出了网络生成与更新过程的逻辑结构。

\begin{figure}[H]
\centering
\includegraphics[width=0.8\textwidth]{PIC-ins/5.png}
\caption{网络生成与更新过程}
\caption*{Figure~\thefigure~ The process of network generation and update}
\label{fig:Network}
\end{figure}

MGNM路径规划采用“多主体任务协同—分层寻优”的策略,将疏散与救援对象划分为居民、应急车辆和救援人员三类。针对不同主体,系统定义差异化的目标函数:居民路径以安全性为核心,应急车辆强调最短时间,而救援人员更关注资源覆盖与通达性。我们认为,这种目标差异化设计更贴合实际应急需求。路径求解在GPU端并行执行,从而显著提升计算效率。

为减少路径冲突,系统在结果整合阶段引入路径重权算法,对道路容量进行动态分配,并根据情势调整通行优先级,使多主体在同一网络中的同时疏散与救援保持协调。规划完成后,路径结果以GeoJSON格式存储,并附带风险等级、预计耗时与安全评分等属性。结果既用于三维可视化模块的动态呈现,也作为资源调度模块的输入,用于后续任务分配与行动指令生成。一个可能的解释是,MGNM模块的引入在很大程度上弥合了室内外空间割裂带来的分析障碍,为应急响应提供了统一且连贯的空间计算基础。

\subsection{协同决策与任务调度逻辑}

系统的协同决策机制用于整合风险预测、路径规划与资源调度三类核心模块,形成可直接执行的应急指挥策略。其运行遵循“事件触发—策略生成—任务调度—反馈修正”的循环逻辑,如图~\ref{fig:Decision}所示。当风险场或路径信息发生更新时,系统自动进入策略生成流程,由策略生成器依据资源状态与任务优先级生成候选策略集,并结合成本、时间与风险权重等指标筛选出较优方案。我们认为,这种多指标权衡方式在复杂情境下更具韧性。策略确定后,系统根据地理位置、道路通行条件与资源可达性生成可执行的任务计划,使调度过程与现场条件保持一致性。

\begin{figure}[H]
\centering
\includegraphics[width=0.8\textwidth]{PIC-ins/6.png}
\caption{决策闭环图}
\caption*{Figure~\thefigure~ Decision Loop Diagram}
\label{fig:Decision}
\end{figure}

任务调度逻辑采用分层控制机制。全局调度层负责跨区域资源协调与任务分配,优先保障关键区域的响应;局部调度层在区域范围内完成路径分配与人员指派;反馈更新层实时接收现场回传的状态信息,包括任务进度、设备状况与交通延误,并在必要时触发任务重分配。系统通过任务队列与优先级调度器组织不同任务的执行顺序,实现多任务场景下的动态调度。当同一区域内并行存在疏散与物资运输等任务时,系统会依据优先级表调整执行顺序,从而减少潜在冲突。我们认为,这种分层调度方式在复杂任务负载下更具稳定性。

人机协同构成系统决策的重要组成部分。指挥中心界面允许用户在自动生成方案基础上进行人工修订与确认,系统会记录每次人工干预及其导致的结果变化,用于后续性能评估与模型优化。界面采用语义化可视化设计,以地图、时间轴与警示图层呈现决策内容,使决策者能够快速理解态势并下达指令。

系统还引入反馈学习机制,以实现自适应的持续优化。灾后阶段,系统自动汇总运行日志、传感器数据与任务执行记录,对决策偏差进行统计分析;根据分析结果动态调整风险分级阈值、路径权重与资源调度规则,并在后续任务中加载更新参数。一个可能的解释是,该机制使系统在多次运行中不断积累经验,从被动响应逐渐向主动优化过渡,形成具备演化能力的应急决策体系。

\section{三维可视化与系统实现验证}

\subsection{三维可视化界面与交互逻辑}

在CIM框架的支撑下,本研究构建了集信息展示、态势认知与决策交互于一体的三维可视化模块。该模块不仅用于呈现洪涝风险与疏散路径的动态演化过程,也是应急指挥、人机协同与数据反馈的核心界面。依托统一的城市信息模型,系统能够将建筑构件、道路网络、地形地貌与风险场景在同一空间环境中整合呈现,使计算结果与决策认知之间形成直接映射。我们认为,这种紧耦合的表达方式有助于减少跨界信息解释带来的偏差。

系统设计强调多尺度融合与一体化表达。可视化模块以三维CIM场景为载体,实现城市级与建筑级空间的连续切换,使风险分布、路径规划与资源调度结果能够在同一视图中协同展示。渲染架构采用分层机制:空间基础层负责加载地形、道路、管网与建筑等几何模型,确保空间精度与坐标一致;动态风险层实时渲染水动力学模型输出的洪涝深度、流速与风险等级,并通过颜色梯度与透明度表达风险演化;路径与资源层叠加MGNM的路径结果与应急资源位置,通过流向动画展示人员疏散与车辆运行状态;交互控制层提供多视角漫游、属性查询与时间轴回放功能,使用户能在空间与时间两个维度上追踪态势变化。界面结构如图~\ref{fig:Platform}所示。

\begin{figure}[H]
\centering
\includegraphics[width=0.8\textwidth]{PIC-ins/7.png}
\caption{整体界面逻辑结构}
\caption*{Figure~\thefigure~ Overall interface logical structure}
\label{fig:Platform}
\end{figure}

为进一步展示本研究所基于CIM平台实现的三维可视化能力,本节选取系统典型场景界面作为示例,用于说明三维场景呈现、图层叠加、指标动态更新及交互操作方式等关键功能。图~\ref{fig:Platform-show}与图~\ref{fig:Platform-show-three}展示了系统在真实城市空间中的可视化渲染效果,有助于理解系统在应急指挥中的信息整合与态势表达方式。


\begin{figure}[H]
\centering
\includegraphics[width=0.9\textwidth]{PIC-ins/8.png}
\caption{环境感知与监测数据叠加的三维场景展示界面示意}
\caption*{Figure~\thefigure~ Schematic diagram of the three-dimensional scene display interface combining environmental perception and monitoring data}
\label{fig:Platform-show}
\end{figure}

图~\ref{fig:Platform-show}进一步展示了系统在同一三维城市场景中叠加环境监测数据的能力。左侧面板以折线图、浓度指示条等形式呈现PM2.5、PM10、气压、风速等环境要素,并支持时间序列回放与周期趋势分析。三维场景中各建筑物上悬浮的图标代表不同监测节点,其颜色和数值随传感器实时数据动态更新,从而实现对城市环境状态的空间化表达。该界面展示了系统对多源数据的可视化融合能力,可用于应急响应中对危险区域环境变化的识别与预估,提升对灾害演化的总体把握。


\begin{figure}[H]
\centering
\includegraphics[width=0.9\textwidth]{PIC-ins/9.png}
\caption{城市三维场景下的人口态势可视化界面示意}
\caption*{Figure~\thefigure~ Schematic diagram of the visualization interface for population trends in a three-dimensional urban scene}
\label{fig:Platform-show-three}
\end{figure}

图~\ref{fig:Platform-show-three}展示了系统在人口与人员态势方面的三维可视化界面。左侧为多维指标面板,包括人口规模、分布结构、年龄构成及重点区域人员密度等内容,以动态柱状图与数值组件实时更新。右侧基于CIM的三维城市模型呈现建筑、道路、水体与地形等要素,采用分层渲染并支持动态图层的灵活切换。通过在建筑表面叠加人口密度或人员聚集度等信息,系统能够同时呈现宏观分布与局部热点。我们认为,这种空间化表达方式在人员疏散研判和区域风险识别中具有较高的实用性。

系统基于多线程渲染与瓦片缓存技术开发,可在常规GPU平台实现秒级场景加载与约30帧/秒的流畅渲染。Shader程序用于控制风险场动画,使洪涝演化以连续图层的方式展现,增强风险传播的可感知性。交互设计根据用户角色实现差异化逻辑:指挥中心端以城市级三维总览为核心,呈现风险区域、资源分布和任务进度;现场端依托移动终端,聚焦任务接收、路径导航与设施上报,并支持语音与文本输入;公众端提供简化的避险路线查询与信息发布,用于面向居民的风险可视化传播。三端经统一的消息中间件互联,形成“命令下达—执行反馈—状态更新”的实时闭环。界面支持时间轴控制与图层切换,使用户能够在不同尺度上观察灾情演变与应急过程,从而强化系统的动态性与交互特征。

\subsection{威海滨海应急服务中心系统部署}

为验证系统的可行性与工程适应性,本研究在威海滨海应急服务中心开展了部署与实际运行测试。部署环境由服务器集群、数据库系统、通信接口与前端渲染平台组成,其整体架构关系如图~\ref{fig:Weihai}所示。系统运行环境包含两台计算节点(每台配备32核CPU、RTX A6000 GPU与128 GB内存)、一台数据库服务器以及一台前端渲染服务器。我们认为,这一配置在城市级场景中能够基本满足模型计算与可视化并行运行的需求。数据库层以 PostgreSQL/PostGIS 管理CIM核心数据,MongoDB用于存储模型中间结果与运行日志。模块间通信通过Kafka消息队列完成事件传递,外部系统则经由RESTful API进行访问。前端界面基于Cesium与Vue开发,可与应急指挥中心的大屏系统直接集成,实现对三维空间数据的实时联动与可视化调度。

\begin{figure}[H]
\centering
\includegraphics[width=0.6\textwidth]{PIC-ins/10.png}
\caption{威海滨海应急服务中心部署架构}
\caption*{Figure~\thefigure~ Weihai Coastal Emergency Service Center Deployment Architecture}
\label{fig:Weihai}
\end{figure}

实验场景选取威海滨海区域典型的低洼易涝地带,范围约3.2平方公里,包含128栋建筑、24条主干道路与 3 处关键应急设施。系统以实时气象数据和地面传感器信息为输入,结合高精度DEM完成洪涝仿真;模拟设置为1分钟时间步长、2米空间分辨率,MGNM网络规模约为1.8万节点和2.15万条边。运行过程遵循“强降雨触发—路径更新—资源调度—疏散指挥”的主线,用于验证多模型协同、实时响应及可视化链路的完整性。我们认为,这一流程在实际场景中具有较好的代表性。

系统运行时,风险图层在三维场景中动态更新,洪涝深度与流速随时间变化实时渲染;路径规划结果以动态箭头与颜色变化叠加展示,受灾道路自动变灰或禁用;调度指令经消息队列下发至现场终端,并实时接收任务状态反馈。整体交互过程在毫秒级延迟下保持连贯,三维渲染帧率稳定在30帧以上,能够满足城市级应急决策的可视化需求。

为展示系统在应急服务中心的实际部署效果,本研究选取指挥大厅大屏端界面作为示例。界面采用多区块拼接布局:左侧为实时更新的数据面板,包含气象要素、环境监测指标、作业状态与人员信息;中部为基于CIM模型的三维地理场景,用于呈现区域环境、基础设施与监测设备;右侧集成监控视频流、风险预警信息与关键指标趋势曲线。通过三维场景与数据面板的联动,大屏支持态势总览、事件定位、信息查询与辅助决策,形成统一直观的指挥展示窗口,如图~\ref{fig:show}所示。

\begin{figure}[H]
\centering
\includegraphics[width=0.9\textwidth]{PIC-ins/11.png}
\caption{威海滨海应急服务中心指挥大厅大屏展示界面示意}
\caption*{Figure~\thefigure~ Illustration of the display interface of the command hall screen in the Weihai Coastal Emergency Response Service Center}
\label{fig:show}
\end{figure}

\subsection{验证指标与结果分析}

系统部署完成后,我们对运行性能、模型协同和整体稳定性进行了定量验证。结果表明,框架在工程化应用中具备较高的可用性与稳健性。根据测试日志与监测数据,系统在洪涝场更新、路径重算、任务下发及渲染性能等方面均保持稳定:洪涝风险场平均更新周期约32秒、峰值45秒;MGNM路径重算平均7.6秒、峰值11秒;指挥端至现场端的调度延迟稳定在1秒以内;三维渲染帧率维持在约31帧。与传统静态路径规划相比,疏散路径的安全性提升约19\%,资源到位时间缩短约23\%。我们认为,这些结果在一定程度上支持系统在实际应急环境中的性能优势。

模型协同验证显示,系统可在洪涝扩展后的两到三分钟内自动完成路径重算与调度更新,风险感知—路径规划—任务调度链路的整体延迟控制在45秒以内。多模型间的数据传递稳定,接口通信未出现丢包或延迟累积,说明集成机制在并行运行条件下具备较好的鲁棒性。系统连续运行48小时未出现宕机或通信阻塞,也进一步证明其在高并发与长周期场景下的结构稳定性。模型算子可按需扩展或替换,为未来引入暴雨—内涝—风暴潮等多灾种耦合模拟提供了技术条件。可视化界面支持模块化扩展,用户可通过插件加载不同主题,实现多场景配置。

总体来看,基于CIM的城市防灾应急协同框架在时效性、协同性和直观性方面展现出明显优势。系统能够在分钟级实现数据更新与路径重算,模型间协同紧凑,任务调度同步推进,并通过三维可视化提供清晰可读的决策支持。威海滨海应急服务中心的应用验证显示,该框架在实际防灾工程中具有较好可行性,也表明CIM平台在应急管理领域具备进一步推广的潜力。

\section{本章小结}

本章围绕基于CIM的城市防灾应急协同框架,从体系设计到系统验证构建了完整的技术链路。通过总体设计与架构实现,论文提出以CIM为核心的数据、模型与业务一体化体系,在此基础上形成可扩展的微服务架构与事件驱动机制。结合动态风险驱动方法与MGNM多主体路径模型,我们认为所构建的协同决策机制能够较好地实现风险识别、路径规划与资源调度之间的自适应联动。在威海滨海应急服务中心的落地部署与实证运行中,系统完成了三维动态可视化与实时响应的整体验证,表现出良好的可行性与稳定性。综上,本研究实现了从理论模型到工程应用的转化,为CIM环境下的城市应急管理提供了统一的技术体系与实现路径。
