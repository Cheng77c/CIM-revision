\chapter{绪论}

\section{研究背景和意义}

进入21世纪以来,全球城市化进程持续加速。据联合国统计,2007年世界城市人口首次超过总人口的50\%,预计到2050年该比例将达到68\%\cite{UNDESA2019}。国家统计局数据显示,中国城市化率从1978年的17.9\%快速提升至2023年的65.2\%\cite{NBS2024},城市人口规模和空间范围持续扩张。城市化进程在促进经济社会发展的同时,也深刻改变了人类聚居环境的空间格局与风险结构。

高密度人口聚集和复杂基础设施系统的相互依存,使城市系统在面临自然灾害和突发事件时展现出显著的系统性脆弱性。洪涝、地震、台风、极端高温等自然灾害频发,对城市安全与可持续发展构成严重威胁。近年来,极端天气事件呈现出强度增强、频率增加的趋势。2021年郑州"7·20"特大暴雨事件造成城区大面积内涝,交通系统瘫痪,人员伤亡重大,充分暴露了超大城市在极端降水事件面前的系统性风险\cite{StateCouncil2022}。类似事件在全球范围内频繁发生,如2011年东日本大地震引发的复合灾害链\cite{CabinetOffice2012}以及2017年美国休斯敦"哈维"飓风造成的城市洪涝灾害\cite{FEMA2018},均表明现代城市复杂系统在面对极端灾害冲击时存在显著的脆弱性。

在传统的城市治理模式下,灾害管理多以单一部门、静态数据和被动响应为主,难以满足超大城市面对极端灾害时的实时感知与动态调度需求。伴随信息技术演进与治理理念更新,从数字城市向智慧城市的转型成为全球城市发展的核心趋势\cite{Batty2018,Cao2023}。智慧城市理念的提出标志着治理逻辑由静态的数据中心化转向动态的系统联动与行为驱动,强调"感知—分析—响应—恢复"的闭环机制。城市韧性治理作为现代城市管理的重要理论范式,强调通过系统性、适应性与协同性机制提升城市应对复杂风险的能力\cite{CN_Li2024CoastalCIM,CN_Qin2024ResilienceIndex}。
 
近年来,中国在国家和地方层面持续推进韧性城市建设。《韧性城市体系构建纲要(2023--2035年)》明确提出要构建以CIM为核心的城市安全运行平台\cite{NDRC2023};《山东省城市应急管理行动计划(2022--2025年)》重点部署滨海城市风暴潮与洪涝灾害的数字化治理体系\cite{Shandong2022};住房和城乡建设部发布的相关指南也要求在应急设施中推广BIM/CIM协同应用\cite{MOHURD2022}。这些政策为本研究提供了重要的制度背景和应用需求,凸显了面向实际治理场景的技术创新迫切性。与此同时,多地积极探索以CIM驱动的综合应急指挥体系和协同治理模式\cite{CN_Tang2024EmergencyPlatform,CN_Lan2024CIMGovernance}。

城市信息建模(City Information Modeling, CIM)作为智慧城市技术体系的核心组成,是指构建覆盖城市空间全域、对象多尺度、语义多层级的综合性信息表达框架。CIM不仅整合了建筑信息模型(BIM)的精细化建筑语义与地理信息系统(GIS)的空间分析能力,还融合了物联网感知数据、社会经济统计信息和实时运行状态,形成城市系统的数字镜像。从技术架构来看,CIM具备四个核心特征:多源异构数据的融合能力、多尺度空间表达的统一性、多层级语义关联的完整性以及多时态动态过程的可模拟性。在应用层面,CIM已广泛应用于城市规划建设、基础设施管理、公共安全应急、环境监测预警等领域,特别是在城市灾害风险管理中展现出独特价值。通过构建高精度的城市三维数字底座,CIM能够支持洪水淹没模拟、建筑脆弱性评估、疏散路径优化、应急资源调度等关键功能的集成运行,为城市韧性治理提供技术支撑\cite{CN_Zhang2023DigitalTwin,CN_Shi2022CommunityTwin}。

随着信息技术的不断演进,海量城市数据的获取与存储已不再是瓶颈,真正的挑战在于如何实现跨部门、跨领域、跨空间尺度的数据融合与建模。传统的地理信息系统(GIS)侧重于空间数据的可视化与空间分析,而建筑信息模型(BIM)则聚焦于建筑单体及设施层面的精细化信息,两者在应用场景、数据结构和服务对象上各自发展,缺乏统一的语义体系和逻辑框架。这种割裂导致城市在灾害风险建模和应急响应时难以实现多尺度联动\cite{CN_Xu2023DataFusion,CN_Ding2023ModelIntegration}。

基于上述CIM的架构特征与应用价值,其在解决数据异构与平台分裂问题方面展现出显著优势。CIM通过构建统一的信息模型和语义框架,为城市治理提供了系统性的解决方案:在技术层面,打破了BIM与GIS之间的数据壁垒,实现了建筑单体到城市空间的无缝连接;在应用层面,将静态的规划设计与动态的运行管理有机结合,为灾害预测、风险评估和应急疏散提供可计算、可模拟的平台;在治理层面,促进了跨部门、跨行业的协同合作,推动城市治理从"部门本位"向"整体治理"转变。

在城市灾害风险评估研究方面,国内外学者已开展了大量工作。在国际上,基于数字技术的城市灾害风险评估研究起步较早,主要围绕模型精度提升、多源数据融合与不确定性量化等方向展开。欧美学者在城市洪水动力学建模领域取得重要进展,英国学者基于浅水方程构建了二维城市洪水模型,美国学者进一步发展了三维计算流体动力学(CFD)模型,通过求解雷诺平均纳维-斯托克斯(RANS)方程,实现对复杂城市环境中垂向流动的精确描述。德国学者提出了基于遥感数据、地面观测与数值模型融合的洪水风险评估框架,通过数据同化技术提高了风险预测的精度\cite{Amirebrahimi2016}。

中国在城市灾害风险评估方面的研究发展迅速,主要集中在洪涝、地震、台风等主要灾种的风险评估与预警系统建设。清华大学、中科院等科研院所基于SWMM与二维水动力模型的耦合,构建了适用于中国城市特征的内涝风险评估模型。北京师范大学在洪水风险评估的社会脆弱性分析方面做出重要贡献,构建了包含人口密度、经济发展水平、基础设施完善程度等指标的综合脆弱性评估体系。近年来,深圳前海、上海临港等地持续推进CIM与应急管理融合示范工程,形成了基于数字孪生的城市水务与交通协同调度体系\cite{Hu2021,Liu2022}。

在滨海城市灾害风险研究方面学者们重点关注风暴潮、海平面上升与城市内涝的复合效应。针对沿海城市的特殊地理位置和复杂地形,研究团队开发了精细化潮汐淹没模型,结合城市建筑布局和排水系统特征,评估不同情景下的灾害风险等级。部分研究还探讨了气候变化背景下极端天气事件频率和强度的变化趋势,以及这对城市防灾减灾规划提出的新的挑战\cite{Brown2007,Fewtrell2008,Gallegos2009}。这些研究为城市韧性评估提供了重要的理论依据和技术支撑,但在多尺度耦合、实时性和智能化方面仍有提升空间。

当前,城市灾害管理面临的核心挑战集中体现在数据异构与平台分裂、建模精度与实时性矛盾、空间语义断裂与响应孤岛等方面。各类数据分散在不同部门与平台,缺乏共享机制;现有灾害风险评估方法多依赖二维模型和静态数据,在描述复杂城市环境的三维流动特征时存在局限;传统疏散规划多割裂室内外空间,难以形成连续的疏散策略\cite{CN_Yu2022KnowledgeGraph,CN_Zhong2024CrossDomain}。基于上述挑战,构建基于CIM的城市灾害风险评估与智能疏散方法体系,实现从"被动响应"向"主动控制"、从"经验决策"向"数据驱动"、从"单点优化"向"系统协同"的转变,具有重要的理论价值与现实意义。


\section{国内外研究现状}

\subsection{CIM 与 BIM/GIS 语义互操作基础}

城市信息建模(City Information Modeling, CIM)旨在弥合建筑信息模型(Building Information Modeling, BIM)与地理信息系统(Geographic Information System, GIS)在空间尺度、语义组织与应用场景上的割裂,通过统一的三维时空数据与语义框架,构建面向城市治理与应急管理的“数据—模型—决策”闭环\cite{Xu2014CIM}。在由建筑域走向城市域的研究演化过程中,CIM 被系统化地界定为一种汇聚 BIM、GIS 以及物联网感知、遥感监测、管理信息系统等多源数据的“城市级数字底座”,既承载精细化几何与语义表达,又集成业务流程与决策规则\cite{zhang2024city}。最新综述进一步指出,CIM 正从早期聚焦三维城市模型语义补全的“模型中心”范式,演变为服务城市韧性治理、基础设施安全和公共服务优化的综合信息基础设施,并通过与大数据、物联网和人工智能等技术深度耦合,为城市规划、运行维护与应急响应提供可追溯、可推演、可反馈的全生命周期支撑\cite{liu2017state}。在这一意义上,CIM 不仅是 BIM 与 GIS 的简单叠加或对接接口,而是承担了从项目级 BIM 向城市级“数字底座”纵向拓展的中枢角色,通过统一的数据模型和语义约束,将规划、建设、运维、应急等多部门、多尺度、多时态需求耦合到同一信息空间中\cite{tan2023citygml}。

在具体的数据与语义建模路径上,现有研究大多从 IFC 与城市三维模型的耦合入手,提出面向 3D 城市 GIS 的统一建筑模型与转换框架,以实现几何与语义的一体化表达\cite{ElMekawy2012UBM}。一类代表性工作通过构建 Unified Building Model(UBM),将 IFC 与 CityGML 中的建筑构件、空间单元及其关系进行统筹建模,在类、属性与拓扑关系层面识别重叠概念并加以合并或扩展,从而为室内外一体化城市模型提供统一的对象体系\cite{ElMekawy2012UBM}。在此基础上,学者们围绕 IFC–CityGML 双向转换形成了较为完整的技术谱系:包括基于三元图文法(Triple Graph Grammar)的可配置转换规则库与工具链,用z以显式维护两类数据模式之间的结构对应与演化\cite{Tauscher2019TGGCnv};基于图匹配与规则模板的“近无损”转换方法及其质量评价指标体系,用于刻画几何精度、语义完整性与拓扑一致性等多维度损失情况\cite{stouffs2018achieving,soini2019investigating};以及面向 IFC 源数据的 CityGML 应用域扩展(Application Domain Extension, ADE),在不破坏既有标准架构的前提下,将建筑细部构件、构造做法和设施信息注入城市级模型,实现从建筑构件到城市对象的多尺度一致表达\cite{biljecki2021extending,tan2023citygml}。在更复杂的空间语义层面,以产权边界、使用功能及权利束(RRR)为代表的三维权属信息被证明可以在 BIM/IFC 数据模型基础上进行扩展,并与 Land Administration Domain Model(LADM)等标准相衔接,用于支撑三维不动产登记、城市更新评估以及精细化空间治理等应用\cite{Atazadeh2017Ownership3D,xie2022automatic}。这些进展共同夯实了 CIM 在“构件级语义—城市级对象—业务规则与治理流程”三层上的互操作基础,使得同一底层模型能够在规划管控、建设许可、运营管理和应急调度等多类场景下被不同部门以各自语义视角按需复用。

进一步而言,CIM 的发展脉络正在从“模型驱动—数据支撑”的初级集成阶段迈向“语义驱动—机制耦合”的深层协同阶段。在这一趋势下,城市对象不再仅被视为具有几何外形与静态语义标签的实体,而被建模为具备状态、行为、约束与演化逻辑的“高语义对象”(high-semantic urban entities),可以在统一的城市知识框架中进行推理、更新与响应。得益于 BIM、GIS 与物联网感知体系的持续耦合,CIM 模型逐步形成了跨尺度、跨模态、跨时态的数据闭环,使得建筑级构件信息、街区级空间结构、城市级运行指标得以在同一数据语义体系下进行关联、复用与动态更新\cite{Xu2014CIM,zhang2024city,liu2017state}。这一过程不仅提升了模型在规划审批、竣工验收、设施管理、灾害模拟等典型业务中的可解释性与可追溯性,也将不同治理部门高度碎片化的知识体系重新组织为可计算、可推理的“城市知识网络”。

在此基础上,CIM 正逐步承担起“城市级数字底座”向“全生命周期数字孪生(City Digital Twin)”演化的关键枢纽功能。传统 BIM-GIS 框架所强调的空间耦合与几何一致性,被进一步拓展为时序一致性、语义可继承性与过程可推演性,使得城市对象能够携带运行状态、历史轨迹、行为规则以及外部约束,在虚拟空间中进行时序演化与情景模拟。例如,在突发事件响应场景中,CIM 可借助物联网实时数据流实现模型即时更新,并通过语义规则体系对城市运行风险进行分析与传播链推理;在城市更新与基础设施韧性评估中,构件级与系统级语义的联动可支撑不同尺度上的状态评估、退化识别与影响预测\cite{liu2017state,tan2023citygml}。这些能力共同将 CIM 从以往偏重“表达与对接”的数据容器,推进为可执行、可推演与可优化的城市运行机制承载框架。

在数据与语义融合路径方面,现有以 IFC–CityGML 为核心的转换与扩展体系,正被广泛应用于构建跨尺度信息一致性的城市对象模型。除前述几何—语义对齐与拓扑一致性维护之外,更细粒度的对象行为与状态语义也正在被纳入统一的建模范式中,使 BIM/IFC 中的构件层级信息能够以可继承的方式映射至城市对象层级,并参与到空间分析、运行模拟与治理流程之中\cite{ElMekawy2012UBM,Tauscher2019TGGCnv,stouffs2018achieving,soini2019investigating,biljecki2021extending,Atazadeh2017Ownership3D,xie2022automatic}。随着三维权属、功能区划、产权束(RRR)及使用场景语义的持续扩展,CIM 的对象体系正不断从建筑—地块—街区—城市的多尺度模型中抽取关键语义结构,并在统一的语义约束之下实现面向任务的灵活拼接,从而真正实现“多源异构数据的模型级融合”与“跨部门业务流程的语义一致协同”。

从治理机制角度看,CIM 的核心价值正在于将规划、建设、运维、应急等城市治理流程的知识逻辑显式化,并以可执行的形式嵌入数据模型自身。这一范式使得模型不仅是信息的载体,也是治理机制的运行场,能够根据法规、管理制度、流程约束进行状态校验、规则推理及任务调度。例如,基于 CIM 的建设许可管理可以在几何、规范与权属语义层面进行多维校验;基于 CIM 的设施运维则可对构件状态、生命周期阶段与故障模式进行时序分析;基于 CIM 的应急管理可实现场景模拟、资源调度与风险传播的模型内闭环\cite{Xu2014CIM,zhang2024city,liu2017state}。在此意义上,CIM 完成了从“技术体系”到“治理体系”间的本体级跃迁,使城市的物理运行逻辑、制度逻辑与数据逻辑得以在统一的语义空间中重构与协同。


\subsection{室内外一体化通行网络与应急疏散建模}

面向室内外一体化通行与路径规划,BIM 导向的多用途几何网络模型(Multi-purpose Geometric Network Model, MGNM)提出了从 IFC 语义自动抽取室内通行网络的技术路线:通过识别房间、门、走廊、楼梯等语义对象及其邻接关系,构建与建筑几何高度一致的网络拓扑;再将其与室外道路网通过“入口—界面—街廊”等关键要素完成连通,从而实现室内外通行空间的无缝衔接\cite{Teo2016BIMIndoorNetwork}。在此基础上,统一的室内—室外网络模型显著降低了传统接口简化带来的可达性偏差,为室内外联合路径规划、可达性分析和拥堵识别提供了统一的网络表达\cite{Claridades2021SeamlessNav}。相关研究进一步指出,通过面向多尺度对象(建筑单体、街区组织与城市结构)的模型抽象与连接策略,同一数据结构能够支持跨尺度路径搜索、区域避障、楼层转换以及多模式出行(步行—室外交通—建筑内部移动)的连贯表达,从而推动跨场域导航由可行性验证走向流程化落地。

在室内网络构建算法方面,最新研究正从二维“房间—门”拓扑向面向复杂三维几何的自动化生成技术演化。Zhang 等提出的 GINIT 方法从 BIM 模型中抽取楼板和开口,通过图像栅格化与骨架细化(image thinning)自动生成三维室内导航网络,对异形空间、中庭和斜坡等复杂几何表现出更高鲁棒性\cite{zhang2023automatic}。与 IFC 语义驱动的 MGNM 形成互补,几何主导型方法能够作为语义残缺或建模风格不一致的 BIM 数据的“兜底”机制,为构建全局统一的城市级通行网络提供更高容错能力。

在室内外过渡空间建模上,传统 BIM/GIS 集成多将入口简化为单节点,难以表达台阶、门厅、站台等结构的连续通行属性。Wang 等基于 IndoorGML 标准提出了室内外过渡空间体素化建模方法,通过点云与几何自动识别可通行区域,并构建室内 Cell Space 与室外道路网之间的细粒度连接图,实现更加真实的方向性、坡度与宽度表达\cite{wang2023automatic}。这一类方法强化了 MGNM 的“界面层”能力,为无障碍通行、多主体换乘与机器人导航提供重要的几何支撑。

针对应急疏散场景,室内外一体化网络不仅用于最短路径规划,还被进一步嵌入到疏散时间评估、安全瓶颈识别和疏散策略优化等关键分析任务中。最新综述强调三维室内环境的几何、拓扑与语义三元并重,提出以 IFC/室内拓扑的自动抽取作为仿真前置步骤,以稳定供给高质量模型输入\cite{Xie2022AutCon3DIndoorEvac}。在这一框架下,自动化的室内拓扑生成不仅保证了节点—边结构与真实建筑空间布局的对应关系,也为基于多主体仿真(agent-based simulation)与基于流体动力学(CFD-like)的人群模型提供了统一空间底图,使疏散路径选择、行进速度估计与拥堵动力学分析得以在一致的数据基础上开展。

进一步的研究指出,最短路径与最安全路径在实际疏散中往往不同。Mazlan 等基于 3D BIM–GIS 集成构建几何网络模型,同时考虑家具布置、烟气暴露、可见度等因素,对最短路径与安全路径进行比较分析,结果显示二者在时间与空间上存在显著差异\cite{mazlana2022emergency}。这说明通行网络应从“单一距离成本”发展为多权重、多目标的风险度量体系,以更真实地模拟疏散行为。

在设计—仿真—合规一体化方面,宏观疏散模型能够直接消费 IFC 语义生成通行网络,并自动执行关键法规条文的校核,例如出口数量与分布、最小疏散宽度、疏散距离与时间等,从而显著降低建模与校核成本\cite{Zhu2018IFCEvac}。更进一步,基于自动化语义解析得到的室内—室外网络可以为多方案设计(design alternatives)提供快速评价能力:通过在同一建筑或片区的不同设计版本上复用疏散模型,可自动生成安全性能差异、潜在瓶颈位置、弱势群体(如行动不便人群)可达性等指标的比较结果,形成面向建筑师、结构工程师与审图机构的评估闭环。

在行为模型层面,Xie 等提出“基于运动概念的空间模型”,将门、楼梯、坡道等构件映射到不同的运动模式空间,使行人流、拥堵形成与垂直交通得以在三维结构中统一建模\cite{xie2023motion}。随后结合体素化三维室内模型,研究者构建了可同时驱动人群流动、烟气扩散与能见度变化的多场耦合模拟框架,为高可信度应急仿真提供了技术支撑\cite{xie2024voxel}。

在交互与可视化方面,室内外一体化导航网络也被加载到 AR/VR 系统中。Ahn 等提出基于 BIM 的增强现实疏散导航,相比传统 2D 示意图显著提升了路径理解效率\cite{ahn2024bim};Valizadeh 等构建了基于 BIM + GIS 的室内 AR 导航系统,实现了动态环境中的实时路径呈现\cite{valizadeh2024indoor}。基于 BIM 的虚拟现实疏散训练系统亦被用于模拟人群行为与验证疏散策略,为网络模型提供了行为实验层面的支撑\cite{wang2014bim}。

综合来看,BIM 驱动的多用途几何网络模型正在推动室内外导航与应急疏散分析从“手工建模—局部分析”的碎片化流程向“语义驱动—自动生成—跨尺度联动”的工程化框架演进。随着 IFC 语义映射规则、自动化拓扑抽取算法以及室内外网络融合策略的不断完善,此类方法将成为城市级导航服务、建筑运维管理(Facility Management)以及城市应急规划的重要基础设施,为真正意义上的城市数字孪生提供可操作的通行与疏散网络底图。

\subsection{群体行为理论与微观疏散模拟方法}

群体行为与应急疏散的理论基底方面,社会力模型(social force model)以简洁的微观相互作用刻画行人之间以及人与环境之间的力学关系,能够自然再现自组织排队、瓶颈拥堵与“越急越慢”等现象,奠定了微观疏散模拟的理论基石\cite{Helbing1995SocialForce}。社会力框架将个体视为在期望速度、社会斥力、界面约束和随机扰动共同作用下的“主动粒子”,使得行人运动的非线性耦合行为可以在连续空间中获得统一描述。该模型不仅为解释局部交互导致的宏观拥堵提供了可计算的动力学体系,也使得行为假设(如风险规避、群体依从、同伴跟从)能够以参数形式嵌入,为情景化疏散研究提供了灵活的理论载体。

在社会力模型发展之后,大量研究开始引入更丰富的行为决策要素,以弥补原模型在复杂场景下的局限。一类重要工作通过显式建模感知域和视域(field of view),将“谁被看见、何时被看见”纳入力学项,使行人决策不再基于理想化的全局信息,而是基于局部可见环境和遮挡关系,从而更符合真实的感知–反应过程\cite{moussaid2011simple}。另一类研究则借鉴交通行为理论,将效用最大化思想引入行人路径选择与行为层决策之中,将环境吸引、风险规避、从众心理等综合为可估计的效用函数,建立“战术层”与“操作层”耦合的行人行为模型\cite{hoogendoorn2004pedestrian}。同时,针对高密度和强接触情境,离心力模型(centrifugal force model)等扩展形式通过将相对速度和间距共同嵌入排斥力项,更细致地刻画了挤压、摩擦与接触冲击等效应,为描述灾害场景下的压强积聚与出口堵塞提供了更具物理含义的表达\cite{yu2005centrifugal,chraibi2011force}。

在此基础上,关于“逃生恐慌”的典型研究揭示了从层流到间歇性“停—走”再到湍动拥堵的相变机制及其与出口几何、密度阈值之间的定量关系,为通行断面设计与组织管理提供了可检验的物理解释\cite{Helbing2000EscapePanic}。这一相变过程本质上源于行人与边界之间的竞争性相互作用,当出口收缩、密度升高或群体焦虑加强时,系统会经历从稳定流动向含周期脉动的 metastable 区域再到完全失稳的“crowd turbulence”的跨越。相关理论结果表明,微观扰动可在高密度下被放大并触发“推挤链式反应”,从而导致剧烈的速度涨落、压力积聚与拥堵自激,这为安全疏散的瓶颈设计、单向限流与破窗通风策略等提供了物理可证的设计依据。


后续的实证研究则通过视频观测、传感器跟踪与实验室控制试验,进一步量化了灾难人群动力学中的临界密度、速度波动和失稳特征,补强了社会力模型的经验基础\cite{Helbing2007CrowdDisasters}。例如,通过高帧率视频重建获得的人群轨迹揭示了湍动状态下速度场的非高斯分布与能谱特征,室内和场地实验通过可控密度布设验证了局部接触力、个体避让策略与拥堵发生概率之间的统计关系,从而为模型参数标定、场景再现以及极端工况下的安全评估提供了可复现实验依据\cite{johansson2008crowd}。

除密度与出口几何外,环境的几何复杂性、拓扑结构与多层建筑的空间连通性也被证明显著影响疏散相变行为。研究表明,局部视线阻断、走廊交叉点、T 型节点、双出口竞争等结构会诱发集聚、反向流和“局部瓶颈迁移”等现象,其机制可通过流体动力学类比与网络路径竞争理论解释\cite{johansson2007specification}。此外,多层建筑中的竖向交通(楼梯、坡道)因步幅变化与空间压缩效应,在疏散过程中可能触发“速度塌陷”与多点拥堵,形成类似交通网络中的“多源—多汇拥塞”问题\cite{kretz2011quickest}。这些研究强调了建筑拓扑对群体动力学的系统性影响,为“复杂建筑疏散”提供了更接近真实运行条件的物理基础,并推动从单断面分析向全建筑、多通道耦合的研究范式转变。

与连续模型相对,基于元胞自动机(Cellular Automata, CA)的离散方法在网格空间上刻画行人状态转移,以更新规则近似群体相互作用,具备实现简单、计算开销可控的优势,适于大尺度人群通行与疏散过程的高效模拟\cite{Burstedde2001FloorField}。CA 模型通过概率化的“地板场(floor field)”将环境吸引、局部占据、群体从众与出口偏好编码到离散格点,使群体动力学得以在复杂拓扑空间中以较低成本重建。在跨楼层、多出口和双向流等情景下,CA 模型表现出与连续模型高度一致的宏观通行规律,因此常在城市级、园区级疏散评估中承担大规模仿真的主体角色。

在工程应用中,社会力模型与 CA 模型常被结合或混合使用,以兼顾行为逼真度与计算效率。随着 BIM 语义抽取和网络生成技术的成熟,建筑几何与功能分区可以直接映射为疏散模拟网格与拓扑结构,实现从建筑设计到疏散仿真的全链路自动化\cite{Zhu2018IFCEvac}。IFC 模型中的房间(Space)、开口(Opening)、门(Door)、楼梯(Stair)等语义对象可被自动识别并转换为节点—通道结构,继而作为连续模型的边界条件或 CA 模型的离散网格。基于此的自动化流程使得“建筑语义—行为模型—群体动力学”之间的耦合更加紧密,为嵌入多行为模式(紧急度差异、家庭结伴、疏散引导)、不同风险偏好以及应急指挥策略(动态指示、单向通行、分区清退)提供了统一计算框架,也为数字孪生环境下的实时疏散评估奠定了技术基础。

随着物联网传感、视频分析与数字孪生技术的不断成熟,疏散模拟的范式正在从传统的离线静态评估跃升至实时预测与动态控制。首先,有研究通过将摄像头及其他轨迹监测数据与代理人模拟平台结合,实现了大型活动场景中人群流动的10 分钟预警能力,并在约3 000 个代理人模拟下将预测时间缩至约1分35秒,从而体现出实时反应的可行性\cite{yasufuku2024development}. 其次,在构建智能建筑的安全管理框架中,已有研究提出了一种基于数字孪生(Digital Twin)的火灾安全管理体系,该体系整合 BIM、IoT、AI 等技术,并形成“感知-仿真-决策-执行”闭环逻辑,虽然聚焦于火安全,但其方法论可自然推广至疏散引导场景\cite{almatared2023digital}. 因此,“建筑语义—人群动力学—实时感知—指挥控制”这一链路正逐步由理念走向可实施的技术路径,为下一代智能建筑与城市级韧性治理提供了新的范式。
综上所述,群体行为与应急疏散的研究已经从早期基于简化假设的微观动力学模型,逐步发展为融合物理机制、行为认知与环境语义的综合建模体系。社会力模型提供了连续空间下描述行人交互与拥堵形成的基础框架,相变研究揭示了高密度条件下的失稳路径与风险规律,实证观测进一步确保了模型的可校准性与外推能力;而元胞自动机等离散方法则以可控的计算复杂度支撑大尺度场景的高效仿真。在工程侧,BIM 语义抽取、网络生成与数字孪生技术的引入,使得建筑信息、环境感知与疏散动力学在统一框架中相互映射、实时联动,从而实现由“几何—行为—控制”贯通的全链路模拟。整体而言,该研究方向已由单一模型驱动的局部分析迈向跨尺度、多源数据支持的智能化疏散体系,为韧性建筑设计、应急指挥优化及城市级安全治理提供了可验证、可计算、可扩展的技术基础。

\subsection{城市洪涝二维水动力模型与实时预报体系}

在城市洪涝风险评估方面,二维浅水方程(Shallow Water Equations, SWE)长期作为主力水动力内核,用于刻画地表径流与淹没过程。传统完整动量方程在高分辨率大范围模拟中往往面临计算成本过高的问题,因此“局部惯性”(local inertial)近似成为兼顾效率与精度的关键路径:通过忽略部分加速度项,在显著降低计算量的同时,仍能保持对亚临界水动力过程的合理刻画,使得在城市或区域尺度开展高分辨率淹没模拟成为可能\cite{Bates2010Inertial}。近年来,该近似方法在数值稳定性与适用性方面得到进一步验证,包括对水深骤变、浅层流态过渡、局部能量耗散等关键机制的敏感性分析,使其在复杂城市空间(如密集建筑群、微地形突变区域)中的应用范围不断扩展。其适用性研究系统比对了在不同糙率、坡降与边界条件下该近似与全动量方程的误差界,给出了在城市、河网和平原等不同地貌下的模型选型建议\cite{deAlmeida2013LocalInertial},并指出在高流速、高Froude数或强惯性主导的水动力条件下需谨慎采用,以保证关键动力过程不被过度简化。此外,有研究进一步从数值算法稳定性角度对“局部惯性”近似形式的 Local Inertial Approximation (LIA) 进行了系统性分析。 Gustavo A. M. de Almeida 等\cite{de2012improving}作者针对基于 LIA 的二维淹没模型在低糙率、平缓坡地条件下可能出现的数值振荡与湿/干界面不稳定问题,提出了改进格式并通过理想算例与城市地形模拟验证了改进后的方案可显著提升稳定性且几乎不增加计算代价。该研究指出,在城市尺度建模场景中,尽管采用了近似简化,但通过适配时间步限制与数值通量处理,仍可保持模型数值可靠性。 

围绕模型能力与透明度,基准化研究通过标准算例与观测资料对多种二维水动力模型开展结构化对比,并在数值稳定性、误差统计指标、计算效率等方面建立统一的评估框架,从而提升了不同模型间的可比性与选型透明度\cite{Hunter2008Benchmark2D}。此类研究不仅包括经典的干湿界面测试、溃坝实验、渠道转弯试验等基础算例,还扩展到高分辨率城市街区、涵洞与排水系统耦合、复杂多源入流(降雨、市政排放、上游来水)等真实场景再现。随着地面激光点云(LiDAR)、高精度DEM/DSM 与无人机影像的普及,模型验证逐渐从单点水位对比向二维空间的淹没范围匹配、动态淹没边界移动速度、空间误差分布结构等更精细的维度转变,从而显著提高了模型性能评价的可解释性。另一方面,针对城市场景下高分辨率地形+建筑耦合条件,最新综述强调仍须警惕“近似形式模型”在极端流动状态下的适用边界。Weiqi Wang 等\cite{wang2021urbanr}研究在中国高度城市化区域构建了基于 LIA 的 2D 地表-地下排水耦合模型,并与完整 SWE 模型对比发现:尽管整体水深与速度结果高度一致,但当局部 Froude 数趋近或超过 1(超临界/冲击流态)时,LIA 模型误差显著增大,水深相对误差可能超过 20%。因此,该文强调:在城市细尺度淹没模拟中,当模型预测区域可能出现强惯性主导、快速水流或垂直跌落情况时,仍应优先采纳完整动量方程或混合方案。 


另一方面,面向预报业务的研究则在保证关键物理过程刻画能力的前提下,提出结构简洁、计算高效的二维淹没模型与降阶方案,为实时或近实时的城市洪涝预报奠定基础\cite{Dottori2013Simple2D}。这些方法融合了参数化摩阻表达、栅格分辨率自适应、基于物理约束的机器学习加速模块,以及多源实时资料(降雨雷达、IoT 水位计、排水管网流量计)的数据同化框架,以实现对不同洪涝触发机制(短时强降水、潮汐顶托、河道溢流、内涝积聚)的快速响应。近年来,研究者还提出了模型集成(model ensemble)与场景库(scenario library)技术,通过预先计算大量可能的入流组合与城市排水工况,形成可在应急条件下即时检索的洪涝响应数据库,使得在复杂城市地表条件下进行多场景、多情景的快速情景演练成为可能,为风险图编制、排水系统设计与应急预案制定提供了更为稳健、可重复和透明的技术支撑。

此外,这些模型体系正逐步与城市数字孪生、在线监测平台及智能排水系统耦合,通过云计算架构实现大规模高分辨率模拟任务的并行化处理,并在水动力学基础上进一步叠加交通、供电、医疗等城市基础设施脆弱性模型,从而支持跨部门、跨时间尺度的综合城市洪涝韧性评估。整体来看,以局部惯性近似为代表的高效水动力方法正在从基础研究走向业务实用化,其在城市极端事件风险分析中的作用愈发突出,并为下一代城市洪灾预报与韧性治理体系奠定了坚实基础。

\subsection{遥感—模型一体化洪涝同化与不确定性刻画}

在实际业务落地层面,遥感—模型一体化正逐步从“概念验证”走向“业务运行”。在流域级与城市级预报系统中,SAR 衍生淹没信息不再仅仅作为独立的地图产品,而是被显式嵌入到整体信息流中:上游以水文与降雨径流模型提供入境流量,中游以二维浅水方程或局部惯性模型构建高分辨率水动力内核,下游则通过 SAR 洪涝反演模块周期性更新水位与淹没边界\cite{Mason2012010TerraSARX,Mason2014DoubleScattering,Giustarini2015UncertaintySARFlood}。在这一链条中,洪水演进的关键阶段往往与防洪工程调度和城市排涝决策高度重叠,因此如何在有限观测频次下最大化利用 SAR 影像成为核心问题:一方面,通过在洪峰前后加密任务规划、结合多星座观测与不同重访周期,尽可能缩短连续影像之间的时间间隔;另一方面,在非成像时段通过模型滚动积分与同化结果外推维持预报连续性,使“观测—同化—预报”的闭环在时间轴上形成尽可能平滑的过渡\cite{Hostache2018WRR,hostache2010assimilation,schumann2009progress}。

从遥感反演到模型同化的接口层面,越来越多工作开始关注“信息类型匹配”的问题,即如何将淹没二值图或概率淹没图转换为水动力模型状态变量与参数的有效约束\cite{Giustarini2015UncertaintySARFlood,giustarini2016probabilistic,schlaffer2017probabilistic}。在状态同化层面,常见做法是构建从水位场到淹没概率场的观测算子,并在像元尺度引入与地形起伏、粗糙度和地物类型相关的似然函数,以缓解“同一淹没模式对应多种水位组合”的非唯一性问题;在参数同化层面,则通过长期序列的 SAR 观测对河道糙率、漫滩阻力、城市局部糙率修正系数等进行反演与更新,使模型在多次洪水事件间完成跨事件标定\cite{wood2016calibrationOOD,DiMauro2021HESS}。这类设计实质上将SAR 观测由“单次事件诊断工具”升级为“跨事件约束源”,有助于构建在多年气候情景下仍具有稳健性的城市洪涝风险评估模型体系\cite{DiMauro2021HESS,DiMauro2022WRR}。

在复杂城市环境中,影像层面不确定性与模型层面不确定性相互叠加,使得“观测噪声—反演误差—同化偏差”的传递链条更为复杂\cite{Giustarini2015UncertaintySARFlood,Shen2019RemoteSensingFlood,Li2019RSUrbanSARFusion}。一方面,高层建筑与密集植被导致的阴影与叠掩区域会在 SAR 影像中形成大面积“不可观测区”,即使引入双程散射与体散射分析,仍难以完全恢复真实淹没状态\cite{Mason2010TerraSARX,Mason2014DoubleScattering};另一方面,城市排水系统的地下管网与明渠系统往往在模型中被粗略简化,其对地表积水演化的影响在 SAR 观测中则以复杂空间纹理的形式被隐性反映。针对这一“结构不完备—观测不完全”的双重不确定性格局,概率洪水图与影像不确定性建模提供了一种相对统一的表述框架:将水体识别结果从“0/1 决策”提升为“像元级置信度场”,并在同化过程的权重更新与观测误差协方差构建中显式引入,从而实现“在可信区域中强约束,在高不确定区域中弱约束”的自适应融合策略\cite{Giustarini2015UncertaintySARFlood,giustarini2016probabilistic,schlaffer2017probabilistic}。

在同化算法层面,调温粒子滤波等先进贝叶斯方法的引入,使得 SAR 概率淹没图与耦合水文—水动力模型之间的“信息匹配”更加细腻\cite{DiMauro2021HESS,DiMauro2022WRR}。通过在权重更新过程中设置逐步增强的“调温系数”,可以将一次观测分解为若干“子观测”逐步注入系统,使粒子集在高维状态—参数空间内有足够时间完成重采样与扩散,避免因单次高置信度观测导致的权重塌缩与样本退化\cite{DiMauro2021HESS}。与此同时,将概率淹没图中的像元级置信度直接纳入似然函数计算,可在空间上实现对观测不确定性的差异化刻画:在噪声较大或地物异质性显著的区域,观测对权重更新的影响被自动削弱,而在水体识别稳定且多源数据一致性较高的区域,则对模型状态施加更强约束\cite{giustarini2016probabilistic,schlaffer2017probabilistic}。这类算法设计使“观测不确定性—模型不确定性—预报不确定性”的耦合成为可能,为后续纳入更多观测类型提供了统一的贝叶斯框架\cite{DiMauro2021HESS,DiMauro2022WRR}。

面向城市韧性治理与风险沟通,SAR—模型一体化预报系统的输出也在从传统的水位、流量时间序列,扩展到面向决策的空间化风险指标\cite{schumann2009progress,schumann2018assisting}。在同化框架下得到的水位轨迹与淹没概率场,可以被进一步转化为“淹没概率—暴露资产—脆弱性函数”的组合,从而构建在像元级或地块级的期望损失分布以及超越概率曲线,为不同预案下的损失对比与成本—效益分析提供基础\cite{Hostache2018WRR,schumann2018assisting}。对堤防漫顶、城市下穿通道、重要交通枢纽与生命线工程等关键基础设施,可在多情景预报结果基础上提取其在不同重现期洪水下的概率淹没时长与最大水深区间,为精细化布防、应急疏散路径优化以及物资与泵站调度提供定量支撑\cite{schumann2009progress,wood2016calibrationOOD}。从这一视角看,SAR 驱动的洪涝同化不仅是数值预报技术的改进,更是将“观测误差—模型误差—决策风险”在城市尺度上贯通起来的重要媒介,为构建面向不确定性管理与多源观测协同的智能化洪涝韧性治理框架奠定了方法与实践基础。

\subsection{城市水系统数字孪生与“模型—数据—治理”闭环}

在“模型—数据—治理”的落地层面,数字孪生(Digital Twin)被视为连接多源观测、物理/数据驱动模型与协同决策的核心承载体,其内涵已从早期的“可视化资产复制”扩展为涵盖数据治理、过程模拟、风险评估与策略推演的综合决策基础设施。流域尺度的前沿研究提出了面向可迁移、可持续与公平治理的数字孪生框架,将多源水文气象观测、流域水动力—水质模型、利益相关方行为模型以及政策规则显式纳入统一的“数据—模型—治理”结构之中,明确了数据、模型、政策与能力建设之间的协同要求与挑战清单\cite{Yang2024npjNH}。该类框架不仅关注灾前的风险识别与情景推演,也强调在灾中通过实时同化与滚动预报支持应急调度,在灾后通过事后评估与经验归纳反哺规划与制度设计,从而在防灾减灾、资源调配与长期情景分析等典型应用中,展示出在提升流域韧性、多方协同决策以及区域尺度公平性方面的应用潜力\cite{Yang2024npjNH,pal2025blueprint}。在工程与运维方向,水务系统数字孪生相关研究则从“运行与维护”的视角抽象出一套较为系统的功能构件与评估维度:包括面向泵站、管网、污水厂与调蓄设施的实时数据采集与多源融合,面向关键运行状态与隐患征兆的状态估计与数据同化,面向供水安全、能耗优化与泄漏控制的主动控制与多目标优化,以及支撑在线优化与情景枚举的降阶加速与模型替代技术\cite{Cavalieri2024SensorsAAS,Bonilla2022WaterDTWDS}。相应研究进一步提出,应通过基于运营闭环的性能提升(如预测精度、响应时间、服务可靠性、资源利用效率)与成本收益分析(包括改造成本、维护成本与治理绩效的综合权衡),在全生命周期视角下检验孪生系统的实际价值与可持续性\cite{Cavalieri2024SensorsAAS,Bonilla2022WaterDTWDS,wang2024digital}。此外,有研究系统地回顾了水务数字孪生在数据融合、标定流程与运行绩效中的实际经验与测量收益\cite{bonilla2022digital}。

综合来看,从 CIM 底座与语义互操作,到室内外一体化网络与疏散行为模型,再到二维洪涝主干模型与 SAR 同化,以及面向城市洪涝风险管理的数字孪生系统综述\cite{hlal2025digital},国内外研究已基本形成面向“风险识别—疏散响应—治理优化”闭环的多层次技术栈:在底层,统一的城市信息建模与多源时空数据体系为洪涝过程模拟、通行网络构建与人群轨迹刻画提供了高一致性的几何与语义基座;在过程层,基于局部惯性近似的二维浅水方程、水动力—水文耦合模型以及室内外一体化疏散模型,实现了从地表径流演化到人群响应行为的联动模拟;在应用层,水系统数字孪生与流域级数字孪生框架则进一步将上述模型与实时观测、运维规程与治理规则耦合在一起,形成支撑多部门协同与多时空尺度分析的综合决策平台\cite{Yang2024npjNH,Cavalieri2024SensorsAAS,Bonilla2022WaterDTWDS,maksoud2023digital}。然而,要实现跨城市、跨区域的规模化落地,仍需在若干关键环节上进一步完善:一是在高保真 BIM/GIS 转换流水线与质量控制方面,构建面向大规模城市群的自动化建模与一致性校核机制,形成可复用、可迁移的三维时空基底;二是在跨尺度与跨模型链的不确定性传递上,建立从观测误差、模型结构偏差到决策输出的系统性量化与分解框架\cite{kim2025stormwater},避免在多次耦合与降阶过程中放大不确定性;三是在“观测—行为—调度”的统一目标函数与实时控制体系上,将风险降低、效率提升、公平性约束与成本控制纳入同一优化框架之中,形成既可解释又可计算的决策准则;四是在现有管理体制与技术条件下,探索分级部署、渐进迭代与示范引领的实施路径,通过试点工程、跨部门协同机制和能力建设计划,将上述技术栈逐步嵌入日常规划、建设与运维流程,真正实现从“技术可行”向“制度可用、长期可持续”的转化\cite{ghorbani2025digital}。


\section{研究理论基础}

为构建面向城市综合防灾减灾与应急管理的韧性治理技术体系,本文主要立足于三个层面的理论基础:城市韧性治理理论、智慧城市与数字孪生理论以及城市信息建模与 BIM--GIS 融合理论。三者分别从治理范式、数字基础设施与数据模型三个维度,为本文提出的“风险评估—智能疏散—构件损伤—协同治理”一体化框架提供支撑。

\subsection{城市韧性治理与复杂系统}

城市韧性治理(urban resilience governance)源于韧性科学与现代治理理论的综合发展,将城市视为由自然环境、社会经济与技术基础设施共同构成的社会--生态--技术复合系统(Social--Ecological--Technical Systems, SETS)。韧性理论经历了工程韧性、生态韧性与演化韧性等多个发展阶段:工程韧性强调系统在遭受扰动后的快速恢复,生态韧性关注系统在跨稳态条件下维持关键功能的能力,而演化韧性则突出系统在持续变化背景下通过学习、创新与制度重构实现“向前恢复”(bounce forward)的能力。

在此基础上,城市综合韧性可抽象表示为
\begin{equation}
R(t) = f\big(C_{a}(t),\, C_{r}(t),\, C_{t}(t)\,\big|\,G(t)\big),
\end{equation}
其中,$C_{a}(t)$ 为吸收能力,刻画城市在冲击发生时削弱损失的能力;$C_{r}(t)$ 为自适应能力,反映系统在不确定情景下调整结构与功能的能力;$C_{t}(t)$ 为转型能力,表示在深度变革压力下进行路径重塑与结构升级的潜力;$G(t)$ 则代表治理体系的制度安排、主体网络、决策机制与知识系统。该表达式揭示了治理结构在韧性生成中的核心调节作用。

在各类自然灾害和突发事件情景下,城市韧性治理强调从单一部门、单一灾种的“事后响应”模式,转向覆盖“事前预防—事中应对—事后恢复”的全过程、多主体协同模式。一方面,基于脆弱性与暴露度的框架,城市风险水平不仅取决于外部扰动强度,还受暴露对象、敏感性以及适应能力的共同影响;另一方面,通过多源数据与模拟模型支持的情景推演,可以在规划、建设与运营阶段提前识别关键基础设施和人员密集区的薄弱环节,从而提升系统整体的吸收、自适应与转型能力。

本文在城市韧性治理理论框架下:(1)在城市、建筑与构件多尺度上刻画灾害风险演化过程,为识别易损单元提供依据;(2)通过室内外一体化疏散网络与动态路径规划,强化应急响应阶段的人群安全保障能力;(3)结合构件级损伤评估与功能恢复分析,为灾后修复与系统转型提供量化支撑,从治理视角实现对 $C_{a}(t)$、$C_{r}(t)$ 与 $C_{t}(t)$ 的系统性提升。

\subsection{智慧城市与数字孪生}

智慧城市(smart city)与数字孪生(digital twin)为城市韧性治理提供了关键的数字基础与技术路径。智慧城市以新一代信息通信技术(ICT)、物联网(IoT)与大数据分析为核心,通过对城市运行状态的感知、传输、计算与反馈,实现城市基础设施和公共服务的智能化与精细化管理。数字孪生在此基础上进一步发展,将物理城市与虚拟城市通过数据流和模型体系紧密耦合,使城市在虚实互动的闭环结构中开展状态监测、趋势研判、情景模拟与优化决策。

从形式化角度,可将智慧城市能力与数字孪生能力分别抽象为
\begin{equation}
S(t) = g\big(I(t),\, T(t),\, G(t),\, C(t)\big),
\end{equation}
\begin{equation}
D(t) = h\big(M(t),\, U(t),\, B(t)\big),
\end{equation}
其中,$S(t)$ 表示城市在时间 $t$ 的智慧化水平,$I(t)$ 代表感知基础设施,$T(t)$ 表示计算与通信平台能力,$G(t)$ 表征治理结构,$C(t)$ 表示公民参与与社会资本;$D(t)$ 表示数字孪生能力,$M(t)$ 为多模型体系,$U(t)$ 表示数据驱动更新的频率与精度,$B(t)$ 代表虚实双向耦合的有效性。城市整体治理绩效可进一步表示为
\begin{equation}
P(t) = F\big(S(t),\, D(t)\big),
\end{equation}
其中 $P(t)$ 反映城市在安全性、可靠性、运行效率与可持续性等方面的综合表现。该框架表明,智慧城市与数字孪生并非彼此独立的概念,而是共同嵌入城市治理体系的双重能力结构:前者构建数据与平台基础,后者则将数据与模型转化为面向具体场景的预测与决策能力。

在城市公共安全场景下,数字孪生城市强调利用高精度三维模型与多源数据,实现对人群活动、基础设施运行状态及多类风险传播过程的动态仿真,并通过可视化与交互分析支撑应急决策。本文提出的基于 CIM 的多尺度风险评估、室内外一体化疏散与构件级损伤评估,实质上是在智慧城市与数字孪生理论框架下,面向城市综合风险治理构建的一套场景化应用:通过数字孪生场景中对风险演化、人群疏散行为和关键设施损伤的联动推演,为应急指挥部门提供“可感知、可计算、可推演、可追溯”的决策支撑环境。

\subsection{城市信息建模与 BIM--GIS 融合}

城市信息建模(City Information Modeling, CIM)作为智慧城市与数字孪生的重要基础设施,是整合多源空间数据与语义信息的统一表达框架。CIM 以城市整体三维几何模型为载体,融合建筑信息模型(Building Information Modeling, BIM)、地理信息系统(Geographic Information System, GIS)、物联网传感数据及统计业务数据等多源异构信息,构建跨尺度、跨领域的城市数字底座。其典型技术架构包括数据采集层、数据管理层和应用服务层:前者负责获取地形地貌、建筑设施、环境要素与人群活动等数据,中间层面向多源异构数据开展存储、索引与质量控制,应用层则提供三维可视化、空间分析、模拟预测与决策支持等功能。

BIM 侧重于建筑与基础设施层面的精细化信息表达,能够描述构件的几何形态、材料特性、构造关系与生命周期属性;GIS 则擅长城市与区域尺度的空间分析与场景表达,强调地形地貌、土地利用、交通网络与环境要素等多维空间关系。BIM 与 GIS 在空间精度、语义层级和应用尺度上具有显著互补性,通过二者的深度融合,可在建筑构件—建筑单体—街区—城市等多尺度上建立连续、一致的空间信息链条。

在城市综合防灾减灾与应急管理研究中,CIM 以及 BIM--GIS 融合理论主要体现在以下三个方面:(1)在数据层面,打通建筑内部构件信息与城市外部地形、基础设施和道路网络数据,实现从城市空间格局到建筑内部空间的全过程统一建模;(2)在模型层面,为三维模拟分析与室内外疏散网络构建提供高精度几何与语义底图,使风险评估结果能够精确映射到建筑与构件层级,并进一步驱动基于语义的疏散路径规划与损伤评估;(3)在应用层面,通过 CIM 平台实现风险评估、路径诱导与损伤分析等多模型的可视化集成和协同调用,为多部门、多层级、多角色参与的综合应急指挥与韧性治理提供统一支撑环境。

综上,城市韧性治理理论为本文提供了目标与评价框架,智慧城市与数字孪生理论为本文构建多模型耦合与虚实互动的技术路径,城市信息建模与 BIM--GIS 融合理论则为本文实现多源数据集成、三维仿真与一体化决策支持奠定了数据与模型基础。三者共同构成本文研究的理论支撑体系。



\section{研究目标和内容}

本研究旨在构建一个CIM驱动的灾害风险评估、智能疏散与灾后损伤评估协同框架,形成“数据汇聚—风险计算—路径诱导—损伤复盘—协同指挥”的闭环流程,并以威海滨海应急服务中心为验证场景。围绕这一总体目标,研究重点回答三个相互递进的问题:(1) 如何实现多源数据的语义一致性表达,整合BIM、GIS、遥感与物联网数据并准确刻画风暴潮洪涝的三维动力学演化;(2) 如何构建风险驱动的室内外一体化疏散模型,将动态洪水风险映射到疏散网络,实现语义一致、拓扑连通且可实时更新的路径规划;(3) 如何在统一的CIM平台上实现风险评估、路径诱导与构件级损伤评估的联动,并通过典型滨海案例验证系统的实用性与可靠性。

针对上述问题,本研究设计了四个层层递进的研究内容:一是建立多源数据底座与高精度洪涝风险评估方法,构建三维水动力模拟与脆弱性指标体系,输出时空连续的危险度产品;二是提出MGNM增强策略与风险驱动代价函数,构建室内外联通的疏散网络并实现动态重规划;三是面向灾后恢复建立构件级洪灾损伤评估与三维可视化方法,实现与风险、疏散模型的多尺度联动;四是构建威海滨海应急服务中心综合验证流程,围绕洪涝模拟、疏散规划与构件损伤联动评估体系适用性。

\subsection{研究路线与数据支持}

本研究遵循"理论分析—方法设计—系统开发—实验验证"的路线展开。数据层面,采集无人机倾斜摄影、激光雷达扫描、楼宇BIM、城市管网GIS、潮位与雨量监测等资料;对象层面,以滨海应急服务中心主体建筑及周边关键设施为研究单元;验证层面,通过历史风暴潮过程与应急演练脚本开展对比验证。每一阶段的核心成果在下一阶段得到继承,实现从机理模型到业务系统的逐步落地。

本研究面向洪水等致灾情景下的人员安全疏散,提出一种以 CIM(City Information Modeling)为核心的室内外一体化疏散与闭环引导系统架构(见图~\ref{fig:tech_architecture})。系统遵循"数据—模型—决策—服务"的流水线思想,自下而上划分为七个层级,并通过标准化数据与控制接口实现解耦与可替换。

\begin{figure}[htbp]
\centering
\includegraphics[width=0.8\textwidth]{PIC/技术架构图.png}
\caption{研究技术路线图}
\caption*{Figure~\thefigure~ Research Technical Roadmap}
\label{fig:tech_architecture}
\end{figure}

\noindent\textbf{L1 数据采集层}:汇聚多源异构数据以保障真实性与时效性,包括:航空/近景数据(无人机倾斜摄影、航测),地形与几何(LiDAR 点云/网格),建筑语义信息(BIM/IFC),城市空间与道路网络(GIS/CityGML/OSM),以及时序监测(潮位、雨量、泵站与闸门状态、人群密度等)。

\noindent\textbf{L2 语义融合与治理(CIM)}:针对入湖前数据执行坐标与高度基准统一、分辨率重采样与时空索引构建,形成城市统一语义模型(建筑—楼层—房间—出入口—道路—广场等实体及拓扑关系),并沉淀于数据湖/仓库,支撑在线与离线计算。

\noindent\textbf{L3 风险评估层}:基于 RANS 的三维水动力模型模拟外部致灾因子演化,输出水深/流速/淹没范围等栅格要素记为 $H(x)$。结合物理、社会、经济维度构建脆弱性指标 $V(x)$,定义复合风险
\begin{equation}
R(x)=f\!\left(H(x),\,V(x)\right),
\end{equation}
常见实现为加权归一化或层次化综合,并可输出风险栅格与等值线以供下游使用。

\noindent\textbf{L4 室内外一体化网络建模}:由 IFC 自动生成多粒度导航网络(MGNM)。通过骨架提取与走廊中心线化(如改进 MAT)得到室内连通图,识别房间/走廊/门窗/楼梯(电梯)等关键节点;再与室外道路/广场网络在出入口—楼梯—街道处拓扑拼接,构建室内外统一可达图。

\noindent\textbf{L5 路径规划与动态引导}:在统一可达图上进行风险感知最短路搜索。对边 $e$ 定义综合成本
\begin{equation}
C(e)=\ell(e)+\lambda\,k_t(e)+\beta\,\rho(e),
\end{equation}
其中 $\ell(e)$ 为几何或时间长度,$k_t(e)$ 来源于人群密度热图与历史吞吐的拥堵惩罚,$\rho(e)$ 为由 $R(x)$ 沿入口/楼梯等结构要素投影或积分得到的风险暴露度,$\lambda,\beta$ 为权重。为兼顾城市级规模与实时性,采用"粗到细/单尺度"可切换策略,并在应急模式下周期刷新 $\tilde{H}(x)$ 与 $k_t(e)$ 实施局部重规划。

\noindent\textbf{L6 应用与可视化层}:面向多角色提供双端服务——指挥与应急响应控制台用于态势总览、告警联动、策略下发与资源调度;公众终端(App/小程序)提供个体/小群体引导与动态提醒;WebGIS/三维可视化展示风险演化、可达性变化与人群分布。

\noindent\textbf{L7 案例与评估}:在真实场景中开展综合演练与对比评估,指标涵盖:总撤离时间、处于高风险边上的路径占比、服务覆盖率、网络冗余度、重规划次数与端到端引导时延等,并结合专家评分与问卷对有效性、可拓展性与适配性进行定量与定性评价。

\noindent\textbf{数据与控制流说明}:系统形成从"采集—治理—评估—建模—规划—引导"的单向数据流与"监测/人流反馈—状态更新—策略下发"的反向闭环控制流。具体为:L1 多源数据经 L2 清洗统一后入湖;L3 结合静态地形与实时水文产生 $H(x)$ 与 $R(x)$;L4 由 BIM/IFC 构建室内网络并与室外拓扑拼接;L5 将 $R(x)$ 投影到网络边并按 $C(e)$ 进行多策略规划,生成指挥端与公众端指令;L6 输出可视化与引导服务;公众与现场终端的位置、速度、拥堵等实时反馈经 L2 更新模型状态,实现闭环优化。权重参数 $\lambda,\beta$ 与阈值的选取依据及灵敏度分析详见第5章。

本研究的主要创新点体现在理论创新、方法创新、技术创新、应用创新四个方面。在理论创新方面,提出基于CIM的城市灾害全过程管理理论框架,建立多源数据融合的统一语义模型,发展覆盖城市、建筑与构件多尺度的空间建模理论;在方法创新方面,建立高精度三维水动力风险评估方法,设计语义驱动的室内外网络自动构建与风险感知疏散算法,并提出结合水动力作用与材料耐水性的构件级洪灾损伤评估方法;在技术创新方面,构建支持实时响应与灾后复盘的集成系统架构,开发面向多用户的可视化交互平台,实现风险评估、路径诱导与损伤评估的一体化应用;在应用创新方面,建立完整的案例验证框架,形成可复制推广的技术方案,推动城市韧性治理能力的系统提升。

本研究对"城市韧性治理"理论的贡献体现在理论层面、方法层面、实践层面三个方面。在理论层面,本研究首次系统性地将CIM技术范式与城市韧性治理理论相结合,提出"数字韧性治理"概念框架;基于复杂系统理论与CIM的多尺度建模能力,发展了多尺度韧性协同治理理论;将动态韧性边界控制理论引入城市治理领域,提出动态韧性边界治理理论。在方法层面,集成数字航空摄影测量、BIM–GIS融合与三维RANS水动力建模,构建具备构件级精度的洪涝风险评估体系;基于MGNM的自动构建与语义增强路径规划算法,实现从IFC到疏散网络的无缝转换;基于CIM平台的统一语义框架,构建了支持多部门、多层级、多角色协同参与的韧性治理方法。在实践层面,通过威海沿海城市典型案例开展全流程验证,构建了从数据采集、模型构建、风险评估到应急响应的完整技术体系;通过将建筑信息学、地理信息科学、水动力学、运筹优化等多学科知识进行系统集成,为跨学科融合研究提供了成功示范;为智慧城市建设提供了韧性导向的发展路径。

综合上述贡献,本研究在理论创新、方法突破与实践示范等方面为城市韧性治理学科发展做出了系统性贡献,对推动韧性治理理论的数字化转型与实践化应用具有重要的学术价值与现实意义。

\subsection{章节安排}

本文围绕“基于 CIM 的城市灾害风险评估与智能疏散”构建多尺度、一体化的技术体系,全篇共分七章,具体内容安排如下:

第一章:从全球城市化加速与极端天气事件频发的时代背景切入,阐明开展城市洪涝风险评估与智能疏散研究的重要现实意义。随后系统回顾 CIM–BIM/GIS 语义互操作、室内外一体化通行网络、群体行为模拟等领域的国内外研究进展,指出现阶段模型割裂、数据标准不统一、室内外协同不足等关键问题。在此基础上,总结本研究的理论基础,明确研究目标、研究内容及技术路线,为全文奠定整体框架。

第二章:从理论建模出发,首先提出“物质—空间—性能—文化—时间”五维城市语义体系,并系统定义各维度所承载的信息内涵。其次,阐明“场景化—参数化—互动化—智能化”四视角的功能定位,构建 CIM 的多层次认知框架。最后,本章建立 CIM 的数据层次结构与对象关系模型,为后续的水动力模拟、路径规划及构件级损伤评估提供统一的语义支撑。

第三章 首先介绍本研究的数据基础,包括高精度数字航空摄影、LiDAR 及 BIM-GIS 融合技术,实现厘米级城市地形与语义信息的构建。其次,基于三维 RANS 方程构建城市级水动力模型,通过精确刻画垂向流动、局地涡旋与建筑扰动等物理机制,有效提升淹没深度、流速与压力等关键参数的模拟精度。然后,构建集成物理脆弱性、社会经济脆弱性与暴露度的综合指标体系,形成多层次洪涝风险评估方法。最后,通过威海典型区域案例验证模型性能,结果表明三维模型相较于二维模型在预测精度与细节刻画方面具有显著优势。

第四章 首先提出 BIM/IFC 向多用途几何网络模型(MGNM)的自动转换方法,实现室内几何—语义信息向可计算拓扑网络的结构化映射。其次,通过“入口—街道”连接策略,将室内 MGNM 与室外道路网络无缝衔接,构建统一的室内外可达性图。然后,设计粗细结合的多尺度路径规划方法及风险感知代价函数,在复杂环境与动态灾害条件下提升路径求解能力。最后,通过多案例实验系统验证模型在计算效率、路径精度与多主体适应性方面的优势,展示其在洪涝应急、公共建筑疏散与特殊人群引导等场景中的应用潜力。

第五章 首先构建 CIM-BIM 融合的构件级洪灾损伤评估体系,通过扩展 IFC 语义建立材料、几何与性能属性之间的多层关联关系。其次,基于多物理作用机制(静水压力、动水压力、浮力、水接触等)提出构件损伤判据,实现构件级的定量推断。然后,引入 Assembly-Based Vulnerability(ABV)方法,建立损伤率、功能恢复时间与经济损失之间的数学映射,实现损伤精细量化。最后,开发三维可视化系统,以多层级方式呈现损伤空间分布,并通过威海应急服务中心案例验证方法的可靠性与适用性。

第六章 首先提出基于 CIM 的城市防灾应急协同系统总体架构,明确数据层、模型层与业务层的功能分工,构建面向应急场景的多源信息流转机制。其次,设计模型集成与服务编排框架,通过微服务化架构提高系统可扩展性与模块解耦能力。然后,将动态风险驱动机制与 MGNM 多主体路径规划相结合,构建自适应协同决策模型,提高路径重算、资源调度与任务协同的智能化水平。最后,通过在威海滨海应急服务中心的实际部署与验证,展示系统在三维可视化、模型联动与实时响应方面的工程可行性,实现“风险识别—路径规划—指挥调度”的闭环协同能力。

第七章 对全文的研究成果进行凝练总结,从理论框架、方法创新、模型性能与工程价值四个方面系统归纳研究贡献。随后分析研究中仍存的不足,包括多灾种耦合能力、数据动态获取的实时性与模型智能化水平等方面的限制。最后展望未来的研究方向,如跨区域应急协同、多源感知融合、自主场景生成以及数字孪生城市体系的进一步完善。
