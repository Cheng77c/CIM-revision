\chapter{绪论}

\section{研究背景和意义}
进入21世纪以来,全球城市化进程持续加速。据联合国统计,2007年世界城市人口首次超过总人口的50\%,预计到2050年该比例将攀升至68\%\cite{UNDESA2019}。
国家统计局数据显示,中国的城市化率也从1978年的17.9\%迅猛提升至2023年的65.2\%\cite{NBS2024}。
快速城市化在促进经济增长的同时,也推动了多中心、高密度与高度耦合的城市空间形态形成,使城市在面对突发事件时呈现出更强的系统性脆弱性。
近年来极端天气事件频发,其强度与频率持续上升,洪涝灾害尤其成为威胁城市安全的主要风险类型之一。2021年郑州“7·20”特大暴雨造成的大范围城市内涝\cite{StateCouncil2022}、
2011年东日本大地震引发的复合灾害链\cite{CabinetOffice2012}以及2017年“哈维”飓风对休斯敦造成的重大洪涝损失\cite{FEMA2018},均深刻揭示出在复杂建成环境中,洪涝灾害具有传播快、链式耦合强、受影响对象多的典型特征。

面对城市洪涝灾害的高频化、多发化与复合化风险,传统灾害治理模式主要依赖静态历史数据、部门割裂的管理机制以及事后响应式的应急体系,
已难以满足在极端天气条件下对全域实时感知、跨部门协同与动态应急决策的需求。随着智慧城市与数字治理理念的快速发展,城市治理正在从静态的、
部门化的、经验驱动的范式,向动态的、系统联通的、数据与行为共同驱动的韧性治理体系转型\cite{Batty2018,Cao2023}。城市韧性治理强调系统的准备能力、
吸收能力、恢复能力与适应能力\cite{CN_Li2024CoastalCIM,CN_Qin2024ResilienceIndex},要求在复杂风险扰动中通过协同机制实现快速响应与高效恢复。
然而,洪涝灾害的多尺度耦合特性(如地形—建筑—水动力相互作用)、受灾主体的异质性(如人群、设施、地下空间)以及风险的链式传播特征,使得传统的单模型、
单部门、单阶段的治理方式难以支撑对洪涝风险的全过程动态管理。

在此背景下,城市信息建模作为智慧城市的重要技术底座,被视为破解现有数据割裂、模型孤岛与协同不足问题的关键路径。
CIM通过融合建筑信息模型(BIM)的精细化几何与语义信息、地理信息系统(GIS)的宏观空间分析能力、物联网(IoT)的实时动态数据以及社会经济统计信息,构建覆盖多尺度、多实体、
多语义层次的城市数字表达框架\cite{CN_Zhang2023DigitalTwin,CN_Shi2022CommunityTwin}。其统一的数据模型与语义体系能够打破BIM/GIS在结构、用途上的壁垒,
实现建筑室内空间与城市室外空间的无缝连接,为开展高精度洪水淹没模拟、构件脆弱性分析、风险动态评估以及应急疏散路径规划提供可计算、可模拟的分析环境\cite{CN_Xu2023DataFusion,CN_Ding2023ModelIntegration}。

同时,中国在国家与地方层面持续推进韧性城市建设的战略部署。从《韧性城市体系构建纲要(2023—2035年)》提出建设以CIM为核心的城市安全运行平台\cite{NDRC2023},
到《山东省城市应急管理行动计划(2022—2025年)》强调推进滨海城市风暴潮与洪涝灾害数字化治理体系建设\cite{Shandong2022},
再到住房和城乡建设部明确要求推动BIM/CIM协同应用于应急设施管理\cite{MOHURD2022},均为本研究提供了清晰的政策背景与现实需求。
综上所述,在极端天气不断加剧、城市洪涝灾害风险持续攀升的大背景下,亟需构建一个集成多源数据、耦合多模型、面向协同决策的城市洪涝灾害风险演化与控制体系,以支撑城市的主动防御、智慧响应与韧性提升。

综上所述,当前城市灾害管理正面临数据异构与平台分裂导致难以协同、建模精度与计算实时性之间存在矛盾、以及空间语义断裂形成响应孤岛等核心挑战\cite{CN_Yu2022KnowledgeGraph,CN_Zhong2024CrossDomain}。
因此,系统性地构建一套基于CIM平台的城市灾害风险评估与智能疏散方法体系,其理论意义在于推动城市风险管理从“被动响应”向“主动控制”、从“经验决策”向“数据驱动”、从“单点优化”向“系统协同”的范式转变,
丰富和发展智慧城市与韧性治理的交叉学科理论;其现实意义在于直接响应国家战略需求,通过技术集成与方法创新,为提升我国超大、特大城市的防灾减灾救灾能力,建设安全、宜居、韧性的智慧城市提供可落地的技术方案与科学决策依据。

\section{城市洪涝灾害模拟、疏散与数字治理体系的研究进展}
\subsection{城市洪涝灾害模拟与风险评估研究现状}

在城市洪涝灾害研究中,二维浅水方程(Shallow Water Equations, SWE)长期作为主力水动力内核,用于刻画地表径流与淹没扩展过程。传统完整动量方程在高分辨率、广域建模中计算成本较高,因此“局部惯性”(local inertial)近似成为兼顾计算效率与物理精度的重要途径:通过忽略部分加速度项,在较低计算代价下仍能保持对亚临界流态的合理刻画,使城市或区域尺度的高分辨率淹没模拟成为可能\cite{de2012improving}。相关研究通过敏感性试验系统评估了该近似在水深突变、流态过渡与局部能量耗散等关键动力过程中的可靠性,进一步拓展了其在密集建筑群、微地形突变区域等复杂城市环境中的适用范围\cite{deAlmeida2013LocalInertial}。同时,针对局部惯性模型在低糙率或湿/干界面条件下可能出现的数值振荡问题,有研究提出了改进格式,从时间步约束和通量处理层面提升稳定性,使其在城市洪涝模拟中能够保持较高的数值可信度\cite{de2012improving}。

随着洪涝模拟在应急业务中的应用不断深化,模型评估与基准化研究成为提升模型透明度与区域可比性的重要方向。Hunter 等\cite{Hunter2008Benchmark2D}通过标准算例与观测资料,对多种二维水动力模型在数值稳定性、误差统计指标和计算效率方面进行了结构化对比;相关工作进一步拓展到高分辨率城市街区、涵洞与排水系统耦合、复杂多源入流(降雨、市政排放、上游来水)等真实场景,从而推动二维模型向更高物理解释性和工程适用性发展。随着 LiDAR、无人机影像和高精度 DEM/DSM 的普及,模型验证方法也从传统的点位水位或淹没深度对比转向二维空间范围匹配、动态淹没边界移动速度及误差场结构分析,实现了洪涝模拟从“单点验证”向“空间验证”与“动力学验证”的范式升级。

尽管局部惯性模型在多数城市情景中具有较高精度,但最新研究指出其在超临界流态或局部冲击区可能产生偏差,尤其当 Froude 数接近或超过 1 时,局部水深误差可能增至 20\% 以上\cite{Hunter2008Benchmark2D},提示在极端事件或强动力条件下需谨慎选用。面向实时预报业务,研究提出了更为简化的二维淹没模型与降阶方案,以保证计算效率;通过参数化摩阻表达、分辨率自适应与多源实时监测资料(降雨雷达、IoT 水位计、排水系统流量计)的融合,使模型能够对不同触发机制(强降雨、潮汐顶托、河道溢流、排水不畅)实现快速响应,为城市洪涝的实时预警奠定了基础\cite{Dottori2013Simple2D}。同时,“模型集成(model ensemble)”与“情景库(scenario library)”技术的发展,使得预先计算的大量可能场景可在应急状态下即时检索与匹配,从而支持多情景、多时段的应急推演与风险图编制,为排水系统设计与应急预案制定提供了更加稳健、透明与可重复的技术支撑。

在更大范围的综合预报系统中,遥感—模型一体化正成为城市洪涝灾害风险评估的重要途径。SAR 衍生淹没信息不再仅作为独立产品,而是被显式嵌入水文—水动力模型链:上游以水文与降雨径流模型提供入境流量,中游以二维浅水方程或局部惯性模型构建高分辨率水动力内核,下游则通过 SAR 洪涝反演模块周期性更新水位与淹没边界,实现“观测—同化—预报”联动的动态循环\cite{Mason2014DoubleScattering,Giustarini2015UncertaintySARFlood}。由于洪水演变的关键时段与排涝调度、工程调控高度重叠,因此如何在有限重访周期下高效利用 SAR 影像成为关键问题。一方面,通过多星座观测、任务加密规划等手段缩短观测间隔;另一方面通过模型滚动积分维持非观测时段的预报连续性,使高频观测与模型预测形成相对平滑的融合链条\cite{Hostache2018WRR,hostache2010assimilation,schumann2009progress}。

从 SAR 观测到模型同化的接口设计方面,研究开始关注不同信息类型间的匹配方式:淹没二值图或概率淹没图如何转换为可直接约束水位或水深的观测量\cite{Giustarini2015UncertaintySARFlood,giustarini2016probabilistic,schlaffer2017probabilistic}。在状态同化中,通过构建从水位场到淹没概率场的观测算子,并引入与地形起伏、粗糙度和地物类型相关的似然函数,从而缓解淹没—水位之间“一对多”的非唯一性问题。在参数同化中,通过长序列 SAR 观测反演河道糙率、漫滩阻力或城市局部糙率修正系数,实现跨事件标定\cite{DiMauro2021HESS},使模型在多次洪水事件之间保持长期稳定性\cite{DiMauro2021HESS,DiMauro2022WRR}。

在复杂城市尺度上,结构不完备的模型与观测不完全的 SAR 影像共同导致更高程度的不确定性传递\cite{Giustarini2015UncertaintySARFlood,Shen2019RemoteSensingFlood,Li2019RSUrbanSARFusion}。高层建筑阴影、叠掩与多路径效应使得 SAR 影像中存在大范围“不可观测区”\cite{Mason2014DoubleScattering,Mason2010TerraSARX},而地下排水网络的简化处理又可能在模型端引入结构偏差。为此,概率洪水图与影像不确定性建模被用于将 SAR 观测从“0/1 分类”扩展为“像元级置信度场”,并在同化权重与观测误差协方差中显式引入,使系统在可信区域施加强约束,在高不确定性区域弱化约束,显著提升了同化的稳健性\cite{Giustarini2015UncertaintySARFlood,giustarini2016probabilistic,schlaffer2017probabilistic}。

在同化算法方面,调温粒子滤波等先进贝叶斯技术通过逐步增强的调温系数,将一次观测分解为多个“子观测”注入系统,使粒子群在高维状态—参数空间中有充足时间完成重采样与扩散,避免因高置信度观测导致的权重塌缩与样本退化\cite{DiMauro2021HESS,DiMauro2022WRR}。同时,将概率淹没图中的置信度直接纳入似然函数,使“不确定性—模型误差—预报风险”的空间耦合得到显式描述,为多源观测的统一融合提供了可靠的贝叶斯框架。

在城市韧性治理与风险沟通中,SAR—模型一体化系统的输出也逐渐由水位或流速等物理量向面向决策的空间化风险指标扩展:如通过“淹没概率—暴露资产—脆弱性函数”构建像元级期望损失场,生成超越概率曲线,为预案比选、损失评估与资源调度提供量化依据\cite{Hostache2018WRR,schumann2018assisting}。在堤防漫顶、下穿通道、交通枢纽与生命线工程等关键高风险区域,可进一步分析多情景洪水下的概率淹没时长与最大水深区间\cite{schumann2009progress},为城市极端事件的应急决策提供精细化支撑。

总体而言,从二维水动力模拟、遥感反演到同化与不确定性量化,城市洪涝灾害模拟已从单一模型向“观测—模型—决策”联动的系统化框架演进。局部惯性近似与降阶算法解决了高分辨率模拟的计算瓶颈,SAR—模型一体化提升了城市洪涝认知的时空完整性,不确定性建模与贝叶斯同化增强了预报体系的稳健性。这一系列发展为城市洪涝灾害风险评估提供了更高精度、更高效率与更高可解释性的模型体系,为下一代城市洪灾预报、风险图编制及韧性治理框架奠定了坚实基础。

\subsection{城市洪涝情景下的疏散行为与室内外一体化通行网络研究现状}

在城市洪涝灾害情景中,室内外通行空间的可达性、路径选择与拥堵形成机制对人员疏散效率和生命安全具有决定性影响。因此,如何构建能够真实反映建筑内部结构特征、室外道路组织以及二者之间连续过渡关系的统一通行网络,成为洪涝情景下开展动态疏散分析的基础。面向这一需求,BIM 导向的多用途几何网络模型提出了从 IFC 语义自动抽取室内通行网络的技术路径:通过识别房间、门、走廊、楼梯等语义对象及其邻接关系,构建与建筑几何高度一致的网络拓扑;再将其与室外道路网通过“入口—界面—街廊”等关键要素完成连通,从而实现室内外通行空间的无缝衔接\cite{Teo2016BIMIndoorNetwork}。在此基础上,统一的室内—室外网络模型显著降低了传统接口简化带来的可达性偏差,为室内外联合路径规划、可达性分析和拥堵识别提供了统一的网络表达\cite{Claridades2021SeamlessNav}。相关研究进一步指出,通过面向多尺度对象(建筑单体、街区组织与城市结构)的模型抽象与连接策略,同一数据结构能够支持跨尺度路径搜索、区域避障、楼层转换以及多模式出行(步行—室外交通—建筑内部移动)的连贯表达,推动跨场域导航由可行性验证走向流程化落地。

在室内网络构建算法方面,最新研究正从二维“房间—门”拓扑向面向复杂三维几何的自动化生成技术演化。Zhang 等提出的 GINIT 方法从 BIM 模型中抽取楼板和开口,通过图像栅格化与骨架细化技术自动生成三维室内导航网络,对异形空间、中庭和斜坡等复杂几何表现出更高鲁棒性\cite{zhang2023automatic}。与 IFC 语义驱动的 MGNM 形成互补,此类几何主导型方法可作为语义残缺或建模风格不一致的 BIM 数据的“兜底”机制,为构建全局统一的城市级通行网络提供更高容错能力。

在室内外过渡空间建模上,传统 BIM/GIS 集成多将入口简化为单节点,难以表达台阶、门厅、站台等结构的连续通行属性。Wang 等基于 IndoorGGML 标准提出了室内外过渡空间体素化建模方法,通过点云与几何自动识别可通行区域,并构建室内 Cell Space 与室外道路网之间的细粒度连接图,实现更加真实的方向性、坡度与宽度表达\cite{wang2023automatic}。此类方法强化了 MGNM 的“界面层”能力,为无障碍通行、多主体换乘与机器人导航提供重要的几何支撑。与此同时,AR/VR 技术也逐渐融入疏散辅助系统中,Ahn 等提出基于 BIM 的增强现实疏散导航方案,相比传统示意图显著提升了路径理解效率\cite{ahn2024bim};Valizadeh 等构建的 BIM–GIS 融合 AR 导航系统进一步提升了动态环境下的室内外路径呈现能力\cite{valizadeh2024indoor}。上述研究表明,随着几何抽取、语义解析与融合建模的发展,室内外通行网络正从“手工建模—局部分析”向“语义驱动—自动生成—跨尺度集成”的工程化体系演化,为洪涝情景下动态通行性评估奠定了重要基础。

在人员疏散行为研究方面,社会力模型(social force model)通过简洁的微观相互作用机制刻画行人之间以及人与环境之间的力学关系,能够自然再现自组织排队、瓶颈拥堵与“越急越慢”等现象,奠定了微观疏散模拟的理论基石\cite{Helbing1995}。该框架将个体视为在期望速度、社会斥力、界面约束和随机扰动共同作用下的“主动粒子”,使得行人运动的非线性耦合行为可以在连续空间中获得统一描述。随后研究逐步引入视域、感知域\cite{moussaid2011simple}、效用最大化决策\cite{hoogendoorn2004pedestrian}、相对速度与间距驱动的排斥力\cite{yu2005centrifugal,chraibi2011force}等机制,以增强模型在复杂疏散场景下的行为表达能力。伴随研究深入,“逃生恐慌”中的相变机制\cite{Helbing2000EscapePanic}、密度与出口几何对拥堵演化的影响\cite{Helbing2007CrowdDisasters,johansson2008crowd}、建筑拓扑结构对行人动力学的系统性调制\cite{johansson2007specification,kretz2011quickest}等规律被逐步揭示,为应急疏散策略、建筑安全设计及风险识别提供了可检验的理论依据。

与连续模型相对,元胞自动机(Cellular Automata, CA)以规则驱动的离散方式模拟行人流动,通过“地板场(floor field)”将环境吸引、局部占据、从众心理与出口偏好编码于网格结构,使群体动力学能够以较低计算成本在复杂拓扑中重建\cite{Burstedde2001FloorField}。在工程应用中,社会力模型与 CA 模型常被结合使用,以兼顾行为逼真度与计算效率。随着 BIM 语义抽取和网络生成技术的发展,IFC 模型中的房间、门、楼梯等语义对象可自动映射为疏散模拟的网格结构或边界条件,实现从建筑设计到疏散仿真的全链路自动化\cite{Zhu2018IFCEvac}。近年来,数字孪生理念也逐步应用于疏散预测与动态控制中,借助传感器监测与视频分析技术,可在大型活动场景实现分钟级预警能力\cite{yasufuku2024development},并在“感知—仿真—决策—执行”闭环逻辑下实现智能建筑安全管理\cite{almatared2023digital}。

总体而言,室内外一体化通行网络研究从几何抽取、语义建模到跨尺度融合,已为复杂建成环境下的路径规划与通行评估提供了丰富工具;群体行为与疏散动力学研究则在连续模型、离散模型与实证观察的交叉发展中构建了较为成熟的行为表达体系。然而,现有研究仍主要关注通行空间与人群动力学本身,对洪涝灾害中的关键影响因素——如积水深度、流速变化、局部水动力危险性、可通行性的动态退化等——缺乏系统的耦合机制。尤其是洪涝情景中“水—路—人”之间的相互作用尚未形成统一的表达框架,洪水对通行网络的实时可达性影响、人群在动态风险下的路径选择行为、建筑内部不同楼层之间的风险传播模式等问题仍有待深入研究。这些瓶颈表明,有必要构建能够同时整合通行网络结构、群体行为动力学与洪涝水动力过程的综合模型体系,为城市洪涝灾害中的动态疏散分析与协同控制提供坚实理论与方法基础。


\subsection{CIM 与数字孪生支撑的城市洪涝灾害治理研究现状}

城市信息建模旨在弥合建筑信息模型与地理信息系统在空间尺度、语义组织与应用场景上的割裂,通过统一的三维时空数据与语义框架,为城市运行、规划管理与应急响应提供“数据—模型—决策”闭环支撑\cite{Xu2014CIM}。随着研究范式由建筑域向城市域扩展,CIM 被系统化界定为汇聚 BIM、GIS、物联网观测、遥感监测与管理信息系统等多源数据的“城市级数字底座”,不仅承担精细化几何与语义表达,更支持业务流程建模与决策规则表达\cite{zhang2024city}。最新综述指出,CIM 已从早期聚焦三维建筑/城市模型整合的“模型中心”范式,演进为面向城市韧性治理、基础设施安全与公共服务优化的综合信息基础设施,其与大数据、物联网和人工智能等技术的深度耦合,使城市治理具备可追溯、可推演与可反馈的全生命周期能力\cite{liu2017state}。在这一框架下,CIM 不再是 BIM 与 GIS 的简单叠加,而是通过统一的数据模型和语义约束,将规划、建设、运维与应急管理等不同部门和时空尺度的需求耦合在共享的信息空间中\cite{tan2023citygml}。

在具体的几何与语义融合路径上,现有研究多以 IFC 与三维城市模型的耦合为突破点,通过构建 Unified Building Model(UBM)实现建筑构件、空间单元及其关系的统一表达\cite{ElMekawy2012}。围绕 IFC-CityGML 双向转换,学界已发展出一套较为成熟的技术体系:包括基于三元图文法(Triple Graph Grammar)的可配置规则库与工具链,用以维护两类数据模式间的结构对应关系\cite{Tauscher2019TGGCnv};基于图匹配与规则模板的近无损转换方法及其质量评价体系,用于刻画几何精度、语义完整性与拓扑一致性等多维度损失\cite{stouffs2018achieving,soini2019investigating};以及通过 CityGML 应用域扩展(ADE)将构件级细部信息映射至城市级模型,实现建筑级到城市级的多尺度一致表达\cite{tan2023citygml,biljecki2021extending}。在更复杂的空间语义层面,产权边界、使用功能与权利束(RRR)等三维权属信息也被纳入 BIM/IFC 的扩展体系中,并与土地管理模型(LADM)衔接,用于支撑三维不动产登记、城市更新与精细化治理\cite{Atazadeh2017Ownership3D,xie2022automatic}。

随着模型从静态几何表达向动态语义推理演化,CIM 正进入“语义驱动—机制耦合”的深层协同阶段。在这一趋势下,城市对象被建模为具备状态、行为、约束与演化逻辑的高语义实体,能够在统一的知识框架中进行动态更新、规则推理与自动响应。得益于 BIM、GIS 与物联网的深度融合,CIM 模型初步形成跨尺度、跨模态、跨时态的数据闭环,使建筑构件信息、街区拓扑结构与城市运行指标能够在同一语义体系下关联复用\cite{Xu2014CIM,zhang2024city,liu2017state}。这不仅增强了模型在规划审批、建设管理与灾害模拟等典型应用中的可解释性、可追溯性,也为构建“可计算的城市知识网络”奠定基础。

在“模型—数据—治理”联动方面,数字孪生(Digital Twin)被视为衔接多源观测、物理/数据驱动模型与协同决策的核心承载体,其内涵已从早期的“可视化复制”扩展为集成数据治理、过程模拟、风险评估与策略推演的综合平台。流域尺度研究提出了面向可迁移、可持续与公平治理的数字孪生框架,将水文气象观测、流域水动力—水质模型、利益相关方行为模型及政策规则显式纳入统一结构,并涵盖灾前识别、灾中调度与灾后评估的全过程循环\cite{Yang2024npjNH}。相关研究指出,这类框架在洪涝风险识别、应急调度、资源配置与长期规划等方面展现显著优势\cite{Yang2024npjNH,pal2025blueprint}。

在城市水务运维场景中,数字孪生进一步扩展为涵盖泵站、管网、污水厂与调蓄设施的实时数据采集、多源融合、状态估计与主动控制的一体化系统,其中模型降阶、替代模型与多目标优化技术为在线优化与情景分析提供支撑\cite{Cavalieri2024SensorsAAS,Bonilla2022WaterDTWDS}。此外,相关研究强调应从运营绩效(预测精度、响应时间、资源利用效率)与成本收益(建设成本、维护成本、治理成效)双维度评估数字孪生的长期价值与可持续性\cite{Cavalieri2024SensorsAAS,Bonilla2022WaterDTWDS,wang2024digital},并结合工程经验与多源测量数据对其标定流程与运行成效进行系统化总结\cite{bonilla2022digital}。

总体来看,从 CIM 底座与语义互操作,到二维洪涝模拟与 SAR 同化,再到面向城市治理的水系统数字孪生,国内外研究已逐步构建起面向“风险识别—疏散响应—治理优化”的多层次技术体系\cite{hlal2025digital}。底层的 CIM 与多源时空数据体系为洪涝过程模拟、通行网络构建与人群行为刻画提供统一几何与语义基座;过程层的二维水动力模型、室内外通行网络与疏散模型实现了从洪水动力到人群响应的联动模拟;应用层的流域与城市数字孪生则进一步将模型、观测与治理规则嵌入统一框架,形成跨部门、跨时空尺度的综合决策平台\cite{Yang2024npjNH,Cavalieri2024SensorsAAS,Bonilla2022WaterDTWDS,maksoud2023digital}。

然而,要实现面向城市洪涝灾害的规模化数字孪生落地,仍有若干挑战亟需突破:其一,在大尺度 BIM/GIS 集成中,仍缺乏自动化、高一致性的建模流水线与质量控制机制,制约城市群级三维时空基底的构建;其二,在跨尺度模型链耦合过程中,观测误差、模型结构偏差与决策输出之间的不确定性传递尚缺乏系统化刻画框架\cite{kim2025stormwater};其三,洪涝场景中的“水—路—人—设施”多过程交互尚未形成统一的语义与目标函数体系,难以支撑实时调度、路径优化与资源配置的协同求解;其四,现有城市治理体系中数字孪生的部署路径仍需在试点工程、跨部门协作与能力建设中不断验证与优化,方能实现从技术可行到制度可用的可持续转化\cite{Douben2006}。这些问题共同构成了本研究的关键切入点,为构建面向城市洪涝灾害风险演化与协同控制的 CIM–数字孪生方法体系提供了明确的理论与工程需求。


\subsection{研究局限与不足}

通过对已有研究的系统梳理可以发现,尽管在城市洪涝水动力模拟、室内外一体化通行网络构建、群体疏散行为建模以及 CIM—数字孪生治理框架等方面已取得重要进展,但面向真实城市洪涝灾害的全流程风险识别与协同控制,现有技术体系仍存在显著局限,主要包括以下五个方面。

首先,在数据与模型层面,“模型孤岛”现象仍然突出。二维水动力模型、疏散模型与 CIM 平台虽然均已实现各自领域的深度发展,但缺乏统一的语义框架将其机制化耦合。当前洪涝模拟通常输出水深、流速等物理量,而疏散模型则依赖“通行可行性”“风险成本”等行为量化指标,二者之间缺乏自动、实时、一致的映射机制,仍需大量人工规则与经验假设进行转化。这种跨模型语义断裂使得“洪涝演化—通行退化—人群响应”难以构成统一的可计算链条,制约了从预警到应急响应的自动化推理与调度。

其次,在动态感知与实时调控方面仍存在明显断层。现有研究大多基于静态或预设情景开展疏散模拟,而实际洪涝过程具有强烈的时序性与不确定性。随着水位上升、局部积水扩张、道路失效或设施故障,通行可达性会持续变化,但现有模型体系尚难以将物联网传感器、遥感监测或排水系统运行状态实时融合入疏散模型,更无法形成“感知—模拟—决策—调控”的闭环反馈,从而降低了应急响应的自适应性和时效性。

第三,在行为建模方面,现有疏散模型的人性化表达不足。社会力模型、元胞自动机等方法虽能刻画群体动力学的宏观模式,但对个体在灾害压力下的心理波动、信息获取能力差异、家庭集结行为、从众与恐慌等因素的建模仍然有限。此外,特殊人群——如老人、儿童、残障人士——在洪涝中的行动能力受限,其决策行为与危险感知显著不同,但现有模型普遍采用同质性假设,导致对真实人群疏散困境的预测偏差,并削弱疏散策略的公平性。

第四,跨尺度一体化建模仍难以兼顾精度与效率。在城市级疏散规划中,为控制计算开销,路网与建筑内部拓扑常被大幅简化,导致对建筑内部真实可达性与通行瓶颈的精细刻画不足;而高保真室内疏散模拟又难以扩展至城市尺度。这种“尺度悖论”使得室内—街区—城市多尺度之间缺乏统一表达,从而影响疏散策略在真实行动层面的可执行性。

最后,实证验证与评价体系尚未成熟。由于缺乏真实洪涝灾害下的人群疏散轨迹与多源观测数据,现有模型多依赖理想算例、实验室数据或小范围演练进行验证,难以形成面向实际场景的系统性评估框架。同时,不同研究缺乏标准化、可量化的性能指标,使得模型优劣难以比较,阻碍了技术从研究原型向工程业务系统的转化。

综上所述,当前研究体系在一定程度上能够支持城市洪涝灾害的模拟、分析与辅助决策,但在模型协同、动态更新、行为真实性、跨尺度一体化与实证验证等关键环节仍存在系统性短板。突破这些瓶颈,构建能够贯通“水—路—人—系统”的城市洪涝风险演化模型与协同控制体系,正是本研究的核心出发点与现实需求。

\section{研究目标、内容与技术路线、创新点与难点}
\subsection{研究总体目标与关键科学问题}
在全球气候变化和高强度城市化叠加作用下,极端降雨事件触发的城市洪涝灾害呈现出多源驱动、多尺度演化与多系统耦合的复杂特征。高密度建成环境、地下空间与综合管廊等新型基础设施形态,使得城市在遭遇洪涝冲击时更易产生级联失效与功能紊乱。现有研究在城市信息建模、水动力模拟、风险评估、应急疏散与灾损分析等方面已取得重要进展,但在统一机理框架下系统刻画“降雨—汇流—积涝—暴露—损伤”的风险演化过程,并将其与城市防灾应急业务深度耦合仍然存在明显不足,难以满足韧性城市建设对精细化风险识别、智能决策与协同控制的综合需求。

基于上述背景,本研究以“城市洪涝灾害的风险演化机理与协同控制系统研究”为主线,立足城市信息建模平台与三维时空语义表达能力,综合运用水动力学、风险科学、交通与人群行为建模等多学科理论,面向典型滨海城市及其重点片区,开展从机理分析、模型构建到系统集成与工程验证的系统研究。总体目标可概括为:构建面向城市洪涝灾害的 CIM 语义建模与风险演化分析框架,形成多尺度水动力—风险耦合评估、风险驱动的室内外一体化疏散与构件级损伤评估的关键方法体系,并在此基础上集成形成支持多部门协同决策的防灾应急协同控制系统,为复杂城市洪涝灾害的机理认知、风险量化与应急管理提供系统化技术路径。

围绕这一总体目标,本研究聚焦以下关键科学问题:

(1)多源异构城市空间数据与洪涝致灾机理如何在统一的 CIM 语义框架下实现一体化表达与机理耦合?城市洪涝灾害涉及地形地貌、建筑形态、地下空间、基础设施网络、人口与资产分布等多源异构数据,以及降雨产汇流、水动力演化、暴露与脆弱性等多尺度、多过程机理。如何构建兼顾对象结构、空间拓扑、状态演化与事件触发关系的 CIM 语义模型,将建筑信息模型、地理信息系统与水动力模型有机融合,在统一的数据—模型—业务框架下刻画城市洪涝灾害的风险演化链条,是实现后续高精度模拟与业务协同的基础性科学问题。

(2)多尺度水动力过程与城市复杂建成环境耦合下,如何实现城市洪涝风险的高精度定量评估与时空演化表征?市空间形态高度复杂,局部关键部位的流场结构与整体汇流过程具有显著多尺度耦合特征。如何在城市尺度上协调三维雷诺时均 Navier–Stokes 方程与二维浅水方程的分区嵌套,合理处理复杂边界条件与地物粗糙度,构建兼顾计算精度与效率的水动力模拟体系,并在此基础上综合水深、流速、持续时间等指标,形成可直接服务于城市规划与应急管理的洪涝危险度与综合风险栅格,是面向工程应用的核心科学问题。

(3)动态演化的洪涝风险场如何映射到室内外疏散网络与人群行为,支撑风险驱动的路径规划与协同疏散决策?城市洪涝灾害情景下,水深与流速的时空演化直接影响道路与建筑内部通行能力与安全程度。如何在 CIM 平台中统一编码城市道路网络与建筑内部拓扑,构建多层广义网络模型,将水动力模拟结果转化为时间依赖的通行约束与风险代价函数,进而刻画人群在风险感知下的路径选择与拥挤效应,实现室内外一体化疏散路径优化与疏散过程仿真,是连通物理过程与人群行为的关键科学问题。

(4)洪涝荷载作用下的构件级损伤如何定量映射至经济损失与功能退化,在统一 CIM 平台上与风险评估和疏散决策实现多尺度联动?城市洪涝灾害不仅造成短期积水风险,还会导致建筑与基础设施构件的物理损伤与功能退化。如何在构件尺度上建立水深、流速等荷载与不同材料、构造形式的损伤关系,构建属性–建筑–体素一体化表示,将构件级损伤进一步映射为经济损失与服务性能损失,并通过 CIM 三维场景实现与城市尺度风险评估、疏散路径与应急资源配置结果的联动表达,是支撑韧性评估与灾后恢复策略制定的重要科学问题。

(5)在实际城市场景中,如何将多源数据、复杂模型与应急业务流程集成为可操作的城市洪涝灾害协同控制系统?多源数据接入、异构模型耦合与多部门业务协同具有显著的复杂性与不确定性。如何在 CIM 平台之上构建数据层—模型层—业务层分层微服务架构,实现洪涝风险评估、疏散引导、构件级损伤评估等模型组件的灵活编排与服务化封装,并将其嵌入预警发布、避险引导、资源调度与灾后评估等业务流程,形成可在典型滨海城市中部署运行的防灾应急协同控制系统,是推动研究成果走向工程应用的综合性科学与工程问题。



\subsection{研究内容与技术路线}

在前文研究总体目标与关键科学问题的基础上,本研究围绕“城市洪涝灾害的风险演化机理与协同控制系统”这一主线,构建由理论方法、模型构建、系统集成与工程应用验证相衔接的研究内容体系与技术路线。整体思路上,以城市信息建模为统一数据与语义底座,以多尺度水动力模拟为洪涝风险演化刻画的物理基础,以室内外一体化疏散与构件级损伤评估为风险响应与灾后评估的核心支撑,通过分层解构与跨尺度耦合,形成从“数据汇聚—风险计算—路径诱导—损伤复盘—协同控制”的完整技术链条,与论文题目中的“风险演化机理”与“协同控制系统”相呼应。

本研究的主要内容可归纳为五个相互关联且逐级递进的方面,各部分之间在 CIM 平台与三维时空语义框架下形成数据流与模型流的有机耦合,共同支撑城市洪涝灾害的系统性防控:

(1)研究内容一:面向城市洪涝灾害的 CIM 语义框架与风险演化机理建模(对应第 2 章)。综合分析城市洪涝灾害演化特征与防灾应急业务需求,构建集成要素维、空间维、性能维、文化维与时间维的 CIM 5D--4V 语义体系,引入对象--关系--事件语义结构,统一表达城市实体的几何形态、功能属性、时空状态与事件驱动行为。围绕“降雨—汇流—积涝—暴露—损伤”的机理链条,构建多维灾害知识模型与风险演化逻辑,为多尺度水动力模拟、室内外疏散建模与构件级损伤评估提供统一的语义与机理基础。

(2)研究内容二:基于多尺度水动力耦合的高精度城市洪涝风险评估方法(对应第 3 章)。面向复杂建成环境下的洪涝过程模拟需求,构建多尺度嵌套的三维雷诺时均 Navier--Stokes 方程与二维浅水方程水动力模拟体系,形成兼顾局部关键区与全域城市尺度的协同计算框架。通过精细化地形—建筑一体化建模、边界条件同化与摩阻参数标定等技术,实现对城市地表径流与积涝演化过程的高精度表征。在此基础上,构建水深、流速、持续时间等指标融合的洪涝危险度量方法,并结合暴露度与脆弱性分析形成城市洪涝综合风险栅格,为后续疏散规划与损伤评估提供统一的风险场输入。

(3)研究内容三:风险驱动的室内外一体化疏散路径规划方法(对应第 4 章)。针对洪涝灾害情景下人群避险疏散需求,基于 CIM 平台统一编码城市道路网络与建筑内部拓扑,构建多层广义网络模型,实现室内外空间的一体化连通。利用多尺度水动力模拟结果构建时间依赖的通行约束与风险感知代价函数,将动态洪涝场映射到疏散网络,建立风险驱动的多目标疏散路径优化模型与人群演化仿真方法,并在典型洪涝情景下对不同疏散策略的安全性与效率进行评估,为协同控制系统中的避险引导与方案比选提供决策依据。

(4)研究内容四:构件级洪灾损伤评估与三维可视化技术(对应第 5 章)。面向建成环境精细化损伤分析需求,提出属性--建筑--体素一体化表达方法,将建筑与基础设施构件分解到体素尺度,实现几何、材料与功能属性的统一编码。结合水深、流速等水动力参数与构件脆弱性曲线,构建构件级物理损伤、经济损失与服务性能损失评估方法,并在 CIM 三维场景中实现构件级损伤状态与城市尺度风险栅格之间的联动可视化表达,为韧性评估、灾后恢复策略制定及协同控制系统中的损伤诊断与优先恢复排序提供量化支撑。

(5)研究内容五:基于 CIM 的城市洪涝灾害协同控制系统构建与应用验证(对应第 6 章)。在上述理论与方法研究基础上,构建数据层—模型层—业务层分层微服务架构,将洪涝风险评估、室内外疏散规划与构件级损伤评估等模型组件进行服务化封装与灵活编排,形成面向预警发布、避险引导、资源调度与灾后评估等典型业务的防灾应急协同控制系统。以典型滨海城市及其重点片区为对象,开展系统部署与情景演练,检验所提出方法体系与系统框架在真实城市场景下的工程适用性与综合效能,实现论文题目中“协同控制系统”的工程化落地。


在上述研究内容基础上,本论文形成了如图\ref{fig:Roadmap}所示,与第 2--6 章严格对应的技术路线,突出各章之间在输入、输出及相互衔接关系上的一致性。

\begin{figure}[htbp]
\centering
\includegraphics[width=1.0\textwidth]{PIC/技术路线图.png}
\caption{技术路线图}
\caption*{Figure~\thefigure~ Research Technology Roadmap}
\label{fig:Roadmap}
\end{figure}

首先,第 2 章以城市基础空间数据、建筑信息模型、地理信息数据、基础设施网络资料以及相关洪涝历史案例与业务需求为输入,输出面向城市洪涝灾害的 CIM 5D--4V 语义体系、O--R--E 语义结构以及“降雨—汇流—积涝—暴露—损伤”风险演化机理模型。这些输出构成统一的语义与机理底座,为后续多尺度水动力模拟中的几何与属性建模、风险指标体系构建、疏散网络拓扑编码与构件级属性定义提供规范化的数据结构与语义约束。

在此基础上,第 3 章以第 2 章构建的 CIM 语义框架和多源城市基础数据为输入,引入降雨情景、边界条件与排水系统参数,构建多尺度三维雷诺时均 Navier--Stokes 方程与二维浅水方程嵌套的水动力模拟体系。第 3 章的主要输出包括城市范围内随时间演化的水深、流速等水动力场,以及基于水深、流速、持续时间、暴露度与脆弱性综合形成的洪涝危险度与风险栅格。这些结果一方面作为第 4 章构建风险驱动疏散代价函数与通行约束的核心输入,另一方面作为第 5 章进行构件级荷载作用与损伤评估的重要边界条件。

第 4 章以第 2 章提供的 CIM 语义框架下的道路网络与建筑内部拓扑数据和第 3 章输出的时空洪涝风险场为输入,在多层广义网络模型中统一编码室内外疏散空间结构,构建时间依赖的风险感知代价函数与通行可行性约束。第 4 章输出包括室内外一体化疏散网络模型、不同情景下的最优或次优疏散路径方案、人群时空分布与拥挤演化过程,以及对不同疏散策略的安全性与效率评价指标。这些输出将作为第 6 章协同控制系统中避险引导模块与预案推演模块的核心输入,为系统级协同控制提供路径决策依据。

第 5 章以第 2 章构建的 ABV 一体化构件表达模型和第 3 章输出的水动力场为输入,在构件尺度上建立洪涝荷载与材料、构造形式之间的损伤关系,形成构件级物理损伤判别模型,并进一步结合修复成本模型与功能退化模型,输出构件级经济损失与服务性能损失评估结果。通过将这些结果映射回 CIM 三维场景,第 5 章还输出多尺度联动的三维可视化表达,为第 6 章协同控制系统中的损伤诊断、恢复优先级排序及资源配置优化模块提供精细化输入。

最后,第 6 章以第 2--5 章形成的 CIM 语义框架、多尺度水动力—风险评估结果、室内外一体化疏散方案与构件级损伤与损失评估结果为综合输入,构建数据层—模型层—业务层分层微服务架构,实现模型组件的服务化封装与业务流程的事件驱动编排。第 6 章的输出包括面向城市洪涝灾害的协同控制系统原型、覆盖预警发布、避险引导、应急调度与灾后评估的协同控制业务流程,以及在典型滨海城市场景下开展的情景演练与效果评估结果。至此,从风险演化机理建模、多尺度物理过程模拟、风险响应决策到协同控制系统集成的技术链条得以闭合,整体体现了论文题目所指向的“城市洪涝灾害的风险演化机理与协同控制系统”的系统化研究路线。

\subsection{论文创新点}
围绕“城市洪涝灾害的风险演化机理与协同控制系统”这一主线,本论文在城市信息建模、多尺度水动力模拟、室内外一体化疏散建模、构件级损伤评估与应急协同系统集成等方面开展系统研究,在统一 CIM 语义框架下打通“风险演化—风险响应—灾损评估—协同控制”的关键环节,主要创新点概括如下:


(1)创新点一:CIM 5D–4V × O–R–E 的洪涝风险演化语义框架。融合 5D–4V 多维语义与 O–R–E 对象—关系—事件结构,统一表达城市实体的几何—属性—时空状态与事件行为;以“降雨—汇流—积涝—暴露—损伤”构建风险演化链条,为异构数据融合、语义互操作与多模型协同计算提供一致语义底座。

(2)创新点二:多尺度三维水动力—风险耦合的机理化精细评估方法。建立三维 RANS—二维浅水分区嵌套模拟,兼顾关键区三维细节与城市尺度汇流;在 CIM 中将水深/流速/历时与暴露度/脆弱性融合映射,生成格网化危险度与综合风险栅格,提升复杂几何条件下的分辨率与物理可信度。

(3)创新点三:风险驱动的室内外一体化疏散与多目标优化框架。在 CIM 统一编码道路网络与建筑内部拓扑,构建室内外贯通的多层广义网络;将动态洪涝场转化为时间依赖的通行约束与风险代价,开展兼顾时间—安全—拥挤的路径优化与人群演化仿真,实现物理过程—风险演化—行为响应耦合。

(4)创新点四:ABV 构件级损伤评估—三维可视化—协同控制联动技术。提出 ABV(属性—建筑—体素)一体化表达,将建筑/基础设施细化到构件尺度并统一建模;耦合洪涝荷载与脆弱性曲线评估构件级损伤及经济/功能损失,联动展示构件损伤与城市风险格网,并接入协同控制流程支撑诊断、恢复排序与资源优化。


\subsection{研究难点与关键技术}
围绕城市洪涝灾害的风险演化机理与协同控制系统构建,研究对象跨越“城市空间—水动力过程—人群行为—构件损伤—业务流程”等多个尺度和层次,涉及多源异构数据融合、多物理场耦合、复杂网络优化与系统集成等一系列问题,整体具有显著的复杂性与交叉性。本研究在统一的 CIM 语义框架下打通“风险演化—风险响应—灾损评估—协同控制”的技术链条,在此过程中面临若干关键难点,需要通过针对性的关键技术设计予以破解,主要体现在以下四个方面。

(1)多源异构数据的统一语义表达与洪涝致灾机理耦合的难点。城市洪涝灾害涉及地形地貌、建筑形态、道路与地下空间、综合管线、人口与资产分布以及监测与遥感数据等多源异构信息,这些数据在空间尺度、几何精度、语义结构与时间分辨率方面存在显著差异。同时,降雨产汇流、水动力演化、暴露与脆弱性等机理模型在参数体系与输入需求上也高度不一致。如何在统一 CIM 框架下,通过语义建模将建筑信息模型、地理信息系统数据以及业务数据有机整合,并构建从对象、关系到事件的多层语义映射,实现多源数据与洪涝致灾机理之间的紧密耦合,是本研究面临的首要难点。拟采用的关键技术包括:CIM 5D--4V 语义建模方法、对象--关系--事件语义结构构建、多源数据本体与映射规则设计,以及面向水动力、疏散与损伤评估的统一语义接口规范等。

(2)三维水动力与城市复杂几何耦合模拟的精度与效率平衡难点。城市建成环境呈现高密度、高立体、多尺度特征,局部关键区域(如低洼片区、重要枢纽、地下出入口)的流场行为与整体城市尺度汇流过程高度耦合。三维雷诺时均 Navier--Stokes 方程具有较高的计算精度,但在大范围城市尺度直接应用会面临网格规模庞大、计算成本高昂与数值稳定性控制困难等问题;二维浅水方程虽具有较高效率,但在建筑物绕流、局部冲刷等场景下又存在精度不足的风险。因此,如何在城市尺度上合理划分三维与二维求解区域,构建稳定可靠的多尺度嵌套计算体系,并通过边界条件传递与参数同化保证不同尺度间的数值一致性,是本研究在洪涝风险演化刻画方面的关键难点之一。拟采用的关键技术包括:城市地形—建筑一体化网格生成、多尺度 3D/2D 水动力模型区域划分与耦合策略、摩阻与湍流模型标定及并行计算加速等。

(3)动态洪涝风险场向室内外疏散网络与人群行为映射的时空耦合难点。洪涝灾害情景下,水深和流速随时间与空间动态变化,直接影响道路与建筑内部空间的通行能力与安全水平。在此背景下,疏散网络的可用性与路径代价具有明显的时间依赖性,传统基于静态路网与固定权重的路径规划方法难以反映真实风险。与此同时,人群在风险感知、拥挤效应与路径选择行为方面具有复杂性与不确定性,如何在网络模型中兼顾安全性、效率与可达性,构建具有物理约束与行为机制支撑的疏散优化模型,成为连接“水动力演化—风险场—人群响应”的关键难点。拟采用的关键技术包括:基于 CIM 的多层广义网络模型构建、动态洪涝场向通行约束与风险代价函数的映射方法、考虑时间依赖与多目标需求的室内外一体化疏散路径优化算法,以及融合拥挤效应的人群演化仿真模型等。

(4)构件级损伤—经济损失—协同控制系统之间多尺度联动的集成难点。城市洪涝灾害的实际影响不仅体现在短期积水风险,还体现在建筑与基础设施构件的长期物理损伤与功能退化。构件级受损状态与城市尺度风险格网、疏散路径选择及应急资源配置之间存在复杂的多尺度关联,传统研究多在宏观尺度进行损失估算,难以支撑精细化的恢复优先级排序与资源优化调度。如何构建能够细致刻画构件级损伤的属性--建筑--体素一体化表达模型,并将损伤评估结果通过 CIM 平台有效上卷至建筑与城市层面,与协同控制系统的数据层、模型层与业务层实现闭环联动,是促使“风险评估—损伤诊断—协同控制”一体化的综合难点。拟采用的关键技术包括:ABV 构件表达与脆弱性曲线标定、洪涝荷载向构件损伤与功能退化的映射方法、损伤结果向经济损失与服务性能损失的多尺度集成模型,以及基于分层微服务架构的模型组件编排与三维可视化联动技术等。


\section{章节安排}
本论文围绕“城市洪涝灾害的风险演化机理与协同控制系统研究”这一主线,按照“理论基础—模型构建—系统集成—应用验证”的逻辑展开,共分为七个章节,各章节内容安排如下:

第 1 章介绍研究背景与意义,分析城市洪涝灾害的系统性特征与现有研究的主要不足,提出本研究的总体目标、研究内容、技术路线、创新点与关键难点,并阐述全文的整体结构安排。

第 2 章面向城市洪涝灾害的应用需求,构建基于 CIM 5D--4V 与对象--关系--事件的语义建模框架,统一表达城市实体、空间拓扑、状态演化与事件触发关系,并系统分析“降雨—汇流—积涝—暴露—损伤”的风险演化机理,为后续模型构建提供语义与机理基础。

第 3 章以第 2 章的语义框架为基础,构建三维雷诺时均 Navier--Stokes 方程与二维浅水方程嵌套的多尺度水动力模拟体系,开展城市洪涝危险度与综合风险栅格的计算,实现洪涝风险的高精度时空刻画,为疏散规划与构件损伤评估提供风险场输入。

第 4 章在 CIM 平台中统一编码城市道路网络与建筑内部拓扑,构建多层广义网络模型,并将第 3 章的动态洪涝场映射为时间依赖的通行约束与风险代价函数,提出风险驱动的室内外一体化疏散路径优化方法,评估不同疏散策略的安全性与效率。

第 5 章基于第 2 章提出的 ABV 构件表达模型和第 3 章的水动力场输出,构建洪涝荷载与构件材料、构造之间的损伤关系,完成构件级物理损伤、经济损失与服务性能损失的评估,并实现构件级与城市尺度之间的三维可视化联动。

第 6 章将第 2--5 章的关键模型与数据成果集成到分层微服务架构中,构建城市洪涝灾害协同控制系统,实现预警发布、避险引导、资源调度与灾后评估等业务流程的协同运行,并在典型滨海城市场景中进行系统验证和效果评估。

第 7 章对全篇研究工作进行系统总结,围绕论文主线提炼约六条具有代表性的研究结论,并分别指出其对应章节与关键内容,形成从语义建模、多尺度水动力模拟、风险评估、疏散优化到构件级损伤评估与协同控制系统构建的完整知识链条。同时,本章还从整体上总结论文的主要研究创新点,明确其在城市洪涝灾害风险演化机理刻画与协同控制体系构建方面的理论贡献与工程价值。最后,本章结合研究中尚待突破的技术瓶颈与趋势,提出未来可在跨灾种耦合、实时数据同化、多主体行为建模与智能决策技术等方向展开的进一步研究展望。

以上章节共同形成从语义建模、物理过程模拟、风险响应到协同控制的完整技术链条,实现对城市洪涝灾害风险演化机理的系统刻画与协同应对能力的集成提升。