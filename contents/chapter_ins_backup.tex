\chapter{基于CIM的城市防灾应急协同系统构建}

\section{系统总体设计与功能框架}

\subsection{研究定位与总体目标}

本章旨在将前述高精度洪涝风险评估、MGNM室内外一体化疏散以及构件级损伤评估与三维可视化等关键研究成果融入统一的CIM(City Information Modeling)平台,构建一个面向城市防灾应急的协同框架。该框架以系统工程的视角为基础,通过实现数据、模型与业务的深度融合,形成可支撑灾前、灾中与灾后全过程管理的综合体系。其总体目标在于在统一的CIM语义与空间坐标体系下,实现多源数据的动态汇聚、多模型协同求解、智能决策生成与现场反馈的闭环运行,确保防灾业务的时效性与自适应性。

系统的设计理念强调分层解耦与职责清晰。数据层、模型层与业务层在功能上各司其职、层次分明,通过标准化的接口实现模块间的松耦合与高可扩展性。所有数据对象均依托CIM语义进行统一建模,确保建筑、道路、管网及地形等多源空间要素在时空维度上的一致性。系统支持模块的可插拔运行与计算编排,不同模型可根据场景需要进行自由组合与调用。传感器、遥测及人机交互端的动态数据被持续用于修正模型参数和路径代价函数,使系统具备持续学习与自我优化的能力。为确保工程可验证性,系统在设计阶段引入了可观测指标体系,既包括路径重算时间、系统响应时延、渲染帧率等性能指标,也包括预测精度、疏散效率及资源到位率等业务指标,从而为后续的系统验证提供量化基础。

\subsection{三层体系框架与模块边界}

系统总体结构采用“三层—四域”的架构形式,即数据支撑层、模型计算层和业务应用层,贯穿风险感知、应急决策、资源调度与可视化展示四个功能域。其总体组织关系可见于图~\ref{fig:System_Structure}所示。

\begin{figure}[htbp]
\centering
\includegraphics[width=0.8\textwidth]{PIC-ins/1.png}
\caption{系统总体结构}
\label{fig:System_Structure}
\end{figure}

数据支撑层以CIM核心数据库为中心,整合BIM、GIS、DEM、气象、水文、遥感及IoT传感器等多源数据,建立统一的语义映射与空间基准,确保地形、水系、建筑与管网等要素的对应关系。模型计算层封装了关键算法模块,涵盖三维水动力学仿真、MGNM室内外路径规划、构件级损伤评估与风险分级预警,实现跨尺度的协同计算。业务应用层则以城市防灾任务为导向,提供态势监测、风险分析、决策推演、资源调度与三维可视化等综合功能,为应急管理提供一体化支撑。

\subsection{系统功能构成与职责划分}

围绕“灾前预警—灾中响应—灾后评估”的完整防灾流程,系统功能被划分为四个互联的子域。风险感知子域依托多源数据与三维水动力学模型,生成洪涝深度、流速及影响区域的时空分布,并结合分级标准形成动态风险图。应急决策子域以MGNM路径引擎为核心,综合风险场结果构建风险感知代价函数,生成多主体、多尺度的疏散与交通组织方案。资源调度子域在道路约束及需求预测的条件下,实现车辆、人员与装备的时空优化分配,与路径计算结果联动完成滚动调度。可视化展示子域则在三维CIM环境中实时呈现风险态势、路径规划、资源分布及构件级损伤信息,为指挥中心、现场终端和公众平台提供分层可交互的可视化服务。

\subsection{运行逻辑与信息流闭环}

系统运行遵循“感知—求解—决策—反馈”的闭环逻辑,整体信息流关系见图~\ref{fig:System_info}。首先,系统从CIM核心库加载城市的静态空间基底(包括建筑、道路、地形、管网等),并持续接入气象、水文、遥感和IoT设备的实时数据流。随后,水动力学模型生成动态风险场,为MGNM路径规划提供代价输入;构件级模型同时对重点设施进行受灾评估,形成多尺度的风险信息集合。决策模块依据模型输出结果,为不同主体(如指挥中心、现场终端与公众平台)生成可执行的疏散和封控方案。最后,现场位置回传、道路通行状态及设施运行状况等实时反馈会触发模型参数与路径代价的动态修正,实现自适应的闭环控制。

\begin{figure}[htbp]
\centering
\includegraphics[width=0.8\textwidth]{PIC-ins/2.png}
\caption{系统整体信息流关系}
\label{fig:System_info}
\end{figure}

\subsection{关键接口与数据契约}

为实现跨层级、跨模型的高效协同,系统在概要层设计了统一的接口与数据契约。所有模块共享统一的城市级空间坐标体系(如CGCS或UTM)和时间标准(UTC+偏移),并依据CIM核心对象Schema进行描述。核心对象包括建筑、道路、水动力网格及传感器等基础实体,每类对象均具备标准化的几何、属性与状态字段。

模型间的数据交互遵循严格的输入输出约定:水动力学模型以地形、糙率、边界及降雨条件为输入,输出包含深度与速度的风险场;MGNM模型以空间网络、风险代价及约束条件为输入,输出多主体路径及耗时指标;构件评估模型则接收水位、持续时间与材料信息,计算对应的损伤指数与结构状态。系统内部的事件通信通过主题机制实现,如 /risk/update、/route/recompute、/dispatch/assign 等主题消息,用于不同模块间的异步触发与数据流转。

\subsection{系统层可观测指标}

为了量化系统的整体性能与业务效果,本研究在概要层定义了两类可观测指标。工程性能指标主要用于评估系统运行效率,包括路径重算时延、资源调度收敛时间、三维场景渲染帧率以及数据传输吞吐量等。业务成效指标则关注模型与业务结果之间的关联性,涵盖风险预测误差、疏散到达时间分布、关键设施损伤判定准确率以及资源到位率与平均响应时间等方面。这些指标在后续章节的系统验证中将作为评估依据,用以衡量系统在不同灾害场景下的实际表现与工程可行性。

\section{系统架构与数据模型集成设计}

\subsection{架构设计思路与实现原则}

在前述总体框架的基础上,本节进一步阐述系统的实现性架构与数据—模型融合机制。整个系统以分布式微服务架构为核心,通过CIM平台实现数据、模型与业务的可插拔、可扩展集成。该架构在保持模块自治的前提下,通过统一的消息通信和数据访问协议实现全局协同,从而兼顾计算性能与灵活扩展性。

系统的架构设计以解耦与模块化为核心理念。模型计算、数据处理、可视化渲染等各组件均以独立微服务形式运行,并通过标准化接口完成通信。不同模块之间不存在直接依赖,而是借助消息总线和接口标准实现松耦合交互。系统的事件处理机制采用事件驱动与异步通信的模式,当风险场更新、路径重算或调度命令触发时,相关事件以主题(Topic)的形式传递至订阅模块,实现高并发场景下的实时响应。

在数据管理方面,系统设计了统一的数据访问层,所有模型计算均通过该层与CIM数据库交互。数据访问层为模型提供一致的空间语义和坐标基准,使得各类算法无需感知底层数据结构即可进行计算与结果回写。为保证高可靠性,系统支持容错与高可用机制,核心模块具备状态恢复与故障转移功能,即使在高负载或突发断链环境下,也能维持持续运行。系统的运行结构与各模块间的交互关系可见于图~\ref{fig:System_topo},该图展示了在实际部署与运行过程中,数据访问层、模型算子、消息总线与应用端之间的通信路径及信息流转方式。


\begin{figure}[htbp]
\centering
\includegraphics[width=0.8\textwidth]{PIC-ins/3.png}
\caption{系统交互关系拓扑}
\label{fig:System_topo}
\end{figure}

\subsection{数据与模型的一体化集成机制}

系统在CIM平台的支撑下实现了数据与模型的深度融合,其核心机制包括统一的数据服务体系、模型算子化封装以及数据到模型的动态映射。

首先,统一的数据服务体系通过分布式数据库结构实现多源信息的集中管理。系统在城市语义模型(City Ontology)的指导下,对BIM、GIS、DEM等静态数据和气象、水文、遥感、传感器等动态数据进行整合。数据通过数据中台汇聚后,形成一个可统一访问的“数据服务层”。该层负责数据注册、版本管理、坐标与时间对齐、索引缓存以及访问权限控制等功能,为上层模型提供高效稳定的数据支撑。所有模型计算均通过该层调用数据,实现了模型与数据的分离,使模型能够在无感知底层结构的情况下完成统一访问与计算。

其次,模型集成机制通过算子(Operator)封装实现模型的可插拔部署。每个模型模块被封装为独立的算子服务,具备明确的输入、输出、运行条件与依赖定义。这些算子注册在任务调度系统中,当相关事件被触发时,调度器会自动调用相应算子完成计算。系统的主要算子包括用于生成洪涝风险场与流速场的水动力学算子、负责室内外路径规划的MGNM算子、进行构件级损伤评估的结构算子以及基于路径和需求预测的资源调度算子。算子之间通过数据引用与消息队列进行交互,彼此之间不存在直接调用关系,构成一个松耦合、可组合的模型运行环境。

最后,数据与模型的映射过程在运行时按照既定流程执行。系统首先在数据准备阶段完成多源数据的同步、清洗与标准化,形成统一的数据包;随后,在模型调用阶段,各算子模块通过接口访问数据服务层并加载运行参数进行计算;模型输出的结果在结果发布阶段通过API网关或消息队列返回数据库或传递至下游模块。该机制确保了各模型在串行或并行运行时,能在语义、时空与数据一致性上保持统一,为多模型协同计算奠定了坚实基础。

\subsection{多源信息融合与接口实现}

在多源信息融合方面,系统以CIM平台为核心枢纽,采用“集中注册—异步订阅—动态更新”的模式实现信息的实时集成。各类数据源首先在注册中心登记元数据及访问方式,模型模块以订阅的方式接收所需主题,当数据发生变化时,系统会自动触发事件通知,实现多模块间的数据联动与实时更新。这种机制不仅保证了风险感知、路径规划和调度优化的时效性,也使系统能够根据场景动态调整数据流的优先级和传输策略。

系统接口的设计注重标准化与互操作性,整体分为内部接口、外部接口与管理接口三类。内部接口用于模型算子之间的通信,依托消息总线(如MQ或Kafka)实现事件与计算结果的异步传递;外部接口主要面向业务应用层,通过RESTful API或WebSocket提供可视化平台与外部系统的交互;管理接口则承担系统监控、日志收集与状态维护等职能。所有接口统一采用JSON或GeoJSON数据格式,支持空间索引与时间戳标识,确保系统在多平台间的兼容与可追溯性。

数据安全与运行监测是系统设计的重要保障。系统通过身份认证与访问控制机制限制不同角色的数据使用范围,对位置、轨迹等敏感信息在公开展示前进行脱敏处理。每次接口调用均自动记录访问日志、执行时长与响应状态,用于后续的性能分析与溯源诊断。通过这一系列安全与监控机制,系统不仅保证了运行的可靠性与数据的保密性,也为后续的扩展与维护提供了坚实的工程基础。

\subsection{关键技术实现与性能优化}

为保证系统在城市级场景下的实时性与稳定性,本研究在架构设计的基础上进一步实现了一系列关键工程技术与性能优化策略。总体上,系统采用“容器化部署+消息驱动+前端增量渲染”的技术路线,在保证模块自治与可扩展性的同时,兼顾高并发场景下的数据吞吐与图形渲染性能。

在部署层面,模型计算服务、数据访问服务与可视化服务均采用Docker容器进行封装,并通过容器编排平台实现统一管理。各微服务通过服务注册与发现机制接入网关,网关基于Nginx实现请求的负载均衡与反向代理,确保在洪涝演化与路径重算高频触发的情况下,模型计算资源能够按需弹性扩展。针对水动力学仿真与MGNM路径规划等计算密集型任务,系统将核心求解模块部署于GPU计算节点,通过CUDA加速矩阵运算与图搜索过程,相比传统CPU实现平均计算时间缩短约35\%。

在数据流转方面,系统依托Kafka搭建统一的消息总线,采用“主题细分+分区并行”的策略管理不同类型的数据流。洪涝风险场更新、路径重算请求与调度指令分别映射到 /risk、/route 与 /dispatch 等主题,每个主题根据城市区域或业务类型进一步划分为多个分区,由独立的消费者组进行订阅与处理。为降低数据库压力,系统在消息总线与数据库之间引入Redis缓存,对高频访问的风险切片、路径片段与任务状态进行短期缓存,典型缓存时长为30~60秒,从而避免在短时间内重复访问同一批数据。

在三维可视化前端,实现了多级瓦片切片与层次细节(LOD)相结合的场景管理策略。底层地形与建筑模型预先切分为多级瓦片,并在Cesium中按需加载;摄像机视锥以外的瓦片不参与渲染,视锥边缘区域采用低分辨率模型进行过渡,以降低GPU负载。动态风险场与路径结果则通过独立的动态图层叠加实现,仅在新一期模型结果生成或用户操作触发时进行增量更新,而不对整个场景进行重绘。对于高频交互操作(如平移、缩放与时间轴拖动),系统采用前后端协同的节流策略,将请求频率控制在5~10次/秒范围内,在保证交互流畅的前提下减少不必要的网络请求与模型调用。

此外,系统在运行监控与故障恢复方面也进行了细致的工程设计。所有微服务均集成统一的日志采集与指标上报模块,关键指标包括CPU/GPU利用率、Kafka消息堆积量、接口调用耗时以及三维渲染帧率等;当指标超过设定阈值时,监控模块会自动触发告警并记录对应时间段的详细运行日志。核心服务支持状态快照与自动重启策略,当单个算子服务异常退出时,编排平台会在数秒内完成实例重建与状态恢复,避免城市级应急任务因局部故障而中断。通过上述关键技术实现与性能优化,系统在复杂工况下仍能保持洪涝更新、路径重算与三维可视化的整体稳定运行。

\section{应急响应与协同决策机制设计}

\subsection{动态风险驱动的预警与响应}

本节构建了一个以动态风险为核心驱动的预警与应急响应机制,实现从灾情监测到决策执行的全流程闭环。系统通过对气象、水文及传感器数据的实时监控,将外部环境变化以数据流的形式持续注入模型计算层。当监测参数触及预设阈值时,系统自动启动应急响应流程。整个逻辑流程包括风险识别、等级预警、动态推演及应急联动四个阶段,具体信息流可参见图~\ref{fig:System_emer}。


\begin{figure}[htbp]
\centering
\includegraphics[width=0.8\textwidth]{PIC-ins/4.png}
\caption{系统应急响应流程}
\label{fig:System_emer}
\end{figure}

在风险识别阶段,水动力学模型根据实时输入数据生成洪涝范围、深度与演化趋势,并将结果输出为动态风险场。系统依据风险等级标准对模拟结果进行分级,当风险等级达到轻微、严重或极端等不同阈值时,自动触发相应的预警事件。随后进入动态推演阶段,模型快速执行预测与风险场更新,以形成新的代价函数与风险区划,为后续路径计算与资源调度提供输入支撑。最终,系统将预警信息同步推送至业务层和外部管理平台,驱动路径重算与资源调度模块的协同执行。

该机制采用事件驱动模式运行:当风险场更新完成后,系统发布 /risk/update 事件,调度器捕获该事件后立即启动路径重算与资源调度。若局部区域风险上升,系统会针对该区域执行局部代价调整与路径重算;若风险扩展至全局范围,则启动全域重规划;若检测到设施失效或交通封闭,则自动修改路网结构并进行避障重构。通过这一机制,系统能在数秒内完成从风险识别到路径输出的响应,实现突发灾情下的快速动态适应。

根据城市防灾体系的分级要求,系统设置了三级响应模式。轻微风险阶段主要发布预警信息,以指导公众和管理部门进行风险关注;严重阶段在洪涝深度超过0.5米且持续时间超过30分钟时,自动触发疏散路径规划与交通调整;极端阶段则在关键设施受淹或通信中断时进入应急调度模式,系统自动切换备用通信链路以维持决策链的连续性。通过分级响应机制,系统实现了“风险监测—模型求解—策略执行”的闭环联动,构建了具备自适应与韧性的城市应急反应体系。

\subsection{基于MGNM的多主体路径规划}

在应急响应过程中,MGNM(Multi-purpose Geometric Network Model)模型承担了系统的核心路径求解任务,是连接风险场与资源调度的关键环节。MGNM模型通过融合室内BIM语义与室外道路拓扑,实现了室内外网络的一体化表达,并以风险感知代价函数为驱动,在不同空间尺度下实现多主体路径规划。

系统首先完成室外道路网络的拓扑构建,利用城市GIS数据生成节点与边集合;随后提取建筑BIM信息,识别楼层、走廊、出口等关键空间要素,并映射为室内节点。通过建筑出入口、地铁通道、廊桥等关键连接点,系统建立了室内外网络的跨域映射,从而实现了全域连通性。动态更新机制使模型能够实时响应外部变化:当水动力学模型检测到道路封闭或积水深度超过阈值时,相应的边会被自动禁用,触发路径重算过程。网络生成与更新过程的逻辑结构见图~\ref{fig:Network}。


\begin{figure}[htbp]
\centering
\includegraphics[width=0.8\textwidth]{PIC-ins/5.png}
\caption{网络生成与更新过程}
\label{fig:Network}
\end{figure}

MGNM路径规划采用“多主体任务协同与分层寻优”的策略,将疏散与救援的参与对象划分为居民、应急车辆与救援人员三类。针对不同主体,系统定义了差异化的目标函数,例如居民路径以安全优先,应急车辆以时间最短为目标,而救援人员则侧重资源覆盖与通达性。路径计算在GPU端并行执行,极大提升了求解效率。为避免路径冲突,系统在结果整合阶段引入路径重权重算法,对道路容量进行动态分配与通行优先级调整,从而确保多主体同时疏散与救援的协调有序。

规划完成后,路径结果以GeoJSON格式存储,并附带风险等级、预计耗时与安全评分等属性。结果数据既用于三维可视化模块进行动态展示,也作为资源调度模块的输入,用于后续任务分配与行动指令生成。MGNM模块的引入有效弥合了室内外空间割裂的问题,为应急响应提供了统一的空间分析与决策基础。

\subsection{协同决策与任务调度逻辑}

系统的协同决策机制旨在整合风险预测、路径规划与资源调度三个核心模块,形成可直接执行的应急指挥策略。其运行流程遵循“事件触发—策略生成—任务调度—反馈修正”的循环逻辑,如图~\ref{fig:Decision}所示。每当风险场或路径信息发生更新,系统自动触发决策生成流程,由策略生成器结合资源状态与任务优先级生成多方案策略集,并依据成本、时间与风险权重等指标选取最优方案。选定策略后,系统依据地理位置、道路通行条件与资源可达性生成详细的执行计划。


\begin{figure}[htbp]
\centering
\includegraphics[width=0.8\textwidth]{PIC-ins/6.png}
\caption{决策闭环图}
\label{fig:Decision}
\end{figure}

任务调度逻辑采用分层控制机制。全局调度层负责跨区域资源协调与任务分配,确保关键区域优先响应;局部调度层在区域范围内执行路径分配与人员指派;反馈更新层实时接收现场回传的状态信息,包括任务进度、设备状态与交通延误情况,并在必要时触发任务重分配。系统通过任务队列与优先级调度器管理不同任务的执行次序,实现对多任务场景的动态调度。当同一区域内同时存在疏散与物资运输等任务时,系统会根据优先级表自动调整顺序,避免执行冲突。

人机协同是系统决策的重要特征。指挥中心界面允许用户在自动生成方案的基础上进行人工修订与确认,系统将每次人工干预的操作记录及结果变化存入日志,用于后续的性能评估与模型优化。界面采用语义化可视化设计,通过地图、时间轴与警示图层展示决策内容,使决策者能够快速理解态势并下达指令。

此外,系统引入了反馈学习机制,以实现自适应的持续优化。灾后阶段,系统会自动收集运行日志、传感器数据与任务执行记录,对决策偏差进行统计分析。根据偏差结果,系统动态调整风险分级阈值、路径权重及资源调度规则,在下一次任务运行时加载更新参数,实现知识迁移与自学习。通过这一机制,系统应急反应能力在多次运行中不断增强,形成了从被动响应到主动优化的智能进化体系。

\section{三维可视化与系统实现验证}

\subsection{三维可视化界面与交互逻辑}

在CIM框架的支撑下,本研究构建了一个集信息展示、态势认知与决策交互于一体的三维可视化模块。该模块不仅用于可视化洪涝风险与疏散路径的动态演化过程,同时也是应急指挥、人机协同与数据反馈的核心接口。通过统一的城市信息模型,系统能够将建筑构件、道路网络、地形地貌以及风险场景在同一空间环境中直观呈现,从而实现从数据计算到决策认知的直接映射。

系统的设计理念强调多尺度融合与一体化表达。可视化模块以三维CIM场景为载体,实现城市级与建筑级空间的连续切换,使风险分布、路径规划与资源调度的结果在同一视图中协同展示。渲染架构采用分层机制:底层空间层负责加载地形、道路、管网及建筑等基础几何模型,确保空间精度与坐标统一;动态风险层实时渲染水动力学模型输出的洪涝深度、流速与风险等级,利用颜色梯度与透明度表达不同风险区的演化;路径与资源层叠加MGNM的路径结果与应急资源位置,通过流向动画直观展示人员疏散与车辆运行状态;交互控制层则提供多视角漫游、信息查询与时间轴回放功能,使用户能够在空间和时间两个维度上探索城市态势的演变。整体界面逻辑结构如图~\ref{fig:Platform}所示。


\begin{figure}[htbp]
\centering
\includegraphics[width=0.8\textwidth]{PIC-ins/7.png}
\caption{整体界面逻辑结构}
\label{fig:Platform}
\end{figure}

为进一步展示本研究所基于CIM平台实现的三维可视化能力,本节选取系统典型场景界面作为示例,用于说明三维场景呈现、图层叠加、指标动态更新及交互操作方式等关键功能。图~\ref{fig:Platform-show}与图~\ref{fig:Platform-show-three}展示了系统在真实城市空间中的可视化渲染效果,有助于理解系统在应急指挥中的信息整合与态势表达方式。


\begin{figure}[htbp]
\centering
\includegraphics[width=0.8\textwidth]{PIC-ins/8.png}
\caption{环境感知与监测数据叠加的三维场景展示界面示意}
\label{fig:Platform-show}
\end{figure}

图~\ref{fig:Platform-show}进一步展示了系统在同一三维城市场景中叠加环境监测数据的能力。左侧面板以折线图、浓度指示条等形式呈现PM2.5、PM10、气压、风速等环境要素,并支持时间序列回放与周期趋势分析。三维场景中各建筑物上悬浮的图标代表不同监测节点,其颜色和数值随传感器实时数据动态更新,从而实现对城市环境状态的空间化表达。该界面展示了系统对多源数据的可视化融合能力,可用于应急响应中对危险区域环境变化的识别与预估,提升对灾害演化的总体把握。


\begin{figure}[htbp]
\centering
\includegraphics[width=0.8\textwidth]{PIC-ins/9.png}
\caption{城市三维场景下的人口态势可视化界面示意}
\label{fig:Platform-show-three}
\end{figure}

图~\ref{fig:Platform-show-three}展示了系统在三维城市空间中的人口与人员态势可视化界面。界面左侧为多维指标面板,包括人口规模、分布结构、年龄构成、重点区域人员密度等指标,采用动态柱状图与数值指示形式对实时状态进行更新。右侧为基于CIM的三维城市模型,其中建筑、道路、水体与地形均以分层方式进行渲染,并支持动态图层的开启与关闭。通过在建筑表面叠加人口密度或人员聚集程度等信息,系统能够实现城市区域内人员态势的宏观展示和局部热点识别,为应急指挥中的人群疏散与风险研判提供重要决策依据。

该系统基于多线程渲染与瓦片缓存技术开发,可在普通GPU平台上实现秒级场景加载与30帧/秒的流畅渲染效果。Shader程序用于动态控制风险场动画,使洪涝演化过程能够以连续图层的形式展示,增强风险传播的直观感受。在交互设计上,系统根据用户角色实现差异化逻辑:指挥中心端以城市级三维总览为核心,展示风险区域、资源分布与任务进度;现场端以移动终端为载体,聚焦任务接收、路径导航与设施上报,支持语音与文本输入;公众端提供简化的避险路线查询与信息发布功能,面向居民实现风险可视化的社会化传播。三端之间通过统一的消息中间件互联,实现“命令下达—执行反馈—状态更新”的实时闭环。可视化界面还支持时间轴控制和图层切换,使用户能够在不同尺度下分析灾情演变与应急响应过程,充分体现系统的动态性与交互性。

\subsection{威海滨海应急服务中心系统部署}

为验证所构建系统的可行性与工程适应性,本研究在威海滨海应急服务中心进行了部署与实际运行测试。部署环境由服务器集群、数据库系统、通信接口和前端渲染平台组成,其整体架构关系如图~\ref{fig:Weihai}所示。系统运行环境包括两台计算节点(每台配备32核CPU、RTX A6000 GPU与128 GB内存)、一台数据库服务器和一台前端渲染服务器。数据库层采用PostgreSQL/PostGIS管理CIM核心数据,MongoDB用于存储模型中间结果与运行日志。模块之间的通信依托Kafka消息队列实现事件传递,外部接口则通过RESTful API实现跨系统访问。前端界面基于Cesium与Vue框架开发,可直接与应急指挥中心的大屏系统集成,实现三维空间数据的实时联动与可视化调度。

\begin{figure}[htbp]
\centering
\includegraphics[width=0.8\textwidth]{PIC-ins/10.png}
\caption{威海滨海应急服务中心部署架构}
\label{fig:Weihai}
\end{figure}

实验场景选取威海滨海区域典型的低洼易涝地带,区域范围约3.2平方公里,包含128栋建筑、24条主干道路与3处关键应急设施。系统以实时气象数据和地面传感器信息为输入,结合高精度DEM模型完成洪涝仿真。模拟参数设置为1分钟时间步长、2米空间分辨率,MGNM路径网络包含约1.8万个节点与2.15万个边。运行过程中,系统以“强降雨触发—路径更新—资源调度—疏散指挥”为主线,全面验证了多模型协同、实时响应及信息可视化的工作流程。

在系统运行中,风险图层可在三维场景中实现动态更新,洪涝深度与流速随时间变化实时渲染;路径规划结果在界面上以动态箭头和颜色变化的方式叠加展示,受灾道路自动变灰或禁用;调度命令经消息队列实时下发至现场终端,终端设备反馈任务执行状态;整个交互过程在毫秒级响应延迟下保持流畅,三维渲染帧率稳定在30帧以上,能够满足城市级应急决策的实时可视化需求。

为展示系统在应急服务中心的实际部署效果,本研究选取指挥大厅大屏端的典型界面作为示例,展示了系统在威海滨海应急服务中心指挥大厅中的大屏展示效果。界面采用多区块拼接布局,左侧为实时更新的多维数据面板,包括气象要素、环境监测指标、作业状态与人员信息等;中部为三维地理空间场景,基于CIM模型实现区域环境、基础设施与监测设备的可视化表达;右侧则集成了监控视频流、风险预警信息以及关键指标的趋势曲线。通过三维场景与数据面板的联动,大屏能够支持态势总览、事件定位、信息查询以及辅助决策,为指挥调度提供直观、统一的信息展示窗口,如图~\ref{fig:show}所示。


\begin{figure}[htbp]
\centering
\includegraphics[width=0.8\textwidth]{PIC-ins/11.png}
\caption{威海滨海应急服务中心指挥大厅大屏展示界面示意}
\label{fig:show}
\end{figure}

\subsection{验证指标与结果分析}

系统部署完成后,对运行性能、模型协同与系统稳定性进行了定量验证,结果显示整体框架具备良好的工程可用性与稳定性。根据测试日志与监测数据,系统在洪涝场更新周期、路径重算时间、任务下发延迟和渲染帧率等方面均表现稳定。洪涝风险场的平均更新周期约为32秒,峰值45秒;MGNM路径重算时间平均7.6秒,峰值11秒;指挥端至现场端的调度延迟保持在1秒以内,三维渲染帧率维持在31帧左右。与传统静态路径规划相比,系统在疏散路径安全性上提升约19\%,资源到位时间缩短约23\%,显著提高了整体应急效率。

模型协同验证表明,系统能够在洪涝扩展后的两到三分钟内自动完成路径重算与调度更新,整个风险感知、路径规划与任务调度的链路延迟控制在45秒以内。多模型之间的数据传递稳定可靠,接口通信未出现丢包或延迟累积,验证了模型集成机制的鲁棒性与并行协同能力。系统连续运行48小时未出现宕机或通信阻塞,表明其架构在高并发与长周期条件下具备良好的稳定性。模型算子可按需扩展或替换,为未来引入暴雨—内涝—风暴潮等多灾种联合模拟提供了技术基础。可视化界面支持模块化扩展,用户可通过插件加载不同数据主题,实现多场景自定义配置。

综合分析可见,本研究提出的CIM驱动城市防灾应急协同框架在时效性、协同性和直观性方面均具有突出优势。系统能够在分钟级时间内完成数据更新与路径重算,实现模型间的同步耦合与任务协同,并以三维可视化方式提供清晰直观的决策支持。威海滨海应急服务中心的应用验证充分证明了该框架在实际防灾工程中的可行性与应用价值,也表明CIM平台在应急管理领域具备良好的集成潜力与推广前景。

\section{本章小结}

本章系统地完成了基于CIM的城市防灾应急协同框架的构建与实现,形成了从体系设计到系统验证的完整闭环。通过总体设计与架构实现,论文确立了以CIM为核心的数据、模型与业务一体化体系,构建了可扩展的微服务架构和事件驱动机制;结合动态风险驱动与MGNM多主体路径模型,提出了自适应的协同决策机制,实现了风险识别、路径规划与资源调度的智能联动;通过威海滨海应急服务中心的部署与验证,系统实现了三维动态可视化与实时响应功能,验证了所提框架在防灾场景中的可行性与稳定性。综上,本研究实现了从理论模型到工程应用的转化,为CIM环境下的城市应急管理提供了统一的技术体系与实现路径。
