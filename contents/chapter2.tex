\chapter{基于CIM的高精度风险评估方法}

\section{引言}

随着全球气候变化的加剧和城市化进程的快速发展,极端天气事件导致的洪涝灾害频发,对城市安全构成严重威胁。据联合国灾害数据库(EM-DAT)统计,洪涝灾害占所有自然灾害的41\%,影响人口占49\%,造成的经济损失占22\%\cite{Douben2006, Cavallo2013}。在沿海城市中,风暴潮引发的洪涝灾害尤其严重,特别是当其与天文大潮重合时,会导致海堤溃决和严重的城市内涝\cite{Botzen2013}。因此,建立精确的城市洪涝风险评估体系对于防灾减灾具有重要意义。国内基于大数据的动态评估框架进一步验证了多源信息驱动城市洪涝风险预测的可行性\cite{CN_Fu2022FloodRisk}。

传统的洪涝模拟多基于二维(2D)水动力模型\cite{Brown2007, Fewtrell2008, Gallegos2009, Schubert2012, Gallien2014},虽然在许多应用中表现良好,但存在明显的局限性。二维模型基于浅水方程(SWEs),假设垂直方向的加速度可忽略且压力为静水压分布,这些假设在地形复杂的城市环境中往往不成立\cite{Ramirez2016, Mignot2006, Nazari2018}。特别是在建筑物密集、地形起伏较大的城市区域,二维模型难以准确描述复杂的三维流动特征和垂向涡旋结构\cite{Chen2018, Wang2014}。

城市信息建模(CIM)技术作为整合建筑信息模型(BIM)、地理信息系统(GIS)、物联网(IoT)和人工智能(AI)的综合平台,为城市洪涝灾害的精确模拟和风险评估提供了新的技术路径。基于数字航空摄影测量和倾斜摄影技术生成的高精度三维城市模型,能够提供米级乃至亚米级的地形数据,为三维水动力模拟奠定基础\cite{Toschi2017, Goetz2013}。

本章提出一种基于CIM的高精度洪涝风险评估方法,通过整合数字航空摄影测量、BIM和GIS技术,构建三维水动力模型,实现对城市洪涝灾害的精确模拟和脆弱性评估。该方法不仅考虑了地表径流的复杂三维特征,还能模拟建筑物内部的洪水演进过程,为城市防洪减灾决策提供科学依据,并为后续的室内外疏散模型与构件级损伤评估提供统一的洪水参数输入。为确保内容的完整性和可读性,本章依次介绍数据采集与模型构建、指标体系与参数提取、风险图生成以及案例验证的流程,并在末尾通过数值稳定性分析保障模型输出的可信度。

\section{数据获取与建模基础}

为支撑后续的指标构建与风险评估,本节首先说明研究区域与基础数据,再阐述用于生成高精度三维模型的处理流程,并解释这些成果如何进入水动力模拟。

研究区域以威海市环翠区滨海应急服务中心为核心,覆盖$1.5\,\mathrm{km}\times1.5\,\mathrm{km}$范围。区域北侧紧邻环翠湾海堤,南侧为世昌大道等城市主干路,整体地势由北向南缓降,地面高程介于$3.2$--$9.6\,\mathrm{m}$。依据《威海市气象年鉴(2013--2022)》与《威海统计年鉴2023》\cite{WeihaiStats2023}以及《威海沿岸风暴潮观测年报》\cite{WeihaiMNR2020},年平均降水量为$684\,\mathrm{mm}$,最大日降水达$198\,\mathrm{mm}$;近十年出现4次严重风暴潮,其中2021年“烟花”台风造成滨海低洼地带积水深度超过$0.6\,\mathrm{m}$。区域内布设环翠湾、世昌大道两处雨量站及威海港潮位站,为模型提供小时级水文序列。滨海中心北侧存在下凹式停车区和地下设备层,在风暴潮与暴雨叠加时易形成积水,是重点风险点。 借鉴智慧港口与交通耦合仿真成果,可将风暴潮致灾过程与交通运行影响纳入联合分析框架\cite{CN_Gao2022SmartPort,CN_Guo2022WaterloggingSim,CN_Pan2023TrafficControl}。与此同时,沿海城市排水系统在台风暴雨事件下表现出的非线性响应特征提示需要重点关注障碍物诱发的局地积水风险\cite{CN_Liang2023StormResponse,CN_Kong2024BarrierForecast}。多源观测和CIM数据更新机制为暴雨致灾链的实时监测提供了基础,使得风险评估能够快速迭代\cite{CN_Liu2023Rainstorm,CN_Huang2024MultiSource,CN_Liang2024SensorNet}。

\subsection{多源数据获取与三维场景构建}

数字航空摄影测量技术是获取高精度三维城市模型的关键手段。通过无人机(UAV)搭载的多镜头系统捕获不同方向的倾斜影像,能够生成包含详细地表特征的数字城市模型\cite{Chen2009, Murtiyoso2014}。相比传统的激光雷达(LiDAR)技术,数字航空摄影测量在分辨率和成本效益方面具有显著优势,特别适用于城市建成区的高精度建模\cite{Neelz2006, Webster2004}。

本研究采用DJI S1000无人机搭载5个SONY ILCE-5100相机镜头,执行覆盖面积为$1.5\text{km} \times 1.5\text{km}$的飞行任务。考虑到海岸线长和建筑密集的特点,航线采用“沿海带状+城区棋盘”组合方式:主航线沿海堤布设以捕获滨海高差,支航线覆盖应急中心周边建筑。飞行高度120m、航速4m/s,纵向与横向重叠度分别为80\%和65\%。像控点与检校点共布设12个,采用RTK接收机测定平面与高程坐标。影像经ContextCapture完成空三加密与密集匹配,并通过自编流程进行色彩均衡与噪声点剔除,输出点云、三角网与正射影像用于后续分析。这些空间数据直接为三维水动力网格生成、建筑物几何抽取及暴露度量化提供输入,确保后续风险指标具备足够的空间分辨率。

数字城市模型虽然包含详细的地表特征,但缺乏建筑物内部结构的描述。为实现室内外一体化的洪涝模拟,本研究整合BIM和GIS技术,构建多层次的空间信息集成框架\cite{Irizarry2013}。BIM模型提供建筑物的详细几何信息、空间拓扑关系和语义属性,包括房间、门窗、楼梯等构件的精确位置和尺寸。GIS系统则提供大尺度的地形、水文和基础设施数据。通过统一的坐标参考系统和数据编码标准,实现两种数据源的无缝融合。

建筑信息模型(BIM)和地理信息系统(GIS)技术的集成应用使得在不同层次的空间信息整合成为可能,使建筑物表示更加方便、合理和标准化\cite{Irizarry2013}。为生成二维地形数据进行对比分析,从数字城市几何模型中提取10,000个特征点来表征地形起伏和地表特征,而标准30m分辨率DEM最多只能提取3,080个特征点。这一对比清楚地展示了高分辨率数字城市模型在地形表征方面的优势,如图\ref{fig:terrain_comparison}所示。

\begin{figure}[htbp]
\centering
\includegraphics[width=0.8\textwidth]{PIC/地形对比图.jpg}
\caption{数字城市模型几何体与标准30m DEM提取的地形对比}
\caption*{Figure~\thefigure~ Comparison of Digital City Model Geometry and Terrain Extracted from Standard 30m DEM}
\label{fig:terrain_comparison}
\end{figure}

为确保建模数据的质量和一致性,建立了“源数据校验—坐标统一—语义补齐—拓扑检查”的预处理流程。首先,对原始摄影测量数据进行几何校正和辐射校正,消除镜头畸变和大气影响;随后利用RTK像控点开展绝对定向,并与CGCS2000基准融合,整体平面误差控制在0.25m以内。对成果点云采用噪声点剔除与地物分类算法,区分建筑、道路、绿地与水体等要素,为后续建模提供可靠语义标签。

在BIM数据处理方面,采用IFC(Industry Foundation Classes)标准格式进行数据交换,结合Revit Model Checker与自编脚本对几何和属性进行双重校核。针对易出现的悬浮面、重复构件、法向错误等问题,建立自动修复规则;对关键构件(楼梯、消防门、竖井)执行抽样复核,并将空间编号、功能类型、楼层编号等属性与城市CIM编码体系映射,确保多源数据语义一致。

质量控制体系涵盖几何精度、属性完整性、拓扑一致性和时间有效性四个维度。为验证数字城市模型的精度,在室外硬质路面与建筑屋顶设置10个RTK检校点,实时动态(RTK)系统定位精度达到$\pm0.05\,\mathrm{m}$。对比检校点,模型平面位置的最大误差为0.463m,最小误差为0.078m,中位数误差为0.238m,满足《倾斜摄影测量技术规范》$0.5\,\mathrm{m}$的要求\cite{Bureau2012}。此外随机抽取20条剖面与既有测量剖面对比,平均高程差小于0.25m,为水动力模型提供了可靠的地形基础。为提升模型的可维护性,本研究将预处理流程沉淀为标准化作业指南,并与威海市自然资源和规划局共享,实现后续项目的快速复用。

\subsection{数值模拟平台与参数配置}

三维洪涝数值模拟采用 MIKE 21 FM HD 模块(2023\,Update2 版),二维浅水对照模型采用 Delft3D-FLOW(2022.02 版)。三维模型网格为结构化嵌套:主区域单元尺寸 1 m,关键建筑周边细化至 0.5 m;二维模型采用不规则三角网格,平均尺度 2 m。参数标定遵循《城市防洪工程设计规范》(GB 50882-2013) 的建议,结合威海市风暴潮历史案例进行调整,其关键参数列于表~\ref{tab:hydro_params}。

\begin{table}[htbp]
  \centering
  \caption{水动力模型主要参数设定}
  \caption*{Table~\thetable~ Main Parameter Settings of the Hydrodynamic Model}
  \label{tab:hydro_params}
  \begin{tabular}{@{}p{3.5cm}p{3cm}p{6cm}@{}}
    \toprule
    参数名称 & 取值 & 说明 \\
    \midrule
    重力加速度 $g$ & $9.81\,\mathrm{m/s^2}$ & 常数 \\
    Manning 粗糙系数 $n$ & 0.025 (路面) / 0.035 (绿地) / 0.040 (建筑外围) & 参考《城市防洪工程设计规范》及现场踏勘 \\
    Nikuradse 粗糙度 $k_s$ & 0.012 m & 依据铺装材料试验数据 \\
    湍流模型 & $k$-$\varepsilon$ (三维) / Smagorinsky (二维) & 与国内海岸工程常用设置一致 \\
    时间步长 $\Delta t$ & 0.1 s & 满足 CFL 条件 \\
    模拟时长 & 36 h & 覆盖风暴潮完整过程 \\
    边界水位 & 式(\ref{eq:bc}) & 统计分析+潮位站记录 \\
    地下排水能力 & 0.6 m$^3$/s & 依据滨海中心雨水泵站设计流量 \\
    \bottomrule
  \end{tabular}
\end{table}

模型标定采用“事件拟合+局部调参”策略:选取2019年10月风暴潮事件实测水位和积水深度作为参考,通过调节 Manning 粗糙系数和地下排水能力,使模拟峰值水深与监测点误差控制在 0.08 m 以内;模型验证则使用2021年“烟花”台风过程,在三处关键路口对比监测与模拟水深,均方根误差为 0.11 m,满足洪涝预警系统对精度的要求。

\section{三维水动力学模型设计}

\subsection{控制方程与物理封闭}

二维水动力模型基于深度平均的浅水方程(SWEs),采用湍流封闭关系和Nikuradse摩擦定律来参数化粗糙度效应。而三维水动力模型则基于三维雷诺平均纳维-斯托克斯(RANS)方程,采用非静水压力近似\cite{Temam1977, Constantin1998}。

浅水方程从纳维-斯托克斯方程导出,用于计算洪水事件在特定位置的到达时间和淹没深度。虽然非线性浅水方程已被证明能够在许多情况下支持对洪涝和排水动力学的准确和稳定预测\cite{Wahl2015, Gallien2014, Ramirez2016},但二维水动力模型中的维度限制和一些经验方程的约束近年来受到了挑战。

城市洪涝淹没过程具有快速流态变化、高地形梯度上的小水深、干湿交替以及与结构的激烈碰撞等特性。这些方面对基于二维SWEs的标准数值建模工具构成了数值挑战,这些工具可能不包括对这些特性的适当数值处理。如果第三维度的波动相对于其余两个维度的波动较小,且相对于感兴趣的时间尺度较小,则二维假设(静压假设、忽略垂直加速度和自由表面曲率)可能确实有效。然而,随着流动偏离这一极限,二维假设的基础变得薄弱,计算得到的二维流动将偏离物理上准确的三维对应物。

为便于复现与对比,本研究采用不可压RANS方程作为三维控制方程,写作:
\begin{equation}
\nabla\cdot \mathbf{u} = 0, \qquad \frac{\partial \mathbf{u}}{\partial t} + (\mathbf{u}\cdot\nabla)\,\mathbf{u} = -\frac{1}{\rho}\nabla p + \nabla\cdot\left[\left(\nu + \nu_t\right)\left(\nabla\mathbf{u}+\nabla\mathbf{u}^{\mathrm{T}}\right)\right] + \mathbf{g},
\label{eq:rans}
\end{equation}
其中$\mathbf{u}=(u,v,w)$为速度,$p$为压力,$\nu$为分子黏性系数,$\nu_t$为湍黏性系数,$\mathbf{g}=(0,0,-g)$为重力加速度项。

作为对照,二维深度平均浅水方程可写为:
\begin{equation}
\begin{aligned}
&\frac{\partial h}{\partial t} + \frac{\partial (hu)}{\partial x} + \frac{\partial (hv)}{\partial y} = q,\\
&\frac{\partial (hu)}{\partial t} + \frac{\partial}{\partial x}\left(hu^2 + \tfrac{1}{2}gh^2\right) + \frac{\partial (huv)}{\partial y} = -gh\,\frac{\partial \eta}{\partial x} - S_{fx},\\
&\frac{\partial (hv)}{\partial t} + \frac{\partial (huv)}{\partial x} + \frac{\partial}{\partial y}\left(hv^2 + \tfrac{1}{2}gh^2\right) = -gh\,\frac{\partial \eta}{\partial y} - S_{fy},
\end{aligned}
\label{eq:swe}
\end{equation}
其中$h$为水深,$\eta$为自由表面,$q$为源汇项,$S_{f\bullet}$为底摩阻项。自由表面运动满足运动学边界条件:
\begin{equation}
\frac{\partial \eta}{\partial t} + u\,\frac{\partial \eta}{\partial x} + v\,\frac{\partial \eta}{\partial y} = w\quad \text{at } z=\eta(x,y,t).
\label{eq:kinematic}
\end{equation}

在三维模型中,剪应力(摩擦力)是影响流体在不平坦地形表面运动的关键因素。城市环境中的湍流特性复杂,涉及建筑物的阻挡效应、街道峡谷的引导作用以及复杂地形的影响。为实现对式(\ref{eq:rans})的闭合,描述这些复杂的湍流现象,本研究采用$k$-$\varepsilon$两方程模型:
\begin{equation}
\nu_t = C_\mu\, \frac{k^2}{\varepsilon},\quad
\frac{\partial k}{\partial t}+\mathbf{u}\cdot\nabla k = \nabla\cdot\left[\left(\nu+\frac{\nu_t}{\sigma_k}\right)\nabla k\right] + P_k - \varepsilon,
\label{eq:k}
\end{equation}
\begin{equation}
\frac{\partial \varepsilon}{\partial t}+\mathbf{u}\cdot\nabla \varepsilon = \nabla\cdot\left[\left(\nu+\frac{\nu_t}{\sigma_\varepsilon}\right)\nabla \varepsilon\right] + C_{\varepsilon 1}\,\frac{\varepsilon}{k}P_k - C_{\varepsilon 2}\,\frac{\varepsilon^2}{k},
\label{eq:eps}
\end{equation}
其中$P_k=\nu_t\,(\nabla\mathbf{u}:\nabla\mathbf{u}^{\mathrm{T}})$为湍动能产生项,$C_\mu,\sigma_k,\sigma_\varepsilon, C_{\varepsilon 1},C_{\varepsilon 2}$取常用常数。

底边界采用粗糙壁面对数定律(以Nikuradse等效粗糙度$k_s$表征):
\begin{equation}
U^+ = \frac{1}{\\kappa}\ln\left(E\,y^+\right),\quad y^+=\frac{y\,u_\tau}{\nu},\quad z_0=\frac{k_s}{30},
\label{eq:wall}
\end{equation}
与二维浅水方程配套的底摩阻采用Manning形式:
\begin{equation}
S_f = g\,\frac{n^2\,\mathbf{u}\,\|\mathbf{u}\|}{h^{4/3}},
\label{eq:manning}
\end{equation}
对应于式(\ref{eq:swe})中的$S_{f\bullet}$项。

时间推进采用显式多步法并满足CFL稳定性约束:
\begin{equation}
\Delta t \le \mathrm{CFL}\; \min\left(\frac{\Delta x}{|u|+\sqrt{gh}},\; \frac{\Delta y}{|v|+\sqrt{gh}}\right),
\label{eq:cfl}
\end{equation}
其中$\mathrm{CFL}\in(0,1)$为安全系数。

\subsection{边界条件与网格划分}

根据威海潮位站近十年的观测资料\cite{WeihaiMNR2020},百年一遇风暴潮峰值增水约为$\Delta\eta_{100}=3.5\,\mathrm{m}$,其演化过程可用正、负半周期不同的余弦函数逼近\cite{Zhang2023}。因此入流边界取分段光滑的时间序列:
\begin{equation}
\eta\big|_{\Gamma_{\mathrm{in}}}(t) = \eta_0 + \Delta\eta_{100}\,s(t),
\end{equation}
其中
\begin{equation}
s(t)=
\begin{cases}
\tfrac{1}{2}\left[1-\cos\left(\pi t/T_{\mathrm{rise}}\right)\right], & 0\le t< T_{\mathrm{rise}},\\[4pt]
1-\tfrac{1}{2}\left[1-\cos\left(\pi (t-T_{\mathrm{rise}})/T_{\mathrm{fall}}\right)\right], & T_{\mathrm{rise}}\le t< T_{\mathrm{rise}}+T_{\mathrm{fall}},\\[4pt]
0, & t\ge T_{\mathrm{rise}}+T_{\mathrm{fall}},
\end{cases}
\label{eq:bc}
\end{equation}
其中$T_{\mathrm{rise}}=6\,\mathrm{h}$、$T_{\mathrm{fall}}=12\,\mathrm{h}$,并在$36\,\mathrm{h}$的模拟窗口内重复施加以覆盖全潮周期。顶部边界设定为自由表面,其余模型边界定义为无流边界:$\mathbf{n}\cdot\mathbf{u}|_{\Gamma_{\mathrm{wall}}}=0,\;\boldsymbol{\tau}|_{\Gamma_{\mathrm{fs}}}=\mathbf{0}$。初始条件取
\begin{equation}
\mathbf{u}(\mathbf{x},0)=\mathbf{0},\qquad \eta(\mathbf{x},0)=\eta_0.
\label{eq:ic}
\end{equation}

边界条件的设置对于确保模拟结果的准确性至关重要。入流边界条件基于威海潮位实测序列与统计分析确定,以反映研究区域可能面临的极端洪水情况。自由表面边界条件允许水位的自然波动,而无流边界则确保计算域边界处没有人为的质量损失或增加。

对于三维计算域,采用不同网格分辨率(1m、2m和5m)进行敏感性分析。最终确定全局网格尺寸为1m,进一步细分为梯度嵌套的结构化网格,最小网格尺寸为0.5m,总单元数量为6580万。关于二维模型,采用约300万个非结构化三角形网格。通过迭代过程确保网格间距和类型不会对模拟结果产生影响。

网格敏感性分析是确保数值解准确性的重要步骤。通过比较不同网格分辨率下的计算结果,确定在计算精度和计算成本之间的最佳平衡点。梯度嵌套网格策略在关键区域(如建筑物附近和复杂地形处)采用更精细的网格,在相对简单的区域采用较粗的网格,从而在保证计算精度的同时降低计算成本。城市排水系统的高分辨率水动力模拟也证明了细网格在动态径流刻画中的必要性\cite{CN_Fan2022Hydrodynamic}。

\subsection{模型机理差异解析与数据耦合机制}

为明确三维水动力模型在复杂城市环境中的适用性,本节从控制方程的物理封闭性和数值表现两个维度,系统对比其与二维浅水模型的差异,并界定模型输出与后续风险模块的接口关系。

二维浅水方程(SWEs)建立在静水压力分布($\partial p/\partial z = -\rho g$)和流速沿垂向均匀分布的假设之上,适用于水平尺度远大于垂直尺度的开阔水域模拟。然而,在城市洪涝演进过程中,特别是洪水冲击建筑物、流经街道峡谷及下凹式立交桥时,流动表现出强烈的非静水特性和三维湍流结构。三维RANS方程通过引入垂向动量方程和全张量湍流封闭,能够捕捉以下关键物理过程:

\begin{itemize}
    \item \textbf{非静水压力效应}:在建筑物迎水面,流体动能转化为压力势能,导致局部压力显著高于静水压力。三维模型直接求解总压力$p = p_{\mathrm{stat}} + p_{\mathrm{dyn}}$,其中$p_{\mathrm{dyn}}$为由垂向加速度引起的动压力分量,这对于评估结构冲击载荷至关重要。
    \item \textbf{垂向流速分量}:二维模型忽略垂向速度($w \approx 0$),无法描述洪水翻越防洪墙或跌入地下空间时的射流与跌水现象。三维模型保留完整的速度矢量$\mathbf{u}=(u,v,w)$,能够模拟最大可达$3.2\,\mathrm{m/s}$的瞬时垂向流速,这对于分析局部冲刷和行进阻力具有决定性作用。
    \item \textbf{三维湍流结构}:城市冠层内的尾流脱落和马蹄涡系统是三维现象。通过$k$-$\varepsilon$模型求解湍动能$k$和耗散率$\varepsilon$的输运方程,三维模型能够准确预测建筑物背水侧的低压区和涡旋回流,而这往往是漂浮物堆积和人员受困的高风险区。
\end{itemize}

表~\ref{tab:model_comparison_summary}总结了两种模型的关键特性差异及其对风险评估的影响。

\begin{table}[htbp]
  \centering
  \caption{二维与三维水动力模型特性对比及适用性分析}
  \caption*{Table~\thetable~ Comparison of Characteristics and Applicability between 2D and 3D Hydrodynamic Models}
  \label{tab:model_comparison_summary}
  \begin{tabular}{p{3cm}p{5cm}p{5cm}}
    \toprule
    特性维度 & 二维浅水模型 (2D SWEs) & 三维 RANS 模型 (3D RANS) \\
    \midrule
    \textbf{压力假定} & 静水压力分布 (Hydrostatic) & 非静水压力分布 (Non-hydrostatic) \\
    \textbf{速度场} & 深度平均速度 $\bar{\mathbf{u}}=(U, V)$ & 全三维速度场 $\mathbf{u}=(u, v, w)$ \\
    \textbf{湍流封闭} & 水平涡粘性 (Smagorinsky等) & 全张量湍流粘性 ($k$-$\varepsilon$等) \\
    \textbf{垂直结构} & 无 (单一水深值) & 解析垂向流速与分层结构 \\
    \textbf{计算代价} & 低 (分钟级-小时级) & 高 (数十小时-天级) \\
    \textbf{适用场景} & 大尺度洪水演进、开阔平原淹没 & 城市局部复杂流场、地下空间进水、结构冲击 \\
    \bottomrule
  \end{tabular}
\end{table}

作为风险评估系统的核心动力引擎,三维水动力模型的输出不仅是可视化的结果,更是后续模块的关键输入参数。具体的数据接口定义如下:
\begin{enumerate}
    \item \textbf{网格级危险度矩阵}:将每一时刻$t$、每一网格$i$处的计算结果映射为向量$\mathbf{H}_i(t) = [h_i(t), \|\mathbf{u}_i(t)\|, p_i(t)]$,直接输入至第四章构建的脆弱性函数$V = f(\mathbf{H}, \mathbf{E})$中,用于计算动态风险指数。
    \item \textbf{疏散路径阻力场}:利用三维流速场$\mathbf{u}(x,y,z)$,特别是底层的流速分布,结合人体稳定性准则(Human Stability Criteria),生成动态的不可通行区域(Blocked Areas)和高阻力区域(High Resistance Areas),为第五章的疏散仿真提供时变的环境势场数据。
    \item \textbf{结构损伤载荷}:提取建筑物表面的总压力积分$\int p \, dS$,作为构件级损伤评估的荷载输入,判断门窗破坏和墙体受损的可能性。
\end{enumerate}

通过上述接口,本章建立的高精度水动力模型不再是孤立的计算环节,而是贯穿“数据—模拟—评估—应对”全链条的数字化底座,确保了最终风险图和应急预案的科学性与物理一致性。

\section{脆弱性评估指标体系构建}

在数据与三维模型准备完成后,需要将物理、社会与经济信息转化为可定量评估的指标体系,以便与水动力模拟输出耦合生成风险结果。本节按照“指标设计—归一化—动力输入”的顺序展开,并在充分考虑数据可获得性与后续模型接口的一致性基础上,对初始指标库进行精简,仅保留参与计算和制图的核心指标。

\subsection{评估指标体系框架}

城市洪涝脆弱性评估需要综合考虑承灾体的暴露度、敏感性和适应能力等多个维度。基于国际减灾框架和风险评估理论,本研究构建了一个包含物理脆弱性、社会脆弱性和经济脆弱性的三维综合评估指标体系,每个维度下设若干指标组和可量化的核心指标。

物理脆弱性(Physical Vulnerability)主要反映建筑物和基础设施在洪涝灾害作用下的本体抗灾能力和受损程度。该维度评估的核心在于量化物理实体对水力荷载的响应特性,重点关注:
\begin{itemize}
  \item 建筑物几何与高程特征:包括建筑高度、建筑平面面积、首层标高等,决定在给定水深条件下的有效淹没程度和受灾规模;
  \item 结构抗洪性能:建筑结构类型、建筑年代与规范等级等,反映在相同荷载作用下的受损概率差异;
  \item 地形地貌与空间布局:地面高程及局部坡度、地表粗糙度、街道走向与宽度等,影响水流路径与局部汇流条件;
  \item 基础设施韧性:排水管网密度与泵站能力等,影响淹没持续时间与排涝能力。
\end{itemize}

近期针对关键基础设施暴雨脆弱性评估的研究为物理维度的指标设计提供了实证基础\cite{CN_Qiao2023InfraAssessment}。

社会脆弱性(Social Vulnerability)反映人口及社会系统在灾害冲击下的暴露程度、敏感性和适应能力。该维度强调“以人为本”的风险评估理念,主要考虑:
\begin{itemize}
  \item 人口暴露特征:常住人口密度、昼夜人口变化和高峰时段人口聚集度;
  \item 人群脆弱性差异:老年人、儿童、残障人士等特殊群体占比;
  \item 社会经济地位与应急能力:教育水平、收入水平、应急避难设施可达性等,影响获取救援资源的能力与自救能力。
\end{itemize}

社区级应急物资数字化管理实践印证了人群脆弱性差异在风险评估中的关键作用\cite{CN_Zhao2023Community}。

经济脆弱性(Economic Vulnerability)评估区域经济系统的资产暴露、潜在经济损失和灾后恢复能力,重点包括:
\begin{itemize}
  \item 资产暴露价值与空间分布:建筑物重置价值、室内财产价值与基础设施投资;
  \item 经济活动集中度与关键节点:商业设施密度、服务业聚集程度、重要基础设施节点;
  \item 恢复重建能力:保险覆盖率、财政支付能力、建设资源可获得性等。
\end{itemize}

该指标体系设计遵循科学性、系统性、可操作性和地域适应性原则。科学性体现在指标选择基于风险理论和实证研究;系统性体现在多维度覆盖和内在逻辑的一致性;可操作性体现在数据的可获得性和计算方法的可实现性;地域适应性体现在考虑研究区域的特定自然环境和社会经济条件。

数据层面,物理指标主要来自无人机影像提取和BIM属性;社会、经济指标使用《威海统计年鉴2023》\cite{WeihaiStats2023}、环翠区社区人口普查资料以及威海市应急管理局发布的能力评估报告\cite{WeihaiEM2022}。社会经济数据经街道尺度加权分配至网格单元;关键基础设施信息来源于滨海中心资产台账及市政管网数据。结合上述基础,采用专家打分构建AHP判断矩阵,邀请应急管理、城市规划、水利工程、社区治理等七名专家参与,矩阵一致性比率$CR=0.062<0.1$。

在初始指标库中,候选指标较多。考虑到数据可获得性、统计稳定性以及与后续疏散与构件损伤模块的接口需求,对指标集合进行了精简,仅保留参与定量计算和制图的核心指标,见表~\ref{tab:core_indicators},其余指标仅用于定性分析,不再进入后续公式计算。

\begin{table}[htbp]
  \centering
  \caption{综合脆弱性与暴露度核心指标说明}
  \caption*{Table~\thetable~ Definition of Core Indicators for Vulnerability and Exposure}
  \label{tab:core_indicators}
  \begin{tabular}{p{1.4cm}p{1.9cm}p{3.0cm}p{3.7cm}p{1.5cm}p{3.5cm}}
    \toprule
    维度 & 指标符号 & 指标名称 & 定义 & 单位 & 数据来源与与水动力结果的耦合方式 \\
    \midrule
    物理 & $H_{\mathrm{b}}$ & 建筑首层标高 & 建筑首层地面相对于基准面的高程 & m &
    BIM/CIM 提供;与三维水动力淹没深度$h_i$叠加计算有效淹没深度$h_i-H_{\mathrm{b}}$,参与危险度指数$H_i$构建 \\[0.3em]
    物理 & $A_{\mathrm{b}}$ & 建筑平面面积 & 单元内建筑投影面积 & m$^2$ &
    BIM/CIM 提供;与$h_i$及暴露度$E_i$共同决定单元内潜在受淹建筑体积和直接经济损失 \\[0.3em]
    物理 & $D_{\mathrm{dr}}$ & 排水管网密度 & 单元内雨水管总长与面积之比 & km/km$^2$ &
    市政管网数据;通过折减有效淹没持续时间$T_i$影响危险度$H_i$ \\[0.3em]
    物理 & $C_{\mathrm{str}}$ & 结构抗洪等级 & 按结构类型、年代和规范综合得到的抗洪能力等级(0--1) & -- &
    设计资料与现场调研;作为物理脆弱性子指标进入$V^{(\mathrm{phy})}_i$,在相同$H_i$下区分易损与耐洪建筑 \\[0.3em]
    社会 & $D_{\mathrm{pop}}$ & 人口密度 & 单元内常住人口折算密度 & 人/ha &
    普查数据;与危险度$H_i$相乘得到人口暴露度$E_i$,直接影响综合风险$R_i$ \\[0.3em]
    社会 & $R_{\mathrm{vul}}$ & 脆弱群体比例 & 老年人、儿童及残障人士占总人口比例 & \% &
    普查数据;调节社会脆弱性指数$V^{(\mathrm{soc})}_i$,在相同$H_i$下提高高龄社区风险权重 \\[0.3em]
    社会 & $A_{\mathrm{shel}}$ & 避难设施可达性指数 & 单元内居民到最近避难点的通行时间归一化指标 & -- &
    基于POI与路网分析;影响应急响应能力子指标,间接影响第4章疏散路径规划中的约束与优先级 \\[0.3em]
    经济 & $V_{\mathrm{build}}$ & 建筑重置价值密度 & 单元内建筑物重置价值总和与面积之比 & 万元/ha &
    资产台账;与$H_i$和$E_i$共同用于估算经济损失风险,并为第5章构件损伤经济折算提供基数 \\[0.3em]
    经济 & $D_{\mathrm{econ}}$ & 经济活动强度 & 综合反映就业密度与商业服务设施密度的指数 & -- &
    统计年鉴与POI数据;在同等$H_i$条件下提高商业/办公区风险权重,反映间接损失 \\[0.3em]
    经济 & $N_{\mathrm{crit}}$ & 关键基础设施数量 & 单元内变电站、泵站等关键基础设施数量 & 座 &
    资产台账;与$h_i$和$T_i$联合用于识别系统性风险突出的单元,为协同防控预案提供重点清单 \\
    \bottomrule
  \end{tabular}
\end{table}

可以看出,表~\ref{tab:core_indicators}中的指标既包含纯静态的脆弱性项,也包含通过$h_i$、$U_i$、$T_i$等水动力指标调节后的“有效脆弱性”与暴露度。在后续计算中,这些指标经归一化和权重组合构成综合脆弱性$V_i$及暴露度$E_i$,并与水动力危险度指数$H_i$共同决定网格级综合风险$R_i$。

\subsection{指标归一化与权重设定}

在表~\ref{tab:core_indicators}所示核心指标基础上,需将不同量纲的数据投射到统一尺度,并界定各类指标的重要性权重,才能与水动力参数进行叠加分析。

不同维度的指标量纲不一致,需先做归一化处理。采用区间缩放的最小-最大归一化:
\begin{equation}
z_{i,j} = \frac{x_{i,j}-x_{j}^{\min}}{x_{j}^{\max}-x_{j}^{\min}},\quad x_{j}^{\min}<x_{i,j}<x_{j}^{\max},
\label{eq:minmax}
\end{equation}
其中$x_{i,j}$为第$i$个单元在第$j$个指标的原始值,$z_{i,j}\in[0,1]$为归一化值。对于“代价型”(值越大越安全)指标使用反向归一化:
\begin{equation}
z_{i,j}^{(\mathrm{inv})} = 1-\frac{x_{i,j}-x_{j}^{\min}}{x_{j}^{\max}-x_{j}^{\min}}.
\label{eq:minmax-inv}
\end{equation}

权重采用专家判定和灵敏度分析相结合的方法设定,记$\mathbf{w}=[w_1,\ldots,w_m]^{\mathrm{T}}$,满足
\begin{equation}
\sum_{j=1}^{m} w_j = 1,\quad w_j\ge 0.
\label{eq:weights}
\end{equation}

为保证权重确定的透明性,本研究邀请来自应急管理、城市规划、水利工程等领域的7名专家开展AHP评判,判断矩阵的一致性比率为0.062,满足$CR<0.1$的要求。表~\ref{tab:vulnerability_weights}给出了指标组层权重结果,同时将物理、社会、经济三类脆弱性在综合指数中的权重设置为$\beta_{\mathrm{p}}=0.40$、$\beta_{\mathrm{s}}=0.35$、$\beta_{\mathrm{e}}=0.25$,危险度分项权重取$\alpha_1=0.45$(水深)、$\alpha_2=0.35$(流速)、$\alpha_3=0.20$(持续时间),后续敏感性分析表明该组合可在风险预测准确率与模型稳定性之间取得平衡。


\subsection{综合指标构建与风险计算}

物理、社会与经济脆弱性分别按加权和得到分项指数:
\begin{equation}
V^{(\mathrm{phy})}_i=\sum_{j\in \mathcal{P}} w_j\, z_{i,j},\quad V^{(\mathrm{soc})}_i=\sum_{j\in \mathcal{S}} w_j\, z_{i,j},\quad V^{(\mathrm{eco})}_i=\sum_{j\in \mathcal{E}} w_j\, z_{i,j}.
\label{eq:subindex}
\end{equation}
综合脆弱性指数定义为
\begin{equation}
V_i=\beta_{\mathrm{p}} V^{(\mathrm{phy})}_i + \beta_{\mathrm{s}} V^{(\mathrm{soc})}_i + \beta_{\mathrm{e}} V^{(\mathrm{eco})}_i,\quad \beta_{\mathrm{p}}+\beta_{\mathrm{s}}+\beta_{\mathrm{e}}=1.
\label{eq:vulnerability}
\end{equation}

基于水动力模拟得到的危险度因子(如水深$h$、流速$U=\|\mathbf{u}\|$、持续时间$T$)构建洪涝危险度指数:
\begin{equation}
H_i = \alpha_1\, \frac{h_i}{h_{\max}} + \alpha_2\, \frac{U_i}{U_{\max}} + \alpha_3\, \frac{T_i}{T_{\max}},\quad \alpha_1+\alpha_2+\alpha_3=1.
\label{eq:hazard}
\end{equation}

曝光度(人口或资产)用$E_i\in[0,1]$表示,最终风险矩阵采用乘性整合:
\begin{equation}
R_i = H_i\, V_i\, E_i,\quad R_i\in[0,1].
\label{eq:risk}
\end{equation}
该形式兼顾危险触发强度、承灾体脆弱性与暴露水平,便于解释与分区。

\subsection{风险等级划分与制图}

为支持应急分级预警,将$R_i$划分为五级:极低、较低、中等、较高、极高。采用分位数阈值$\{q_{0.2},q_{0.4},q_{0.6},q_{0.8}\}$:
\begin{equation}
\text{等级}(R_i)=\begin{cases}
\text{极低}, & 0\le R_i< q_{0.2},\\
\text{较低}, & q_{0.2}\le R_i< q_{0.4},\\
\text{中等}, & q_{0.4}\le R_i< q_{0.6},\\
\text{较高}, & q_{0.6}\le R_i< q_{0.8},\\
\text{极高}, & R_i\ge q_{0.8}.
\end{cases}
\label{eq:grading}
\end{equation}
据此生成风险热力图,并与历史淹没记录或实测点进行一致性检验,以保证分级的可解释性与稳健性。

脆弱性评估需要考虑多个空间尺度和时间尺度的因素。在空间尺度上,从个体建筑物到街区、从街区到城市的不同层次都需要进行评估。在时间尺度上,需要考虑短期冲击和长期恢复能力。

\subsection{水动力耦合指标}

上述脆弱性框架需要来自三维水动力模型的危险度输入,本小节对关键水力指标的提取方式进行说明,作为指标体系的补充参数。

基于三维水动力模拟结果,对每一个网格单元$i$提取代表洪涝强度与持续性的水动力统计量,组成指标向量
\begin{equation}
\mathbf{Q}_i = \big(h_i,\; U_i,\; P_i,\; T_i\big),
\end{equation}
其中$h_i$为峰值淹没深度,$U_i$为峰值或特征深度平均流速,$P_i$为空间平均水动力压力,$T_i$为淹没持续时间。经归一化处理后,$h_i$、$U_i$和$T_i$按照式(\ref{eq:hazard})构成危险度指数$H_i$,而$P_i$主要作为第5章构件级损伤评估的荷载输入。

淹没深度($H$)直接影响建筑物和人员安全,可按以下标准进行分级:
\begin{itemize}
\item $H>0.3\,\mathrm{m}$:高危险区域,对人员和建筑物构成严重威胁;
\item $0.2\,\mathrm{m}\le H\le 0.3\,\mathrm{m}$:中等危险区域,对弱势群体有较大影响;
\item $0.1\,\mathrm{m}\le H<0.2\,\mathrm{m}$:低危险区域,主要影响交通和日常活动;
\item $H<0.1\,\mathrm{m}$:轻微影响区域。
\end{itemize}

在威海案例研究中,三维模型淹没深度分布显示:$H>0.3\,\mathrm{m}$区域占21.3\%,$0.2$--$0.3\,\mathrm{m}$区域占54.3\%,$0.1$--$0.2\,\mathrm{m}$区域占18.3\%,其余区域占6.1\%。这一分布特征反映了沿海城市洪涝灾害的显著空间差异性,为风险等级划分提供了依据。

流速($V$)影响人员疏散和结构安全,本研究同时关注深度平均流速和垂直流速分量。当遇到障碍物时垂直流速占主导地位,瞬时垂直流速可达$3.2\,\mathrm{m/s}$,不可忽视其对结构的冲击作用。在特征点分析中,深度平均三维速度约为$2.6\,\mathrm{m/s}$,垂直速度分量的平均值为$2.37\,\mathrm{m/s}$,流速的空间分布主要集中在地形起伏明显及街道峡谷区域,这些区域是人员通行风险较高的地带。

水动力压力($P$)反映洪水对建筑物的冲击强度。研究发现,在典型点位,水动力压力比静水压力高约$5000\,\mathrm{Pa}$,这种额外动压力对结构安全具有重要影响。在10个特征点的分析中,三维模型的平均水动力压力为$21{,}037\,\mathrm{Pa}$,而二维模型仅为$18{,}143\,\mathrm{Pa}$。$P_i$不直接进入式(\ref{eq:hazard}),而是作为第5章构件级损伤评估中计算构件荷载和破坏概率的关键输入。

淹没持续时间($T$)则影响灾害损失程度和恢复难度,尤其是在风暴潮情景下,尾水位居高不下导致长时间积水。沿海城市中,由于风暴潮造成的高尾水位和海防工程的存在,淡水无法通过重力排入海湾,高水位可能维持数小时甚至更久,从而显著增加建筑物受损和室内财产损失风险。$T_i$在危险度指数$H_i$中以持续时间因子的形式体现,同时也被用于识别长时间淹没的高风险构件与街区。

综合以上水动力指标,可将网格级危险度信息$\mathbf{Q}_i$与前述脆弱性指标$V_i$和暴露度$E_i$共同输入风险计算与热力图生成模型,实现“水动力—脆弱性—暴露度”的显式耦合。


\subsection{风险热力图生成}

在完成指标归一化与水动力参数提取后,本研究基于式(\ref{eq:vulnerability})--(\ref{eq:risk})构建网格尺度的综合洪涝风险指数,并生成相应的风险热力图。整体流程可概括为以下六个步骤:

\begin{enumerate}
  \item \textbf{水动力模拟与指标提取}:基于三维RANS控制方程对百年一遇风暴潮情景进行数值模拟,得到随时间变化的水深场$h(\mathbf{x},t)$、速度场$\mathbf{u}(\mathbf{x},t)$与压力场$p(\mathbf{x},t)$。在每个网格单元$i$内提取峰值水深$h_i$、特征流速$U_i$、淹没持续时间$T_i$及水动力压力$P_i$等统计量。
  \item \textbf{水动力危险度归一化}:对$h_i$、$U_i$和$T_i$进行归一化处理,得到对应的无量纲变量$h_i/h_{\max}$、$U_i/U_{\max}$和$T_i/T_{\max}$,并按式(\ref{eq:hazard})计算网格级危险度指数$H_i$。
  \item \textbf{脆弱性与暴露度计算}:利用表~\ref{tab:core_indicators}中的核心指标,按式(\ref{eq:subindex})和式(\ref{eq:vulnerability})计算物理、社会、经济三个维度的分项脆弱性指数$V^{(\mathrm{phy})}_i$、$V^{(\mathrm{soc})}_i$、$V^{(\mathrm{eco})}_i$及综合脆弱性$V_i$;同时基于人口密度与资产价值构造暴露度$E_i$。
  \item \textbf{综合风险指数计算}:将水动力危险度$H_i$、综合脆弱性$V_i$与暴露度$E_i$代入式(\ref{eq:risk}),得到每个网格单元的综合洪涝风险指数$R_i\in[0,1]$。$R_i$越大,表示在给定情景下该单元遭受严重损失的可能性越大。
  \item \textbf{风险等级划分}:根据式(\ref{eq:grading})中给出的分位数阈值$\{q_{0.2},q_{0.4},q_{0.6},q_{0.8}\}$,将连续变量$R_i$划分为“极低、较低、中等、较高、极高”五个等级,实现从连续风险到离散预警等级的映射。
  \item \textbf{空间可视化与多尺度聚合}:将风险等级栅格映射为颜色梯度,生成百年一遇风暴潮情景下的综合洪涝风险热力图;同时按社区边界与功能分区对$R_i$进行聚合,得到社区级和设施级风险统计结果,为后续疏散预案和构件加固方案提供空间依据。
\end{enumerate}

通过上述流程,三维水动力模拟与多维脆弱性指标在网格尺度上实现了统一,得到的风险热力图既保留了高分辨率水动力场的空间细节,又综合考虑了人口与资产暴露,为城市精细化防洪提供了直观的决策支撑。在下一小节中,将对风险热力图及相关图件进行空间分布分析与成因解读。


\section{应用案例分析}

为验证上述方法的适用性,本节将风险评估流程应用于威海市滨海应急服务中心案例,演示数据采集、指标计算和热力图展示的完整链条。

\subsection{研究区域概况}

选择威海市环翠区作为典型案例区域,该区域位于渤海沿岸,经济发达、人口密集,受地理位置、天气条件和海底地形影响,经常遭受风暴潮袭击。当风暴潮发生时,特别是与天文大潮重合时,海水位急剧上升,大量海水冲入城市造成内涝。

研究区域覆盖面积为$1.5\text{km} \times 1.5\text{km}$,包含了典型的沿海城市特征:密集的建筑群、复杂的街道网络、重要的基础设施以及脆弱的滨海地形。该区域具有典型的喇叭状地形特征,当洪水来临时,几何形状首先较宽,更有利于水体汇集,如图\ref{fig:study_area}所示。

\begin{figure}[htbp]
\centering
\includegraphics[width=0.8\textwidth]{PIC/研究区域位置.jpg}
\caption{研究区域位置及数字城市模型}
\caption*{Figure~\thefigure~ Study Area Location and Digital City Model}
\label{fig:study_area}
\end{figure}

由于地形平缓和排水出口较小,在风暴潮造成的高尾水位作用下,淡水无法通过重力排入海湾。高水位可能维持较长时间,进一步加剧灾害程度。统计数据显示,该地区每年都会遭受不同程度的风暴潮损害。过去500年中,山东省沿海记录的风暴潮达96次,其中33次被归类为严重灾害。

研究区域沿海建有由钢筋混凝土墙和滨海建筑物支撑结构组成的海防工程网络。历史上最强的风暴潮发生在2007年3月4-5日,海平面上升3.5m,造成大面积淹没,尽管该地区受到海防工程网络的保护。

\subsection{模型验证}

风险评估依赖的水动力结果必须经过实验与现场数据双重校核,本小节通过物理实验和实测潮位对模型进行验证,确保热力图输入的可靠性。

由于沿海观测数据相对稀缺且往往无法捕获复杂的沿海动力学,使得模型校准变得不可行。通常使用洪涝范围来验证洪涝模型的准确性。本研究利用具有确切流深数据的城市洪涝实验和研究区域观测到的潮位上升过程来验证模型的准确性。

物理实验验证:首先构建了一个具有10个探测点充分实验数据的洪涝模型,采用相同的建模方法验证三维模型的适用性和准确性。物理实验比例为1:100,设置10个探测点观测水深变化\cite{Testa2007},探测点布置如图\ref{fig:probe_points}所示。

\begin{figure}[htbp]
\centering
\includegraphics[width=0.8\textwidth]{PIC/探测点布置图.jpg}
\caption{探测点布置及t=14.5s时的流深分布}
\caption*{Figure~\thefigure~ Probe Point Layout and Flow Depth Distribution at t = 14.5s}
\label{fig:probe_points}
\end{figure}

对物理实验和所有10个探测点流深的模拟结果进行比较。结果显示,所有探测点的流深平均误差均在15\%以内。因此,本文建立的三维水动力模型适用于城市洪涝淹没模拟。

现场观测验证:对2019年4月16日的潮位上升过程进行了调查,以验证模型的准确性。测量物理量时空变化的困难阻碍了现场尺度的调查。只获得了沿岸潮位上升过程中的最大洪涝范围,选择威海潮汐站获得的海水位数据作为边界条件来强迫水动力模型。

验证结果表明,模拟的最大洪涝范围与现场调查结果(红线)一致,表明建模方法有效,如图\ref{fig:validation}所示。这种验证方法虽然有限,但在缺乏详细现场观测数据的情况下,提供了模型可靠性的重要证据。

此外,将模拟得到的洪水峰值水深与2019年威海市应急管理局记录的积水监测数据进行对比\cite{WeihaiEM2022},在滨海中心北侧停车区、应急广场和西南出入口三处监测点的差异分别为0.08m、0.11m和0.09m,均方根误差为0.10m,说明模型能够再现实际洪涝分布特征,与政府公开资料相吻合。

\begin{figure}[htbp]
\centering
\includegraphics[width=0.8\textwidth]{PIC/验证对比图.jpg}
\caption{洪涝预测最大范围与现场验证数据对比及不同网格分辨率三维建模结果对比}
\caption*{Figure~\thefigure~ Comparison of Maximum Flood Prediction Range with Field Validation Data and 3D Modeling Results at Different Grid Resolutions}
\label{fig:validation}
\end{figure}

\subsection{网格敏感性分析}

为了保证风险评估结果对网格划分具有稳健性,进一步开展敏感性分析,说明数据分辨率对淹没范围和风险热力图的影响。

建模不确定性可能来自广泛的物理过程和广泛的时空尺度。在缺乏现场观测的情况下,不确定性的概率评估可能很困难。然而,敏感性分析和更基本的不确定性分析形式可能有助于识别模型预测中不确定性的关键来源和规模。

城市环境的复杂性通常需要在科学可信度和计算可行性之间找到平衡的网格。网格应在描述城市环境物理现实和计算可行性之间提供妥协。因此,采用不同网格分辨率(1m、2m和5m)进行洪涝范围敏感性分析。

大部分淹没区域在不同网格分辨率下相互吻合,被分类为匹配并标记为黄色。5m网格分辨率的淹没范围在地形急剧变化的地方(如东北角)覆盖更大面积,而在地形平缓的地方,似乎更精细的网格产生更大的淹没面积。

这一结果的可能解释是,粗糙网格无法准确表示带有小障碍物的急剧地形,水流可能被阻挡,因此在这些区域粗糙网格提供淹没结果而精细网格不提供。另一个可能原因是在这些区域,单个粗糙网格覆盖更大面积,即使网格未完全填满,如果网格中的水量超过预设阈值,仍会表示该网格被水淹没\cite{Begnudelli2008}。

\subsection{模拟结果分析}

采用DHI MIKE 21和Flow-3D软件包分别进行二维和三维沿海洪涝淹没模拟。三维计算域总单元数为6580万,使用Intel core E5-1603四核处理器(@2.80 GHz)、32GB内存和并行软件许可证,三维模拟计算时间约80小时。二维模型采用约300万个非结构化三角形网格,最大计算时间为4小时。

\subsubsection{二维与三维模拟结果比较}

二维和三维模型都在具有相似喇叭状地形的区域预测出三个相对较大的淹没区域。当洪水来临时,由于几何形状首先较宽,更有利于获得大量水体。随着洪水向前移动,通向喇叭状地形内部的出口逐渐变窄,洪水别无选择只能加深,冲击海防工程,大量水体冲入城市内部,使洪水危险加剧,如图\ref{fig:trumpet_flow}所示。

\begin{figure}[htbp]
\centering
\includegraphics[width=0.8\textwidth]{PIC/喇叭状地形流场图.jpg}
\caption{喇叭状地形周围的流场及中间淹没区域的三维透视放大图}
\caption*{Figure~\thefigure~ 3D Perspective Magnified View of the Flow Field Around the Funnel-Shaped Terrain and the Central Inundation Area}
\label{fig:trumpet_flow}
\end{figure}

二维和三维结果覆盖相似的洪涝淹没范围,除了复杂结构和狭窄街道区域。可以观察到,二维模型无法预测洪水流向内陆区域的路径,淹没区域沿海岸呈带状分布。相比之下,三维模型能够准确区分高程梯度,因此洪水可以沿街道进一步流向城市内部。二维和三维模型的最终淹没面积分别为$0.37\text{km}^2$和$0.41\text{km}^2$。差异源于床面高程快速变化和狭窄街道附近的复杂流动行为,如图\ref{fig:flow_distribution}所示。

\begin{figure}[htbp]
\centering
\includegraphics[width=0.8\textwidth]{PIC/最终流深分布图.jpg}
\caption{最终流深分布及流场的三维局部放大视图}
\caption*{Figure~\thefigure~ 3D Local Magnified View of the Final Flow Depth Distribution and Flow Field}
\label{fig:flow_distribution}
\end{figure}

再现非常小尺度的涡流通常不被认为是为了提供流场信息而建模大型海湾和河口所必需的。然而,正确获得垂直湍流和再现残余环流对于预测流动路径极其重要\cite{Teeter2001}。

当洪水淹没城市区域时,地面复杂几何形状的相互作用使其发展出三维特征,如垂直加速度和非静水压力分布。在这种情况下,需要三维建模技术来描述翻转冲击过程复杂动力学的主要特征并捕获垂直涡旋。由于这种现象局部明显呈三维且自由表面呈强烈曲率,垂直方向压力分布非静水,垂直平均方法在重建剧烈流动对障碍物冲击方面存在局限性。

当反射波遇到前进洪水时,形成的垂直涡旋进一步扰乱流场。三维模型捕获了大自由表面变形和飞溅现象。在这些区域明显的三维效应不能忽视,因为湍流能量主要通过垂直方向的碰撞耗散。如果在模拟问题中只考虑二维,则忽略湍流扩散的三维特征,导致不现实的流动运动\cite{Li2008},二维与三维湍流对比如图\ref{fig:2d_3d_comparison}所示。

\begin{figure}[htbp]
\centering
\includegraphics[width=0.8\textwidth]{PIC/二维三维对比图.jpg}
\caption{二维流深分布及研究区域部分区域二维与三维湍流对比}
\caption*{Figure~\thefigure~ 2D Flow Depth Distribution and Comparison of 2D and 3D Turbulence in Subregions of the Study Area}
\label{fig:2d_3d_comparison}
\end{figure}

垂直速度分量主要集中在地形波动明显的区域,在这些区域由于遇到障碍而改变流向。从结果可以观察到,瞬时垂直速度大小可达3.2m/s,这是可观的且不能为了准确性而忽略,而二维深度平均浅水方程无法捕获准确的流动动力学\cite{Ozmen2011},垂直速度分量分布如图\ref{fig:vertical_velocity}所示。

\begin{figure}[htbp]
\centering
\includegraphics[width=0.8\textwidth]{PIC/垂直速度分布图.jpg}
\caption{局部放大区域垂直速度分量分布}
\caption*{Figure~\thefigure~ Vertical Velocity Component Distribution in the Local Magnified Area}
\label{fig:vertical_velocity}
\end{figure}

沿海岸选择10个特征点来表征流场。这些点的结果变化趋势相似。以P06为例,水动力压力比静水压力高约5000Pa。关于速度,在洪水到达时观察到非常大的波动。在早期阶段,深度平均三维速度大于二维值,因为遇到障碍时垂直速度占主导地位。深度平均三维速度比二维值下降得更快。对此的可能解释是二维模型中忽略了垂直湍流扩散,P06点处的压力和速度对比如图\ref{fig:pressure_velocity}所示。

\begin{figure}[htbp]
\centering
\includegraphics[width=0.8\textwidth]{PIC/压力速度对比图.jpg}
\caption{P06点处压力(左)和速度(右)对比}
\caption*{Figure~\thefigure~ Pressure (Left) and Velocity (Right) Comparison at Point P06}
\label{fig:pressure_velocity}
\end{figure}

在选定点,三维模型和二维模型计算的流深、压力、深度平均速度的平均绝对误差(MAE)值分别为0.05-0.21m、1847-5589Pa、0.03-0.24m/s。差异源于二维假设,这些假设不适用于具有巨大垂直变化的湍流。具体的误差分析结果如表\ref{tab:error_analysis}所示。

\begin{table}[htbp]
\centering
\caption{二维与三维模拟结果的流深、压力和深度平均流速相对误差}
\caption*{Table~\thetable~ Relative Errors in Flow Depth, Pressure and Depth-Averaged Velocity between 2D and 3D Simulation Results}
\label{tab:error_analysis}
\begin{tabular}{c@{\hspace{0.5em}}c@{\hspace{0.5em}}c@{\hspace{0.5em}}c@{\hspace{1em}}c@{\hspace{0.5em}}c@{\hspace{0.5em}}c@{\hspace{0.5em}}c}
\toprule
\multirow{2}{*}{探测点} & \multicolumn{3}{c}{MAE} & \multirow{2}{*}{探测点} & \multicolumn{3}{c}{MAE} \\
\cmidrule(lr){2-4} \cmidrule(lr){6-8}
& 流深 & 压力 & 流速 & & 流深 & 压力 & 流速 \\
\midrule
P01 & 0.05 & 3680 & 0.21 & P06 & 0.17 & 3758 & 0.14 \\
P02 & 0.13 & 2325 & 0.12 & P07 & 0.19 & 5589 & 0.15 \\
P03 & 0.15 & 1847 & 0.03 & P08 & 0.14 & 2592 & 0.08 \\
P04 & 0.07 & 2958 & 0.12 & P09 & 0.06 & 2315 & 0.13 \\
P05 & 0.21 & 4258 & 0.17 & P10 & 0.12 & 2796 & 0.24 \\
\bottomrule
\end{tabular}
\begin{flushleft}
\footnotesize \textit{注}:MAE = $\frac{1}{n}\sum_{i=0}^{n}|X_i - Y_i|$。$X_i$:t时刻二维结果;$Y_i$:t时刻三维结果 (m, Pa, m/s)
\end{flushleft}
\end{table}

在二维模型中,底部剪应力直接转化为缓解深度平均速度并因此增加流深\cite{Molinari2019}。水平速度是深度平均的(即在垂直方向上均匀),垂直剪切不存在\cite{National2009}。然而,底部剪应力在三维模型中平衡垂直剪切,只直接影响底部单元附近的水平速度。在高流量条件下,二维模型中的深度平均速度大于三维模型中的近壁速度,底部剪应力和自由表面高程也更大,导致二维模型性能不佳。

\subsubsection{地形影响对比分析}

为讨论数字城市模型的重要性,使用Flow-3D水动力模型分别基于标准和改进的高程数据集进行洪涝淹没模拟。基于标准地形数据集的结果明显受到DEM误差的影响:由于标准DEM以整数格式提供,流动的扩展和收缩以阶梯式方式模拟,而基于高分辨率地形数据,淹没区域的范围以更现实的方式变化。标准DEM中详细地表特征之间的平滑高程过渡导致稳态流动条件。没有形成湍流或垂直涡旋。这是不现实的结果,表明过度简化的模型无法准确代表真实世界条件,如图\ref{fig:dem_comparison}所示。

\begin{figure}[htbp]
\centering
\includegraphics[width=0.8\textwidth]{PIC/DEM对比图.jpg}
\caption{基于标准30m DEM的三维模拟结果及数字城市模型与标准DEM洪涝范围差异}
\caption*{Figure~\thefigure~ 3D Simulation Results Based on Standard 30m DEM and Differences in Flood Extent Between the Digital City Model and Standard DEM}
\label{fig:dem_comparison}
\end{figure}

对于分布在研究区域左侧的低洼植被区域,数字城市模型足够精细,能够呈现每条街道和结构,因此计算了内陆洪涝过程,甚至建筑物内部洪涝过程。标准DEM通过大梯度间的平滑插值忽略了所有地形特征,因此无法正确表示陡坡的阻挡,流动可能比详细地形数据前进得更远。这一结果表明,我们需要详细的地形数据来进行更好的洪涝淹没模拟。

此外,使用标准DEM数据分别设置二维和三维水动力模型。由于几何形状足够简单且不包括狭窄街道或复杂结构,二维和三维模拟结果之间没有观察到明显差异。在这种情况下波动不明显,深度平均浅水方程确实产生可接受的结果,计算需求和初步努力要少得多;在这种情况下,三维模拟的好处不太明显,二维模型是三维模型的有吸引力的替代方案。

二维和三维模型中建筑物的不同处理方式对流动运动也有很大影响。在二维模型中建筑物被视为水无法流过的块体,忽略垂直方向的壁面摩擦,而与数字城市模型耦合的三维水动力模型可以模拟表面粗糙度。因此,三维模型中洪水的摩擦阻力可以准确模拟,这对流动动力学非常重要。

洪水流在约58秒后到达建筑围护结构并冲入房屋。经过约70秒模拟,一楼被填满,水流继续流向建筑内部。最大淹没深度约1.8m。流入建筑内部的洪涝过程强调了由于建筑围护墙障碍导致的垂直速度。复杂性和瞬态特征是结构周围流场的主要特征。二维模拟和过度简化的地形无法捕获这些流动特征的详细信息,建筑物内部流场如图\ref{fig:building_interior}所示。

\begin{figure}[htbp]
\centering
\includegraphics[width=0.8\textwidth]{PIC/建筑内部流场图.jpg}
\caption{t=90s时建筑物内部流场}
\caption*{Figure~\thefigure~ Flow Field Inside the Building at t = 90s}
\label{fig:building_interior}
\end{figure}高分辨率地形显然是精细城市洪涝模拟所必需的,其中狭窄街道和建筑物对流动运动有大尺度影响。它们不仅阻挡水流并改变洪水路径,还对洪水产生摩擦阻力。与数字城市模型耦合,三维方法有望准确表示地面反应和建筑阻挡相互作用的复杂三维现象。

\subsection{洪涝风险分析}

二维和三维结果都表明,研究区域具有很大的洪涝危险潜力,主要原因是地形分布特征和这些区域的位置。一旦发生洪涝,居民的紧急救援主要依靠社区,因此整个区域按社区边界划分为47个部分来评估洪涝危险。洪涝危险主要发生在海岸200m范围内,人口分布相对较少。海防工程大大减弱洪涝能量,导致洪涝危险逐渐减少。流深>0.2m的淹没面积为$0.31\text{km}^2$,占研究区域的16.9\%,淹没影响约400人。在三维模拟中约有4座建筑物实际被淹没。

总体而言,该区域容易受到风暴潮引起的严重洪涝影响,尽管内城受到海防工程和滨海建筑物保护。洪泛区范围有限,损失集中在沿海区域。一旦发生严重风暴潮,首先影响交通,对研究区域人员疏散措施产生负面影响。确实,二维模型可以给出洪涝范围的粗略近似。没有相关建筑物高程数据的二维建筑物表示不能完全对应实际受洪涝影响的建筑物,而三维模型可以提供更准确和现实的结果来应对潜在的洪涝危险。

通过透视图可以观察到,压倒性的洪水首先击中具有喇叭状地形的低地区域,然后沿着道路和裸地向前推进。由于建筑物的阻挡作用随着深入而变得越来越重,因此主导水流无法在城市中进一步深入。结果,水流必须改变方向扩散到整个滨水区域。由于风暴潮造成的高尾水位和海防工程的存在,淡水无法通过重力排入海湾。持续的洪水从海防缺口涌入城市、城市内部建筑物的阻挡以及高尾水位加剧了洪涝危险的情况。

从透视图可以观察到,三维模型可以提供洪涝淹没过程的非常直观的呈现,包括洪涝范围、水深、流速和任何关注变量的信息。这在决策者和管理者的实际工作中非常有用。显然,二维模型无法做到这一点,如图\ref{fig:perspective_views}所示。

\begin{figure}[htbp]
\centering
\includegraphics[width=0.8\textwidth]{PIC/透视图.jpg}
\caption{不同位置t=100s和500s时洪涝淹没透视图}
\caption*{Figure~\thefigure~ Flood Inundation Perspective Views at t = 100s and 500s at Different Locations}
\label{fig:perspective_views}
\end{figure}

二维和三维水动力模拟的洪涝危险分析比较显示,虽然两种模型都能预测基本的淹没范围,但三维模型在以下方面具有明显优势。详细的对比分析结果如表\ref{tab:flood_analysis}所示。

\begin{table}[htbp]
\centering
\caption{基于二维和三维水动力模拟的洪涝危险分析对比}
\caption*{Table~\thetable~ Comparison of Flood Risk Analysis Based on 2D and 3D Hydrodynamic Simulations}
\label{tab:flood_analysis}
\tiny
\begin{tabular}{p{3.5cm}p{4cm}p{5cm}}
\toprule
二维模型能力 & 三维模型额外提供的信息 & 三维模型额外信息对洪涝危险分析的重要性 \\
\midrule

淹没面积为0.37km²,占研究区域的20.2\% & 淹没面积为$0.41\text{km}^2$(包括城市内部和建筑内部的淹没区域) & 洪涝范围是量化洪涝危险的主要因素 \\
\midrule
估计7座建筑物被淹没 & 实际4座建筑物被淹没 & 三维水动力模型可以区分流向城市内部的洪水路径,并识别建筑物是否被淹没。没有相关建筑物高程数据的二维建筑物表示不能完全对应实际受洪水影响的建筑物。三维水动力模型可以得出更准确的建筑物损害评估 \\
\midrule
淹没区域流深(H)分布:H>0.3m占24\%,H=0.2-0.3m占62.4\%,H=0.1-0.2m占10\%,其他占3.6\% & 三维模型淹没区域流深(H)分布:H>0.3m占21.3\%,H=0.2-0.3m占54.3\%,H=0.1-0.2m占18.3\%,其他占6.1\% & 流深是评估洪涝危险的关键因素。应根据流深分布实施不同的救援措施 \\
\midrule
10个探测点的深度平均流速为2.5m/s & 深度平均流速为2.6m/s。瞬时垂直速度大小的平均值为2.37m/s & 遇到障碍时垂直速度占主导地位。损害评估应考虑不同水位的流速 \\
\midrule
10个探测点的平均流动压力为18,143Pa & 10个探测点的平均流动压力为21,037Pa & 水动力压力用于识别面临猛烈洪水冲击的最危险区域 \\
\bottomrule
\end{tabular}
\end{table}

人口暴露分析结果如图\ref{fig:population_exposure}所示。

\begin{figure}[htbp]
\centering
\includegraphics[width=0.8\textwidth]{PIC/人口暴露图.jpg}
\caption{百年一遇洪水人口暴露分析}
\caption*{Figure~\thefigure~ Population Exposure Analysis for a 100-Year Flood}
\label{fig:population_exposure}
\end{figure}

\subsection{洪涝风险分析}

根据第\ref{sec:脆弱性评估指标体系构建}节构建的综合指标体系,首先利用式(\ref{eq:hazard})--(\ref{eq:risk})计算每个$10\,\mathrm{m}\times 10\,\mathrm{m}$网格的危险度指数$H_i$、综合脆弱性$V_i$与暴露度$E_i$,得到百年一遇风暴潮情景下的综合风险指数$R_i$并生成风险热力图。整体来看,风险空间分布呈现出“滨海高危险—内陆中等风险—局部高值斑块”的格局,与地形特征和人口、资产分布高度耦合。

二维和三维水动力结果都表明,研究区域具有较大的洪涝危险潜力,主要原因是地形分布特征和沿海位置。一旦发生风暴潮,洪水首先通过喇叭状低洼地形迅速汇入滨海带状区域,形成大范围、高水深的淹没带。综合风险热力图显示,\emph{极高风险}栅格主要集中在海堤内侧约$200\,\mathrm{m}$范围内的喇叭状低地和下凹式停车区,兼具“高水深($H>0.3\,\mathrm{m}$)+长持续时间($T>3\,\mathrm{h}$)+一定人口和资产暴露”的特点;其成因主要包括地势低洼、排水出口受高尾水位“顶托”以及海防工程局部缺口引导的集中溃入。

\emph{较高风险}区域则沿主要道路和建筑密集区向内陆延伸,一方面受三维水动力场中“沿街道内涝通道”的影响,危险度$H_i$沿道路方向具有明显条带状特征;另一方面,这些街廓往往兼具较高人口密度$D_{\mathrm{pop}}$与较大建筑重置价值$V_{\mathrm{build}}$,导致$V_i$和$E_i$叠加后形成高风险斑块。相比之下,部分滨海绿地和低密度仓储区虽危险度较高,但人口与资产暴露较低,综合风险等级多为“中等”或“较高”,而非“极高”。

从定量指标来看,流深$H>0.2\,\mathrm{m}$的淹没面积约为$0.31\,\text{km}^2$,占研究区域的16.9\%,对应受影响人口约400人。在三维模拟中约有4座建筑物实际被淹没,这些建筑物所在栅格在风险热力图中均呈“较高”及以上等级,印证了三维模型在识别建筑级高风险单元方面的优势。二维模型在同一情景下估计有7座建筑物被淹没,空间位置与实际受淹建筑存在明显偏差,说明仅基于二维水深场难以精确刻画复杂城市结构中的局部风险。

从人口暴露角度看,人口暴露分析结果如图\ref{fig:population_exposure}所示。可以发现,\emph{最高危险度区域并不一定对应最大人口风险}:沿海200m范围内的极高危险区人口相对较少,而部分略微内陆、地势相对较高但人口稠密的社区,则在综合风险热力图中表现为“中等—较高风险”条带。这一结果提示,在制定疏散预案时需要兼顾水动力危险与人口分布,不宜仅以水深作为唯一决策依据。

结合三维流场结果可以进一步解释高风险斑块的物理成因。透视图表明,压倒性的洪水首先击中具有喇叭状地形的低地区域,然后沿着道路和裸地向前推进。由于建筑物的阻挡作用随着深入而变得越来越强,主导水流无法在城市中进一步深入,必须改变方向扩散到整个滨水区域;同时,在建筑物迎水侧形成的高压区和背水侧的回流区往往对应高水动力压力$P_i$和较大垂直流速,是构件损伤和人员受困的潜在高风险点。

二维和三维风险分析的对比结果表明,二维模型可以给出洪涝范围和大致危险区的粗略近似,但由于无法刻画建筑物内部与街道峡谷内的三维流动特征,在建筑级风险识别方面存在局限。三维模型可以提供洪涝淹没过程的直观三维呈现,包括洪涝范围、水深、流速和压力等信息,为辨识关键疏散通道、危险驻留区和结构薄弱点提供了更精细的依据。

从后续章节的角度看,本节得到的网格级危险度指数$H_i$与综合风险栅格$R_i$承担了“承前启后”的作用:一方面,第4章将在此基础上将$H_i$与$R_i$映射为室内外一体化疏散网络上的“代价场”和禁行区域,实现\emph{风险驱动的路径规划};另一方面,第5章将利用$h_i$、$T_i$和$P_i$等三维水动力指标筛选高风险构件,并将风险热力图中“极高/较高风险”区域内的关键建筑作为构件级损伤分析和经济损失评估的重点对象,从而形成“风险评估—疏散引导—构件损伤”的闭环。

\section{本章小结}

本章提出了基于CIM的高精度城市洪涝风险评估方法,通过整合数字航空摄影测量、BIM和GIS技术,构建三维水动力模型与多维脆弱性指标体系,实现了对城市洪涝灾害的精细化模拟与综合风险量化。主要结论与特点如下:

\begin{enumerate}
\item \textbf{建立了高精度三维建模技术体系。} 通过数字航空摄影测量技术获取高精度三维城市模型,相比传统30m分辨率DEM能够提取10{,}000个特征点(相比3{,}080个),显著提升了地形表征的精度和细节,为复杂城市环境下的洪涝模拟提供了可靠几何基础。

\item \textbf{构建了适用于城市洪涝情景的三维水动力学模型。} 基于三维雷诺平均纳维–斯托克斯方程,采用$k$–$\varepsilon$湍流模型和非静水压力近似,能够准确描述建筑物密集区域的三维流动特征,包括垂直涡旋(最大垂直速度约$3.2\,\mathrm{m/s}$)、自由表面剧烈变形和高水动力压力区等现象,克服了传统二维浅水模型在复杂城市空间中的适用性局限。

\item \textbf{构建了多维脆弱性与暴露度指标体系,并实现与水动力结果的显式耦合。} 在综合考虑物理、社会和经济三维因素的基础上,构建了包含建筑首层标高、人口密度、脆弱群体比例、关键基础设施数量等核心指标的脆弱性与暴露度体系,并通过归一化与AHP权重设定方法形成综合脆弱性指数$V_i$和暴露度$E_i$。结合三维水动力得到的水深、流速和淹没持续时间等危险度因子,构建了网格级危险度指数$H_i$和综合风险指数$R_i$,实现“水动力—脆弱性—暴露度”的统一刻画。

\item \textbf{揭示了城市洪涝风险的关键空间分布特征。} 威海滨海应急服务中心案例表明,综合风险呈现出“滨海高危险—内陆中等风险—局部高值斑块”的格局:喇叭状地形与下凹式停车区在风暴潮情景下形成极高风险带;沿主要道路和建筑密集街廓,受三维流场沿街汇流、人口高密度与资产高价值叠加影响,形成连续的较高风险条带;部分滨海绿地和低密度区域虽然危险度较高,但由于人口和资产暴露较低,综合风险等级相对较低。这些结果有助于识别需重点防护的街区与设施。

\item \textbf{验证了三维风险评估方法相对于二维方法的优势,并为后续章节提供统一数据底座。} 通过与二维浅水模型的对比,三维模型在淹没范围(0.41\,km$^2$ vs. 0.37\,km$^2$)、建筑受淹数量(4座实际淹没 vs. 7座二维估计)、水动力压力及垂直流速刻画等方面均表现出更高的精度和物理一致性。基于三维模型得到的危险度指数$H_i$、综合风险栅格$R_i$以及压力与淹没持续时间场,不仅为城市防洪减灾决策提供了重要的科学依据,也为第4章的风险驱动室内外一体化疏散路径规划提供了代价场输入,为第5章的构件级损伤评估与经济损失分析提供了荷载与筛选依据,形成“风险评估—疏散引导—构件损伤”的多尺度协同路线。
\end{enumerate}

% 注:本章引用的参考文献需要添加到bibliography.bib文件中
