% ********************************************************************
% ****************** Free to modify the content below ******************
% ********************************************************************

Driven by climate change and intensified urbanization, urban flooding has shown a simultaneous increase in frequency and severity. The intricate coupling among high-density built environments, interdependent infrastructure networks, and dynamic hydrometeorological processes amplifies cascading failures and systemic risks during extreme rainfall events. Although substantial progress has been made in City Information Modeling (CIM), hydrodynamic simulation, emergency evacuation, and damage assessment, existing studies still suffer from fragmented model development, insufficient semantic interoperability, and incomplete mechanistic linkages from risk identification to coordinated control. These limitations hinder fine-grained characterization of flood risk evolution and impede cross-department emergency decision-making in complex urban environments. To address these scientific and engineering challenges, this research centers on the “risk evolution mechanisms and collaborative control system for urban flooding,” and systematically investigates CIM semantic modeling, multi-scale hydrodynamics–risk coupling, integrated indoor–outdoor evacuation, component-level damage assessment, and emergency coordination system integration.

First, at the theoretical and methodological level, a CIM-based 5D–4V semantic modeling framework and an Object–Relation–Event (O–R–E) structure are developed for urban flood disaster representation. By integrating material, spatial, performance, cultural, and temporal dimensions, a comprehensive multi-dimensional disaster knowledge model is constructed to depict the evolution chain of “precipitation–runoff–inundation–exposure–damage.” This unified semantic and mechanistic foundation enables data aggregation, model collaboration, and business-driven decision processes across subsequent chapters (Chapter~2).

Second, for quantitative characterization of flood-induced processes, a multi-scale hydrodynamic modeling system is established by nesting three-dimensional Reynolds-averaged Navier–Stokes equations with two-dimensional shallow water equations. High-fidelity geometric modeling and boundary condition assimilation are incorporated to jointly capture localized flow structures and city-scale surface runoff. Building upon the simulated hydrodynamic fields, multi-indicator metrics for flood hazard, exposure, and vulnerability are formulated to support grid-based urban risk assessment, which is validated in a representative coastal city case (Chapter~3).

Third, to address evacuation challenges under dynamic flood conditions, a Multi-layer Generalized Network Model (MGNM) is introduced to achieve integrated indoor–outdoor evacuation modeling. Building topology derived from IFC and city road networks are encoded consistently within the CIM environment. Risk-aware, time-dependent cost functions are developed based on evolving flood conditions, enabling multi-objective evacuation optimization and crowd dynamics simulation. This framework supports safety and efficiency evaluation across different evacuation strategies (Chapter~4).

Fourth, to represent fine-grained impacts of flooding on the built environment, an Attribute–Building–Voxel (ABV) method is proposed for unified component-level representation and damage assessment. Hydrodynamic parameters such as water depth and flow velocity are mapped to component-level physical degradation and functional loss. Combined with vulnerability curves and recovery cost models, a component-level economic and service performance loss assessment system is constructed and visualized within a CIM-based 3D environment, providing quantitative support for resilience evaluation and recovery planning (Chapter~5).

Finally, at the system integration level, the above methods are incorporated into a collaborative emergency management system built on a layered microservice architecture. Through a data–model–business division, the system supports multi-source spatiotemporal data ingestion, modular model orchestration, and event-driven coordination of emergency workflows. A set of application scenarios—including early warning dissemination, evacuation guidance, resource allocation, and post-disaster assessment—is implemented and demonstrated in real urban settings (Chapter~6).

In summary, this research establishes a new framework for CIM-based semantic modeling and risk evolution analysis of urban flooding, proposes multi-scale hydrodynamics–risk coupling assessment techniques, introduces a risk-driven integrated indoor–outdoor evacuation modeling and optimization approach, and develops a component-level damage assessment and 3D visualization method. Building on these foundations, an engineered collaborative control system for urban flood emergency management is implemented. The research outcomes provide a systematic technical pathway for mechanism understanding, quantitative risk assessment, and coordinated disaster prevention in urban flooding scenarios, offering substantial theoretical value and practical significance for resilient city development and integrated watershed–urban flood mitigation.