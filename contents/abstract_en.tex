% ********************************************************************
% ****************** Free to modify the content below ******************
% ********************************************************************

This study addresses the growing compound risks that urban floods pose to densely built coastal cities under climate change and rapid urbanization. Extreme rainfall and storm surge events increasingly threaten population safety and critical infrastructure, while traditional 2D risk assessment methods and building-level evacuation designs fail to capture the three-dimensional hydrodynamic processes and cross-building crowd behavior in high-density environments. As a result, disaster risk assessment, evacuation planning and damage estimation are often fragmented, and emergency decision-making lacks a unified digital foundation and coupling mechanism. To overcome these limitations, this work takes City Information Modeling (CIM) as the core and develops an integrated, multi-scale technical framework for flood risk assessment and intelligent evacuation in coastal cities, enabling a closed loop of pre-event early warning, real-time response and post-event recovery analysis.

At the theoretical level, the study introduces the concepts of urban resilience governance under a complex systems perspective and digital twin cities into CIM research, and proposes a “digital resilience governance” framework. A “5D–4V” modeling approach is established, covering material, spatial, performance, cultural and temporal dimensions, and describing urban objects from four perspectives: scenarized, parameterized, interactive and intelligent. By constructing a unified semantic matrix and object relationship model for the city, the framework realizes cross-scale representation from building components and single buildings to blocks and the city as a whole, thus providing a consistent data and knowledge base for hydrodynamic simulation, evacuation network construction and component-level damage assessment.

For risk assessment, the study fuses multi-source data including UAV oblique photography, airborne LiDAR, CityGML and BIM/IFC to build a centimeter-level 3D terrain and semantic building model, supporting risk raster calculations at 0.5 m resolution. Based on the three-dimensional Reynolds-Averaged Navier–Stokes (RANS) equations, a nested hydrodynamic solver is developed with 65.8 million structured cells to resolve complex flow fields around building clusters. Compared with conventional 2D shallow-water models, the proposed method significantly improves the accuracy of inundation extent and pressure estimation, reducing errors by 10.8\% and 16.0\%, respectively. On this basis, a composite indicator system combining physical exposure, infrastructure vulnerability and socio-economic sensitivity is constructed to generate spatiotemporally continuous maps of flood hazard and population exposure, providing quantitative constraints for evacuation strategy design and damage analysis.

In the intelligent evacuation module, a Multi-purpose Geometric Network Model (MGNM) is proposed. Semantic objects such as rooms, corridors, doors, windows and stairs, as well as their topological relations, are automatically extracted from IFC models to form an indoor network strictly consistent with building geometry. This network is then seamlessly connected with the outdoor road network via building entrances and street nodes, yielding an integrated indoor-outdoor accessibility graph. A risk-aware cost function is designed that dynamically couples water depth, flow velocity and crowd density, together with a coarse-to-fine parallel path-searching strategy, enabling rapid global updates and local re-routing at the city scale. In a test case involving five interconnected buildings, the path computation time is reduced from 485.2 s to 127.6 s, improving efficiency by about 74\%. Under typical flood scenarios, risk-driven evacuation reduces total evacuation time by 15.3\%, while the share of paths running along high-risk edges falls to 9.4\%, significantly enhancing safety and robustness of crowd evacuation.

For post-event damage assessment, the study extends BIM component semantics under the Assembly-Based Vulnerability theory framework. Quantitative relationships are established between hydrostatic pressure, hydrodynamic pressure and buoyancy, and the geometry and material properties of components. Component-level damage criteria and cost mapping methods for repair or replacement are developed. By combining indicators such as component damage ratio, functional recovery time and investment payback, a component-level flood damage assessment and 3D visualization framework is built. It supports filtering, statistics and spatial querying by component type, damage level and economic loss, thereby enabling fine-grained analysis “from risk field to component.” Using the Weihai Coastal Emergency Service Center as a case study, the system identifies an overall component damage ratio of about 27\% (including 62\% for doors and windows and 54\% for wall finishes), estimates a direct economic loss of 514,000 RMB, and predicts a 21-day recovery period for key functions, thus providing quantitative support for phased resilience retrofitting and asset-management planning.

At the system level, high-resolution hydrodynamic simulation, MGNM-based evacuation modeling and component-level damage assessment are integrated into a unified CIM platform, forming a closed collaborative chain of “data acquisition–semantic governance–risk assessment–path planning–damage analysis–coordinated command.” The system adopts a layered microservice architecture with an event-driven mechanism, supporting online updating and coupled invocation of risk rasters, evacuation networks and damage results. For the Weihai case, the platform completes the entire workflow of “risk assessment–evacuation planning–damage analysis” within 14.6 minutes and simultaneously presents risk evolution and guidance routes to commanders and public terminals through a 3D visualization interface. Field deployment and drills demonstrate that the system meets practical emergency-management requirements in terms of minute-level timeliness, cross-department collaboration and intuitive scenario representation.

Overall, this study proposes a systematic CIM-based framework for full-process urban disaster management and digital resilience governance; establishes a multi-scale, coupled technical chain linking high-precision 3D flood risk assessment, risk-driven indoor–outdoor integrated evacuation and component-level flood damage analysis; and implements and validates a prototype system for coastal emergency management in a real-world scenario. The findings provide a replicable technical pathway for building integrated digital disaster-prevention systems that combine risk early-warning, intelligent evacuation and post-disaster recovery in coastal cities, and lay a foundation for future research on multi-hazard coupling, real-time data assimilation and refined behavior modeling.