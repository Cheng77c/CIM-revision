% ********************************************************************
% ****************** Free to modify the content below ******************
% ********************************************************************

Under the dual pressures of global climate change and rapid urbanization, compound disasters triggered by extreme rainfall and storm surges are occurring with increasing frequency, posing severe cascading risks to high-density urban built-up areas. Existing studies are often constrained by 2D terrain generalization and static building models, which makes it difficult to accurately capture complex three-dimensional flow fields, cross-building evacuation behaviors, and component-level damage evolution mechanisms at fine scales. This leads to a pronounced "semantic gap” and "model islands” between risk assessment, emergency decision-making, and post-disaster recovery, and highlights the lack of a unified digital foundation. To address these scientific bottlenecks, this study adopts City Information Modeling (CIM) as the digital base and introduces the concept of resilience governance from a complex systems perspective. It proposes a "Mechanism–Data–Knowledge–Model” (MDKM) four-engine integrated technical framework for urban flood digital resilience governance, achieving full-chain integration and closed-loop validation from macro-scale urban substrates to micro-scale component damage.

First, a CIM-driven MDKM theoretical framework and modeling paradigm is developed. In response to the multi-source and heterogeneous characteristics of urban complex systems, a "5D–4V” CIM modeling system is proposed, covering material, space, performance, culture, and time. On this basis, the MDKM methodology is constructed using an Object–Relation–Event (O–R–E) model to establish cross-scale semantic mapping from building components to urban blocks. This framework provides a unified spatiotemporal reference and semantic foundation for multi-physics simulation, behavioral dynamics modeling, and knowledge reasoning, effectively enabling the collaboration of mechanism-based models and data-driven methods.

Second, a three-dimensional hydrodynamic risk assessment method based on a high-precision CIM base is proposed. By integrating UAV oblique photogrammetry, airborne LiDAR, CityGML, and BIM/IFC data, a centimeter-level urban 3D terrain and building semantic model is constructed. An embedded-grid hydrodynamic solver based on the 3D Reynolds-Averaged Navier–Stokes (RANS) equations is developed to finely characterize complex flow fields and vertical vortex structures around dense building clusters. Validation shows that, compared with traditional 2D shallow-water models, the errors in inundation extent and water pressure estimation are reduced by 10.8\% and 16.0\%, respectively, demonstrating deep coupling between "mechanism” and "data” and providing physically constrained quantitative inputs for subsequent decision-making.

Third, a risk-driven integrated indoor–outdoor intelligent evacuation model is established. To overcome spatial disconnection in evacuation path planning, a Multi-purpose Geometric Network Model (MGNM) is proposed. By automatically parsing IFC semantics to extract building topology and constructing entrance–street connection strategies, the model achieves seamless integration of indoor and outdoor circulation networks. A risk-aware cost function that incorporates water depth, flow velocity, and crowd density is introduced, and a parallel path-solving strategy combining coarse- and fine-grained computations is designed to realize coordinated "knowledge–model” driving. Experiments show that under typical flood scenarios, the risk-driven evacuation strategy shortens total evacuation time by approximately 15.3\% and reduces the proportion of high-risk paths to 9.4\%, significantly improving both efficiency and safety.

Fourth, a multi-physics coupled, component-level flood damage assessment system is constructed. The Assembly-Based Vulnerability (ABV) theory is extended by embedding multiple physical mechanisms—hydrostatic pressure, hydrodynamic pressure, buoyancy, and water contact effects—into BIM semantics. A quantitative mapping model among hydraulic loads, component attributes, and economic losses is established, enabling refined classification and 3D visualization of building component damage states. Taking the Weihai Coastal Emergency Service Center as a case study, the overall component damage rate is identified as approximately 27\%, and high-incidence damage areas for doors, windows, and finish materials are accurately located, verifying the effectiveness of "mechanism–knowledge” integration in post-disaster recovery assessment.

Fifth, an urban disaster prevention and emergency coordination system is developed and validated in engineering practice. The high-precision hydrodynamic simulation, MGNM-based evacuation network, and component-level damage assessment modules are integrated into a unified CIM platform to form a closed-loop workflow of "data acquisition–semantic governance–risk assessment–path planning–damage analysis–collaborative command." The system adopts a layered microservice architecture, can complete the full computational workflow in about 14.6 minutes, and supports multi-terminal collaboration through a 3D visualization interface. Field exercises demonstrate that the system provides significant engineering value in terms of response timeliness, cross-departmental coordination, and scenario intuitiveness.

In summary, this study proposes an MDKM-driven digital resilience governance framework at the theoretical level, overcomes key technical challenges in cross-scale disaster modeling and multi-model coupling at the methodological level, and forms a replicable and scalable urban emergency management solution at the engineering level, thereby providing scientific support and technical foundations for building smart and resilient cities.