% ********************************************************************
% ****************** Free to modify the content below ******************
% ********************************************************************

Under the coupled effects of global climate change and rapid urbanization, compound extreme rainfall–storm surge events are increasing significantly, and the cascading risks posed by urban flooding to population safety, critical infrastructure, and urban operation systems continue to accumulate. However, existing studies mostly conduct risk assessment and evacuation design based on two-dimensional terrain and static building models, which makes it difficult to capture three-dimensional hydrodynamic processes in high-density built environments, cross-building evacuation behavior, and component-level damage mechanisms. This leads to a broken chain linking risk assessment, evacuation decision-making, and damage analysis, and emergency management lacks a unified digital foundation and coordination mechanism. To address these scientific bottlenecks, this study adopts City Information Modeling (CIM) as the digital foundation and proposes a multi-scale integrated technical framework for urban flood disasters, achieving end-to-end integration of flood risk assessment, risk-driven intelligent evacuation, and post-disaster damage assessment, and completing system integration and engineering validation in a typical urban district.

At the theoretical and modeling level, this study introduces the concepts of urban resilience governance from a complex systems perspective and the digital twin city, and proposes an overall digital resilience governance framework for full-process management of urban flooding. A 5D–4V CIM modeling paradigm is developed, covering the five dimensions of material, spatial, performance, cultural, and temporal attributes, and systematically organizing urban objects and their behavioral semantics from four perspectives: scenario-based, parametric, interactive, and intelligent. By designing a unified urban semantic matrix and object relationship model, the framework enables consistent cross-scale representation from building components to individual buildings, blocks, and the entire city, thereby providing homogeneous data and a knowledge base for three-dimensional hydrodynamic simulation, evacuation network construction, and component-level damage analysis.

In terms of flood risk assessment, the study integrates multi-source spatial information such as UAV oblique photogrammetry, airborne LiDAR, CityGML, and BIM/IFC to construct a high-precision urban three-dimensional terrain and building semantic model, forming a CIM data foundation that supports fine-grid computation. Based on the three-dimensional Reynolds-averaged Navier–Stokes (RANS) equations, a nested hydrodynamic solver is developed to accurately characterize complex flow fields around building clusters. Compared with traditional two-dimensional shallow-water models, the errors in inundation extent and water pressure estimation are reduced by approximately 10.8\% and 16.0\%, respectively. On this basis, a composite indicator system coupling physical exposure, infrastructure vulnerability, and socio-economic sensitivity is constructed to generate spatio-temporally continuous maps of flood hazard and population exposure, providing a physically constrained quantitative basis for subsequent evacuation path planning and component-level damage assessment.

In terms of intelligent evacuation, this study proposes a Multi-purpose Geometric Network Model (MGNM), which automatically extracts semantic objects such as rooms, corridors, doors and windows, and staircases, as well as their topological relationships, from IFC models to construct indoor circulation networks that are highly consistent with building geometry. These networks are then automatically connected to the urban road network through entrance and exit nodes to form an integrated indoor–outdoor reachability graph. A risk-aware cost function is constructed based on water depth, flow velocity, and crowd density, and a coarse-to-fine parallel path-solving strategy is designed to support dynamic updating and local replanning of the city-scale evacuation network. Case study results show that, in a scenario with five interconnected buildings, path computation time is reduced from 485.2 s to 127.6 s, improving computational efficiency by about 74\%. Under typical flood scenarios, risk-driven evacuation can shorten total evacuation time by approximately 15.3\%, while the proportion of paths along high-risk edges decreases to 9.4\%, significantly enhancing the safety and robustness of crowd evacuation.

In terms of post-disaster damage assessment and system integration, this study extends the Assembly-Based Vulnerability (ABV) theory by coupling hydrostatic pressure, hydrodynamic pressure, and buoyancy loads into BIM component semantics, establishing quantitative mappings among water loads, geometric properties, and material parameters. A component-level flood damage classification criterion and a repair/replacement cost mapping method are proposed, and a three-dimensional visual evaluation framework is developed, enabling filtering, statistics, and spatial queries by component type, damage level, and economic loss. Using the Weihai Coastal Emergency Service Center as a case study, the system identifies an overall component damage ratio of about 27\%, with particularly severe damage to doors, windows, and wall finishes, and estimates direct economic losses at approximately 514,000 CNY; key functions are expected to be restored within 21 days. Building on this, the high-precision hydrodynamic simulation, MGNM-based evacuation network, and component-level damage assessment modules are integrated into a unified CIM platform, forming a closed-loop collaborative system encompassing data acquisition, semantic governance, risk assessment, path planning, damage analysis, and coordinated command. The system adopts a layered microservice architecture and an event-driven mechanism, and can complete the full computational workflow of risk assessment, evacuation planning, and damage analysis for the study area in about 14.6 minutes. Through a three-dimensional visualization interface, it simultaneously presents risk evolution and evacuation guidance paths to commanders and public terminals. Practical deployment and multiple drills verify the system’s engineering applicability in terms of response timeliness, cross-departmental collaboration, and scenario intelligibility.

In summary, this study proposes a CIM-driven framework for digital resilience governance of urban flooding at the theoretical level, establishes a multi-scale coupled technical chain of high-precision three-dimensional hydrodynamic risk assessment, risk-driven integrated indoor–outdoor evacuation, and component-level flood damage assessment at the methodological level, and develops and validates a prototype urban emergency management system that can serve real-world operational scenarios at the engineering level. The results provide systematic technical support for building a digital disaster prevention and mitigation system that integrates risk early warning, intelligent evacuation, and post-disaster recovery, and lay a methodological and engineering foundation for subsequent research on multi-hazard coupling, real-time data assimilation, and refined modeling of crowd behavior.