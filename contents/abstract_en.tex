% ********************************************************************
% ****************** Free to modify the content below ******************
% ********************************************************************

Under the combined effects of global climate change and rapid urbanization, extreme rainfall and storm surge events are occurring with increasing frequency, leading to accumulating compound flood risks for urban populations, critical infrastructure, and urban operation systems. Existing studies mostly conduct risk assessment and evacuation design on the basis of two-dimensional topography and static building units, which are insufficient to capture three-dimensional hydrodynamic processes in dense built environments, inter-building evacuation behavior, and component-level damage mechanisms. As a result, flood risk assessment, evacuation decision-making, and damage analysis remain largely decoupled, and emergency management lacks a unified digital foundation and coordination mechanism. To address these issues, this dissertation develops a City Information Modeling (CIM)–based multi-scale integrated technical framework for urban flood disasters, which links flood risk assessment, risk-driven intelligent evacuation, and post-disaster damage assessment, and is implemented and validated in a real engineering context.

In terms of theory and modeling, a “digital resilience governance” framework for the whole life cycle of urban flood management is proposed by introducing the perspectives of complex adaptive urban systems and digital twin cities. A “5D–4V” CIM modeling methodology is established, covering five dimensions—material, spatial, performance, cultural, and temporal—and organizing urban objects and their behavioral semantics from four perspectives: scenario-oriented, parametric, interactive, and intelligent. A unified urban semantic matrix and object relationship model are designed to achieve consistent cross-scale representation from building components and individual buildings to blocks and the entire city, thereby providing a common data and knowledge basis for three-dimensional hydrodynamic simulation, evacuation network construction, and component-level damage analysis.

For flood risk assessment, this study integrates multi-source spatial data including UAV-based oblique photogrammetry, airborne LiDAR, CityGML, and BIM/IFC to construct a high-precision three-dimensional urban terrain and building semantic model, forming a CIM data backbone that supports fine-resolution grid-based computations. A nested hydrodynamic solver is developed based on the three-dimensional Reynolds-Averaged Navier–Stokes (RANS) equations to resolve complex flow fields around building clusters. Compared with traditional two-dimensional shallow-water models, the proposed approach reduces the errors in inundation extent and water pressure estimation by approximately 10.8\% and 16.0\%, respectively. On this basis, a composite indicator system is built by coupling physical exposure, infrastructure vulnerability, and socio-economic sensitivity, yielding spatio-temporally continuous maps of flood hazard and population exposure, which provide a physically constrained quantitative basis for subsequent evacuation path planning and component-level damage assessment.

For intelligent evacuation, a Multi-purpose Geometric Network Model (MGNM) is proposed. Semantic objects such as rooms, corridors, doors, windows, and staircases, together with their topological relations, are automatically extracted from IFC models to construct an indoor circulation network that is highly consistent with the as-built building geometry. This indoor network is then automatically connected to the outdoor road network via entrance and exit nodes, forming an integrated indoor–outdoor accessibility graph. A risk-aware cost function that jointly considers water depth, flow velocity, and crowd density is designed, together with a coarse-to-fine parallel path-search strategy, enabling dynamic updating and local re-routing of evacuation paths at the city scale. Numerical experiments on a scenario with five interconnected buildings show that the path computation time is reduced from 485.2 s to 127.6 s, improving computational efficiency by about 74\%. Under a typical flood scenario, risk-driven evacuation reduces the total evacuation time by approximately 15.3\%, while the proportion of paths traversing high-risk edges decreases to 9.4\%, significantly enhancing the safety and robustness of pedestrian evacuation.

For post-disaster damage assessment and system integration, this study extends BIM component semantics on the basis of the Assembly-Based Vulnerability theory, establishing quantitative relationships between hydrostatic pressure, hydrodynamic pressure, buoyancy, and component geometry and material properties. Component-level flood damage criteria and repair/replacement cost mapping methods are proposed, and a three-dimensional visualization framework is developed that supports filtering, statistical analysis, and spatial queries by component type, damage state, and economic loss. Using the Weihai Coastal Emergency Service Center as a case study, the system identifies an overall component damage ratio of about 27\%, with doors, windows, and wall finishes being particularly affected, and estimates direct economic losses of approximately 0.514 million CNY, while the recovery time for key functions is about 21 days. Building on this, the high-precision hydrodynamic simulation, MGNM-based evacuation network, and component-level damage assessment modules are integrated into a unified CIM platform, forming a closed-loop collaborative system of “data acquisition – semantic governance – risk assessment – path planning – damage analysis – coordinated command.” The system adopts a layered microservice architecture and an event-driven mechanism, and can complete the full workflow of “risk assessment – evacuation planning – damage analysis” for the study area within about 14.6 minutes. Through a three-dimensional visualization interface, the system simultaneously presents risk evolution and evacuation guidance paths to commanders and public terminals. Field deployment and multiple drills demonstrate its engineering applicability in terms of response timeliness, cross-departmental coordination, and scenario intelligibility.

In summary, this dissertation proposes a CIM-driven digital resilience governance framework for urban flooding at the theoretical level, develops a multi-scale coupled technical chain of “high-resolution three-dimensional hydrodynamic risk assessment – risk-driven integrated indoor–outdoor evacuation – component-level flood damage assessment” at the methodological level, and constructs and validates a prototype urban emergency management system that serves real operational scenarios at the engineering level. The research outcomes provide systematic technical support for building a digital disaster prevention and mitigation system that integrates risk early warning, intelligent evacuation, and post-disaster recovery, and lay a methodological and engineering foundation for future studies on multi-hazard coupling, real-time data assimilation, and fine-grained modeling of crowd behavior.