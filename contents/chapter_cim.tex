\chapter{CIM五维四视角建模方法}

如前章所述,传统的二维地理信息模型和静态风险评估框架往往难以全面刻画复杂城市系统中多要素、多时态的动态风险过程。随着城市空间形态、基础设施系统与社会活动数据的持续丰富,构建一种可表达物理、空间、性能、文化与时间多维语义的统一模型,已成为实现高精度风险感知与智能决策的关键前提。基于此,本研究提出了一种面向城市灾害风险分析的CIM(City Information Modeling)语义建模框架,以实现从数据集成到语义推理的多层次映射,为后续的风险计算与动态仿真提供结构化支撑。

CIM 模型不仅是对城市要素的几何与属性描述,更是一个跨尺度、跨领域的语义融合系统。它在空间上整合建筑、基础设施与自然环境的几何特征;在时间上追踪对象状态的演化过程;在逻辑上通过对象—关系—事件(Object--Relation--Event)模型建立起实体之间的动态语义联系。通过对不同层级数据的统一建模与关联推理,CIM 能够支持从局部设施到整体城市的多尺度认知,为灾害场景中的水动力学计算、结构响应分析和人群疏散模拟提供统一的数据基础。

相比传统的 BIM(Building Information Modeling)或 GIS(Geographic Information System),CIM 更加强调语义完整性与时空一致性。它不仅关注对象的几何属性和空间位置,还强调运行状态、性能指标、交互关系等多维语义的持续更新,从而实现对象—过程—知识一体化的数字孪生映射。在这一基础上,本研究进一步构建了五维—四视角(5D--4V)建模方法,以体系化地揭示 CIM 模型的结构层次与认知机制。

五维体系分别对应城市系统的物质、空间、性能、文化与时间五个核心要素;四视角则从场景化、参数化、互动化与智能化四个方向刻画模型的认知与推理机制。二者相互作用,构成了从静态实体到动态认知、从几何数据到语义知识的多层级建模框架。该框架既可作为后续灾害风险模拟的语义支撑,也为城市运行状态的实时感知与智能推理提供统一逻辑基础。

综上所述,本节提出的 CIM 建模框架为高精度风险评估提供了统一的语义底座与数据认知结构。下一节将进一步阐述五维四视角建模方法的内在逻辑与层次结构,揭示其在风险建模与智能决策中的支撑作用。

\section{五维四视角建模方法}

在前节的 CIM 建模概述中,构建了城市信息模型(CIM)的整体语义框架,明确了其作为多源数据融合与语义推理载体的基础作用。为进一步提升模型的表达能力与智能化水平,本节提出一种面向城市系统的五维四视角建模方法(5D--4V Modeling Framework),以实现从物理实体到认知语义的多层级映射。该方法以五维刻画城市对象的多重属性空间,以四视角揭示模型认知、分析与学习的多元路径,从而在结构层面与语义层面上支撑复杂场景下的风险计算与动态决策。

如图~\ref{fig:five_dimensions}所示,五维体系包括物质维度(Material Dimension)、空间维度(Spatial Dimension)、性能维度(Performance Dimension)、文化维度(Cultural Dimension)与时间维度(Temporal Dimension),分别对应城市系统中结构—空间—功能—人文—演化的核心构成要素。

这些维度不仅刻画了城市对象的静态形态与动态过程,也反映了灾害风险形成的多因素耦合机理。例如,物质维度支撑几何与材料属性的高精度表达;性能维度刻画运行状态与能耗响应;时间维度用于记录生命周期演化与历史版本;而文化维度与空间维度共同描述社会行为与场地关系的动态适应性。通过在 CIM 模型中引入这五个维度,可形成从几何实体到语义知识的完整映射链,为灾害过程的可计算建模提供数据与语义双重支撑。

\begin{figure}[htbp]

\centering

\includegraphics[width=0.8\textwidth]{PC-CIM/图1.png}

\caption{五维体系示意图}

\caption*{Figure~\thefigure~ Schematic Diagram of the Five-Dimensional System}

\label{fig:five_dimensions}

\end{figure}

如图~\ref{fig:four_views}所示,与五维体系相对应,四视角描述了模型认知与交互的四种机制:

\begin{itemize}

\item 场景化视角(Scenario-based View)——以具体的应用语境为驱动,将模型嵌入灾前、灾中与灾后的全过程场景中,通过任务定义与语境绑定,建立情境–事件–响应–反馈的闭环逻辑;

\item 参数化视角(Parametric View)——通过多源数据参数化表达,实现对象属性、关系及约束条件的形式化描述,支撑模型的灵活组合与批量生成;

\item 互动化视角(Interactive View)——基于人–机–物多层交互机制,将 IoT 感知、可视化控制与仿真反馈纳入统一语义框架,实现模型与现实的实时对齐;

\item 智能化视角(Intelligent View)——借助知识图谱、深度学习与强化学习技术,对对象行为与系统演化进行预测与优化,使模型具备自学习、自校正的认知能力。

\end{itemize}

\begin{figure}[htbp]

\centering

\includegraphics[width=0.8\textwidth]{PC-CIM/图2.png}

\caption{四视角体系示意图}

\caption*{Figure~\thefigure~ Schematic Diagram of the Four-Perspective System}

\label{fig:four_views}

\end{figure}

上述五维与四视角相互映射,构成一个维度—视角双层语义矩阵(表~\ref{tab:mapping_matrix}所示),揭示了从结构到智能的多层建模逻辑。例如,在物质维下,场景化对应建筑与设施场景,参数化体现 BIM 几何约束,互动化反映 IoT 感知反馈,而智能化则表现为结构健康监测与状态预测。同样,在性能维下,四视角分别映射为运行工况、IoT 参数、实时监控与调度优化。

这种矩阵化映射不仅确保了 CIM 模型的语义完整性,也为跨维度推理、跨场景分析提供了结构化路径。

\begin{table}[htbp]

\centering

\caption{五维与四视角映射矩阵}

\caption*{Table~\thetable~ Mapping Matrix Between the Five-Dimensional System and the Four-Perspective System}

\label{tab:mapping_matrix}

\small

\begin{tabular}{p{2cm}p{2.2cm}p{2.2cm}p{2.2cm}p{2.2cm}}

\toprule

维度 / 视角 & 场景化 & 参数化 & 互动化 & 智能化 \\

\midrule

物质维 & 建筑与设施场景 & BIM 参数绑定 & IoT 感知反馈 & 智能结构监测 \\

空间维 & 城市空间场景 & GIS 空间参数 & 空间交互控制 & 空间动态优化 \\

性能维 & 能源 / 交通运行场景 & IoT 传感参数 & 实时监控反馈 & 预测与调度学习 \\

文化维 & 文化遗产保护场景 & 文化属性参数化 & 群众参与与展示 & 舆情 / 行为数据分析 \\

时间维 & 生命周期建模 & 时间参数 & 历史与实时融合 & 演化规律学习 \\

\bottomrule

\end{tabular}

\normalsize

\end{table}

通过上述映射关系,CIM 模型在结构上实现了多维一致性,在功能上实现了多层交互性。其核心思想是将城市系统的多源异构数据经由维度建模—视角推理—语义映射转化为可计算、可演化、可学习的认知实体。这种方法不仅拓展了传统 CIM 在语义表达与时序管理上的边界,也为后续章节中基于 CIM 的高精度风险评估提供了坚实的语义与逻辑支撑。

\subsection{维度定义与内涵}

\subsubsection{物质维度}

物质维度是 CIM 模型的基础层,主要用于刻画城市物理实体的结构组成、材料属性与几何形态。它通过对建筑、基础设施与地形环境的高精度数字化建模,建立起城市系统在形态—结构—材料层面的统一描述,为风险模拟与结构分析提供可计算的物理支撑。

在数据来源层面,物质维度整合了多源异构的基础数据,包括建筑信息模型(BIM)、地理信息系统(GIS)、激光雷达(LiDAR)点云、摄影测量(SfM/MVS)与遥感影像等。通过几何对齐与拓扑重构,形成具有一致空间基准的三维物理场景。在此基础上,可进一步结合工程参数(如结构类型、层数、高程、材料强度、孔隙率等),生成具备物理语义的对象属性表。该过程不仅保证模型几何精度,也使其具备面向灾害场景的结构可计算性与属性可推理性。

在模型表达层面,物质维度通过 BIM 与 GIS 的深度融合实现了从设施级到城市级的多尺度映射。BIM 模型提供建筑与构筑物的部件级几何细节与材料信息,而 GIS 则负责空间定位与环境关系表达。两者的融合使得 CIM 能够在统一的坐标体系中实现建筑物、道路、管网、地形等要素的关联表达。这种结构化的物理建模不仅支持局部尺度下的结构力学分析(如有限元应力计算、振动响应评估),还可与城市级的水动力学模拟(RANS 模型)耦合,实现跨尺度、多物理场的联动计算。

在语义拓展层面,物质维度提供了风险评估所需的物理现实底座。通过与性能维度中的运行状态、时间维度中的演化过程相结合,可实现建筑及基础设施在灾害作用下的动态响应分析。例如,在台风、暴雨或地震等灾害场景下,物质维度的结构信息可作为输入条件,参与对受力状态、失稳风险与次生灾害传播路径的预测。通过这种从几何实体到力学语义的扩展,CIM 模型在灾害风险表征中由可视化模型转化为可计算模型。

综上,物质维度不仅是 CIM 的几何基础,更是城市风险建模的物理核心。它通过高精度的结构表达、属性融合与语义对齐,建立了从实体建模到风险计算的逻辑通道,为后续空间关系推理与动态模拟提供坚实的底层支撑。

\subsubsection{空间维度}

空间维度是 CIM 模型的组织层,负责描述城市对象之间的空间关系、拓扑连接与场景分布规律。相较于物质维度的几何与结构建模,空间维度更关注对象之间的相对位置、邻接关系与功能区逻辑,是城市系统实现语义一致性与场景表达的关键环节。通过对空间网络的多尺度建模,CIM 能够支持从局部设施到区域空间的统一认知,为灾害传播模拟与应急路径规划提供空间语义支撑。

在空间表达上,CIM 模型采用基于拓扑图(Topological Graph)与空间网格(Spatial Mesh)的联合建模方式。建筑、道路、管网与自然地形等对象在此层被抽象为节点(Node)与边(Edge),节点代表实体要素,边表示其空间或功能关联关系,从而形成一个具备方向性与层级结构的城市语义图(City Semantic Graph)。

在该语义图中,不同类型的空间关系可被明确定义,如:

\begin{equation}

R_s = \{ (o_i, o_j, \text{adjacent}), (o_i, o_k, \text{contains}), (o_i, o_l, \text{connected}) \}

\end{equation}

其中 $R_s$ 表示空间关系集,涵盖相邻、包含、连通等多种拓扑类型。这使 CIM 能够在统一语义框架下实现空间对象的检索、定位与动态更新。

在模型实现层面,空间维通过 BIM--GIS 融合与网络化建模(MGNM, Multi-Granularity Network Modeling)实现多尺度的空间统一。微观层面,建筑与设施通过 BIM 模型精确描述结构内部的空间构成;中观层面,街区、道路与地形通过 GIS 拓扑表达地理布局与功能分区;宏观层面,城市整体以空间网络形式抽象出不同功能单元(居住、交通、商业、应急等)的关联模式。这种多粒度的空间建模方法确保了 CIM 能在灾害模拟中实现从局部到整体的空间推理能力,支持如洪涝水流扩散、地震波传播与人员疏散路径优化等过程的动态计算。

此外,空间维还引入了语义分区与可达性分析机制。通过构建空间可达性矩阵:

\begin{equation}

A_{ij} =

\begin{cases}

1, & \text{if } d(o_i, o_j) \leq \tau \\

0, & \text{otherwise}

\end{cases}

\end{equation}

其中 $d(o_i, o_j)$ 为两对象间的最短空间距离,$\tau$ 为可达性阈值。该矩阵用于动态评估灾害场景中不同功能区的连通性与可达性,为应急路径规划、避险区选址及资源调度提供空间分析基础。在此基础上,CIM 模型可结合 IoT 感知数据、交通流监测与人群迁移仿真,形成具备实时感知与预测能力的空间认知网络。

从语义层面看,空间维度是连接物理实体与语义推理的中介层。它不仅定义了对象在何处,更揭示了对象与对象之间如何相互作用。通过空间维度的逻辑建模,CIM 实现了从静态几何结构到动态场景网络的过渡,为性能维度中的系统行为建模奠定了空间拓扑基础。在后续章节中,空间维度与性能维度将共同支撑灾害场景下的流体传播模拟、路径优化与风险扩散分析,构成城市系统时空一体化的核心框架。

\subsubsection{性能维度}

性能维度是 CIM 模型中连接结构实体与运行行为的关键层,用于表征城市系统在不同工况下的动态响应、能效状态与服务性能。在灾害风险评估场景中,性能维度不仅反映基础设施与环境系统的物理运行特征,也刻画其在外界扰动下的功能退化与恢复能力,是实现从几何到机理的核心纽带。

在数据层面,性能维度以传感器、监控系统和仿真模型为主要数据源,构建多维时序数据集,涵盖能源消耗、交通流量、水文变化、结构应变、温度分布等多类型运行参数。这些数据经由 IoT 感知网络与数据中台汇聚至 CIM 模型,实现对物理对象状态的实时监控与历史追溯。性能数据在模型中以参数化形式存储:

\begin{equation}

P_i(t) = \{ p_{i1}(t), p_{i2}(t), \dots, p_{in}(t) \}

\end{equation}

其中 $P_i(t)$ 表示对象 $o_i$ 在时刻 $t$ 的性能向量,涵盖能耗、载荷、流速、位移等多维状态变量。这种结构化表达使 CIM 模型能够在灾害模拟中实现动态场景驱动的状态更新。

在模型机制上,性能维通过多物理场联合建模(Multiphysics Coupling),将结构、能量、流体及交通等要素纳入统一求解框架。例如,在城市内涝模拟中,CIM 模型可将建筑体与地形信息作为边界条件输入至雷诺平均纳维–斯托克斯方程(RANS)系统:

\begin{equation}

\nabla \cdot \mathbf{u} = 0, \quad

\rho (\mathbf{u} \cdot \nabla)\mathbf{u} = -\nabla p + \mu \nabla^2 \mathbf{u} + \mathbf{F}

\end{equation}

其中,速度场 $\mathbf{u}$、压力场 $p$ 与外力项 $\mathbf{F}$ 由 CIM 物理层提供的几何与材料信息共同决定。这种多源数据与物理方程的协同求解,使 CIM 模型能够在动态条件下描述洪水流速、结构受力、交通拥堵等复杂现象,形成数据—模型—语义三位一体的性能分析机制。

性能维还引入了数据同化(Data Assimilation)与模型校准(Model Calibration)机制,以缩小仿真与实测之间的偏差。通过集合卡尔曼滤波(EnKF)等方法,可在每个时间步根据观测数据修正模型状态向量:

\begin{equation}

\hat{x}_k = x_k + K_k (z_k - Hx_k)

\end{equation}

其中,$\hat{x}_k$ 为修正后的状态,$K_k$ 为增益矩阵,$z_k$ 为实测观测量。该过程确保 CIM 在动态模拟中具备自适应校正能力,从而提升风险评估的预测精度。

在语义层面,性能维度为灾害风险建模提供了从静态结构风险到动态功能风险的扩展路径。它能够定量描述不同系统在灾害作用下的承载能力、响应延迟与恢复过程,为后续章节中高精度风险评估方法的脆弱性指标构建与时序仿真提供直接支撑。同时,性能维度与时间维度相结合,可进一步形成基于时序学习与预测优化的动态风险认知体系,为城市系统的主动防御与自适应调度奠定基础。综上,性能维度是 CIM 模型的动力核心,体现了系统从物理状态到功能表现的映射机制。通过数据驱动与物理驱动的融合,它不仅实现了城市多系统的实时监测与响应预测,也为风险评估、灾害模拟和智能决策提供了坚实的模型支撑。

\subsubsection{文化维度}

文化维度是 CIM 模型中体现人文语义与社会行为特征的重要组成部分,旨在捕捉城市系统中人、物、空间之间的认知联系与行为逻辑。与物质、空间和性能等物理维度不同,文化维度强调社会要素、文化价值与公众行为模式对城市运行和风险响应的深层影响。在灾害风险评估语境下,文化维度不仅用于刻画城市形态与历史文脉的延续性,更用于表达社会系统的韧性特征与行为反馈机制。

在数据层面,文化维度整合了来自城市遗产保护、社会治理与公众参与的多源信息。包括遗产清单、风貌控制线、历史地段保护范围、非遗活动数据,以及社交媒体、舆情平台、问卷调查等反映公众认知与行为的数据集。通过对这些数据的结构化处理与语义标签化,可建立多层文化属性向量:

\begin{equation}

\kappa_i = \{ \text{symbolicity}, \text{legibility}, \text{identity}, \text{attachment} \}

\end{equation}

其中,$\kappa_i$ 表示空间对象 $o_i$ 的文化特征向量,分别对应其象征性、可读性、认同度与场所依恋度。这种参数化的表达形式使文化语义能够与物理与空间特征协同建模,形成多维度的综合认知空间。

在建模机制上,文化维基于多主体系统(ABM, Agent-Based Modeling)与多层网络(MLN, Multi-Layer Network)构建人—空间—事件耦合关系图:

\begin{equation}

G_C = (V_P, V_S, V_E, E_{PS}, E_{PE}, E_{SE})

\end{equation}

其中,$V_P, V_S, V_E$ 分别代表人群、空间与事件节点,$E_{PS}, E_{PE}, E_{SE}$ 表示不同类型的交互关系。该图模型能够模拟个体与群体在灾害场景中的感知、决策与迁移行为,实现文化层面对风险认知与响应模式的量化表达。例如,不同人群的风险感知阈值、避险倾向或信息传播速度,可通过行为参数设定和网络权重学习自动调整,从而形成动态的社会行为场(Social Behavior Field)。

在语义层面,文化维度使 CIM 模型具备社会智能性。通过引入舆情分析与行为挖掘机制,可实时捕捉公众对灾害风险的认知态势与参与反馈,从而支撑多主体协同决策。这种机制使模型从物理空间重建扩展到社会语义表达,实现从客观环境到主观认知的跨域映射。在灾害应对中,文化维度不仅有助于识别高风险区域中的社会脆弱群体,也可为风险沟通与公众教育提供数据支撑。

此外,文化维度在空间治理与城市设计中还具有重要意义。通过文化语义参数与空间形态数据的耦合,可构建基于场所认知的空间优化模型,评估城市更新或灾后重建方案对历史风貌与公众认同的影响。这种结合人文、空间与行为的多层语义建模,使 CIM 模型超越传统的工程化范式,具备社会可解释性与决策辅助功能。

综上,文化维度在 CIM 模型中承担着连接社会认知—空间表达—风险响应的桥梁作用。它使模型从结构性系统演化为认知性系统,从而在风险评估、灾害响应及城市可持续治理中实现更高层次的智能化与人本化表达。

\subsubsection{时间维度}

时间维度是 CIM 模型的动态核心,用于刻画城市对象、关系与事件在时间轴上的演化规律与生命周期过程。它不仅反映对象状态的时间连续性,还表达不同要素之间的时序关联与演化逻辑,从而支撑城市系统在建造—运营—更新—再生全周期中的动态管理与预测分析。时间维度的引入,使 CIM 模型从静态结构描述转化为可演化的时空知识体系,为高精度风险评估提供了历史、实时与预测三层信息支撑。

在数据形态层面,时间维度整合了多源时序数据,包括施工与维护日志、传感器时序数据、历史事件记录、应急响应轨迹、版本化几何与语义信息等。这些数据通过统一的时间戳机制与事务日志体系进行管理,从而形成支持双时态(Bitemporal)存储的多层数据库。该数据库既保留了对象属性随时间的变化轨迹,也支持事件驱动的状态更新与回溯分析。其形式化表示为:

\begin{equation}

L = \{ (o_i, [t_s, t_e), \text{state}) \}

\end{equation}

其中,$L$ 表示对象生命周期集合,$[t_s, t_e)$ 为对象有效时间区间,$\text{state}$ 为该期间的状态描述。通过此结构,CIM 能够追踪对象从创建、运行到退役的全过程,支持灾害前后状态对比与时序分析。

在模型机制上,时间维度引入时序数据建模与演化预测机制,结合统计学习与深度时序网络,实现城市系统状态的动态重构与趋势预测。典型方法包括循环神经网络(RNN)、时间卷积网络(TCN)与基于状态空间模型(SSM, State Space Model)的混合结构,用于捕捉复杂时序中的长期依赖与非线性演化特征。在灾害风险评估场景中,时间维度可用于分析雨量序列与流量变化、交通流密度演化、建筑结构老化与性能衰减等过程,并基于模型预测未来可能的风险趋势。其动态更新过程可形式化表示为:

\begin{equation}

x_{t+1} = f(x_t, u_t, \eta_t), \quad y_t = h(x_t) + \epsilon_t

\end{equation}

其中,$x_t$ 为系统状态,$u_t$ 为外部输入,$\eta_t, \epsilon_t$ 分别为过程与观测噪声。该结构可与 CIM 的物理、空间与性能维度信息协同求解,实现时序驱动的多维风险演化模拟。

在语义层面,时间维度实现了 CIM 模型的演化智能化。通过事件触发机制(Event-Driven Update),CIM 可在感知数据或系统状态变化时自动更新模型结构与语义内容。例如,当传感器检测到积水阈值突破、交通中断或结构异常时,模型会自动触发事件实例化过程,对相关对象的属性、状态及风险值进行更新与传播。这种事件驱动的动态联动机制,使 CIM 模型具备了实时响应与自我演化的能力。

此外,时间维度还支撑灾后重建与长期管理的数字记忆机制。通过版本化几何与语义模型(Versioned Geometry and Semantic Model),可实现模型在不同时刻的快照存储与差异分析,为城市规划、基础设施维护与风险复盘提供依据。同时,时间维度与文化维度结合,可研究社会行为、舆情态势与文化认知在时间维度上的变迁,实现社会—空间—时间一体化的认知建模。

综上,时间维度使 CIM 模型从静态描述走向动态认知,从结构建模演化为时序推理系统。它为灾害风险评估提供了多时间尺度(历史、实时、预测)的统一分析框架,并通过事件驱动与学习机制赋予模型自适应演化能力。通过这一维度的扩展,CIM 不再仅是信息集成的容器,而成为具备记忆、预测与优化功能的城市智能体。

\subsection{视角定义与功能}

\subsubsection{场景化视角}

场景化视角是 CIM 模型的认知起点,它以语境—事件—响应三元逻辑为核心,通过对现实城市问题的情境化建模,构建面向任务与语义的动态场景结构。在传统的模型体系中,城市信息模型往往局限于静态结构或功能层的表达,而场景化视角则赋予模型以语义情境性,使其能够在特定情境下表现出对应的逻辑、规则与动态演化特征。这种情境化机制是 CIM 从数据集成系统向语义认知系统转变的关键步骤。

在方法层面,场景化视角将城市运行和灾害管理过程划分为不同语义阶段,如灾前准备、灾中响应与灾后恢复。每一阶段均以事件(Event)为驱动单元,通过事件触发、状态转移与反馈更新的方式组织模型的语义流。场景可形式化表示为:

\begin{equation}

S = \{ (E_i, C_i, A_i, R_i) \mid i = 1, 2, \dots, n \}

\end{equation}

其中,$E_i$ 为事件(Event),$C_i$ 为上下文条件(Context),$A_i$ 为响应动作(Action),$R_i$ 为系统结果(Result)。这一结构体现了 CIM 模型的动态语义逻辑:当特定事件在特定语境下发生时,系统将根据规则或学习策略执行响应并更新状态。通过该机制,CIM 不仅记录静态信息,还具备了语义层面的情境感知与自适应推理能力。

在模型应用层面,场景化视角可支持多类型的复杂灾害模拟与城市运行分析。在洪涝风险评估中,可定义强降雨事件—排水能力不足—积水形成—交通中断的情境链;在地震模拟中,可描述震源触发—结构响应—人员疏散—次生风险传播的事件序列。这些场景通过与物质维、空间维和时间维的联动,实现从物理要素到语义过程的映射,从而形成城市风险叙事的可计算模型。此外,基于场景定义的任务可被用于模型驱动的自动仿真(Model-driven Simulation)或策略优化(Decision-driven Planning),实现风险态势的预测与应急资源的优化分配。

在语义表达上,场景化视角强调模型的情境连贯性与语义闭环性。通过构建场景本体(Scenario Ontology)与语义图谱(Semantic Graph),可将不同维度的数据、事件与关系嵌入统一语义空间。这种结构使 CIM 模型能够通过上下文推理机制(Context Reasoning)实现情境关联分析与事件驱动更新。例如,当降雨监测节点触发阈值事件时,系统可自动识别相关区域的排水设施、交通网络与人群分布数据,从而即时生成风险预警与响应策略。该机制显著提升了模型的语义完整性与响应智能性。

场景化视角还为后续三种视角——参数化、互动化与智能化——提供了认知入口。它定义了何时何地为何的情境框架,而参数化视角回答如何建模,互动化视角回答如何响应,智能化视角则回答如何学习与演化。四者协同作用,形成从语境识别到知识生成的闭环认知结构,使 CIM 模型在灾害风险评估与城市运行管理中具备全流程语义支撑。

综上,场景化视角通过语义驱动的情境建模机制,使 CIM 模型具备对城市复杂系统的动态感知与情境理解能力。它是实现场景驱动—数据响应—模型演化的核心桥梁,为后续参数化建模与智能推理提供了情境基础。

\subsubsection{参数化视角}

参数化视角是 CIM 模型实现形式化建模与计算推理的核心环节。它将场景化视角中抽象的语义要素转化为可量化的参数集合,通过定义变量、约束与函数关系,实现模型在数学与语义层面的统一表达。在灾害风险评估语境下,参数化视角使 CIM 从语义可理解转变为机理可计算,为多维指标体系构建、模型仿真与优化提供了统一的计算框架。

在建模机制上,参数化视角通过对象—属性—约束(Object--Attribute--Constraint)三元组结构对城市要素进行形式化描述。每个对象 $o_i$ 被定义为一组时变参数的集合:

\begin{equation}

o_i = \{ (a_{i1}, v_{i1}(t)), (a_{i2}, v_{i2}(t)), \dots, (a_{in}, v_{in}(t)) \}

\end{equation}

其中,$a_{ij}$ 表示属性名称,$v_{ij}(t)$ 为其在时间 $t$ 的取值。约束条件以函数或逻辑规则形式定义对象间的依赖与耦合,如:

\begin{equation}

C_k : f(a_{i1}, a_{j2}) \leq \theta_k

\end{equation}

通过该结构,CIM 模型可在不同语义层次上实现参数的继承、绑定与动态更新。这使得模型不仅能够静态描述建筑、管网、道路等实体特征,还可动态反映能耗、流速、承载率等运行状态的变化规律。

在实现方式上,参数化视角通过多维参数空间映射(Multidimensional Parameter Mapping)建立不同维度间的逻辑关联。例如,在物质维与性能维之间,参数化机制可实现材料属性与结构响应的动态绑定:

\begin{equation}

\sigma(t) = E(t) \cdot \varepsilon(t)

\end{equation}

其中,$\sigma(t)$ 为应力,$\varepsilon(t)$ 为应变,$E(t)$ 为随时间变化的材料模量。在空间维与时间维之间,参数化模型可定义流场与时序的耦合关系,如:

\begin{equation}

Q(x, t) = \int_S u(x, t) \, dS

\end{equation}

表示流量 $Q$ 随空间位置与时间的动态变化。通过这种跨维度参数映射,CIM 模型能够在语义一致的框架下进行多物理、多时态的综合分析。

参数化视角的另一个关键特征是可扩展性与可继承性。通过统一的参数模板与本体结构,模型可支持多源数据的自动对接与模型实例化。例如,当接入新的感知数据流(如实时降雨量、交通流速或能耗监测值)时,系统可依据参数定义自动更新相关对象的属性值与约束条件。这种机制使 CIM 模型具备数据驱动自演化的能力,即模型结构可随外部数据变化而动态调整,从而维持全局一致性与实时性。

在语义层面,参数化视角为 CIM 模型提供了精确计算的语言。它通过将抽象语义映射为参数化表达,实现了从语义建模到数值计算的无缝衔接。在风险评估场景中,这意味着模型能够直接量化建筑脆弱性、排水能力、交通通行率等指标,并在仿真过程中实时调整参数以反映灾害演化的非线性动态。参数化不仅是一种表达方式,更是一种认知机制,使模型能够在复杂约束下保持可解释性与可验证性。

综上,参数化视角通过变量定义、约束推理与函数映射,实现了 CIM 模型的形式化与可计算化。它在模型层次上为后续互动化视角提供输入通道,使模型能够通过参数变化驱动与外部环境交互;同时,在语义层次上为智能化视角提供可学习特征空间,支撑机器学习与优化算法在 CIM 环境中的应用。

\subsubsection{互动化视角}

互动化视角是 CIM 模型实现人—机—物协同与多层动态反馈的关键环节。它通过将参数化模型与传感器网络、仿真引擎及人机界面(HMI)深度融合,使模型不仅具备被动的数据承载能力,还具备主动的感知、响应与干预能力。在灾害风险评估与应急管理中,互动化视角使 CIM 成为贯通数据—模型—决策的中枢,实现虚实融合的智能响应体系。

在体系结构上,互动化视角以 IoT 感知层—模型交互层—用户反馈层为三层架构核心。感知层负责采集环境与设备的实时状态数据,如雨量、风速、水位、结构应变、交通流量等;模型交互层承担数据融合、参数更新与仿真计算功能,将感知数据实时映射到模型对象中,驱动模型状态变化;用户反馈层通过可视化与交互界面展示仿真结果与风险态势,并允许用户进行干预、调整或策略推演。该体系形成一个完整的数据输入—模型响应—人机反馈—策略修正闭环,使 CIM 具备连续感知与自适应调整的能力。

在建模机制上,互动化视角通过事件驱动(Event-driven)与反馈控制(Feedback Control)双机制实现模型动态更新。当外部感知事件触发(如雨量超过阈值、交通中断、传感器告警),模型自动识别相关对象集合并执行状态更新:

\begin{equation}

S_{t+1} = F(S_t, E_t, U_t)

\end{equation}

其中,$S_t$ 为系统状态,$E_t$ 为触发事件,$U_t$ 为人工或算法控制输入。这一更新机制保证 CIM 能够在快速变化的灾害场景中维持模型一致性与响应实时性。同时,通过反馈控制环节,系统根据实时数据与仿真输出计算误差项,并自动优化参数或策略,使模型在多轮迭代中逐步逼近真实状态。

在技术实现上,互动化视角依托可视化与交互技术(如 WebGL、Unity3D、Cesium、Qt/PySide 等)实现三维动态展示与操作。通过空间可视化平台,用户可实时观察城市要素的运行状态与风险演化过程,例如雨水流动路径、交通拥堵动态或结构应变变化。同时,系统提供多维交互接口,支持用户进行模型控制、情景推演与策略测试。例如,在城市内涝场景下,管理者可在界面中调整排水口开度、修改地形参数或模拟不同降雨强度,以观察风险变化趋势并优化应急方案。这种可视—可控—可验证的互动机制,使 CIM 成为一种具有解释力与执行力的智能决策支撑平台。

在语义层面,互动化视角强化了 CIM 模型的协同智能(Collaborative Intelligence)。通过人与模型的持续交互,系统不仅能够接收用户输入,还能在长期运行中学习用户决策规律与响应模式,从而逐步形成基于知识图谱与强化学习的反馈优化机制。这意味着 CIM 不再是被动的信息容器,而是一个具有自适应行为与协同学习能力的城市认知代理(Urban Cognitive Agent)。

综上,互动化视角实现了 CIM 模型在感知、计算与决策层的动态耦合。它通过数据流、模型流与知识流的双向融合,构建了一个既能模拟灾害演化过程、又能支持人机协同决策的智能系统。在此基础上,模型的智能化学习机制得以展开,为下一节智能化视角提供行为模式与认知演化的语义基础。

\subsubsection{智能化视角}

智能化视角是 CIM 模型实现自学习、自优化与知识推理的核心层,标志着模型从感知与响应迈向认知与决策。它通过集成知识图谱、机器学习与强化学习机制,使模型具备对复杂城市系统进行模式识别、预测分析与策略优化的能力,从而实现语义驱动的智能演化。在灾害风险评估与应急响应领域,智能化视角不仅能够识别潜在风险与关键脆弱环节,还能基于经验数据与仿真结果生成最优干预策略,形成从认知到决策的闭环。

在模型体系上,智能化视角以三层智能结构为核心:语义认知层(Semantic Cognition Layer)——通过构建知识图谱(Knowledge Graph)与语义网络,将城市对象、事件与关系转化为可推理的知识单元;学习感知层(Learning Perception Layer)——通过机器学习(ML)与深度神经网络(DNN)对多维数据进行特征提取与模式学习;策略决策层(Policy Decision Layer)——采用强化学习(RL)或基于优化的决策算法,实现模型的自适应优化与策略生成。

在语义认知层,CIM 模型以知识图谱为核心结构,通过实体(Entity)、属性(Attribute)与关系(Relation)三元组形式化表示城市系统的逻辑网络:

\begin{equation}

K = \{ (e_i, r_j, e_k) \mid e_i, e_k \in E, r_j \in R \}

\end{equation}

其中,$E$ 表示对象集合,$R$ 表示语义关系集合。通过对该结构的图神经网络(Graph Neural Network, GNN)建模,可在语义空间中实现跨领域关联推理。例如,模型能够自动识别建筑结构退化与环境湿度之间的因果相关性,或在应急场景中推断交通拥堵与避难路径选择的动态关系。这种语义图谱驱动的推理机制,使 CIM 模型具备知识层自组织与语义层自解释的特征。

在学习感知层,模型通过多源异构数据驱动的机器学习过程实现经验积累与状态识别。典型方法包括卷积神经网络(CNN)用于空间模式识别、循环神经网络(RNN)或时间卷积网络(TCN)用于时序预测,以及自监督学习(Self-supervised Learning)用于语义特征抽取。在灾害风险预测中,这些学习机制可用于识别风险热点、估计水流传播路径或预测建筑结构失稳趋势。此外,通过集成Lipschitz 正则化的强化学习算法(Lipschitz-Regularized RL),模型能够在不确定性环境下实现策略收敛与风险约束优化,提升决策的稳定性与可解释性。

在策略决策层,CIM 模型通过强化学习框架实现基于奖励信号的最优策略生成:

\begin{equation}

\pi^*(s_t) = \arg \max_\pi \mathbb{E} \left[ \sum_{k=0}^\infty \gamma^k R_{t+k+1} \mid s_t \right]

\end{equation}

其中,$\pi^*$ 表示最优策略,$s_t$ 为状态,$R$ 为奖励函数,$\gamma$ 为折扣因子。该机制可在灾害应对场景中学习多阶段决策模式,如避难路径规划、排水阀门控制或交通引导策略,从而实现实时优化与动态适应。与互动化视角结合后,智能化视角可通过持续反馈与在线学习机制,不断更新模型权重与决策规则,使系统具备长时记忆与自我改进能力。

在语义层面,智能化视角实现了 CIM 模型从数据驱动到知识驱动的转变。模型不再仅依赖规则或经验,而是通过学习与推理实现跨尺度认知与主动决策。这使得 CIM 能够在灾害风险管理、城市运营优化与资源调度中实现多层自适应能力。例如,在内涝场景中,模型可自动识别易涝区、预测积水深度,并提出基于实时监测的排水优化策略;在地震或火灾场景中,系统可基于结构损伤预测与人群分布信息生成最优疏散方案,形成从数据采集到智能响应的闭环链条。

综上,智能化视角赋予 CIM 模型以自主学习、语义推理与动态决策的综合能力。它不仅是模型的顶层智能单元,也是实现CIM 驱动的城市智能体(CIM-driven Urban Agent)的关键路径。通过与前述三种视角的协同作用,智能化视角使 CIM 模型能够在复杂环境中实现持续演化,完成从结构信息系统到认知智能系统的跨越。

\section{数据模型}

\subsection{概述}

在前章五维四视角语义建模框架的基础上,CIM 模型已建立了从结构到语义、从静态到动态、从数据到智能的认知体系。然而,要实现这一体系在工程环境中的可计算与可扩展性,必须构建与之相匹配的数据组织与管理机制。因此,本章提出 CIM 的数据模型(Data Model),旨在通过多层次的数据结构、对象关系机制与时序管理策略,将语义建模框架落地为可操作的数字化数据体系。

CIM 数据模型的核心思想是:以对象为核心、以关系为纽带、以事件为驱动、以时间为约束。它通过定义对象(Object)的结构与属性,建立对象间的语义关系(Relation),并以事件(Event)为动态更新的触发机制,形成贯通空间、时间与语义的多层次信息流。这一思想将复杂城市系统抽象为一个可解析、可演化的对象—关系—事件(O--R--E)网络,使模型既能描述静态实体,又能表达动态行为。在灾害风险评估中,该数据模型支撑了从几何特征到风险传播的全过程计算,使 CIM 不仅是数据集成平台,更成为语义驱动的知识表达与决策系统。

在数据组织层面,CIM 数据模型采用分层架构(Layered Architecture)设计,以支持不同层级的数据管理与语义推理:几何层(Geometric Layer):负责空间对象的几何形态与拓扑结构描述;属性层(Attribute Layer):记录对象的物理、性能与社会属性参数;语义层(Semantic Layer):定义对象间的逻辑关系与行为规则,实现语义融合;知识层(Knowledge Layer):通过本体与知识图谱结构组织高阶认知与推理逻辑。该多层架构确保了模型既具备工程可实现性,又能支撑高层次语义计算与智能分析。

此外,CIM 数据模型在管理机制上引入了双时态数据存储(Bitemporal Storage)与事件驱动更新(Event-driven Update)机制。双时态结构使模型能够同时管理事务时间(Transaction Time)与有效时间(Valid Time),从而实现对历史状态的追溯与未来状态的预测。事件驱动机制则通过监听模型内外部的状态变化(如感知数据更新、模拟结果生成或外部任务触发),自动更新模型实例,维持模型语义与数据的一致性。这种时序感知与事件响应机制,为风险态势感知与应急决策提供了实时、可信与可回溯的数据支撑。

\begin{figure}[htbp]

\centering

\includegraphics[width=0.8\textwidth]{PC-CIM/图3.png}

\caption{CIM 数据模型体系示意图}

\caption*{Figure~\thefigure~ Schematic Diagram of the CIM Data Model System}

\label{fig:cim_data_model}

\end{figure}

如图~\ref{fig:cim_data_model}所示,CIM 数据模型体系由几何层、属性层、语义层与知识层构成多层语义映射结构,并通过时间与演化机制实现动态更新。几何层以空间形态(点、线、面、体)为基础,描述对象的几何拓扑关系与多尺度表达;属性层在此基础上承载物理、环境与社会经济等多维属性,实现对对象特征的量化与动态修正;语义层通过对象—关系—事件网络定义行为规则与约束逻辑,使模型能够响应事件触发并实现语义抽象;最上层的知识层则整合知识图谱、本体模型与逻辑推理机制,完成从数据到知识的升华与决策支撑。右上方的时间与演化机制模块提供双时态存储与多源同步更新能力,使模型在演化过程中保持一致性与可追溯性。整体上,该模型实现了从几何结构到语义认知再到知识推理的多层递进,支撑了 CIM 在风险评估与智能决策中的动态语义计算框架。

综上,CIM 数据模型是连接语义建模与工程实现的关键桥梁。它通过多层结构化组织与动态管理机制,实现了语义框架的数字化落地,为后续的数据融合、语义映射、时序管理与质量控制提供了统一的逻辑与数据基础。接下来的章节将分别阐述该数据模型的层次结构、对象关系模型(O--R--E Model)、语义融合机制以及动态更新与治理策略,构成完整的 CIM 数据管理体系。

\subsection{数据层次与体系结构}

CIM 数据模型的核心在于构建一个可扩展、可演化且具备语义一致性的多层数据体系,以支撑从物理实体到智能决策的全链路建模与推理。其层次结构遵循从结构到语义、从数据到知识的递进逻辑,通过不同层级的抽象与封装,将城市复杂系统的多维数据有序组织起来,形成具有清晰分工与逻辑闭环的数据架构。整体体系通常划分为四个主要层次:几何层、属性层、语义层与知识层。

在最底层的几何层(Geometric Layer),数据模型关注对象的空间形态与拓扑结构,是实现空间一致性和可视化建模的基础。该层定义了城市对象的几何要素(点、线、面、体)及其空间关系(邻接、包含、交叉、覆盖等),用于支撑三维场景建模与物理仿真。几何层不仅存储静态形态信息,还支持多尺度表达(如建筑级、街区级、城市级)与版本化管理,以便在不同精度与时间尺度下实现对象的动态可视化和几何演化。

在其上构建的属性层(Attribute Layer)负责对象的物理、环境、性能及社会属性描述,是连接几何实体与语义推理的中间层。属性层的数据通常来源于 IoT 传感器、监测系统、遥感影像及各类业务系统(如能源、交通、气象、人口等),并通过统一的数据接口与标准(如 SensorThings API、CityGML、IFC)进行整合。该层的关键特征是支持参数化管理与动态更新,即对象的每个属性值 $a_i(t)$ 都可随时间变化而被实时修正,使模型在面对复杂动态环境时保持准确性与可溯性。

进一步向上的语义层(Semantic Layer)是 CIM 数据模型的核心,它实现了从结构化数据到知识表达的转化。语义层通过对象间的逻辑关系、约束规则和行为模式建立语义网络,用以刻画对象—关系—事件的多维交互逻辑。例如,排水管网与地表高程之间存在功能依赖关系,建筑物与避难点之间存在安全可达关系,这些语义连接均通过关系图(Relation Graph)与约束算子(Constraint Operator)形式化定义。语义层同时集成了事件驱动机制(Event-driven Mechanism),当模型检测到状态变更或外部触发事件时,可自动更新相关对象与关系,实现语义层级的自洽与动态响应。

最高层的知识层(Knowledge Layer)是实现智能推理与决策支持的关键环节。该层通过知识图谱、本体模型及规则库组织城市运行的高阶认知逻辑,将对象及其语义关系抽象为可推理的知识节点与边结构。例如,在灾害风险场景中,知识层可基于历史数据与专家知识,推理出地形低洼 + 降雨量上升 → 内涝风险上升的因果链,并将该推理结果反馈至语义层触发模型更新。知识层的设计使 CIM 具备语义学习、模式发现与决策生成的能力,从而支撑模型的智能化演化。

在层间关系上,CIM 数据体系采用自底向上的依赖映射与自顶向下的语义约束双向机制。下层为上层提供数据支撑与结构基础,上层对下层施加语义约束与逻辑推理,从而构成一个闭环的动态知识体系。与此同时,体系内部引入了时间索引机制与多源同步更新策略,使不同层级的数据在时间维度上保持一致性。这种层次化架构不仅保证了数据管理的高内聚与低耦合,也使模型能够在多源异构环境下实现可扩展、可维护与可验证的特性。

综上,CIM 的数据层次与体系结构通过几何、属性、语义与知识四个层级的有机衔接,实现了从空间表示到智能推理的全链路支撑。该体系既符合工程实现的分层设计原则,又契合语义建模的逻辑层次划分,为后续章节中对象关系模型(O--R--E Model)、数据融合与语义映射提供了结构化基础,也为城市级风险评估与智能决策构建了统一的数据语义支撑框架。

\subsection{对象关系模型}

对象—关系—事件模型(Object--Relation--Event, 简称 O--R--E 模型)是 CIM 数据模型的语义核心结构,用于统一描述城市实体(Object)的存在状态、要素间的语义联系(Relation)以及系统运行过程中的动态行为(Event)。该模型突破了传统以几何或属性为中心的静态数据范式,通过引入事件驱动机制与关系约束逻辑,使 CIM 能够表达对象间的时空耦合与演化关系,从而实现语义层的动态计算与认知推理。

在模型结构上,对象(Object)是 O--R--E 模型的基本单元,代表城市系统中具有独立属性与行为的实体,如建筑、道路、管网、传感器节点或功能区块等。每个对象不仅包含几何与物理属性,还具有描述其运行状态与生命周期的动态参数,可形式化表示为 $O_i = (G_i, A_i, S_i, T_i)$,其中 $G_i$ 为几何形态,$A_i$ 为属性集合,$S_i$ 为状态变量,$T_i$ 为时间索引。对象的设计遵循语义自洽与上下文可扩展原则,使其既能作为独立的数据单元被引用,又能在语义网络中与其他对象形成关系集群。

关系(Relation)定义对象之间的语义与逻辑联系,是支撑模型知识表达与约束求解的关键要素。关系可分为结构性关系(如包含、邻接、相交)、功能性关系(如依赖、供给、约束)与语义性关系(如影响、触发、反馈)三大类。形式化地,关系可表示为有向图结构 $R = (O_i, O_j, \rho_{ij}, w_{ij})$,其中 $\rho_{ij}$ 表示对象间的关系类型,$w_{ij}$ 表示关系强度或影响权重。关系层不仅刻画静态依存结构,也支持约束传播与多级推理机制。例如,若雨量传感器与排水节点存在供给关系,则当降雨数据触发阈值时,该关系可通过事件机制传播到排水系统,促使模型自动更新流量或积水状态。

事件(Event)是 O--R--E 模型的动态驱动力,用于描述对象状态随时间或外部条件变化的过程。事件可分为自然事件(如降雨、地震)、系统事件(如设备故障、参数更新)与人为事件(如决策干预、调度命令)。其形式化表达为 $E_k = (t_k, O_i, \Delta S_i, C_k)$,其中 $t_k$ 为事件时间,$\Delta S_i$ 表示对象状态的变化量,$C_k$ 为触发条件。事件机制使模型具备实时响应能力:当事件触发时,系统根据既定逻辑规则或学习策略调整相关对象的属性与关系状态,实现状态—行为—反馈的闭环更新。

O--R--E 模型的核心特征是语义可计算性与演化一致性。一方面,它通过统一的数据接口与逻辑算子,使对象、关系与事件能够在语义层面进行自动推理与更新,从而支撑复杂系统的动态仿真。另一方面,该模型采用多层时序同步机制,确保对象状态、关系网络与事件流在时间维度上的一致性,避免信息漂移与逻辑冲突。例如,当多个事件在同一时间段内作用于同一对象时,系统可根据优先级或依赖关系进行顺序调度,保证全局状态更新的确定性与可重现性。

在实现方式上,O--R--E 模型可与图数据库(Graph Database)及事件流引擎(Event Stream Engine)协同运行,以实现高效的语义存储与实时计算。对象与关系以图节点与边的形式存储于知识图谱中,事件则作为触发节点在图上驱动状态转移与语义传播。该机制使 CIM 模型能够在面对复杂灾害或多系统耦合场景时,快速识别风险传播路径、资源依赖链与响应优先级,支持高精度的动态风险推理与智能决策。

综上,O--R--E 模型通过对象—关系—事件的统一表达,实现了 CIM 从静态结构向动态语义的跃迁。它不仅为模型提供了结构化的逻辑骨架,也为语义推理、时序更新与知识学习提供了计算基础。作为数据模型的核心中枢,O--R--E 框架贯穿了从底层数据到高层智能的全过程,为后续的数据融合与语义映射、动态更新与时序管理奠定了严谨而灵活的语义结构基础。