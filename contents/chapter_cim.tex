\chapter{城市洪涝灾害协同防控的理论基础与框架}

\section{引言}

城市灾害风险治理正从被动防御转向主动韧性,需要一条贯通“目标—方法—实现”的统一框架。本文以三层逻辑奠基全文:顶层以城市韧性治理为价值导向,强调吸收、适应、转型三类能力的全生命周期协同;方法层提出“机理—数据—知识—模型”(MDKM)四驱融合,以机理揭示规律、数据驱动感知、知识约束推理、多模型协同仿真;实现层以 CIM 的五维四视角为数字孪生平台,物质、空间、性能、文化、时间五维统一表达城市实体,场景化、参数化、互动化、智能化四视角组织模型认知与交互。三者形成从理念到实践的闭环,为后续章节的风险评估、疏散规划与构件损伤分析提供共同语义与数据底座。

韧性治理明确了城市—建筑—构件多尺度的研究目标,要求从事前预防到事后恢复形成贯通的能力链。智慧城市与数字孪生提供感知、汇聚与计算的大脑,使风险评估、疏散和损伤分析能够在同一虚实耦合环境中实现“可感知—可推演—可决策”的闭环。CIM 则通过融合 BIM 的精细构件语义与 GIS 的空间分析,连接“建筑内部—城市外部”的几何与语义,承载 MDKM 的多模型协同和事件驱动更新。基于这一统一框架,第三章展开高精度风险评估,第四章构建室内外一体化疏散网络并与风险结果联动,第五章实现构件级损伤评估,第六章完成系统集成与业务化验证。

\section{城市洪涝防控的MDKM四驱融合方法论}

\subsection{四驱融合内涵与接口}

在过去十年间,城市安全领域经历了从机理方法到数据驱动,再到知识与模型融合的技术演进路径。然而,现有进展多呈现“单点主导、其他辅助”的松散协作模式。具体而言,机理方法虽能提供良好的物理过程可解释性,但在建模精度与参数漂移方面存在显著局限;数据驱动方法虽追求端到端的预测准确性,但其在样本不足或开放场景下的泛化能力仍有待提升;知识方法虽可将人类经验融入系统,但隐性知识的提取与动态更新面临挑战;模型方法(尤以数字孪生为代表)虽擅长可视化映射,却常在高保真度与低时效性之间面临难以兼顾的权衡困境。

基于上述背景,本研究从城市系统复杂性出发,提出了一种“齿轮式”深度融合架构,如图~\ref{fig:gear-framework} 所示。该架构将机理、数据、知识与模型(Mechanism--Data--Knowledge--Model,简称 MDKM)四大要素类比为相互精密啮合的齿轮,围绕基于 CIM 的数字孪生这一“时空轴”运转,形成一个动态、协同、高效的系统。通过硬约束、软规则及反向校准三类力矩传递机制,这些齿轮能在不确定的环境下实现动态平衡与自适应演化。

\begin{figure}[htbp]
    \centering
    \includegraphics[width=0.85\textwidth]{PC-CIM/四驱融合齿轮式框架.png}
    \caption{四驱融合齿轮式框架}
    \caption*{Figure~\thefigure~ Four-Drive Integrated Gear Framework}
    \label{fig:gear-framework}
\end{figure}

该架构强调“多域协同”和“迭代自校正”:在高确定性区域,机理模型主导系统运行;而在高不确定性场景下,数据与知识则驱动系统的补偿与调整。这一框架不仅是技术层面的深度整合,也体现了系统协同运作的哲学理念,形成了一个完整的从机制到模型的反馈闭环。

\begin{itemize}
\item \textbf{机理齿轮}:承载物理、工程与社会等领域的先验定律与第一性原理(如浅水方程、结构力学、行为经济学),确保模型的可解释性与外推稳健性,是系统认知的物理根基。
\item \textbf{数据齿轮}:负责接入与处理多源异构的实时与历史数据流(如IoT传感、遥感影像、社交数据),为系统提供持续的态势感知与经验证据,是系统演化的现实感知源。
\item \textbf{知识齿轮}:封装专家经验、领域规则、政策法规与逻辑约束(如应急预案、建筑规范、疏散行为规则),为系统注入人类的常识与智慧,是决策合理性与合规性的保障。
\item \textbf{模型齿轮}:是仿真、分析与优化算法的执行主体(如计算流体动力学模型、多智能体仿真、优化算法),构成数字孪生的计算核心,是各类洞察与策略的生成器。
\end{itemize}

四驱融合齿轮式框架图采用机械齿轮的比喻阐释了四类驱动要素之间的信息耦合关系。每个齿轮既可能作为主动轮也可能作为从动轮,其角色取决于当前的轨道交通风险的不确定性水平——更高可靠性的组件将主导力矩输出。如表~\ref{tab:meshing_interfaces} 所示,汇总了六组齿轮副的啮合功能、数学表达形式及其关键超参数。

\begin{table}[htbp]
\centering
\caption{六组啮合面的数学接口与可调超参数}
\caption*{Table~\thetable~ Mathematical Interfaces of Six Meshing Surfaces and Tunable Hyperparameters}
\label{tab:meshing_interfaces}
\small
\begin{tabular}{p{2.4cm} p{3.4cm} p{5.2cm} p{2.5cm}}
\toprule
齿轮关系 & 啮合功能 & 数学形式 & 关键超参数 \\midrule
机理 $\rightarrow$ 数据 & 物理方程正则化深度学习 & $\mathcal{L}_{physics}} = \lvert f(\theta)-\hat{x} \rvert^{2}$ & 残差权重 $\lambda_{1}$ \\
数据 $\rightarrow$ 知识 & 数据驱动结果受规则检验 & $\mathcal{L}_{logic}} = -\log P(\text{rule}\mid \text{graph})$ & 规则置信度 $\beta$ \\
知识 $\rightarrow$ 模型 & 语义与几何空间对齐 & $\mathcal{L}_{align}} = \lvert Emb_{sem}} - Emb_{geo}} \rvert_{1}$ & 嵌入维度 $d$ \\
模型 $\rightarrow$ 机理 & 观测残差反向校准参数 & $\mathcal{L}_{calib}} = \lvert y_{sim}} - y_{obs}} \rvert^{2}_{\Sigma}$ & 协方差 $\Sigma$ \\
数据 $\rightarrow$ 模型 & 实时同步保持虚实一致 & $\mathcal{L}_{sync}} = \lvert x_{digital}} - x_{physical}} \rvert^{2}$ & 延迟容忍 $T$ \\
机理 $\rightarrow$ 知识 & 因果链一致性检查 & $\mathcal{L}_{causal}} = \Sigma \lvert P_{cause}} - P_{effect}} \rvert$ & 因果强度 $\gamma$ \\
\bottomrule
\end{tabular}
\normalsize
\end{table}

\subsection{动力机制与协同优化}

四个齿轮通过三类核心的“力矩传递”机制实现深度啮合与协同工作,这是 MDKM 框架的灵魂所在:

\textbf{1. 物理力矩(机理 $\rightarrow$ 数据 / 模型):}
机理模型提供系统运行的物理规律约束,是保证模型可解释性与稳健性的基础。在样本稀缺或噪声场景下,通过物理信息神经网络等方式,将控制方程、本构关系(如守恒律)嵌入数据驱动模型的损失函数中作为正则项,构建“数据项+物理残差”联合优化。这在数据稀缺或噪声环境下保障了解的物理合理性,在保证精度的同时提升外推与可解释性。

\textbf{2. 逻辑力矩(知识 $\rightarrow$ 数据 / 模型):}
知识层承载专家经验与逻辑规则,其关键作用在于对数据驱动过程进行语义约束与逻辑校验。将领域知识编码为“if--then”类逻辑规则(如知识图谱的三元组)或政策约束,转化为损失函数或推理规则注入训练与推理。这种机制在模型训练与决策过程中施加语义约束,使模型在未见场景下仍保持安全与业务一致性,确保结果符合业务逻辑与安全规范。

\textbf{3. 反馈力矩(模型 $\rightarrow$ 机理 / 数据):}
模型层通过数字孪生体的实时计算,利用模型输出与实时观测数据之间的残差,通过数据同化、参数反演(如梯度校准、EnKF)等技术,反向校准机理模型中的关键参数(如糙率系数、材料性能退化)。这一机制实现参数在线校准与模型自演化,使模型随观测实时贴合物理状态,避免因环境漂移导致的偏差。

多力矩联合求解可抽象为“物理—数据—逻辑—同步”多目标优化,由自适应权重调度不同信息源的可信度。同时引入不确定性闸门,对输入流进行置信筛选,阻断误差级联,保证在复杂环境下的稳健决策。

本节构建的MDKM方法论为后续章节提供了核心理论支撑:其中“机理”与“数据”齿轮的融合将在第三章的高精度洪涝风险评估中得到应用;“知识”与“模型”齿轮的协同则直接支撑第四章的智能疏散规划;而全齿轮的综合作用将在第五章的构件级损伤评估中体现。

\section{基于CIM的城市洪涝数字孪生建模架构}

为实现MDKM方法论在城市尺度的落地,为其提供一个强大的数字载体。本研究通过深化与拓展传统CIM概念,提出了“五维四视角”语义建模理论,旨在将CIM从一个数据集成平台,升维为能够承载并执行MDKM四驱融合的数字孪生操作平台。五维体系定义了城市系统的核心构成要素,如图~\ref{fig:five_dimensions} 所示,包括物质维度、空间维度、性能维度、文化维度与时间维度。

\begin{figure}[htbp]
\centering
\includegraphics[width=0.8\textwidth]{PC-CIM/图1.png}
\caption{五维体系示意图}
\caption*{Figure~\thefigure~ Schematic Diagram of the Five-Dimensional System}
\label{fig:five_dimensions}
\end{figure}

\subsection{物质维度:物理实体映射}
物质维度聚焦城市物理实体的几何、材料与结构组成。通过融合 BIM、GIS、LiDAR 等数据,形成高精度三维场景,为机理齿轮提供边界与属性输入,是风险模拟与构件损伤分析的基础。这些维度不仅刻画了城市对象的静态形态与动态过程,也反映了灾害风险形成的多因素耦合机理。

\subsection{空间维度:拓扑与网络}
空间维度是 CIM 模型的组织层,负责描述城市对象之间的空间关系、拓扑连接与场景分布规律。在空间表达上,CIM 模型采用基于拓扑图(Topological Graph)与空间网格(Spatial Mesh)的联合建模方式。微观层面,建筑与设施通过 BIM 模型精确描述结构内部的空间构成;中观层面,街区、道路与地形通过 GIS 拓扑表达地理布局与功能分区;宏观层面,城市整体以空间网络形式抽象出不同功能单元的关联模式。这种多粒度的空间建模方法确保了 CIM 能在灾害模拟中实现从局部到整体的空间推理能力,支持如洪涝水流扩散、地震波传播与人员疏散路径优化等过程的动态计算。

\subsection{性能维度:动态运行响应}
性能维度是 CIM 模型中连接结构实体与运行行为的关键层,用于表征城市系统在不同工况下的动态响应、能效状态与服务性能。在数据层面,以传感器、监控系统和仿真模型为主要数据源,构建多维时序数据集。在模型机制上,通过多物理场联合建模(Multiphysics Coupling),将结构、能量、流体及交通等要素纳入统一求解框架。例如城市内涝模拟中,可基于 RANS 等方程组结合高精度几何与材料信息,描述洪水流速、结构受力、交通拥堵等复杂现象。性能维度能够定量描述不同系统在灾害作用下的承载能力、响应延迟与恢复过程,为高精度风险评估与时序仿真提供直接支撑。

\subsection{文化维度:社会行为与韧性}
文化维度旨在捕捉城市系统中人、物、空间之间的认知联系与行为逻辑。在灾害风险评估语境下,文化维度不仅用于刻画城市形态与历史文脉的延续性,更用于表达社会系统的韧性特征与行为反馈机制。在建模机制上,文化维基于多主体系统(ABM)与多层网络(MLN)构建人--空间--事件耦合关系图,模拟个体与群体在灾害场景中的感知、决策与迁移行为。这种机制使模型从物理空间重建扩展到社会语义表达,实现从客观环境到主观认知的跨域映射,有助于识别高风险区域中的社会脆弱群体,并为风险沟通与公众教育提供数据支撑。

\subsection{时间维度:全生命周期演化}
时间维度是 CIM 模型的动态核心,用于刻画城市对象、关系与事件在时间轴上的演化规律与生命周期过程。它不仅反映对象状态的时间连续性,还表达不同要素之间的时序关联与演化逻辑。在数据形态层面,时间维度整合了多源时序数据,形成支持双时态(Bitemporal)存储的多层数据库。在模型机制上,引入时序数据建模与演化预测机制,结合统计学习与深度时序网络,实现城市系统状态的动态重构与趋势预测。通过事件触发机制(Event-Driven Update),CIM 可在感知数据或系统状态变化时自动更新模型结构与语义内容,使 CIM 不再仅是信息集成的容器,而成为具备记忆、预测与优化功能的城市智能体。

本节建立的五维建模体系为后续研究提供了统一的数据底座:物质与空间维度的高精度几何信息是第三章水动力模拟的直接输入;空间与性能维度构建的动态网络是第四章疏散路径规划的基础;而物质维度的材料属性与时间维度的演化机制则是第五章构件级损伤评估的关键依据。

\section{灾害情境下的CIM四视角交互机制}

四视角是驱动MDKM四个齿轮在CIM平台上精密啮合、协同工作的“传动机制”与“控制逻辑”。如图~\ref{fig:four_views} 所示,与五维体系相对应,四视角描述了模型认知与交互的四种机制。

\begin{figure}[htbp]
\centering
\includegraphics[width=0.8\textwidth]{PC-CIM/图2.png}
\caption{四视角体系示意图}
\caption*{Figure~\thefigure~ Schematic Diagram of the Four-Perspective System}
\label{fig:four_views}
\end{figure}

上述五维与四视角以及四驱齿轮相互映射,构成一个维度与视角以及维度与齿轮双重映射矩阵(表~\ref{tab:five_dim_four_view},表~\ref{tab:five_dim_gear}所示)。

\begin{table}[htbp]
\centering
\caption{五维与四视角映射矩阵}
\caption*{Table~\thetable~ Mapping Matrix Between the Five-Dimensional System and the Four-Perspective System}
\label{tab:five_dim_four_view}
\small
\begin{tabular}{p{2.0cm} p{3.0cm} p{3.0cm} p{3.0cm} p{3.0cm}}
\toprule
维度 / 视角 & 场景化 & 参数化 & 互动化 & 智能化 \\midrule
物质维 & 建筑与设施场景 & BIM 参数绑定 & IoT 感知反馈 & 智能结构监测 \\
空间维 & 城市空间场景 & GIS 空间参数 & 空间交互控制 & 空间动态优化 \\
性能维 & 能源 / 交通运行场景 & IoT 传感参数 & 实时监控反馈 & 预测与调度学习 \\
文化维 & 文化遗产保护场景 & 文化属性参数化 & 群众参与与展示 & 舆情 / 行为数据分析 \\
时间维 & 生命周期建模 & 时间参数 & 历史与实时融合 & 演化规律学习 \\
\bottomrule
\end{tabular}
\normalsize
\end{table}

\begin{table}[htbp]
\centering
\caption{五维与齿轮映射矩阵}
\caption*{Table~\thetable~ Five-Dimension-to-Gear Mapping Matrix}
\label{tab:five_dim_gear}
\small
\begin{tabular}{p{2.0cm} p{4.0cm} p{3.5cm} p{3.5cm}}
\toprule
维度 & 角色(语义职责) & 数据锚点 & 支撑齿轮 \\midrule
物质维 & 几何 + 材料 & BIM / LiDAR & 机理 \\
空间维 & 拓扑 + 场域 & GIS 语义图 & 机理 + 模型 \\
性能维 & 实时状态 & IoT 时序 & 数据 \\
文化维 & 规则 + 行为 & 知识图谱 & 知识 \\
时间维 & 生命周期 & 版本 + 事件 & 全部(横贯轴) \\
\bottomrule
\end{tabular}
\normalsize
\end{table}

\subsection{场景化:任务驱动}
场景化视角为多齿轮协同提供任务语境。它通过定义灾害应急的特定情境(如台风、内涝),动态组装所需的模型、数据与知识资源,设定各齿轮的参与方式与交互规则。在方法层面,将城市运行和灾害管理过程划分为灾前准备、灾中响应与灾后恢复,以“事件—上下文—动作—结果”链条组织语义流。通过构建场景本体与语义图谱,可将不同维度的数据、事件与关系嵌入统一语义空间,实现情境关联分析与事件驱动更新。

\subsection{参数化:计算基础}
参数化视角实现从物理规律、业务规则到可计算参数的转化。它将五维语义统一表达为模型可调用的参数与约束,是“机理→模型”与“知识→模型”啮合面的具体实现桥梁。通过对象—属性—约束(Object—Attribute—Constraint)三元组结构对城市要素进行形式化描述,每个对象由一组时变参数及其约束组成。这使得模型不仅能够静态描述建筑、管网、道路等实体特征,还可动态反映能耗、流速、承载率等运行状态的变化规律。

\subsection{互动化:虚实交互}
互动化视角实现数字孪生与物理实体的实时双向交互。它通过IoT感知层获取数据驱动模型更新(数据→模型),并通过仿真结果反馈至控制系统或指挥界面(模型→机理/知识)。在体系结构上,以 IoT 感知层—模型交互层—用户反馈层为三层架构核心。通过事件驱动与反馈控制双机制实现模型动态更新,保证 CIM 在快速变化的灾害场景中维持模型一致性与响应实时性。这种可视—可控—可验证的互动机制,使 CIM 成为一种具有解释力与执行力的智能决策支撑平台。

\subsection{智能化:决策支持}
智能化视角作为整个CIM-MDKM系统的“自主智能中枢”,集成知识图谱、机器学习与强化学习机制,从数据中发现新模式、优化机理参数,提炼新知识,并生成自适应策略。模型体系包括语义认知层(构建知识图谱与语义网络)、学习感知层(模式学习与状态识别)和策略决策层(自适应优化与策略生成)。这一视角赋予 CIM 模型以自主学习、语义推理与动态决策的综合能力,实现跨尺度认知与主动决策。

四视角机制明确了模型在不同灾害阶段的交互逻辑:场景化与参数化视角定义的灾害情景与计算参数,直接服务于第三章的风险指标计算;互动化视角支撑了第四章中疏散路径的动态引导;而智能化视角则贯穿于第五章的损伤诊断与第六章的系统集成决策中。

\section{支撑多维协同的O-R-E数据模型}

为解决城市多源异构数据的语义融合与动态交互难题,本研究构建了对象—关系—事件(Object-Relation-Event, O-R-E)数据模型。该模型不仅是CIM的数据底座,更是MDKM四驱融合方法论在数据层面的具体投影与执行载体。

\subsection{数据层级架构}
CIM 数据模型的核心在于构建一个可扩展、可演化且具备语义一致性的多层数据体系。如图~\ref{fig:cim_data_model}所示,其层次结构遵循从结构到语义、从数据到知识的递进逻辑,通常划分为四个主要层次:
\begin{itemize}
\item \textbf{几何层}:关注对象的空间形态与拓扑结构,定义城市对象的几何要素(点、线、面、体)及其空间关系,支撑三维场景建模与物理仿真。
\item \textbf{属性层}:负责对象的物理、环境、性能及社会属性描述,连接几何实体与语义推理。支持参数化管理与动态更新,使模型在动态环境中保持准确性。
\item \textbf{语义层}:通过对象间的逻辑关系、约束规则和行为模式建立语义网络,刻画对象—关系—事件的多维交互逻辑,实现从结构化数据到知识表达的转化。
\item \textbf{知识层}:通过知识图谱、本体模型及规则库组织城市运行的高阶认知逻辑,实现智能推理与决策支持。
\end{itemize}

\begin{figure}[htbp]
\centering
\includegraphics[width=0.8\textwidth]{PC-CIM/图3.png}
\caption{CIM 数据模型体系示意图}
\caption*{Figure~\thefigure~ Schematic Diagram of the CIM Data Model System}
\label{fig:cim_data_model}
\end{figure}

\subsection{对象-关系-事件动态模型}
O-R-E模型通过将静态实体与动态过程解耦又耦合的方式,实现了对MDKM四驱齿轮的底层承载:

\begin{itemize}
\item \textbf{对象(Object)—— 数据与模型齿轮的载体}:
对象是城市实体的数字化映射,形式化定义为 $O_i = (G_i, A_i, S_i, T_i)$,其中包含几何($G$)、属性($A$)、状态($S$)与时间索引($T$)。在MDKM架构中,\textbf{数据齿轮}负责实时更新对象的属性(如水位传感器更新 $A_{level}$),而\textbf{模型齿轮}则基于对象的状态进行仿真运算(如基于 $G_{building}$ 计算阻水效应)。对象实现了“数据被动记录”与“模型主动计算”的统一封装。

\item \textbf{关系(Relation)—— 知识与机理齿轮的纽带}:
关系定义了对象间的语义拓扑与逻辑约束,表示为 $R = (O_i, O_j, \rho_{ij}, w_{ij})$。它承载了\textbf{机理齿轮}的物理连接(如“上游网格流入下游网格”的水力学关系)和\textbf{知识齿轮}的规则约束(如“水深>0.3m $\rightarrow$ 道路不可通行”的业务规则)。通过关系的定义,独立的城市部件被编织成具备因果逻辑的语义网络。

\item \textbf{事件(Event)—— 驱动系统演化的动力源}:
事件是系统状态跃迁的触发器,定义为 $E_k = (t_k, \Omega_{target}, \Delta S, \Phi_{logic})$,其中 $\Omega$ 为受影响对象集,$\Phi$ 为处理逻辑。事件机制通过时间轴驱动MDKM齿轮转动:外部事件(如暴雨预警)触发机理模型解算,内部事件(如节点失效)触发知识推理与状态更新,从而实现从“静态快照”到“动态过程”的演进。
\end{itemize}

\subsection{洪涝场景下的O-R-E实例}

为具体阐释O-R-E模型在本文后续章节中的支撑作用,本节选取三个典型的洪涝灾害防控场景实例进行解析,如表~\ref{tab:ore_instances}所示。这些实例分别对应风险评估、疏散规划与损伤分析三个关键环节,验证了数据模型在跨尺度、跨业务流程中的通用性与适配性。

\begin{table}[htbp]
\centering
\caption{洪涝场景下的O-R-E实例定义}
\caption*{Table~\thetable~ O-R-E Instances Definition and Application Mapping in Flood Scenarios}
\label{tab:ore_instances}
\small
\begin{tabular}{p{2cm} p{3.5cm} p{4cm} p{4cm}}
\toprule
\textbf{应用环节} & \textbf{核心对象 ($O$)} & \textbf{关键关系 ($R$) / 语义约束} & \textbf{驱动事件 ($E$) 与响应} \\
\midrule
\textbf{第三章} \newline \textbf{高精度风险评估} & 
\textbf{计算网格对象} \newline $O_{cell}=\{z_{bed}, h, \vec{u}\}$ \newline (底高程, 水深, 流速) & 
\textbf{邻接通量关系} ($R_{flux}$) \newline 基于浅水方程或RANS方程定义的质量与动量交换关系(机理约束)。 & 
\textbf{边界入流事件} ($E_{inflow}$) \newline 触发:潮位站实时水位更新。 \newline 响应:激活水动力模型解算,更新全域网格($O_{cell}$)的$h, \vec{u}$状态。 \\
\midrule
\textbf{第四章} \newline \textbf{一体化疏散规划} & 
\textbf{交通路段对象} \newline $O_{road}=\{L, v_{lim}, \kappa\}$ \newline (长度, 限速, 通行度) & 
\textbf{时空可达关系} ($R_{access}$) \newline 受洪涝风险场约束的动态拓扑连接:“若$h(t)>h_{safe}$,则$w_{ij} \to \infty$”(知识规则)。 & 
\textbf{路网阻断事件} ($E_{block}$) \newline 触发:网格水深超过车辆/行人涉水阈值。 \newline 响应:动态重构路网拓扑图,触发A*算法重新规划疏散路径。 \\
\midrule
\textbf{第五章} \newline \textbf{构件损伤评估} & 
\textbf{建筑构件对象} \newline $O_{wall}=\{Mat, \sigma_{yield}, D\}$ \newline (材料, 屈服强度, 损伤度) & 
\textbf{流固耦合关系} ($R_{FSI}$) \newline 映射流体压力场至结构表面的力学传递关系。 & 
\textbf{冲击峰值事件} ($E_{impact}$) \newline 触发:流速$u$与水深$h$组合产生最大动水压力。 \newline 响应:计算构件应力状态,更新损伤度$D$,并可视化损伤等级。 \\
\bottomrule
\end{tabular}
\normalsize
\end{table}

通过上述实例化映射,O-R-E数据模型有效地将第三章的连续流场数据、第四章的离散网络数据以及第五章的精细构件属性统一在同一个语义框架下。这种一致性设计确保了上一阶段的输出(如网格水深)能够无损地转化为下一阶段的输入(如路段阻断判定条件),从而在底层数据逻辑上保障了城市洪涝灾害协同防控全流程的连贯性与闭环。

\section{本章小结}

本章构建了面向城市洪涝灾害协同防控的理论基础与技术框架。首先,确立了以城市韧性治理为核心的研究导向,明确了从灾害防御向全过程适应与转型转变的治理目标。其次,提出了机理—数据—知识—模型(MDKM)四驱融合方法论,阐释了四类要素在基于CIM的数字孪生环境中的协同机制,并通过物理、逻辑与反馈力矩的定义,解决了多源异构模型在不确定性环境下的深度融合问题。再次,建立了基于CIM的五维四视角建模体系,通过物质、空间、性能、文化与时间维度的语义映射,以及场景化、参数化、互动化与智能化视角的交互设计,实现了从物理实体到数字孪生体的全要素映射与认知演化。最后,构建了对象—关系—事件(O-R-E)数据模型,为多维语义数据的动态组织与时序管理提供了底层支撑。本章所建立的理论体系为后续章节的高精度风险评估、动态疏散规划及构件级损伤分析提供了统一的语义标准与计算框架,确保了研究内容的逻辑一致性与数据贯通性。